\documentclass{article}

\usepackage[utf8]{inputenc}
\usepackage[T1]{fontenc}
\usepackage[ngerman]{babel}
\usepackage{amsmath, amsfonts, amsthm, mathtools, amssymb}
\usepackage{stmaryrd}
\usepackage{enumerate}
\usepackage{cases}
\usepackage{fancyhdr}
\usepackage{comment}
%\usepackage{xcolor}
\usepackage{tikz}
\usepackage{pgf}
\usepackage{pgfplots}
\pgfplotsset{compat=1.16}
\usepackage{cases}
\usepackage{listings}
\usepackage{siunitx}
\usepackage[left = 3cm, bottom =3cm]{geometry}
\usepackage[hidelinks]{hyperref}
\usepackage{subcaption}
\usepackage{gauss}
\newtheorem{satz}{Satz}[section]
\newtheorem{lemma}[satz]{Lemma}
\newtheorem{korollar}[satz]{Korollar}
\newtheorem{proposition}[satz]{Proposition}
\theoremstyle{definition}
\newtheorem{definition}[satz]{Def.}
\newtheorem{axiom}[satz]{Axiom}
\newtheorem{bsp}[satz]{Bsp.}
\newtheorem*{anmerkung}{Anm}
\newtheorem{bemerkung}[satz]{Bem}
\newtheorem*{notatio}{Notation}
\newcommand{\obda}{O.B.d.A. }
\newcommand{\equals}{\Longleftrightarrow}
\newcommand{\N}{\mathbb{N}}
\newcommand{\Q}{\mathbb{Q}}
\newcommand{\R}{\mathbb{R}}
\newcommand{\Z}{\mathbb{Z}}
\newcommand{\C}{\mathbb{C}}
\newcommand{\intd}{\mathrm{d}}
\newcommand{\Pot}{\operatorname{Pot}}
\newcommand{\mychar}{\operatorname{char}}
\newcommand{\myker}{\operatorname{ker}}
\newcommand{\induktion}[3]
{\begin{proof}\ \\
	\noindent\textbf{Induktionsanfang:}\ #1\\
	\noindent\textbf{Induktionsvoraussetzung:}\ #2\\
	\noindent\textbf{Induktionsschluss:}\ #3
\end{proof}}

\newcommand{\rg}{\operatorname{rg}}
\newcommand{\im}{\operatorname{im}}
\newcommand{\End}{\operatorname{End}}
\newcommand{\abb}{\operatorname{Abb}}
\newcommand{\re}{\operatorname{Re}}
\newcommand{\Ima}{\operatorname{Im}}



\newcommand{\numlayout}[1]
{	
	\pagestyle{fancy}
	\fancyhead[L]{Einführung in die Numerik, Blatt #1}
	\fancyhead[R]{David Wesner, Josua Kugler}
	\fancypagestyle{firstpage}{%
		\fancyhf{}
		\lhead{Professor: Peter Bastian\\
			Tutor: Ernestine Großmann}
		\rhead{Einführung in die Numerik, Übungsblatt #1\\ David, Josua}
		\cfoot{\thepage}
	}
\thispagestyle{firstpage}
}

\newcommand{\analayout}[1]
{	
	\pagestyle{fancy}
	\fancyhead[L]{Analysis 2, Blatt #1}
	\fancyhead[R]{David Wesner, Josua Kugler}
	\fancypagestyle{firstpage}{%
		\fancyhf{}
		\lhead{Professor: Ekaterina Kostina\\
			Tutor: Julian Matthes}
		\rhead{Analysis 1, Übungsblatt #1\\ David Wesner, Josua Kugler}
		\cfoot{\thepage}
	}
	\thispagestyle{firstpage}
}
\newcommand{\lalayout}[1]
{	
	\pagestyle{fancy}
	\fancyhead[L]{Lineare Algebra 2, Blatt #1}
	\fancyhead[R]{David Wesner, Josua Kugler}
	\fancypagestyle{firstpage}{%
		\fancyhf{}
		\lhead{Professor: Denis Vogel\\
			Tutor: Marina Savarino}
		\rhead{Lineare Algebra 2, Übungsblatt #1\\ David Wesner, Josua Kugler}
		\cfoot{\thepage}
	}
	\thispagestyle{firstpage}
}

\lstset{
    frame=tb, % draw a frame at the top and bottom of the code block
    tabsize=4, % tab space width
    showstringspaces=false, % don't mark spaces in strings
    numbers=left, % display line numbers on the left
    commentstyle=\color{green}, % comment color
    keywordstyle=\color{blue}, % keyword color
    stringstyle=\color{red} % string color
}
\setlength{\headheight}{25pt}

\makeatletter \renewcommand\d[1]{\ensuremath{%
		\;\mathrm{d}#1\@ifnextchar\d{\!}{}}}
\makeatother

% maximum norm
\newcommand{\maxnorm}[1]{\left|\left|#1\right|\right|_\infty}
\renewcommand{\epsilon}{\varepsilon}

\begin{document}
\numlayout{2}
\section*{Aufgabe 1}
\begin{itemize}
	\item $F(x,y)=x\cdot y$. Die partiellen Ableitungen lauten: $\frac{\partial F}{\partial x}=y$ und $\frac{\partial F}{\partial y}=x$. Somit gilt   
		$$\frac{F(x+\triangle x, y + \triangle y)-F(x,y)}{F(x,y)}=\underbrace{\left(\frac{yx}{xy}\right)}_{ k_x= 1}\cdot\frac{\triangle x}{x}+\underbrace{\left(\frac{xy}{xy}\right)}_{k_y=1}\frac{\triangle y}{y}.$$
		$F(x,y)$ ist deswegen gut konditioniert. 
	\item $F(x,y)=\frac{x}{y}$. Die partiellen Ableitungen lauten: $\frac{\partial F}{\partial x}=\frac{1}{y}$ und $\frac{\partial F}{\partial y}=-\frac{x}{y^2}$. Somit gilt:
		$$\frac{F(x+\triangle x, y + \triangle y)-F(x,y)}{F(x,y)}=\underbrace{\left(\frac{xy}{xy}\right)}_{ k_x= 1}\cdot\frac{\triangle x}{x}+\underbrace{\left(\frac{-xy^2}{xy^2}\right)}_{|k_y|=1}\frac{\triangle y}{y}.$$
		$F(x,y)$ ist deswegen gut konditioniert.
	\item $F(x,y)=\sqrt{x}$. Es ist $F'(x)= \frac{1}{2\sqrt{x}}$. Somit gilt:
	$$\frac{F(x+\triangle x)-F(x)}{F(x)}=\underbrace{\left(\frac{1\cdot x}{2\sqrt{x}\sqrt{x}}\right)}_{k_x=\frac{1}{2}}\cdot\frac{\triangle x}{x}.$$
		$F(x,y)$ ist deswegen gut konditioniert.
		
\end{itemize}
\section*{Aufgabe 2}
\begin{figure}
	\begin{tikzpicture}
		\begin{axis}[
			xmin=0.001,
			xmax=0.5,
			ymin=0,
			ymax=0.6,
			smooth,
			samples=200,
			domain=0.001:0.5,
			legend pos=north west,
			width = \textwidth,
			height = 0.6\textwidth,
		]
		\addplot[color=red]{2*1*x^3 +x^2};
		\addlegendentry{$f_a(h) = 2ph^3 + h^2$}
		\addplot[color=orange]{2*x^2};
		\addlegendentry{$c\cdot h^2$ mit $c = \frac{1}{2p}$}
		\addplot[color=blue]{-(x^2)/(ln(x))};
		\addlegendentry{$f_b(h) = -\frac{h^2}{\ln(h)}$\hspace*{4mm}}
		\addplot[color=green]{x^2};
		\addlegendentry{$c\cdot h^2$ mit $c = 1$}
		\end{axis}
	\end{tikzpicture}
	\caption{Wir wählen hier $p = 1$. Für $h < \frac{1}{2p} = 0.5$ lässt sich $f_a(h)$ also wie erwartet durch $\frac{1}{2p}h^2= 0.5h^2$ abschätzen (bei $h=0.5$ schneiden sich die Funktionen). Da $f_b(h)$ sogar $o(h^2)$ ist, genügt ein beliebig kleines $c$, um die Funktion für $h\to 0$ nach oben abzuschätzen, hier ist $c = 1$}
\end{figure}
Wir behandeln die Funktionen $f(h) = (ph^2 + h)^2 - p^2h^4$ in (a), $f(h) = -\frac{h^2}{\ln(h)}$ in (b) und $f(h) = \frac{\sin(x+h) - 2\sin(x) + \sin(x-h)}{h^2} + \sin(x)$ in (c).
\begin{enumerate}[(a)]
	\item Es gilt $f(h) = p^2h^4 + 2ph^3 + h^2 - p^2 h^4 = h^2(2ph +1)$. Für $h \leq \frac{1}{2p}$ gilt $f(h) \leq 2h^2$, also $f(h) = O(h^2)$. Die Aussage $f(h) \leq c\cdot h^3$ lässt sich umformen zu
	$$h^2(2ph+1) \leq ch^3 \equals 2ph +1\leq ch \equals 1 \leq (c-2p)h \equals \frac{1}{c-2p} \leq h$$ und ist daher für $h\to 0$ nicht mehr wahr, folglich ist $f(h) = O(h^2)$ die höchste Potenz, die wir finden können.
	\item Es gilt $\lim\limits_{h\to 0}-\frac{1}{\ln(h)} = 0$ und daher $\forall \varepsilon > 0: \exists h > 0: -\frac{1}{\ln(h)} \leq \varepsilon \equals \forall \varepsilon > 0: \exists h> 0: -\frac{h^2}{\ln(h)} \leq \varepsilon \cdot h^2 \implies f(h) = O(h^2)$. Allerdings ist $\lim\limits_{h\to 0} -\frac{1}{h\ln(h)} = \lim\limits_{h\to 0} \frac{1}{h\ln(\frac{1}{h})} = \lim\limits_{x\to\infty}\frac{x}{\ln(x)} = \infty$. Also gilt: $\forall C > 0: \exists h >0: -\frac{1}{h\ln(h)} \geq C$ und daher auch $\forall C> 0: \exists h>0: -\frac{h^2}{\ln(h)} \geq Ch^3$. Also ist $f(h)$ nicht $O(h^3)$ und somit ist auch hier die größte Potenz, die wir finden können $h^2$. $f(h)$ ist sogar $o(h^2)$, da $f(h)  \leq c(h)h^2$ mit $c(h) = -\frac{1}{\ln(h)} \xrightarrow{h \to 0} 0$.
	\item Wir wenden zunächst die Additionstheoreme an und erhalten
	\begin{align*}
		&\frac{\sin(x+h) - 2\sin(x) + \sin(x-h) + h^2 \sin(x)}{h^2}\\
		=&\frac{\sin(x)\cos(h) + \cos(x)\sin(h) - 2\sin(x) + \sin(x)\cos(h) - \cos(x)\sin(h) + h^2 \sin(x)}{h^2}\\
		=& \sin(x)\frac{2\cos(h) - 2 + h^2}{h^2}
	\end{align*}
	Um $f(0)$, $f'(0)$ und $f''(0)$ zu berechnen, wenden wir die Regel von $L'Hospital$ an. Damit erhalten wir 
	\begin{align*}
		&\lim\limits_{\epsilon\to0}f(\epsilon)\\ 
		=&\lim\limits_{\epsilon\to0}\sin(x) \frac{2\cos(\epsilon) - 2 + \epsilon^2}{\epsilon^2}\\
		=&\sin(x)\lim\limits_{\epsilon\to0} \frac{-2\sin(\epsilon) + 2\epsilon}{2\epsilon}\\
		=&\sin(x)\lim\limits_{\epsilon\to0} \frac{-2\cos(\epsilon) + 2}{2}\\
		=&\sin(x) \lim\limits_{\epsilon\to0} 1 - \cos(\epsilon)\\
		=& 0
	\end{align*}
	Für $f'(0)$ erhalten wir
	\begin{align*}
		&\lim\limits_{\epsilon\to0}f'(\epsilon)\\ 
		=&\lim\limits_{\epsilon\to0}\sin(x) \frac{-2\sin(\epsilon)\epsilon - 2(\cos(\epsilon) -1)\cdot 2}{\epsilon^3}
		\intertext{3-maliges Anwenden von L'Hospital liefert}
		=&\sin(x)\lim\limits_{\epsilon\to0} \sin(\epsilon) + \frac{2}{3}\epsilon \cos(\epsilon)\\
		=&0
	\end{align*}
	Für $f''(0)$ erhalten wir
	\begin{align*}
		&\lim\limits_{\epsilon\to 0}f''(\epsilon)\\
		=&-\sin(x)\lim\limits_{\epsilon\to 0} 4\frac{\epsilon^3\cos(\epsilon) - 3\epsilon^2\sin(\epsilon)}{\epsilon^6} + 2\frac{\epsilon^2\sin(\epsilon)+ 2\epsilon(\cos(\epsilon)-1)}{\epsilon^4}\\
		=&-\sin(x)\lim\limits_{\epsilon\to 0} 4\frac{\epsilon\cos(\epsilon) - 3\sin(\epsilon)}{\epsilon^4} + 2\frac{\epsilon\sin(\epsilon)+ 2(\cos(\epsilon)-1)}{\epsilon^3}\\
		\intertext{Der hintere Term ist gleich $f'(\epsilon)$ und verschwindet daher}
		=& -\sin(x)\lim\limits_{\epsilon\to 0} 4\frac{\epsilon\cos(\epsilon) - 3\sin(\epsilon)}{\epsilon^4}\\
		\intertext{Nach zahlreichem Anwenden von L'Hospital kommt raus, dass hier nicht 0 rauskommt...}
	\end{align*}
	Nun bilden wir die Taylorreihe dieses Ausdrucks an der Stelle $h = 0$.
	\begin{align*}
		\frac{f(\epsilon + h)}{\sin(x)} &= f(\epsilon) + f'(\epsilon)\cdot h + \frac{f''(\epsilon)}{2}h^2 + O(h^3)\\
		&= 0 + 0 \cdot h +O(h^2)\\
	\end{align*}
	Also muss die gesamte Funktion $O(h^2)$ sein, sie kann aber nicht $O(h^3)$ sein, da es Terme $\neq 0$ gibt, die von der Ordnung $h^2$ sind.
	Die Skizze der ersten beiden Funktionen findet sich in Abbildung 1.
\end{enumerate}
\section*{Aufgabe 3}
Es gilt (für $p \neq 1$)$$k_{1/2} = \frac{\partial (p \pm \sqrt{p^2 - 1})}{\partial p}\cdot \frac{p}{x_{1/2}} = \left(1 \pm \frac{p}{\sqrt{p^2-1}}\right)\cdot \frac{p}{x_{1/2}} = \frac{\sqrt{p^2 - 1} \pm p}{\sqrt{p^2 - 1}} \cdot \frac{p}{p \pm \sqrt{p^2 - 1}} = \pm \frac{p}{\sqrt{p^2-1}}$$
Die Lösung liegt in den reellen Zahlen, solange $p \neq 1$ ist.
Die Lösungen des Polynoms $x^2 - \frac{t^2 + 1}{t}x + 1 = 0$ sind, wie man leicht durch Einsetzen verifiziert, $t$ und $\frac{1}{t}$. Also ist hier
$$k_1 = \frac{\partial t_1}{\partial t} \cdot \frac{t}{t_1} = \frac{\partial t}{\partial t}\cdot \frac{t}{t} = 1$$
$$k_2 =  \frac{\partial t_1}{\partial t} \cdot \frac{t}{t_1} = \frac{\partial \frac{1}{t}}{\partial t}\cdot \frac{t}{\frac{1}{t}} = -\frac{1}{t^2} \cdot t^2 = -1$$
Die erste Parametrisierung ist gut konditioniert, wenn $$|\frac{p}{\sqrt{p^2-1}}| \leq 1 \equals \frac{p^2}{p^2-1} \leq 1 \equals p^2 \leq p^2 - 1 \equals 1 \leq 0.$$ Folglich ist die erste Parametrisierung immer schlecht konditioniert.
Die zweite Parametrisierung ist hingegen stets gut konditioniert. Mit der zweiten Parametrisierung hat man also eine wesentlich bessere Parametrisierung gefunden (zumindest hinsichtlich der Konditionierung).
\section*{Aufgabe 4}
\begin{enumerate}[(a)]
	\item \lstinputlisting[language=C++, caption={Programm für die Zeltabbildung}]{tent_map.cc}
	\begin{figure}
		\begin{tikzpicture}
			\begin{axis}
				\addplot[only marks, mark options={scale=0.2}] table[col sep=comma, header = false, x index = 0, y index = 1, ] {daten.dat};
			\end{axis}
		\end{tikzpicture}
		\begin{tikzpicture}
			\begin{axis}
				\addplot[only marks, mark options={scale=0.2}] table[col sep=comma, header = false, x index = 0, y index = 1, ] {daten2.dat};
			\end{axis}
		\end{tikzpicture}
		\caption{Links: Daten generiert mit \lstinline{tent_map.cc}, offensichtlich wird ab ca. $i=60$ nur noch der Wert 0 angenommen. Rechts: Daten generiert mit modifiziertem Programm wie in (c) beschrieben, hier erhält man durchgehend chaotisches Verhalten}
	\end{figure}
	\item Behauptung: Für alle $i \leq r$ sind bei $x_i$ alle Nachkommastellen $m_{r-j} = 0 \forall j\in \Z: j < i$.
	\induktion{$i=0$. Dann sind nach Aufgabenstellung alle Stellen ab $m_{r+1}$ gleich 0}{Sei die Behauptung für $i=k < r$ erfüllt, also alle Nachkommastellen von $x_k$ ab $m_{r+1 - k}$ sind 0.}{Wir betrachten zunächst die $r-k$-te Nachkommastelle von $2 \cdot x_k$. Da alle späteren Nachkommastellen offensichtlich von $x_k$ und daher auch von $2\cdot x_k$ 0 sein müssen, haben sie keinen Einfluss auf die $r-k$-te Nachkommastelle von $2 \cdot x_k$. Ist nun in $x_k$ $m_{r-k} = 0$, so ist $m_{r-k}$ auch bei $2\cdot x_k$ 0. Ist sie hingegen 1, so bleibt trotzdem $2\cdot 1= 0$ in $Z/2Z$. Also ist die $r-k$-te Nachkommastelle von $2\cdot x_k$ 0. Ist $x_k < 0.5$ folgt daraus sofort die Behauptung für $x_{k+1} = 2\cdot x_k$, ist $x_k \geq 0.5$, so ist $x_{k+1} = 2 - 2x_k$. Allerdings sind in der Binärdarstellung von $2$ insbesondere alle Nachkommastellen ab $m_{r-k} = 0$, sodass auch in $2 - 2x_k$ gilt $m_{r-k} = 0$. Also gilt auch für $x_{k+1}$ die Aussage.}
	Mit dieser Aussage erhalten wir nun sofort, dass bei $x_r$ alle Nachkommastellen $m_{r - j} = 0\forall j \in \Z: j < r$, also müssen alle Nachkommastellen $=0$ und damit $x_r$ und $x_{r+1} = 2\cdot x_r = 0$ sein.
	\item Siehe Abbildung 2
\end{enumerate}

\end{document}