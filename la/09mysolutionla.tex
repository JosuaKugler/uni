\documentclass{article}
\usepackage[utf8]{inputenc}
\usepackage[T1]{fontenc}
\usepackage[ngerman]{babel}
\usepackage{amsmath, amsfonts, amsthm, mathtools, amssymb}
\usepackage{stmaryrd}
\usepackage{enumerate}
\usepackage{cases}
\usepackage{fancyhdr}
\usepackage{comment}
%\usepackage{xcolor}
\usepackage{tikz}
\usetikzlibrary{calc,intersections,through,backgrounds}
\usepackage{cases}
\usepackage{listings}
\usepackage{siunitx}
\usepackage[left = 2cm, right = 2cm, top=2.5cm, bottom=2.5cm]{geometry}
\usepackage[hidelinks]{hyperref}
\usepackage{subcaption}
\usepackage{gauss}
\usepackage{environ}
\newtheorem{satz}{Satz}[section]
\newtheorem{lemma}[satz]{Lemma}
\newtheorem{korollar}[satz]{Korollar}
\newtheorem{proposition}[satz]{Proposition}
\theoremstyle{definition}
\newtheorem{definition}[satz]{Def.}
\newtheorem{axiom}[satz]{Axiom}
\newtheorem{bsp}[satz]{Bsp.}
\newtheorem*{anmerkung}{Anm}
\newtheorem{bemerkung}[satz]{Bem}
\newtheorem*{notatio}{Notation}
\newcommand{\obda}{O.B.d.A. }
\newcommand{\equals}{\Longleftrightarrow}
\newcommand{\N}{\mathbb{N}}
\newcommand{\Q}{\mathbb{Q}}
\newcommand{\R}{\mathbb{R}}
\newcommand{\Z}{\mathbb{Z}}
\newcommand{\C}{\mathbb{C}}
\newcommand{\intd}{\mathrm{d}}
\newcommand{\Pot}{\operatorname{Pot}}
\newcommand{\mychar}{\operatorname{char}}
\newcommand{\myker}{\operatorname{ker}}
\newcommand{\induktion}[3]
{\begin{proof}\ \\
	\noindent\textbf{Induktionsanfang:}\ #1\\
	\noindent\textbf{Induktionsvoraussetzung:}\ #2\\
	\noindent\textbf{Induktionsschluss:}\ #3
\end{proof}}

\newcommand{\rg}{\operatorname{rg}}
\newcommand{\im}{\operatorname{im}}
\newcommand{\End}{\operatorname{End}}
\newcommand{\abb}{\operatorname{Abb}}
\newcommand{\re}{\operatorname{Re}}
\newcommand{\Ima}{\operatorname{Im}}
\newcommand{\Fit}{\operatorname{Fit}}
\newcommand{\ggt}{\operatorname{GGT}}
\newcommand{\id}{\operatorname{id}}

\let\oldstackrel\stackrel
\renewcommand{\stackrel}[2]{%
    \oldstackrel{\mathclap{#1}}{#2}
}%
\ExplSyntaxOn

% S-tackrelcompatible ALIGN environment
% some might also call it the S-uper ALIGN environment
% uses regular expressions to calculate the widest stackrel
% to put additional padding on both sides of relation symbols
\NewEnviron{salign}
{
    \begin{align}
        \lec_insert_padding:V \BODY
    \end{align}
}
% starred version that does no equation numbering
\NewEnviron{salign*}
{
    \begin{align*}
        \lec_insert_padding:V \BODY
    \end{align*}
}

% some helper variables
\tl_new:N \l__lec_text_tl
\seq_new:N \l_lec_stackrels_seq
\int_new:N \l_stackrel_count_int
\int_new:N \l_idx_int
\box_new:N \l_tmp_box
\dim_new:N \l_tmp_dim_a
\dim_new:N \l_tmp_dim_b
\dim_new:N \l_tmp_dim_needed

% function to insert padding according to widest stackrel
\cs_new_protected:Nn \lec_insert_padding:n
 {
  \tl_set:Nn \l__lec_text_tl { #1 }
  % get all stackrels in this align environment
  \regex_extract_all:nnN { \c{stackrel}{(.*?)}{(.*?)} } { #1 } \l_lec_stackrels_seq
  % get number of stackrels
  \int_set:Nn \l_stackrel_count_int { \seq_count:N \l_lec_stackrels_seq }
  \int_set:Nn \l_idx_int { 1 }
  \dim_set:Nn \l_tmp_dim_needed { 0pt }
  % iterate over stackrels
  \int_while_do:nn { \l_idx_int <= \l_stackrel_count_int }
  {
      % calculate width of text
      \hbox_set:Nn \l_tmp_box {$\seq_item:Nn \l_lec_stackrels_seq { \l_idx_int + 1 }$}
      \dim_set:Nn \l_tmp_dim_a {\box_wd:N \l_tmp_box}
      % calculate width of relation symbol
      \hbox_set:Nn \l_tmp_box {$\seq_item:Nn \l_lec_stackrels_seq { \l_idx_int + 2 }$}
      \dim_set:Nn \l_tmp_dim_b {\box_wd:N \l_tmp_box}
      % check if 0.5*(a-b) > minimum padding, if yes updated minimum padding
      \dim_compare:nNnTF
        { 1pt * \dim_ratio:nn { \l_tmp_dim_a - \l_tmp_dim_b } { 2pt } } > { \l_tmp_dim_needed }
        { \dim_set:Nn \l_tmp_dim_needed { 1pt * \dim_ratio:nn { \l_tmp_dim_a - \l_tmp_dim_b } { 2pt } } }
        { }
      \quad
      % increment list index by three, as every stackrel produces three list entries
      \int_incr:N \l_idx_int
      \int_incr:N \l_idx_int
      \int_incr:N \l_idx_int
  }
  % replace all relations with align characters (&) and add the needed padding
  \regex_replace_all:nnN
      { (\c{approx}&|&\c{approx}|\c{equiv}&|&\c{equiv}|=&|&=|\c{le}&|&\c{le}|\c{ge}&|&\c{ge}|&\c{stackrel}{.*?}{.*?}|\c{stackrel}{.*?}{.*?}&|&\c{neq}|\c{neq}&) }
      { \c{kern} \u{l_tmp_dim_needed} \1 \c{kern} \u{l_tmp_dim_needed} }
      \l__lec_text_tl
  \l__lec_text_tl
 }
\cs_generate_variant:Nn \lec_insert_padding:n { V }
\ExplSyntaxOff



\newcommand{\lalayout}[1]
{	
	\pagestyle{fancy}
	\fancyhead[L]{Lineare Algebra 1, Blatt #1}
	\fancyhead[R]{David Wesner, Josua Kugler}
	\fancypagestyle{firstpage}{%
		\fancyhf{}
		\lhead{Dozent: Denis Vogel\\
			Tutor: Marina Savarino}
		\rhead{Lineare Algebra 2, Übungsblatt #1\\ David Wesner, Josua Kugler}
		\cfoot{\thepage}
	}
	\thispagestyle{firstpage}
}

\lstset{
    frame=tb, % draw a frame at the top and bottom of the code block
    tabsize=4, % tab space width
    showstringspaces=false, % don't mark spaces in strings
    numbers=left, % display line numbers on the left
    commentstyle=\color{green}, % comment color
    keywordstyle=\color{blue}, % keyword color
    stringstyle=\color{red} % string color
}
\setlength{\headheight}{25pt}
\begin{document}
\lalayout{9}
\section*{Aufgabe 32}
\begin{enumerate}[(a)]
    \item Es gilt \[
              \begin{pmatrix}
                  0 & 2 \\ 1 & 0
              \end{pmatrix}
              \cdot \begin{pmatrix}
                  \sqrt{2} \\
                  1
              \end{pmatrix}
              = \begin{pmatrix}
                  2 \\\sqrt{2}
              \end{pmatrix} = \sqrt{2}\cdot \begin{pmatrix}
                  \sqrt{2} \\
                  1
              \end{pmatrix}
          \]
          und
          \[
              \begin{pmatrix}
                  0 & 3 \\ 1 & 0
              \end{pmatrix}
              \cdot \begin{pmatrix}
                  \sqrt{3} \\
                  1
              \end{pmatrix}
              = \begin{pmatrix}
                  3 \\\sqrt{3}
              \end{pmatrix} = \sqrt{3}\cdot \begin{pmatrix}
                  \sqrt{3} \\
                  1
              \end{pmatrix}
          \]
          \item Wir erhalten \[
                C = \begin{pmatrix}
                    0 & 2 \\ 1 & 0
                \end{pmatrix} \oplus \begin{pmatrix}
                    1 & 0\\0 & 1
                \end{pmatrix} + \begin{pmatrix}
                    1 & 0\\0 & 1
                \end{pmatrix} \oplus \begin{pmatrix}
                    0 & 3 \\ 1 & 0
                \end{pmatrix} =  \begin{pmatrix}
                    0 & 0 & 2 & 0\\
                    0 & 0 & 0 & 2\\
                    1 & 0 & 0 & 0\\
                    0 & 1 & 0 & 0
                \end{pmatrix} + \begin{pmatrix}
                    0 & 3 & 0 & 0\\
                    1 & 0 & 0 & 0\\
                    0 & 0 & 0 & 3\\
                    0 & 0 & 1 & 0
                \end{pmatrix}
                = \begin{pmatrix}
                    0 & 3 & 2 & 0\\
                    1 & 0 & 0 & 2\\
                    1 & 0 & 0 & 3\\
                    0 & 1 & 1 & 0
                \end{pmatrix}
          \]
          \item Für das charakteristische Polynom ergibt sich \[
                \chi^{\operatorname{char}}_C = \det \begin{pmatrix}
                    t & -3 & -2 & 0\\
                    -1 & t & 0 & -2\\
                    -1 & 0 & t & -3\\
                    0 & -1 & -1 & t
                \end{pmatrix} = t^4 - 10t^2 + 1.
          \]
            Es gilt nun $(\sqrt{2} + \sqrt{3})^4 - 10(\sqrt{2} + \sqrt{3})^2 + 1 = 0$. Das muss schon nach Aufgabe 31(c) so sein, da die Einheitsmatrix natürlich die Darstellungsmatrix der Identität ist und wir mit $f$ und $g$ die zugehörigen Endomorphismen der Matrizen $A$ und $B$ bezeichnen können. Als Eigenwert von $f$ erhalten wir $\lambda = \sqrt{2}$ und als Eigenwert von $g$ ergibt sich $\mu = \sqrt{3}$. Damit lässt sich die Aussage direkt anwenden.
\end{enumerate}
\section*{Aufgabe 33}
\begin{enumerate}[(a)]
    \item Definiere $\mu\colon V^n \to K,\; (x_1,\dots, x_n) \mapsto f_1(x_1)\dots f_n(x_n)$. Diese Abbildung ist multilinear, da
          \begin{salign*}
              \mu(x_1,\dots,x_i+ \lambda x_i',\dots,x_n) &= f_1(x_1)\dots f_i(x_i + \lambda x_i')\dots f_n(x_n)\\
              &\stackrel{f_i\text{ linear}}{=} f_1(x_1)\cdot f_i(x_i)\dots f_n(x_n) + \lambda f_1(x_1)\cdot \dots\cdot f_i(x_i') \cdot f_n(x_n)\\
              &= \mu(x_1,\dots,x_i,\dots,x_n) + \lambda \mu(x_1,\dots,x_i',\dots,x_n)
          \end{salign*}
          Nach der universellen Eigenschaft (UM) existiert also eine eindeutige lineare Abbildung $\varphi_{f_1,\dots,f_n}\colon V^{\otimes n} \to K$ mit
          \[\varphi_{f_1,\dots,f_n}(x_1\otimes \dots\otimes x_n) = \mu(x_1, \dots, x_n) = f_1(x_1)\dots f_n(x_n),\] was zu zeigen war.
    \item Definiere $\mu\colon (V^*)^n \to (V^{\otimes n})^*,\; (f_1,\dots, f_n) \mapsto \varphi_{f_1,\dots,f_n}$. Diese Abbildung ist multilinear, da
          \begin{align*}
              \mu(f_1,\dots,f_i+ \lambda f_i',\dots,x_n) & = \varphi_{f_1,\dots,f_i+ \lambda f_i',\dots,f_n}                                                                                                                         \\
                                                         & = \big(x_1\otimes\dots\otimes x_n \mapsto f_1(x_1)\cdots (f_i+ \lambda f_i')(x_i) \cdots f_n(x_n)\big)                                                                    \\
                                                         & = \big(x_1\otimes\dots\otimes x_n \mapsto f_1(x_1)\cdots (f_i(x_i) + \lambda f_i'(x_i)) \cdots f_n(x_n)\big)                                                              \\
                                                         & = \big(x_1\otimes\dots\otimes x_n \mapsto f_1(x_1)\cdots f_i(x_i) \cdots f_n(x_n) + \lambda f_1(x_1)\cdots f_i'(x_i) \cdots f_n(x_n)\big)                                 \\
                                                         & = \big(x_1\otimes\dots\otimes x_n \mapsto f_1(x_1) \cdots  f_n(x_n)\big) + \lambda \big(x_1\otimes\cdots\otimes x_n \mapsto f_1(x_1)\cdots f_i'(x_i) \cdots f_n(x_n)\big) \\
                                                         & = \varphi_{f_1,\dots,f_i,\dots,f_n} + \lambda \varphi_{f_1,\dots, f_i',\dots,f_n}                                                                                         \\
                                                         & = \mu(f_1,\dots,f_i,\dots,x_n) + \lambda \mu(f_1,\dots, f_i',\dots,x_n)
          \end{align*}
          Nach der universellen Eigenschaft (UM) existiert also eine eindeutige lineare Abbildung $\Phi_n\colon (V^*)^{\otimes n} \to (V^{\otimes n})^*$ mit
          \[\Phi_n(f_1\otimes \dots\otimes f_n) = \mu(f_1, \dots, f_n) = \varphi_{f_1,\dots,f_n},\] was zu zeigen war.
    \item Sei $(x_1,\dots,x_n)$ eine Basis von $V$. Dann ist $f\in V^*$ eindeutig durch die Werte auf den Basisvektoren definiert. Wir erhalten daher mit $\psi_i\colon V\to K,\quad x_j \mapsto \delta_{ij}$ eine Basis von $V^*$ (siehe LA1). Es gilt nun für ein $f\in V^*:$
          \[
              f = \sum_{i = 1}^{n} f(x_i)\cdot \psi_i,
          \] da nämlich
          \[
              f(x) = f\left(\sum_{j = 1}^{n}\alpha_jx_j\right) = \sum_{j = 1}^{n}\alpha_jf(x_j) = \sum_{j = 1}^{n}\alpha_j \sum_{i = 1}^{n}f(x_i)\psi_i(x_j) = \sum_{j = 1}^{n}\sum_{i = 1}^{n}\alpha_jf(x_i)\delta_{ij} = \sum_{i = 1}^{n}\alpha_if(x_i).
          \]
          Wir betrachten nun $f\otimes g\in V^*\otimes V^*$ mit $\Phi_2(f\otimes g) = 0$. Dann gilt
          \begin{align*}
              0 & = \Phi_2(f\otimes g)                                 \\
                & = \varphi_{f,g}                                      \\
                & = (x_i\otimes x_j \mapsto f(x_i)\cdot g(x_j))
              \intertext{Damit diese Abbildung gleich der Nullabbildung wird, muss gelten}
              0 & = f(x_i)\cdot g(x_i)\qquad \forall 1\le x_i,x_j\le n
          \end{align*}
          Allerdings gilt auch
          \begin{align*}
              f \otimes g & = \left(\sum_{i = 1}^{n}f(x_i)\psi_i\right) \otimes \left(\sum_{j = 1}^{n}g(x_j)\psi_j\right) \\
                          & = \sum_{i = 1}^{n} \sum_{j = 1}^{n} g(x_j)f(x_i) \big(\psi_i\otimes\psi_j\big)
              \intertext{Da $\Phi_2(f,g) = 0$ ist, muss $f(x_i)\cdot g(x_i)\; \forall 1\le x_i,x_j\le n$ gelten}
                          & = \sum_{i = 1}^{n} \sum_{j = 1}^{n} 0 \big(\psi_i\otimes\psi_j\big)                           \\
                          & = 0
          \end{align*}
          Also ist $\ker \Phi_2 = 0$. Ist nun ein $F\in (V\otimes V)^*$ vorgeben durch die Werte an den Basisvektoren $F(x_i\otimes x_j) = \alpha_{ij}$, so ist durch $G\in V^*\otimes V^*$ mit
          \[
              G = \sum_{i = 1}^{n}\sum_{j = 1}^{n}\alpha_{ij}(\psi_i\otimes\psi_j)
          \] ein Urbild gegeben, da
          \begin{align*}
              \Phi_2(G)(x_k\otimes x_l) & = \Phi_2\left(\sum_{i = 1}^{n}\sum_{j = 1}^{n}\alpha_{ij}(\psi_i\otimes\psi_j)\right)(x_k\otimes x_l) \\
                                        & = \sum_{i = 1}^{n}\sum_{j = 1}^{n}\alpha_{ij}\Phi_2(\psi_i\otimes\psi_j)(x_k\otimes x_l)              \\
                                        & = \sum_{i = 1}^{n}\sum_{j = 1}^{n}\alpha_{ij}\psi_i(x_k)\cdot\psi_j(x_l)                              \\
                                        & = \sum_{i = 1}^{n}\sum_{j = 1}^{n}\alpha_{ij}\delta_{ik}\cdot\delta_{jl}                              \\
                                        & = \alpha_{kl}                                                                                         \\
                                        & = F(x_k\otimes x_l)
          \end{align*}
          Also ist $\Phi_2$ sowohl injektiv als auch surjektiv und damit bijektiv.
\end{enumerate}
\section*{Aufgabe 34}
\begin{enumerate}[(a)]
    \item Sei $y \in \bigwedge^n M$. Dann gilt
    \begin{align*}
        y &= y_1\wedge \dots\wedge y_n
        \intertext{Da $(x_1,\dots,x_n)$ ein Erzeugendensystem von $M$ ist, können wir schreiben}
        &= \sum_{i_1 = 0}^{n}\alpha_{1,i} x_{i_1} \wedge \dots \wedge \sum_{i_n = 0}^{n}\alpha_{n,i} x_{i_n}
        \intertext{Aufgrund der Linearität von $\wedge$ ist das gleich}
        &= \sum_{i_1 = 0}^{n}\alpha_{1,i} \dots \sum_{i_n = 0}^{n}\alpha_{n,i}  \big(x_{i_1} \wedge \dots \wedge x_{i_n}\big)
    \end{align*}
    Da Ausdrücke mit $x_{i_j} = x_{i_k}$ für $k \neq j$ gleich $0$ sind, müssen wir nur Ausdrücke mit paarweise verschiedenen $x_{i_j}$ betrachten. Durch Umsortieren, wobei Vorzeichenwechsel in den Koeffizienten berücksichtigt werden sollen, sehen wir ein,dass es o.B.d.A. genügt, solche Ausdrücke zu summieren, in denen $x_{i_1} < \dots < x_{i_n}$ gilt.
    \item Da $I$ von $2$ und $1 + \sqrt{-5}$ erzeugt wird, genügt es zu zeigen, dass $2 \wedge 1 + \sqrt{-5} = 0$, da aufgrund der Bilinearität von $\wedge$ alle Elemente in $\bigwedge^2 I$ durch Linearkombination mit Skalaren aus $R$ aus diesem Element erzeugt werden können. Es gilt 
    \[
        3 \cdot (2 \wedge (1 + \sqrt{-5})) = 6 \wedge 1 + \sqrt{-5} = (1 + \sqrt{-5})(1 - \sqrt{-5}) \wedge 1 + \sqrt{-5} = (1-\sqrt{-5}) \cdot (1 + \sqrt{-5}\wedge 1 + \sqrt{-5}) = 0  
    \]
    und
    \[
        2 \cdot (2 \wedge (1 + \sqrt{-5})) = 2 \wedge 2\cdot (1 + \sqrt{-5}) = (2\wedge 2)\cdot (1 + \sqrt{-5}) = 0
    \]
    Aus der Differenz der beiden Aussagen folgt die Behauptung.
\end{enumerate}
\section*{Aufgabe 35}
\begin{enumerate}[(a)]
    \item Definiere $\mu\colon M\times M \to M \otimes M,\; (a,b) \mapsto a\otimes b - b \otimes a$.
          Es gilt $\mu(\lambda a, b) = (\lambda a)\otimes b - b\otimes (\lambda a) = \lambda (a\otimes b - b\otimes a) = \lambda\mu(a,b)$ und analog $\mu(a, \lambda b) = \lambda\mu(a,b)$. Außerdem ist $\mu(a + a',b) = (a + a')\otimes b - b\otimes (a + a') = a\otimes b - b\otimes a + a'\otimes b - b\otimes a' = \mu(a,b) + \mu(a',b)$ genauso wie $\mu(a, b + b') = \mu(a,b) + \mu(a, b')$. Schließlich gilt $\mu(a,a) = a\otimes a - a\otimes a = 0$. $\mu$ ist also multilinear und alternierend.
          Nach der universellen Eigenschaft (UA) existiert daher eine eindeutige lineare Abbildung $f\colon \bigwedge^2 M \to M \otimes M$ mit
          \[f(a\wedge b) = \mu(a,b) = a\otimes b - b \otimes a,\] was zu zeigen war.
    \item Sei $x_1,\dots,x_n$ eine Basis von $M$. Es genügt, zu zeigen dass $f(x_i\hat x_j) \neq 0$ ist, da die Familie $(x_i,x_j)_{1\le i\neq j\le n}$ eine Basis von $\bigwedge^2 M$ bildet, sodass wir daraus folgern können, dass $\ker f = 0$ ist. Wir nehmen an, es gäbe $i\neq j$ derart, dass $0 = f(x_i \wedge x_j) = x_i\otimes x_j - x_j\otimes i \implies x_i\otimes x_j = x_j\otimes x_i$. Allerdings bilden $(x_i\otimes x_j)_{1\le i, j\le n}$ eine Basis von $M\otimes M$, sodass $x_i \otimes x_j = x_j\otimes x_i$ ein Widerspruch zur linearen Unabhängigkeit dieser Basis wäre. Damit haben wir unsere Annahme zum Widerspruch geführt und die Behauptung gezeigt.
\end{enumerate}
\end{document}