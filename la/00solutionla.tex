\documentclass{article}

\usepackage[utf8]{inputenc}
\usepackage[T1]{fontenc}
\usepackage[ngerman]{babel}
\usepackage{amsmath, amsfonts, amsthm, mathtools, amssymb}
\usepackage{stmaryrd}
\usepackage{enumerate}
\usepackage{cases}
\usepackage{fancyhdr}
\usepackage{comment}
%\usepackage{xcolor}
\usepackage{tikz}
\usepackage{cases}
\usepackage{listings}
\usepackage{siunitx}
\usepackage[left = 3cm]{geometry}
\usepackage[hidelinks]{hyperref}
\usepackage{subcaption}
\newtheorem{satz}{Satz}[section]
\newtheorem{lemma}[satz]{Lemma}
\newtheorem{korollar}[satz]{Korollar}
\newtheorem{proposition}[satz]{Proposition}
\theoremstyle{definition}
\newtheorem{definition}[satz]{Def.}
\newtheorem{axiom}[satz]{Axiom}
\newtheorem{bsp}[satz]{Bsp.}
\newtheorem*{anmerkung}{Anm}
\newtheorem{bemerkung}[satz]{Bem}
\newtheorem*{notatio}{Notation}
\newcommand{\obda}{O.B.d.A. }
\newcommand{\equals}{\Longleftrightarrow}
\newcommand{\N}{\mathbb{N}}
\newcommand{\Q}{\mathbb{Q}}
\newcommand{\R}{\mathbb{R}}
\newcommand{\Z}{\mathbb{Z}}
\newcommand{\C}{\mathbb{C}}
\newcommand{\intd}{\mathrm{d}}
\newcommand{\Pot}{\operatorname{Pot}}
\newcommand{\mychar}{\operatorname{char}}
\newcommand{\myker}{\operatorname{ker}}
\newcommand{\induktion}[3]
{\begin{proof}\ \\
	\noindent\textbf{Induktionsanfang:}\ #1\\
	\noindent\textbf{Induktionsvoraussetzung:}\ #2\\
	\noindent\textbf{Induktionsschluss:}\ #3
\end{proof}}

\newcommand{\rg}{\operatorname{rg}}
\newcommand{\im}{\operatorname{im}}
\newcommand{\End}{\operatorname{End}}
\newcommand{\abb}{\operatorname{Abb}}
\newcommand{\re}{\operatorname{Re}}
\newcommand{\Ima}{\operatorname{Im}}



\newcommand{\ipilayout}[1]
{	
	\pagestyle{fancy}
	\fancyhead[L]{Einführung in die praktische Informatik, Blatt #1}
	\fancyhead[R]{Josua Kugler, Jan Metzger, David Wesner}
	\fancypagestyle{firstpage}{%
		\fancyhf{}
		\lhead{Professor: Peter Bastian\\
			Tutor: Frederick Schenk}
		\rhead{Einführung in die praktische Informatik, Übungsblatt #1\\ David, Jan, Josua}
		\cfoot{\thepage}
	}
\thispagestyle{firstpage}
}

\newcommand{\analayout}[1]
{	
	\pagestyle{fancy}
	\fancyhead[L]{Analysis 1, Blatt #1}
	\fancyhead[R]{Alexander Bryant, Josua Kugler}
	\fancypagestyle{firstpage}{%
		\fancyhf{}
		\lhead{Professor: Ekaterina Kostina\\
			Tutor: Philipp Elja Müller}
		\rhead{Analysis 1, Übungsblatt #1\\ Alexander Bryant, Josua Kugler}
		\cfoot{\thepage}
	}
	\thispagestyle{firstpage}
}
\newcommand{\lalayout}[1]
{	
	\pagestyle{fancy}
	\fancyhead[L]{Lineare Algebra 1, Blatt #1}
	\fancyhead[R]{David Wesner, Josua Kugler}
	\fancypagestyle{firstpage}{%
		\fancyhf{}
		\lhead{Professor: Denis Vogel\\
			Tutor: Marina Savarino}
		\rhead{Lineare Algebra 2, Übungsblatt #1\\ David Wesner, Josua Kugler}
		\cfoot{\thepage}
	}
	\thispagestyle{firstpage}
}

\lstset{
    frame=tb, % draw a frame at the top and bottom of the code block
    tabsize=4, % tab space width
    showstringspaces=false, % don't mark spaces in strings
    numbers=left, % display line numbers on the left
    commentstyle=\color{green}, % comment color
    keywordstyle=\color{blue}, % keyword color
    stringstyle=\color{red} % string color
}
\setlength{\headheight}{25pt}
\begin{document}
\lalayout{0}
\section{Aufgabe 1}
	Es bezeichne $\abb(\R,\C)$ den $\C$-Vektorraum aller Abbildungen von $\R$ nach $\C$.
	Sei $$V := \{f\in\abb(\R,\C)| \text{ es gibt }a,b,c \in\C \text{ sodass }f(x)=a+bx+cx^2 \text{ für alle }x\in \R\}$$
	\begin{enumerate}[(a)]
		\item \textbf{Z.Z.:} $V$ ist ein endlich-dimensionaler Untervektorraum von $\abb(\R,\C)$
			\begin{proof}
			\end{proof}
		\item \textbf{Z.Z.:} $h: V \times V \longrightarrow \C$ gegeben durch 
		$$h(f,g)=\int_0^1\!f(x)\overline{g(x)}\,dx:= \int_0^1\! \re(f(x)\overline{g(x)}) \,dx+i\int_0^1\! \Ima(f(x)\overline{g(x)}) \,dx\in\C$$
		ist ein Skalarprodukt auf $V$. $(V,h)$ ist ein unitärer Raum.
			\begin{proof}
			\end{proof}
	\end{enumerate}
 
\section{Aufgabe 2}

\section{Aufgabe 3} 
Sei $(V,h)$ ein unitärer Raum. Für $f\in \End(V)$ bezeichne wie in der Vorlesung 
$f^{*}\in \End(V)$ die zu $f$ adjungierte Abbildung. 
\begin{enumerate}[(a)]
	\item \textbf{Z.Z.:} Für alle $f,g\in \End(V)$ gilt: $(f\circ g)^{*} = g^{*} \circ f^{*}$.
		\begin{proof}
			Es gilt für $x,y \in V$: $$h((f\circ g)^{*}(x),y)=h(x,f(g(y)))=h(f^{*}(x),g(y))=h(g^{*}(f^{*}(x)),y)=h((g^{*}\circ f^{*})(x),y)$$
		\end{proof}
	\item \textbf{Z.Z.:} Für alle $f\in\End(V),\lambda \in \C$ gilt: $(\lambda f)^{*} = \overline{\lambda}f^{*}$.
		\begin{proof}
			Es gilt für $x,y\in V$: $$h((\lambda f)^{*}(x),y) = h(x,(\lambda f(y))) \overset{\text{h semilinear}}{=} \overline{\lambda} h(x,f(y))= \overline{\lambda}=h(f^{*}(x),y)$$
		\end{proof}
	\item \textbf{Z.Z.:} Für alle $f\in \End(V)$ gilt: $f\circ f^{*}$ und $f^{*}\circ f$ sind selbstadjungiert.
		\begin{proof} Es gilt für $x,y\in V$: 
			\begin{align*}
			h((f\circ f^{*})^{*}(x),y)&=h((f(f^{*}(x)))^{*},y)=h(x,f(f^{*}(y)))=h(f^{*}(x),f^{*}(y))\\
			&=h(f(f^{*}(x)),y)=h((f\circ f^{*})(x),y)\\
			\end{align*}
			und
			\begin{align*}
			h((f^{*}\circ f)^{*}(x),y)&=h((f^{*}(f(x)))^{*},y)=h(x,f^{*}(f(y)))=h(f(x),f(y))\\
			&=h(f^{*}(f(x)),y)=h((f^{*}\circ f)(x),y)
			\end{align*}
		\end{proof}
\end{enumerate}
		

\end{document}