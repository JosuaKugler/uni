\documentclass{article}

\usepackage[utf8]{inputenc}
\usepackage[T1]{fontenc}
\usepackage[ngerman]{babel}
\usepackage{amsmath, amsfonts, amsthm, mathtools, amssymb}
\usepackage{stmaryrd}
\usepackage{enumerate}
\usepackage{cases}
\usepackage{fancyhdr}
\usepackage{comment}
%\usepackage{xcolor}
\usepackage{tikz}
\usetikzlibrary{calc,intersections,through,backgrounds}
\usepackage{cases}
\usepackage{listings}
\usepackage{siunitx}
\usepackage[left = 2cm, right = 2cm, top=2.5cm, bottom=2.5cm]{geometry}
\usepackage[hidelinks]{hyperref}
\usepackage{subcaption}
\usepackage{gauss}
\usepackage{environ}
\newtheorem{satz}{Satz}[section]
\newtheorem{lemma}[satz]{Lemma}
\newtheorem{korollar}[satz]{Korollar}
\newtheorem{proposition}[satz]{Proposition}
\theoremstyle{definition}
\newtheorem{definition}[satz]{Def.}
\newtheorem{axiom}[satz]{Axiom}
\newtheorem{bsp}[satz]{Bsp.}
\newtheorem*{anmerkung}{Anm}
\newtheorem{bemerkung}[satz]{Bem}
\newtheorem*{notatio}{Notation}
\newcommand{\obda}{O.B.d.A. }
\newcommand{\equals}{\Longleftrightarrow}
\newcommand{\N}{\mathbb{N}}
\newcommand{\Q}{\mathbb{Q}}
\newcommand{\R}{\mathbb{R}}
\newcommand{\Z}{\mathbb{Z}}
\newcommand{\C}{\mathbb{C}}
\newcommand{\intd}{\mathrm{d}}
\newcommand{\Pot}{\operatorname{Pot}}
\newcommand{\mychar}{\operatorname{char}}
\newcommand{\myker}{\operatorname{ker}}
\newcommand{\induktion}[3]
{\begin{proof}\ \\
	\noindent\textbf{Induktionsanfang:}\ #1\\
	\noindent\textbf{Induktionsvoraussetzung:}\ #2\\
	\noindent\textbf{Induktionsschluss:}\ #3
\end{proof}}

\newcommand{\rg}{\operatorname{rg}}
\newcommand{\im}{\operatorname{im}}
\newcommand{\End}{\operatorname{End}}
\newcommand{\abb}{\operatorname{Abb}}
\newcommand{\re}{\operatorname{Re}}
\newcommand{\Ima}{\operatorname{Im}}
\newcommand{\Fit}{\operatorname{Fit}}
\newcommand{\ggt}{\operatorname{GGT}}

\let\oldstackrel\stackrel
\renewcommand{\stackrel}[2]{%
    \oldstackrel{\mathclap{#1}}{#2}
}%
\ExplSyntaxOn

% S-tackrelcompatible ALIGN environment
% some might also call it the S-uper ALIGN environment
% uses regular expressions to calculate the widest stackrel
% to put additional padding on both sides of relation symbols
\NewEnviron{salign}
{
    \begin{align}
        \lec_insert_padding:V \BODY
    \end{align}
}
% starred version that does no equation numbering
\NewEnviron{salign*}
{
    \begin{align*}
        \lec_insert_padding:V \BODY
    \end{align*}
}

% some helper variables
\tl_new:N \l__lec_text_tl
\seq_new:N \l_lec_stackrels_seq
\int_new:N \l_stackrel_count_int
\int_new:N \l_idx_int
\box_new:N \l_tmp_box
\dim_new:N \l_tmp_dim_a
\dim_new:N \l_tmp_dim_b
\dim_new:N \l_tmp_dim_needed

% function to insert padding according to widest stackrel
\cs_new_protected:Nn \lec_insert_padding:n
 {
  \tl_set:Nn \l__lec_text_tl { #1 }
  % get all stackrels in this align environment
  \regex_extract_all:nnN { \c{stackrel}{(.*?)}{(.*?)} } { #1 } \l_lec_stackrels_seq
  % get number of stackrels
  \int_set:Nn \l_stackrel_count_int { \seq_count:N \l_lec_stackrels_seq }
  \int_set:Nn \l_idx_int { 1 }
  \dim_set:Nn \l_tmp_dim_needed { 0pt }
  % iterate over stackrels
  \int_while_do:nn { \l_idx_int <= \l_stackrel_count_int }
  {
      % calculate width of text
      \hbox_set:Nn \l_tmp_box {$\seq_item:Nn \l_lec_stackrels_seq { \l_idx_int + 1 }$}
      \dim_set:Nn \l_tmp_dim_a {\box_wd:N \l_tmp_box}
      % calculate width of relation symbol
      \hbox_set:Nn \l_tmp_box {$\seq_item:Nn \l_lec_stackrels_seq { \l_idx_int + 2 }$}
      \dim_set:Nn \l_tmp_dim_b {\box_wd:N \l_tmp_box}
      % check if 0.5*(a-b) > minimum padding, if yes updated minimum padding
      \dim_compare:nNnTF
        { 1pt * \dim_ratio:nn { \l_tmp_dim_a - \l_tmp_dim_b } { 2pt } } > { \l_tmp_dim_needed }
        { \dim_set:Nn \l_tmp_dim_needed { 1pt * \dim_ratio:nn { \l_tmp_dim_a - \l_tmp_dim_b } { 2pt } } }
        { }
      \quad
      % increment list index by three, as every stackrel produces three list entries
      \int_incr:N \l_idx_int
      \int_incr:N \l_idx_int
      \int_incr:N \l_idx_int
  }
  % replace all relations with align characters (&) and add the needed padding
  \regex_replace_all:nnN
      { (\c{approx}&|&\c{approx}|\c{equiv}&|&\c{equiv}|=&|&=|\c{le}&|&\c{le}|\c{ge}&|&\c{ge}|&\c{stackrel}{.*?}{.*?}|\c{stackrel}{.*?}{.*?}&|&\c{neq}|\c{neq}&) }
      { \c{kern} \u{l_tmp_dim_needed} \1 \c{kern} \u{l_tmp_dim_needed} }
      \l__lec_text_tl
  \l__lec_text_tl
 }
\cs_generate_variant:Nn \lec_insert_padding:n { V }
\ExplSyntaxOff



\newcommand{\lalayout}[1]
{	
	\pagestyle{fancy}
	\fancyhead[L]{Lineare Algebra 1, Blatt #1}
	\fancyhead[R]{David Wesner, Josua Kugler}
	\fancypagestyle{firstpage}{%
		\fancyhf{}
		\lhead{Dozent: Denis Vogel\\
			Tutor: Marina Savarino}
		\rhead{Lineare Algebra 2, Übungsblatt #1\\ David Wesner, Josua Kugler}
		\cfoot{\thepage}
	}
	\thispagestyle{firstpage}
}

\lstset{
    frame=tb, % draw a frame at the top and bottom of the code block
    tabsize=4, % tab space width
    showstringspaces=false, % don't mark spaces in strings
    numbers=left, % display line numbers on the left
    commentstyle=\color{green}, % comment color
    keywordstyle=\color{blue}, % keyword color
    stringstyle=\color{red} % string color
}
\setlength{\headheight}{25pt}
\begin{document}
\lalayout{7}
\section*{Aufgabe 26}
Sei $R$ ein Ring, $M, N$ $R$-Moduln und $\varphi:M\longrightarrow N$ ein $R$-Modulhomomorphismus. Sei $\iota:\ker \varphi\longrightarrow M$ die kanonische Inklusion. \newline
Behauptung: Zu jedem $R$-Modul $U$ und jedem $R$-Modulhomomorphismus $f:U\longrightarrow M$ mit $\varphi\circ f=0$ existiert einen eindeutig bestimmten $R$-Modulhomomorphismus $g:U\longrightarrow \ker \varphi$ mit $f=\iota \circ g.$ 
\begin{proof}
    \begin{enumerate}[(i)]
        \item \begin{align*}
            \intertext{Existenz: Wir definieren die Funktion}
            g:U&\longrightarrow \ker \varphi\\
            u&\longrightarrow f(u).\\
            \intertext{$g$ ist wohldefiniert, denn $\varphi\circ f=0,$ also ist $\im f\subseteq \ker \varphi.$ Weil $f$ ein Modulhomomorphismus ist, ist somit auch $g$ einer. Für $u\in U$ ist $f(u)=g(u)=\iota(g(u))\Leftrightarrow f=\iota \circ g.$}
        \end{align*}
        \item \begin{align*}
            \intertext{Eindeutigkeit von $g$: Seien $g,g':U\longrightarrow \ker \varphi$ zwei $R$-Modulhomomorphismen mit $f=\iota\circ g=\iota\circ g'.$ Dann gilt für alle $u\in U:$}
                g'(u)s=\iota(g'(u))&=f(u)=\iota(g(u))=g(u).\\
                \intertext{Es gilt also }
                g&=g'
        \end{align*}
        
    \end{enumerate}
\end{proof}
\section*{Aufgabe 27}
\begin{enumerate}[(a)]
    \item Das ist wörtlich die universellen Eigenschaft freier Moduln (UF) für das Tupel $(M_i, (x_{i,j})_{j\in J_i})$.
    \item Wir zeigen, dass $M$ frei ist mit Basis $q_i(x_{i,j})_{(i,j)\in K}$. Sei also ein $R$-Modul $N$ und eine Familie $(y_{i,j})_{(i,j)\in K}$. Zu zeigen: $\exists ! f : M \to N$ mit $f(q_i(x_{i,j})) = y_{i,j} \forall (i,j)\in K$.
    \begin{proof}
        Nach Aufgabe $a$ gibt es für alle $i\in I$ einen eindeutigen $R$-Modulhomomorphismus $f_i : M_i \to N$ mit $f_i(x_{i,j})= y_{i,j}$ für alle $j\in J_i$. Die universelle Eigenschaft der direkten Summe besagt, dass für jeden $R$-Modul $N$ und jede Familie $(f_i)_{i\in I}$ von $R$-Modulhomomorphismen $f_i:N\mapsto M_i$ genau ein $R$-Modulhomomorphismus $f:N\mapsto M $ mit $f_i=f\circ q_i$ für alle $i\in J$ existiert, insbesondere also auch für die Familie $(f_i)_{i\in i}$ aus Teilaufgabe~(a). Es gilt demnach \[f(q_i(x_{i,j})) = f_i(x_{i,j}) = y_{i,j}.\]
        Also erfüllt $f$ die geforderte Eigenschaft. Angenommen, es gäbe nun ein $f'$ mit $f'(q_i(x_{i,j})) = y_{i,j}$. Sei dann $f_i' = f'\circ q_i$. Dann gilt $f_i' (x_{i,j}) = y_{i,j}$. Allerdings ist nach Teilaufgabe~(a) $f_i$ eindeutig über diese Eigenschaft bestimmt. Also ist $f_i' = f_i$. Also gilt $f_i = f' \circ q_i$. Nach der universellen Eigenschaft der direkten Summe ist $f$ aber darüber eindeutig definiert, also ist $f' = f$. 
    \end{proof}
\end{enumerate}
\end{document}