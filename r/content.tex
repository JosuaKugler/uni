\documentclass{article}
\usepackage[margin=2cm]{geometry}
\usepackage{enumitem}
\setlist{noitemsep}

\begin{document}
\section{Basics}
\begin{enumerate}
    \item (1) Intro
    \item (2) Operatoren
    \item (4) Variablen (typen, gültige Namen)
    \item (6) atomare vektoren (seq, rep, indizierung, NA)
    \item (11) text (print, cat)
    \item (12) funktionen
    \item (13) listen (str, c, unlist, as.list)
    \item (16) if, while, for (seq\_along, letters)
    \item (18) Zufall
    \item (20) summary statistics (mean, median, var, sd, max, which.max, summary)
    \item (21) Matrizen
    \item (23) basegraphics plot
    \item (26) pakete
    \item (27) tibble (erstellung, indizierung, 
    \item (28) apply Funktionen (lapply, sapply, apply(mat, 1/2, f)
    \item (29) Statistik
\end{enumerate}


\section{Datenstrukturen}
\begin{enumerate}
    \item (1) Objekte (typeof, class, length)
    \item (2) Vektoren 
    \begin{itemize}
        \item atomare (basistypen, initialisierung BASETYPE(n), typtest)
        \item logical(3), integer(4), double(4), character(6) (letters, month.name)
        \item coercion(8)
        \item listen(8) (initialisierung list()
    \end{itemize}
    \item (10) Attribute (attr, attributes, structure, str)
    \subitem names (11) (atomare vektoren oder listen)
    \item (14) Matrizen und Arrays (dim, nrow, ncol, rbind, cbind, abind::abind())
    \item (19) S3-Objekte (factor, date time zeug)
    \item (22) Tibbles
    \item (25) Na zeug
\end{enumerate}


\section{Subsetting}
\begin{enumerate}
    \item (1) Intro
    \item (1) mehrere Elemente
    \begin{itemize}
    \item 2.1(1) atomare Vektoren
    \item 2.2(4) Arrays	
    \item 2.3(8) Tibbles
    \end{itemize}
    \item (9) einzelne Elemente ([[, \$]])
    \item (11) fehlende indizes
    \item (12) subsetting und zuweisung, out-of-bounds-zuweisung 
    \item (15) Anwendungen
    \begin{itemize}
        \item (15) Matching und Merging
        \item (16) Random Samples
        \item (17) Sortieren (sort, order, rank)
        \item (18) Aggregierte Zeilen expandieren
        \item (18) Spalten aus Tibble entfernen
        \item (19) Konditionales Subsetting
        \item (19) Boolean/Integer Subsetting (which)
    \end{itemize}
\end{enumerate}

\section{Funktionen}
\begin{enumerate}
\item (1) nomenklatur, syntax
\item (3) Funktionsaufruf (präfix, infix, ersetzungsfunktionen, vordefinierte spezialfunktionen, do.call)
\item (7) Klassifizierung (class function, typeof closure/builtin/special)
\item (8) Komponenten
\item (9) Funktionsargumente (lazy evaluation, default arguments, missing arguments, ...)
\item (12) Beenden einer Funktion
\end{enumerate}

\section{stringr}
\begin{enumerate}
    \item (1) Intro
    \item (1) Basics (str\_length, str\_c, str\_sub, str\_trim, str\_squish, str\_to\_upper, ...)
    \item (3) Formatieren (str\_glue, str\_glue\_data, ...)
    \item (4) regexp (str\_view, str\_view\_all)
        \subitem (5) Funktionen (str\_detect, str\_subset, str\_which, str\_count)
        \subitem (7) Syntax (siehe stringr\_cheatsheet.pdf)
\end{enumerate}



\section{tibble}
\begin{enumerate}
    \item (1) Tabellen (erstellen, as\_tibble, add\_column, add\_row, subsetting, bind\_rows, bind\_cols)
    \item (6) Relationale Datenbanken (union, intersect, setdiff, crossing)
\end{enumerate}

\section{dplyr}
\begin{enumerate}
    \item (1) Verben für eine Tabelle (filter, arrange, select, mutate, summarize, pipe, group\_by, column/row-wise operations: across/rowwise, non-standard-evaluation)
    \item (23) Verben für zwei Tabellen (mutating: left/right/full/inner\_join, filtering: semi/anti\_join)
    \item (28) Relationale Datenbanken
\end{enumerate}

\section{tidyr}
\begin{itemize}
    \item 1.1(3) pivoting
    \item 1.2(5) separate and unite
    \item 1.3(7) Fehlende Werte
    \item 2(9) Normalisierung Relationaler Datenbanken (mega unnötig)
\end{itemize}

\section{Umgebungen}
\begin{enumerate}
\item (2) Umgebungsobjekte (rlang ist auch klausurrelevant)
\item (6) Umgebungsdiagramme
\item (8) Funktionsumgebungen (funktions-, aufrufende, ausführungs-)
\item (14) Scoping-Prinzipien (überdeckung, dynamic lookup, ausführungsumgebung)
\item (17) Zuweisungsoperatoren (<-, <<-, ->, ->>, =)
\item (18) Elternumgebungen, suchpfad
\item (19) paketumgebungen
\item (22) Übersicht
\item (22) Lazy evaluation
\end{enumerate}

\section{Funktionale Programmierung}
\begin{enumerate}
    \item (1) Funktionale Programmierung, rein(3), 
    \item (5) Funktionale (apply-funktionen, rep, outer(12), integrate, optim)
    \item (15) Funktionsfabriken (Counter, MLS, Interpolation)
    \item (22) Funktionsoperatoren
\end{enumerate}

\section{Objektorientierte Programmierung}
\begin{enumerate}
\item (1) Intro
\item (3) S3
\item (13)S4
\item (18)R6
\end{enumerate}

\section{Meta-Programmierung}
\begin{enumerate}
    \item (2) Ausdrücke, Syntaxbaum, parse, base-R 
    \item (7) Ausdrücke verwenden, enexpr, eval (mit Umgebung), quosures(9)
    \item (11)Anwendung
    \item (12)tidy-eval-framework, !!, !!!, {{}}, :=
\end{enumerate}
\end{document}