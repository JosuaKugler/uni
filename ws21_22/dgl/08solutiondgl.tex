\documentclass{article}
\usepackage{josuamathheader}
\begin{document}
\noindent
    Für das linearisierte System gilt
    \[
        \begin{cases}
            r' &= \frac{1}{2} r\\
            \theta' &= 1
        \end{cases}
    \]
    Beide Differentialgleichungen besitzen offensichtlich eine eindeutige Lösung, wir erhalten
    \[
        \psi_t \begin{pmatrix}
            r\\\theta
        \end{pmatrix} = \begin{pmatrix}
            r \cdot e^{\frac{1}{2}t}\\
            \theta + t
        \end{pmatrix}.
    \]
    Da hier stets $r \in (0,1)$ liegt, ist $\tau(r, \theta)$ wohldefiniert.
    Wir führen eine kurze Nebenrechnung durch:
    \begin{align*}
        \tau(\phi_t(r, \theta)) &= \log \left(\frac{1 - \frac{r^2e^t}{1 - r^2 + r^2e^t}}{3\frac{r^2e^t}{1 - r^2 + r^2e^t}}\right)\\
        &= \log \left(\frac{1 - r^2 + r^2e^t - r^2 e^t}{3r^2e^t}\right)\\
        &= \log\left(\frac{1 - r^2}{3r^2}\right) - \log(e^t)\\
        &= \tau(r, \theta) - t
    \end{align*}
    Insbesondere ist auch $\tau(\phi_t(r, \theta))$ ein sinnvoller Wert.
    Es gilt nun
    \begin{align*}
        h \circ \phi_t &= \psi_{- \tau(\phi_t(r, \theta))} \circ \phi_{\tau(\phi_t(r, \theta))} \left( \phi_t(r, \theta)\right)\\
        \intertext{Nach Teilaufgabe (iv) und (iii) folgt}
        &= \psi_{- \tau(\phi_t(r, \theta))} \begin{pmatrix}
            \frac{1}{2}\\ \tau(\phi_t(r, \theta)) + t + \theta
        \end{pmatrix}\\
        &= \begin{pmatrix}
            \frac{1}{2}e^{-\frac{1}{2}\tau(\phi_t(r, \theta))}\\
            - \tau(\phi_t(r, \theta)) + \tau(\phi_t(r, \theta)) + t + \theta
        \end{pmatrix}
        \intertext{Unter Benutzung der Nebenrechung erhalten wir}
        &= \begin{pmatrix}
            \frac{1}{2}e^{-\frac{1}{2}\tau(r, \theta) + \frac{1}{2}t}\\
            t + \theta
        \end{pmatrix}\\
        &= \psi_t\begin{pmatrix}
            \frac{1}{2}e^{-\frac{1}{2}\tau(r, \theta)}\\
            \theta
        \end{pmatrix}\\
        &= \psi_t\circ \psi_{-\tau(r, \theta)}\begin{pmatrix}
            \frac{1}{2}\\
            \tau(r, \theta) + \theta
        \end{pmatrix}
        \intertext{Wieder nutzen wir Teilaufgabe 4}
        &= \psi_t \circ \psi_{-\tau(r, \theta)} \circ \phi_{\tau(r, \theta)}(r, \theta)\\
        &= \psi_t \circ h
    \end{align*}
    Folglich sind $\psi_t$ und $\phi_t$ auf der Einheitskreisscheibe konjugiert.
\end{document}