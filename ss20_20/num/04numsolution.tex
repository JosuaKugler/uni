\documentclass{article}

\usepackage[utf8]{inputenc}
\usepackage[T1]{fontenc}
\usepackage[ngerman]{babel}
\usepackage{amsmath, amsfonts, amsthm, mathtools, amssymb}
\usepackage{stmaryrd}
\usepackage{enumerate}
\usepackage{cases}
\usepackage{fancyhdr}
\usepackage{comment}
%\usepackage{xcolor}
\usepackage{tikz}
\usepackage{pgf}
\usepackage{pgfplots}
\pgfplotsset{compat=1.16}
\usepackage{cases}
\usepackage{listings}
\usepackage{siunitx}
\usepackage[left = 3cm, bottom =3cm]{geometry}
\usepackage[hidelinks]{hyperref}
\usepackage{subcaption}
\usepackage{gauss}

\usepackage{environ}
\newtheorem{satz}{Satz}[section]
\newtheorem{lemma}[satz]{Lemma}
\newtheorem{korollar}[satz]{Korollar}
\newtheorem{proposition}[satz]{Proposition}
\theoremstyle{definition}
\newtheorem{definition}[satz]{Def.}
\newtheorem{axiom}[satz]{Axiom}
\newtheorem{bsp}[satz]{Bsp.}
\newtheorem*{anmerkung}{Anm}
\newtheorem{bemerkung}[satz]{Bem}
\newtheorem*{notatio}{Notation}
\newcommand{\obda}{O.B.d.A. }
\newcommand{\equals}{\Longleftrightarrow}
\newcommand{\N}{\mathbb{N}}
\newcommand{\Q}{\mathbb{Q}}
\newcommand{\R}{\mathbb{R}}
\newcommand{\Z}{\mathbb{Z}}
\newcommand{\C}{\mathbb{C}}
\newcommand{\intd}{\mathrm{d}}
\newcommand{\Pot}{\operatorname{Pot}}
\newcommand{\mychar}{\operatorname{char}}
\newcommand{\myker}{\operatorname{ker}}
\newcommand{\induktion}[3]
{\begin{proof}\ \\
	\noindent\textbf{Induktionsanfang:}\ #1\\
	\noindent\textbf{Induktionsvoraussetzung:}\ #2\\
	\noindent\textbf{Induktionsschluss:}\ #3
\end{proof}}

\newcommand{\rg}{\operatorname{rg}}
\newcommand{\im}{\operatorname{im}}
\newcommand{\End}{\operatorname{End}}
\newcommand{\abb}{\operatorname{Abb}}
\newcommand{\re}{\operatorname{Re}}
\newcommand{\Ima}{\operatorname{Im}}

\let\oldstackrel\stackrel
\renewcommand{\stackrel}[2]{%
    \oldstackrel{\mathclap{#1}}{#2}
}%


\newcommand{\numlayout}[1]
{	
	\pagestyle{fancy}
	\fancyhead[L]{Einführung in die Numerik, Blatt #1}
	\fancyhead[R]{David Wesner, Josua Kugler}
	\fancypagestyle{firstpage}{%
		\fancyhf{}
		\lhead{Professor: Peter Bastian\\
			Tutor: Ernestine Großmann}
		\rhead{Einführung in die Numerik, Übungsblatt #1\\ David, Josua}
		\cfoot{\thepage}
	}
\thispagestyle{firstpage}
}

\newcommand{\analayout}[1]
{	
	\pagestyle{fancy}
	\fancyhead[L]{Analysis 2, Blatt #1}
	\fancyhead[R]{David Wesner, Josua Kugler}
	\fancypagestyle{firstpage}{%
		\fancyhf{}
		\lhead{Professor: Ekaterina Kostina\\
			Tutor: Julian Matthes}
		\rhead{Analysis 1, Übungsblatt #1\\ David Wesner, Josua Kugler}
		\cfoot{\thepage}
	}
	\thispagestyle{firstpage}
}
\newcommand{\lalayout}[1]
{	
	\pagestyle{fancy}
	\fancyhead[L]{Lineare Algebra 2, Blatt #1}
	\fancyhead[R]{David Wesner, Josua Kugler}
	\fancypagestyle{firstpage}{%
		\fancyhf{}
		\lhead{Professor: Denis Vogel\\
			Tutor: Marina Savarino}
		\rhead{Lineare Algebra 2, Übungsblatt #1\\ David Wesner, Josua Kugler}
		\cfoot{\thepage}
	}
	\thispagestyle{firstpage}
}

\lstset{
    frame=tb, % draw a frame at the top and bottom of the code block
    tabsize=4, % tab space width
    showstringspaces=false, % don't mark spaces in strings
    numbers=left, % display line numbers on the left
    commentstyle=\color{green}, % comment color
    keywordstyle=\color{blue}, % keyword color
    stringstyle=\color{red} % string color
}
\setlength{\headheight}{25pt}

\makeatletter \renewcommand\d{\ensuremath{%
		\;\mathrm{d}}}
\makeatother

\ExplSyntaxOn

% S-tackrelcompatible ALIGN environment
% some might also call it the S-uper ALIGN environment
% uses regular expressions to calculate the widest stackrel
% to put additional padding on both sides of relation symbols
\NewEnviron{salign}
{
    \begin{align}
        \lec_insert_padding:V \BODY
    \end{align}
}
% starred version that does no equation numbering
\NewEnviron{salign*}
{
    \begin{align*}
        \lec_insert_padding:V \BODY
    \end{align*}
}

% some helper variables
\tl_new:N \l__lec_text_tl
\seq_new:N \l_lec_stackrels_seq
\int_new:N \l_stackrel_count_int
\int_new:N \l_idx_int
\box_new:N \l_tmp_box
\dim_new:N \l_tmp_dim_a
\dim_new:N \l_tmp_dim_b
\dim_new:N \l_tmp_dim_needed

% function to insert padding according to widest stackrel
\cs_new_protected:Nn \lec_insert_padding:n
 {
  \tl_set:Nn \l__lec_text_tl { #1 }
  % get all stackrels in this align environment
  \regex_extract_all:nnN { \c{stackrel}{(.*?)}{(.*?)} } { #1 } \l_lec_stackrels_seq
  % get number of stackrels
  \int_set:Nn \l_stackrel_count_int { \seq_count:N \l_lec_stackrels_seq }
  \int_set:Nn \l_idx_int { 1 }
  \dim_set:Nn \l_tmp_dim_needed { 0pt }
  % iterate over stackrels
  \int_while_do:nn { \l_idx_int <= \l_stackrel_count_int }
  {
      % calculate width of text
      \hbox_set:Nn \l_tmp_box {$\seq_item:Nn \l_lec_stackrels_seq { \l_idx_int + 1 }$}
      \dim_set:Nn \l_tmp_dim_a {\box_wd:N \l_tmp_box}
      % calculate width of relation symbol
      \hbox_set:Nn \l_tmp_box {$\seq_item:Nn \l_lec_stackrels_seq { \l_idx_int + 2 }$}
      \dim_set:Nn \l_tmp_dim_b {\box_wd:N \l_tmp_box}
      % check if 0.5*(a-b) > minimum padding, if yes updated minimum padding
      \dim_compare:nNnTF
        { 1pt * \dim_ratio:nn { \l_tmp_dim_a - \l_tmp_dim_b } { 2pt } } > { \l_tmp_dim_needed }
        { \dim_set:Nn \l_tmp_dim_needed { 1pt * \dim_ratio:nn { \l_tmp_dim_a - \l_tmp_dim_b } { 2pt } } }
        { }
      \quad
      % increment list index by three, as every stackrel produces three list entries
      \int_incr:N \l_idx_int
      \int_incr:N \l_idx_int
      \int_incr:N \l_idx_int
  }
  % replace all relations with align characters (&) and add the needed padding
  \regex_replace_all:nnN
      { (\c{approx}&|&\c{approx}|\c{equiv}&|&\c{equiv}|=&|&=|\c{le}&|&\c{le}|\c{ge}&|&\c{ge}|&\c{stackrel}{.*?}{.*?}|\c{stackrel}{.*?}{.*?}&|&\c{neq}|\c{neq}&) }
      { \c{kern} \u{l_tmp_dim_needed} \1 \c{kern} \u{l_tmp_dim_needed} }
      \l__lec_text_tl
  \l__lec_text_tl
 }
\cs_generate_variant:Nn \lec_insert_padding:n { V }
\ExplSyntaxOff


% norm
\newcommand{\norm}[1]{\left\Vert#1\right\Vert}
\newcommand{\maxnorm}[1]{\norm{#1}_\infty}
\renewcommand{\epsilon}{\varepsilon}

\begin{document}
\numlayout{4}
\section*{Aufgabe 1}
\begin{enumerate}[a)]
	\item Beträge sind stets größer 0, also ist insbesondere auch $\norm{f} = \max\limits_{x \in [0,1]}|f(x)| > 0$. Gilt andererseits $\norm{f} = \max\limits_{x \in [0,1]}|f(x)| = 0$, so muss $\forall x\in [0,1]: |f(x)| = 0 \implies f \equiv 0$ auf $[0,1]$. Nach der positiven Definitheit folgt auch die Homogenität sehr leicht:
	$$\max\limits_{x \in [0,1]}|a\cdot f(x)| = |a| \cdot \max\limits_{x \in [0,1]}|f(x)| = |a| \cdot \norm{f}$$
	Die Dreiecksungleichung folgt sofort aus der Dreiecksungleichung für Beträge:
	$$\norm{f + g} = \max\limits_{x \in [0,1]}|f(x) + g(x)| \leq \max\limits_{x \in [0,1]}|f(x)| + \max\limits_{x \in [0,1]}|g(x)| = \norm{f} + \norm{g}$$
	\item Beträge sind stets größer 0, also ist insbesondere auch $\norm{f} = \int_0^1 |f(x)| \d x > 0$. Gilt andererseits $\norm{f} = \int_0^1 |f(x)| \d x| = 0$, so muss aufgrund der Definitheit des Integrals und des Betrags $f \equiv 0$ auf $[0,1]$ sein. Nach der positiven Definitheit folgt auch die Homogenität sehr leicht:
	$$\int_0^1 |a\cdot f(x)| \d x = |a| \cdot \int_0^1 |f(x)| \d x= |a| \cdot \norm{f}$$
	Die Dreiecksungleichung folgt sofort aus der Dreiecksungleichung für Beträge:
	$$\norm{f + g} = \int_0^1 |a\cdot |f(x) + g(x)| \d x \leq \int_0^1 |f(x)| \d x + \int_0^1 |g(x)| \d x = \norm{f} + \norm{g}$$
	\item Es gilt $\sin(x) \leq 1 \forall x$. Außerdem ist $u_k\left(\frac{x_k + x_{k+1}}{2}\right) = \sin\left(\frac{\frac{2x_k - x_k - x_{k+1}}{2}}{x_k -x_{k+1}}\cdot \pi\right) = \sin\left(\frac{\pi}{2}\right) = 1$.
	Also ist $\norm{u_k}_\infty = 1 \forall k\in \N$. 
	Um die 1-Norm zu berechnen, müssen wir zunächst zeigen, dass $u_k$ stetig ist. 
	Konstante Funktionen sowie der Sinus sind stetig, also bleiben noch die Punkte $x_k$ und $x_{k+1}$. 
	Es gilt $u_k(x_k) = 0 = u_k(x_{k+1})$. 
	Mit $\lim\limits_{x \searrow x_k}0 = 0$ und $\lim\limits_{x\nearrow x_{k+1}}0 = 0$ folgt die Stetigkeit von $u_k$. 
	Nun können wir das Integral ausführen und erhalten
	\begin{salign*}
		\norm{u_k}_1 &= \int_0^{x_{k+1}} 0 \d x+ \int_{xk}^1 0 \d x + \int_{x{k+1}}^{x_{k+1}} \sin\left(\frac{x_k -x}{x_k-x_{k+1}}\cdot \pi\right) \d x\\
		&= \left[\frac{x_k - x_{k+1}}{\pi}\cos\left(\frac{x_k -x}{x_k-x_{k+1}}\cdot \pi\right)\right]_{x_{k+1}}^{x_k}\\
		&= \frac{x_k - x_{k+1}}{\pi} \cos(0) - \frac{x_k - x_{k+1}}{\pi}\cos(\pi)\\
		&= 2 \frac{x_k - x_{k+1}}{\pi}
	\end{salign*}
	Also ist $\lim\limits_{k\to\infty}\norm{u_k}_1 = \lim\limits_{k\to\infty}2 \frac{x_k - x_{k+1}}{\pi} = \frac{2}{\pi} \lim\limits_{k\to\infty} \frac{1}{k} - \frac{1}{k+1} = 0$.
\end{enumerate}
\section*{Aufgabe 2}
\begin{enumerate}[(i)]
	\item Da stets der Betrag genommen wird, ist die Norm stets positiv. Da über alle Werte summiert wird, bedeutet $\norm{A}_F = 0$, dass alle Einträge 0 sind, also $A = 0$. Damit haben wir bereits die positive Definitheit gezeigt. Die Homogenität gilt wegen $$\norm{b\cdot A}_F = \left(\sum_{i,j = 1}^{n}|b\cdot a_{ij}|^2\right)^\frac{1}{2} = |b| \cdot \left(\sum_{i,j = 1}^{n}|a_{ij}|^2\right)^\frac{1}{2} = |b| \cdot \norm{A}_F.$$ Nun müssen wir nur noch die Dreiecksungleichung zeigen. Es gilt 
	\begin{align*}
		\norm{A + B}_F^2 &= \sum_{i,j = 1}^{n}|a_{ij} + b_{ij}|^2\\
		&\leq \sum_{i,j = 1}^{n} |a_{ij}|^2 + 2\sum_{i,j = 1}^{n} |a_{ij}b_{ij}| + \sum_{i,j = 1}^{n} |b_{ij}|^2\\
		&\stackrel{\text{C.S.U.}}{\leq} \sum_{i,j = 1}^{n} |a_{ij}|^2 + 2 \cdot \sqrt{\sum_{i,j = 1}^{n} |a_{ij}|^2 \cdot \sum_{i,j = 1}^{n} |b_{ij}|^2} + \sum_{i,j = 1}^{n} |b_{ij}|^2\\
		&= \left(\sqrt{\sum_{i,j = 1}^{n} |a_{ij}|^2} + \sqrt{\sum_{i,j = 1}^{n} |b_{ij}|^2}\right)^2\\
		&= \left(\norm{A}_F + \norm{B}_F\right)^2
		\intertext{Wurzelziehen ergibt}
		\norm{A + B}_F &\leq \norm{A}_F + \norm{B}_F
	\end{align*}
	\item Es gilt
	\begin{salign*}
		\norm{Ax}_2^2 &= \sum_{j = 1}^{n} \left(\sum_{i = 1}^{n} a_{ij}x_i\right)^2\\
		&\stackrel{\text{C.S.U.}}{\leq} \sum_{j = 1}^{n} \left(\sum_{i = 1}^{n} a_{ij}^2 \cdot \underbrace{\sum_{i = 1}^{n}x_i^2}_{\text{unabhängig von $j$}}\right)\\
		&=\sum_{i = 1}^{n}x_i^2 \cdot \sum_{i,j = 1}^{n}a_{ij}^2\\
		&= \norm{x}_2 \cdot \norm{A}_F
	\end{salign*}
	\item 
	\begin{salign*}
		\norm{A\cdot B}_F^2 &= \sum_{i,j = 1}^{n} \left(\sum_{k = 1}^{n} a_{ik}b_{kj}\right)^2\\
		&\stackrel{\text{C.S.U.}}{\leq} \sum_{i,j = 1}^{n} \left(\underbrace{\sum_{k = 1}^{n} a_{ik}^2}_{\text{unabhängig von $j$}} \cdot \underbrace{\sum_{i = 1}^{n}b_{kj}^2}_{\text{unabhängig von $i$}}\right)\\
		&=\sum_{i,k = 1}^{n}a_{ik}^2 \cdot \sum_{k,j = 1}^{n}b_{kj}^2\\
		&= \norm{A}_F \cdot \norm{B}_F
	\end{salign*}
\end{enumerate}
\section*{Aufgabe 3}
Kommt dann nächste Woche.
\end{document}