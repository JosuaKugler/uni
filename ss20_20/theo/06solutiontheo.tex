\documentclass{article}

\usepackage[utf8]{inputenc}
\usepackage[T1]{fontenc}
\usepackage[ngerman]{babel}
\usepackage{amsmath, amsfonts, amsthm, mathtools, amssymb}
\usepackage[arrowdel]{physics} % derivatives
\usepackage{stmaryrd}
\usepackage{enumerate}
\usepackage{cases}
\usepackage{fancyhdr}
\usepackage{comment}
%\usepackage{xcolor}
\usepackage{tikz}
\usepackage{pgf}
\usepackage{pgfplots}
\pgfplotsset{compat=1.16}
\usepackage{cases}
\usepackage{listings}
\usepackage{siunitx}
\usepackage[left = 3cm]{geometry}
\usepackage[hidelinks]{hyperref}
\usepackage{subcaption}
\usepackage{gauss}
\newtheorem{satz}{Satz}[section]
\newtheorem{lemma}[satz]{Lemma}
\newtheorem{korollar}[satz]{Korollar}
\newtheorem{proposition}[satz]{Proposition}
\theoremstyle{definition}
\newtheorem{definition}[satz]{Def.}
\newtheorem{axiom}[satz]{Axiom}
\newtheorem{bsp}[satz]{Bsp.}
\newtheorem*{anmerkung}{Anm}
\newtheorem{bemerkung}[satz]{Bem}
\newtheorem*{notatio}{Notation}
\newcommand{\obda}{O.B.d.A. }
\newcommand{\equals}{\Longleftrightarrow}
\newcommand{\N}{\mathbb{N}}
\newcommand{\Q}{\mathbb{Q}}
\newcommand{\R}{\mathbb{R}}
\newcommand{\Z}{\mathbb{Z}}
\newcommand{\C}{\mathbb{C}}
\newcommand{\intd}{\mathrm{d}}
\newcommand{\Pot}{\operatorname{Pot}}
\newcommand{\mychar}{\operatorname{char}}
\newcommand{\myker}{\operatorname{ker}}
\newcommand{\induktion}[3]
{\begin{proof}\ \\
	\noindent\textbf{Induktionsanfang:}\ #1\\
	\noindent\textbf{Induktionsvoraussetzung:}\ #2\\
	\noindent\textbf{Induktionsschluss:}\ #3
\end{proof}}

\newcommand{\rg}{\operatorname{rg}}
\newcommand{\im}{\operatorname{im}}
\newcommand{\End}{\operatorname{End}}
\newcommand{\abb}{\operatorname{Abb}}
\newcommand{\re}{\operatorname{Re}}
\newcommand{\Ima}{\operatorname{Im}}
\newcommand{\Lagrange}[1]{\frac{\d }{\d t}\frac{\partial L }{\partial \dot #1} - \frac{\partial L}{\partial #1}}
\let\oldstackrel\stackrel
\renewcommand{\stackrel}[2]{%
    \oldstackrel{\mathclap{#1}}{#2}
}%


\newcommand{\ipilayout}[1]
{	
	\pagestyle{fancy}
	\fancyhead[L]{Einführung in die praktische Informatik, Blatt #1}
	\fancyhead[R]{Josua Kugler, Jan Metzger, David Wesner}
	\fancypagestyle{firstpage}{%
		\fancyhf{}
		\lhead{Professor: Peter Bastian\\
			Tutor: Frederick Schenk}
		\rhead{Einführung in die praktische Informatik, Übungsblatt #1\\ David, Jan, Josua}
		\cfoot{\thepage}
	}
\thispagestyle{firstpage}
}

% integral d sign
\makeatletter \renewcommand\d[1]{\ensuremath{%
		\;\mathrm{d}#1\@ifnextchar\d{\!}{}}}
\makeatother

\newcommand{\analayout}[1]
{	
	\pagestyle{fancy}
	\fancyhead[L]{Analysis 1, Blatt #1}
	\fancyhead[R]{Alexander Bryant, Josua Kugler}
	\fancypagestyle{firstpage}{%
		\fancyhf{}
		\lhead{Professor: Ekaterina Kostina\\
			Tutor: Philipp Elja Müller}
		\rhead{Analysis 1, Übungsblatt #1\\ Alexander Bryant, Josua Kugler}
		\cfoot{\thepage}
	}
	\thispagestyle{firstpage}
}
\newcommand{\lalayout}[1]
{	
	\pagestyle{fancy}
	\fancyhead[L]{Lineare Algebra 1, Blatt #1}
	\fancyhead[R]{David Wesner, Josua Kugler}
	\fancypagestyle{firstpage}{%
		\fancyhf{}
		\lhead{Professor: Denis Vogel\\
			Tutor: Marina Savarino}
		\rhead{Lineare Algebra 2, Übungsblatt #1\\ David Wesner, Josua Kugler}
		\cfoot{\thepage}
	}
	\thispagestyle{firstpage}
}

\lstset{
    frame=tb, % draw a frame at the top and bottom of the code block
    tabsize=4, % tab space width
    showstringspaces=false, % don't mark spaces in strings
    numbers=left, % display line numbers on the left
    commentstyle=\color{green}, % comment color
    keywordstyle=\color{blue}, % keyword color
    stringstyle=\color{red} % string color
}
\setlength{\headheight}{25pt}

\renewcommand{\phi}{\varphi}
\renewcommand{\theta}{\vartheta}
\begin{document}
\section*{Aufgabe 1}
\begin{enumerate}[(a)]
    \item Es gilt, wie erwartet $p = m\dot x$. Daher gilt 
    $$-2m\frac{\d V}{\d t} = -2m \vec\nabla V \frac{\d x}{\d t} = 2m \frac{\partial L}{\partial \dot{\vec{x}}} \dot{\vec{x}} = 2m \dot{\vec{x}} \frac{\d}{\d t} \frac{\partial L}{\partial \dot{\vec{x}}} = 2m \dot{\vec{x}}\frac{\d}{\d t} \vec{p} = 2\vec{p} \dot{\vec{p}} = \frac{\d{} \vec{p}^{\ 2}}{\d{} t}$$
    \item $$m\dot r = m \cdot \vec{\nabla} |\vec{x}| \cdot \frac{\d{}\vec{x}}{\d{} t} = m \cdot \frac{\vec{x}}{r}\cdot \dot{\vec{x}} = \frac{\vec{x}\vec{p}}{r}$$
    \item 
    \begin{align*}
        \frac{\d{}}{\d{t}} \delta x_i^{(j)} &= \frac{1}{2}\frac{\d{}}{\d{t}}\left(2p_ix_j - x_ip_j - \delta_{ij}(\vec{x}\cdot \vec{p})\right)\delta a\\
        &= \frac{1}{2} \left(2\dot p_i x_j + 2p_i\dot x_j - \dot x_ip_j - x_i\dot p_j - \delta_{ij}(\dot{\vec{x}}\vec{p} + \vec{x}\dot{\vec{p}})\right)
    \end{align*}
    \item Es gilt
    \begin{align*}
        \delta L^{(j)} &= \sum_{i = 1}^{3}\left(\frac{\partial L}{\partial x_i}\delta x_i^{(j)} + \frac{\partial L }{\partial \dot x_i}\delta \dot x_i^{(j)}\right)\\
        &= \frac{1}{2}\sum_{i = 1}^{3}\left(\frac{\d}{\d{t}}\frac{\partial L}{\partial \dot x_i}\left(2p_ix_j - x_ip_j - \delta_{ij}(\vec{x}\cdot \vec{p})\right) + \frac{\partial L }{\partial \dot x_i} \left(2\dot p_i x_j + 2p_i\dot x_j - \dot x_ip_j - x_i\dot p_j - \delta_{ij}(\dot{\vec{x}}\vec{p} + \vec{x}\dot{\vec{p}})\right)\right)\delta a\\
        &=\frac{1}{2}\sum_{i = 1}^{3}\left(\dot p_i\left(2p_ix_j - x_ip_j - \delta_{ij}(\vec{x}\cdot \vec{p})\right) + p_i \left(2\dot p_i x_j + 2p_i\dot x_j - \dot x_ip_j - x_i\dot p_j - \delta_{ij}(\dot{\vec{x}}\vec{p} + \vec{x}\dot{\vec{p}})\right)\right)\delta a\\
        &=\frac{1}{2}\left( - \dot p_j \vec{x}\cdot \vec{p} - p_j (\dot{\vec{x}}\vec{p} + \vec{x}\dot{\vec{p}}) + \sum_{i = 1}^{3}\left(2 \dot p_ip_ix_j - \dot p_ix_ip_j + 2p_i\dot p_i x_j+ 2p_i^2\dot x_j - p_i \dot x_ip_j - p_ix_i\dot p_j\right)\right)\delta a\\
        &=\frac{1}{2}\left( - (\vec{x}\vec{p})\dot p_j -  (\dot{\vec{x}}\vec{p})p_j - (\vec{x}\dot{\vec{p}})p_j + 2 (\dot{\vec{p}}\vec{p})x_j  - (\dot{\vec{p}}\vec{x})p_j + 2(\dot{\vec{p}}\vec{p})x_j+ 2(\vec{p}^{\;2})\dot x_j - (\vec{p} \dot{\vec{x}})p_j - (\vec{p}\vec{x})\dot p_j\right)\delta a\\
        &=\left( - (\vec{x}\vec{p})\dot p_j -  (\dot{\vec{x}}\vec{p})p_j  - (\vec{x}\dot{\vec{p}})p_j + 2 (\dot{\vec{p}}\vec{p})x_j + (\vec{p}^{\;2})\dot x_j\right)\delta a\\
        &=\left( - (\vec{x}\vec{p})\dot p_j + 2 (\dot{\vec{p}}\vec{p})x_j - (\vec{x}\dot{\vec{p}})p_j -  (\dot{\vec{x}}\vec{p})p_j + (\dot{\vec{x}}\vec{p})p_j\right)\delta a
        \intertext{Es gilt $\vec{x}\dot{\vec{p}} = \vec{x} \cdot \frac{\partial H}{\partial \vec{x}} = \vec{x} \cdot \vec{\nabla} V = -V$, da $V$ homogen vom Grad -1 in $\vec{x}$ ist.}
        &=- \frac{\partial L}{\partial x_j} (\vec{x}\vec{p})\delta a - \frac{\d{\vec{p}^{\; 2}}}{\d t} x_j\delta a + Vp_j \delta a\\
        &=- \frac{k}{r^3} (\vec{x}\vec{p})\cdot x_j\delta a + 2m\frac{\d V}{\d t} x_j\delta a + m V \dot{x_j} \delta a\\
        &= -m k \frac{\dot r}{r^2} \cdot x_j\delta a + 2m\frac{\d V}{\d t} x_j\delta a+ m V \dot{x_j} \delta a\\
        &= m \frac{\d{V}}{\d{t}} x_j \delta a + m V \frac{\d{x_j}}{\d{t}} \delta a\\
        &= \frac{\d{}}{\d{t}} \left(m V x_j \delta a\right) = \frac{\d{}}{\d{t}} \delta f(x_j)
    \end{align*}
    \item Nach Noether ist der folgende Ausdruck erhalten.
    \begin{align*}
        &\vec{p}\delta \vec{x} - H\delta t - f(\delta \vec{x}, \delta t) 
        \intertext{In unserem Fall gilt also}
        &\sum_{i = 1}^{3} p_i\delta x_i - H\delta t - f(\delta \vec{x}, \delta t) = \text{const}
        \intertext{Wir betrachten nun die infinetisimale Transformation $x^{(j)} \to x^{(j)} + \delta x^{(j)}$.}
        &=\sum_{i = 1}^{3} p_i\frac{1}{2}\left(2p_ix_j - x_ip_j - \delta_{ij}(\vec{x}\cdot \vec{p})\right)\delta a - mV x_j \delta a\\
        &= \frac{1}{2} \sum_{i = 1}^{3}\left( 2p_i^2x_j - x_ip_ip_j - p_j(x_ip_i)\right)\delta a - mV x_j \delta a\\
        &= x_j \cdot \vec{p}^{\; 2} \delta a- \sum_{i = 1}^{3}\left(x_ip_ip_j\right) \delta a - m\frac{k}{r} x_j \delta a\\
        &= (x_j \cdot \vec{p}^{\; 2} - p_j \cdot \vec{x}\vec{p} - m\frac{k}{r} x_j)\delta a\\
        &= (\vec{x} \times (\vec{x}\times \vec{p}) - m\frac{k}{r} \vec{x})_j \cdot \delta a\\
        &= \left(\vec{p}\times \vec{L} - \frac{mk}{r} \vec{x}\right)_j \delta a
        \intertext{Da $a$ konstant ist, muss auch der erste Faktor konstant sein.}
        &\implies \left(\vec{p}\times \vec{L} - \frac{mk}{r} \vec{x}\right)_j = \text{const}
        \intertext{Dies können wir für alle $j$ durchführen. Dann erhalten wir}
        & \vec{p}\times \vec{L} - \frac{mk}{r} \vec{x} = \text{const}
    \end{align*}
\end{enumerate}

\section*{Aufgabe 2}
\begin{enumerate}[(a)]
    \item Es gilt $\beta = \frac{3}{5},\; \gamma = (1 - \beta^2)^{-1/2} = \frac{5}{4}$. Wir bezeichnen das Bezugssystem von Bob mit $S'$ und das Bezugssystem von Alice mit $S$. Zu Beginn gilt $t = t' = 0$. Beim Erreichen der Raumstation gilt $x_3' = 3 Ly$ und $x_3 = 0$, da sich Alice ja zu dem Zeitpunkt an der Raumstation befindet. Also gilt $x_3' = 3 Ly = \gamma \beta x_0 + \gamma x_3 \implies x_0 = 4 Ly$ und $x_0' = \gamma x_0 + \beta \gamma x_3 = \gamma x_0 = 5 Ly$. In Bobs Bezugssystem sind also 5 Jahre vergangen, bis Alice zur Raumstation gereist ist, während bei Alice nur $4$ Jahre vergangen sind. Für die Rückreise erhalten wir aus Symmetriegründen dieselbe Zeit, sodass Alice um 8 Jahre altert, während Bob um 10 Jahre altert.
\end{enumerate}
\section*{Aufgabe 3}
    \begin{align*}
        \Phi_3(p_1, \dots, p_f, Q_1, \dots, Q_f) &= \Phi_1 - \sum_{i = 1}^{f} \pdv{\Phi_1}{q_i}q_i\\
        \intertext{gegeben. Das totale Differential auf beiden Seiten ist}
        \sum_{i = 1}^{f} \qty(\pdv{\Phi_3}{p_i}\dd{p_i} + \pdv{\Phi_3}{Q_i}\dd{Q_i}) + \pdv{\Phi_3}{t} \dd{t} &= \sum_{i = 1}^{f} \qty(\pdv{\Phi_1}{Q_i}\dd{Q_i} - q_i \dd (\pdv{\Phi_1}{q_i})) + \pdv{\Phi_1}{t} \dd{t} 
        \intertext{Durch Koeffizientenvergleich erhalten wir}
        \mathclap{\pdv{\Phi_1}{t} = \pdv{\Phi_3}{t},\qquad \pdv{\Phi_1}{Q_i} = \pdv{\Phi_3}{Q_i}, \qquad \pdv{\Phi_1}{q_i} = p_i,\qquad -q_i = \pdv{\Phi_3}{p_i}}
        \intertext{Diese Identitäten können wir nun ausnutzen.}
        \dv{\Phi_3}{t} &= \sum_{i = 1}^{f} \qty(\pdv{\Phi_3}{p_i}\dot{p_i} + \pdv{\Phi_3}{Q_i}\dot{Q_i}) + \pdv{\Phi_3}{t}\\
        &= \sum_{i = 1}^{f} \qty(-q_i\dot{p_i} + \pdv{\Phi_1}{Q_i}\dot{Q_i}) + \pdv{\Phi_3}{t}
        \intertext{Wir wissen aus der Vorlesung, dass $\pdv{\Phi_1}{Q_i} = -P_i$ gilt}
        \dv{\Phi_3}{t} &= \sum_{i = 1}^{f} \qty(\dot{q_i}p_i -\dv{t}(q_ip_i) - P_i\dot{Q_i}) + \pdv{\Phi_3}{t}
        \intertext{Daher gilt}
        \sum_{i = 1}^{f} p_i\dot{q_i} - H &=  \sum_{i = 1}^{f}\qty(P_i\dot{Q_i}) - \underbrace{\qty(H + \pdv{\Phi_3}{t})}_{K} +  \dv{t}(\Phi_3 + \sum_{i = 1}^{f}\qty(q_ip_i))
    \end{align*}
    Insgesamt ist also 
    \[ K= H + \pdv{\Phi_3}{t}, \qquad P_i = -\pdv{\Phi_3}{Q_i},\qquad q_i = - \pdv{\Phi_3}{p_i}\]
\section*{Aufgabe 4}
\begin{enumerate}[(a)]
    \item \begin{align*}
        \qty{q_i, q_j} &= \sum_{k = 1}^{3} \qty(\pdv{q_i}{q_k} \underbrace{\pdv{q_j}{p_k}}_{=0} - \underbrace{\pdv{q_i}{p_k}}_{=0} \pdv{q_j}{q_k}) = 0\\
        \qty{p_i, p_j} &= \sum_{k = 1}^{3} \qty(\underbrace{\pdv{p_i}{q_k}}_{=0} \pdv{p_j}{p_k} - \pdv{p_i}{p_k} \underbrace{\pdv{p_j}{q_k}}_{=0}) = 0\\
        \qty{q_i, p_j} &= \sum_{k = 1}^{3} \qty(\pdv{q_i}{q_k} \pdv{p_j}{p_k} - \pdv{q_i}{p_k} \pdv{p_j}{q_k})=\sum_{k = 1}^{3} \qty(\delta_{ik} \delta_{jk} - 0) = \delta_{ij}\\ 
    \end{align*}
    \item \begin{align*}
        \qty{L_i, p_j} &= \qty{\varepsilon_{ilk} q_lp_k,p_j} = \varepsilon_{ilk} \qty{q_lp_k,p_j} = -\varepsilon_{ilk} \qty{p_j,q_lp_k} = -\varepsilon_{ilk} \qty(p_k\qty{p_j,q_l} + q_l\qty{p_j,p_k})\\
        &= \varepsilon_{ilk} p_k \qty{q_l,p_j} = \varepsilon_{ilk} p_k \delta_{lj} = \varepsilon_{ijk} p_k
    \end{align*}
    \begin{align*}
        \qty{L_i, q_j} &= \qty{\varepsilon_{ilk} q_lp_k,q_j} = \varepsilon_{ilk} \qty{q_lp_k,q_j} = -\varepsilon_{ilk} \qty{q_j,q_lp_k} = -\varepsilon_{ilk} \qty(q_k\qty{p_j,q_l} + q_l\qty{q_j,p_k})\\
        &= -\varepsilon_{ilk} q_l \qty{q_j,p_k} = \varepsilon_{ilk} q_k \qty{q_j,p_l} = \varepsilon_{ilk} q_k \delta_{jl} = \varepsilon_{ijk} q_k
    \end{align*}
    \begin{align*}
        \qty{L_i, L_j} &= \varepsilon_{ilk}\varepsilon_{jmn} \qty{q_lp_k, q_mp_n} = \varepsilon_{ilk}\varepsilon_{jmn} \qty(p_n\qty{q_lp_k, q_m} + q_m\qty{q_lp_k, p_n})\\
        &= -\varepsilon_{ilk}\varepsilon_{jmn} \qty(p_n\qty{q_m, q_lp_k} + q_m\qty{p_n, q_lp_k})\\
        &= -\varepsilon_{ilk}\varepsilon_{jmn} \qty(p_nq_l\qty{q_m, p_k} + p_np_k\qty{q_m, q_l}+ q_mq_l\qty{p_n, p_k} + q_mp_k\qty{p_n, q_l})\\
        &= -\varepsilon_{ilk}\varepsilon_{jmn} \qty(p_nq_l\delta_{mk} - q_mp_k\delta_{nl})\\
        &= -\varepsilon_{ilk}\varepsilon_{jkn} p_nq_l + \varepsilon_{ilk}\varepsilon_{jml} q_mp_k\\
        &= -\varepsilon_{ilk}\varepsilon_{knj} p_nq_l - \varepsilon_{ikl}\varepsilon_{ljm} q_mp_k\\
        &= -\qty(\delta_{in}\delta_{lj} - \delta_{ij}\delta_{ln})(p_nq_l) - \qty(\delta_{ij}\delta_{km} - \delta_{im}\delta_{kj})(q_mp_k)\\
        &= - p_iq_j + \delta_{ij}\delta_{nl}p_nq_l - \delta_{ij}\delta_{km} p_kq_m + q_ip_j\\
        &= q_ip_j - p_iq_j + \delta_{ij}(p_lq_l - p_kq_k)\\
        &= q_ip_j - q_jp_i\\
        &= (\delta_{il}\delta_{jm} - \delta_{im}\delta_{jl}) q_lp_m\\
        &= \varepsilon_{ijk} \varepsilon_{klm} q_lp_m\\
        &= \varepsilon_{ijk} L_k
    \end{align*}
    \begin{align*}
        \qty{L_i, \vec{L}^2} &= \qty{L_i, L_jL_j}\\
        &= 2\qty{L_i,L_j}L_j\\
        &= 2 \varepsilon_{ijk}L_jL_k\\
        &= 2 \vec{L} \times \vec{L}\\
        &= 0
    \end{align*}
    \item Aufgrund der Definition des Drehimpulses ist $\pdv*{L}{t} = 0$. Es genügt also, die Poisson-Klammer zu berechnen.
    \begin{align*}
        \qty{\vec{L}, H}_i &= \qty{L_i,H} = \qty{L_i,T} + \qty{L_i,V}\\
        &= \sum_{j = 1}^{3} \qty(\pdv{L_i}{q_j} \pdv{T(p_1,p_2,p_3)}{p_j} - \pdv{L_i}{p_j} \pdv{T(p_1,p_2,p_3)}{q_j}) + \sum_{j = 1}^{3} \qty(\pdv{L_i}{q_j} \pdv{V(q_1,q_2,q_3)}{p_j} - \pdv{L_i}{p_j} \pdv{V(q_1,q_2,q_3)}{q_j})\\
        &= \sum_{j = 1}^{3} \qty(\pdv{L_i}{q_j} \pdv{T}{p_j}) - \sum_{j = 1}^{3} \qty(\pdv{L_i}{p_j} \pdv{V}{q_j})\\
        &= \epsilon_{ikl} \pdv{T}{p_k}p_l - \epsilon_{ikl} q_k\pdv{V}{q_l}
        %= \sum_{j = 1}^{3} \qty(\pdv{L_i}{q_j} \pdv{H}{p_j}) - \sum_{j = 1}^{3} \qty(\pdv{L_i}{p_j} \pdv{H}{q_j})\\
        %&= \sum_{j = 1}^{3} \qty(\pdv{\varepsilon_{ikl}q_kp_l}{q_j} \dot q_j) + \sum_{j = 1}^{3} \qty(\pdv{\varepsilon_{ikl}q_kp_l}{p_j} \dot p_j)\\
        %&= \varepsilon_{ikl}p_l \dot q_k + \varepsilon_{ikl}q_k \dot p_l\\
        %&= \dv{t}(\varepsilon_{ikl}p_l q_k)\\
        %&= \dv{t}(L_i)
    \end{align*}
\end{enumerate}
\end{document}