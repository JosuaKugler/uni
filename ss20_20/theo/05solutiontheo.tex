\documentclass{article}

\usepackage[utf8]{inputenc}
\usepackage[T1]{fontenc}
\usepackage[ngerman]{babel}
\usepackage{amsmath, amsfonts, amsthm, mathtools, amssymb}
\usepackage{stmaryrd}
\usepackage{enumerate}
\usepackage{cases}
\usepackage{fancyhdr}
\usepackage{comment}
%\usepackage{xcolor}
\usepackage{tikz}
\usepackage{cases}
\usepackage{listings}
\usepackage{siunitx}
\usepackage[left = 3cm]{geometry}
\usepackage[hidelinks]{hyperref}
\usepackage{subcaption}
\usepackage{gauss}
\newtheorem{satz}{Satz}[section]
\newtheorem{lemma}[satz]{Lemma}
\newtheorem{korollar}[satz]{Korollar}
\newtheorem{proposition}[satz]{Proposition}
\theoremstyle{definition}
\newtheorem{definition}[satz]{Def.}
\newtheorem{axiom}[satz]{Axiom}
\newtheorem{bsp}[satz]{Bsp.}
\newtheorem*{anmerkung}{Anm}
\newtheorem{bemerkung}[satz]{Bem}
\newtheorem*{notatio}{Notation}
\newcommand{\obda}{O.B.d.A. }
\newcommand{\equals}{\Longleftrightarrow}
\newcommand{\N}{\mathbb{N}}
\newcommand{\Q}{\mathbb{Q}}
\newcommand{\R}{\mathbb{R}}
\newcommand{\Z}{\mathbb{Z}}
\newcommand{\C}{\mathbb{C}}
\newcommand{\intd}{\mathrm{d}}
\newcommand{\Pot}{\operatorname{Pot}}
\newcommand{\mychar}{\operatorname{char}}
\newcommand{\myker}{\operatorname{ker}}
\newcommand{\induktion}[3]
{\begin{proof}\ \\
	\noindent\textbf{Induktionsanfang:}\ #1\\
	\noindent\textbf{Induktionsvoraussetzung:}\ #2\\
	\noindent\textbf{Induktionsschluss:}\ #3
\end{proof}}

\newcommand{\rg}{\operatorname{rg}}
\newcommand{\im}{\operatorname{im}}
\newcommand{\End}{\operatorname{End}}
\newcommand{\abb}{\operatorname{Abb}}
\newcommand{\re}{\operatorname{Re}}
\newcommand{\Ima}{\operatorname{Im}}
\newcommand{\Lagrange}[1]{\frac{\d }{\d t}\frac{\partial L }{\partial \dot #1} - \frac{\partial L}{\partial #1}}
\let\oldstackrel\stackrel
\renewcommand{\stackrel}[2]{%
    \oldstackrel{\mathclap{#1}}{#2}
}%


\newcommand{\ipilayout}[1]
{	
	\pagestyle{fancy}
	\fancyhead[L]{Einführung in die praktische Informatik, Blatt #1}
	\fancyhead[R]{Josua Kugler, Jan Metzger, David Wesner}
	\fancypagestyle{firstpage}{%
		\fancyhf{}
		\lhead{Professor: Peter Bastian\\
			Tutor: Frederick Schenk}
		\rhead{Einführung in die praktische Informatik, Übungsblatt #1\\ David, Jan, Josua}
		\cfoot{\thepage}
	}
\thispagestyle{firstpage}
}

% integral d sign
\makeatletter \renewcommand\d[1]{\ensuremath{%
		\;\mathrm{d}#1\@ifnextchar\d{\!}{}}}
\makeatother

\newcommand{\analayout}[1]
{	
	\pagestyle{fancy}
	\fancyhead[L]{Analysis 1, Blatt #1}
	\fancyhead[R]{Alexander Bryant, Josua Kugler}
	\fancypagestyle{firstpage}{%
		\fancyhf{}
		\lhead{Professor: Ekaterina Kostina\\
			Tutor: Philipp Elja Müller}
		\rhead{Analysis 1, Übungsblatt #1\\ Alexander Bryant, Josua Kugler}
		\cfoot{\thepage}
	}
	\thispagestyle{firstpage}
}
\newcommand{\lalayout}[1]
{	
	\pagestyle{fancy}
	\fancyhead[L]{Lineare Algebra 1, Blatt #1}
	\fancyhead[R]{David Wesner, Josua Kugler}
	\fancypagestyle{firstpage}{%
		\fancyhf{}
		\lhead{Professor: Denis Vogel\\
			Tutor: Marina Savarino}
		\rhead{Lineare Algebra 2, Übungsblatt #1\\ David Wesner, Josua Kugler}
		\cfoot{\thepage}
	}
	\thispagestyle{firstpage}
}

\lstset{
    frame=tb, % draw a frame at the top and bottom of the code block
    tabsize=4, % tab space width
    showstringspaces=false, % don't mark spaces in strings
    numbers=left, % display line numbers on the left
    commentstyle=\color{green}, % comment color
    keywordstyle=\color{blue}, % keyword color
    stringstyle=\color{red} % string color
}
\setlength{\headheight}{25pt}

\renewcommand{\phi}{\varphi}
\renewcommand{\theta}{\vartheta}
\begin{document}
\section*{Aufgabe 1}
\begin{enumerate}[(a)]
    \item Bekanntlich ist in Kugelkoordinaten 
    $$\vec x = R \begin{pmatrix}
        \sin(\theta)\cos(\phi)\\
        \sin(\theta)\sin(\phi)\\
        \cos(\theta)
    \end{pmatrix}$$ und daher bei konstantem Radius $$\dot \vec x = \begin{pmatrix}
        R \begin{pmatrix}
            \dot \theta \cos(\theta)\cos(\phi) - \dot \phi \sin(\theta)\sin(\phi)\\
            \dot \theta \cos(\theta)\sin(\phi) + \dot \phi \sin(\theta)\cos(\phi)\\
            - \dot \theta \sin(\theta)
        \end{pmatrix}
    \end{pmatrix},$$ woraus wir folgern, dass
    \begin{align*}
        \dot{\vec{x}}^2 &= R^2\left[\dot \theta^2 \cos^2(\theta)\cos^2(\phi) + \dot \phi^2 \sin^2(\theta)\sin^2(\phi) + \dot \theta^2 \cos^2(\theta)\sin^2(\phi) + \dot \phi^2 \sin^2(\theta)\cos^2(\phi) + \dot \theta^2 \sin^2(\theta)\right]\\
        &= R^2\left[\dot \theta^2 \cos^2(\theta) + \dot \phi^2 \sin^2(\theta) + \dot \theta^2 \sin^2(\theta)\right]\\
        &= R^2 \left(\dot \theta^2 + \dot \phi^2 \sin^2(\theta)\right)
        \intertext{Laut Aufgabenstellung ist $\dot \phi = \omega$}
        &= R^2 \left(\dot \theta^2 + \omega^2 \sin^2(\theta)\right)
    \end{align*}
    Nun gilt 
    $$L = T - V = \frac{1}{2}m \dot{\vec{x}}^2 - m \cdot g \cdot x_3 = \frac{1}{2}m R^2 \left(\dot \theta^2 + \omega^2 \sin^2(\theta)\right) - mgR\cos(\theta)$$
    \item Der kanonisch konjugierte Impuls ist 
    $$p_\theta = \frac{\partial L }{\partial \dot \theta} = mR^2 \dot \theta.$$ Es gilt $$E = T+ V = \frac{1}{2}m R^2 \left(\dot \theta^2 + \omega^2 \sin^2(\theta)\right) + mgR\cos(\theta)$$ und 
    \begin{align*}
        H &= p_\theta \dot \theta - L\\
        &= mR^2 \dot \theta^2 - \frac{m R^2}{2} \left(\dot \theta^2 + \omega^2 \sin^2(\theta)\right) + mgR\cos(\theta)\\
        &= \frac{mR^2}{2} \left(\dot \theta^2 - \omega^2 \sin^2(\theta)\right) + mgR\cos(\theta)\\
        &= \frac{p_\theta}{2} - \frac{mR^2}{2} \omega^2 \sin^2(\theta) + mgR\cos(\theta)
    \end{align*}
    Also ist $H\neq E$.
    \item Die Energie ist nicht erhalten, da die Zeitableitung von $E$ nicht 0 ist.
    \item Die kanonischen Gleichungen lauten
    $$\dot \theta = \frac{\partial H}{\partial p_\theta}$$ und $$-\dot p_\theta = \frac{\partial H}{\partial \theta} = -mR^2\omega^2\sin(\theta)\cos(\theta) - mgR \sin(\theta)$$
    \item Setzt man den kanonisch konjugierten Impuls in die zweite der kanonischen Gleichungen ein, so erhält man
    $$- \frac{\d}{\d t}mR^2 \dot \theta = \frac{\partial H}{\partial \theta} = - mR^2\omega^2\sin(\theta)\cos(\theta) - mgR \sin(\theta).$$ Mit $\dot \theta = 0$ folgt daraus
    $$m\omega^2R^2\sin(\theta)\cos(\theta) + mgR \sin(\theta) = 0.$$
    Für $\theta = k\cdot \pi, k \in \Z$ ist diese Bedingung erfüllt. Sonst teilen wir durch $mR\sin(\theta)$ und erhalten
    $$R\omega^2\cos(\theta) = -g.$$
\end{enumerate}
\section*{Aufgabe 2}
\begin{enumerate}[(a)]
    \item $L = T - V = \frac{m}{2}\dot{q}^2 - \frac{m}{2}\omega^2 q^2$, $H = T + V = \frac{m}{2}\dot{q}^2 + \frac{m}{2}\omega^2 q^2$.
    \item Die kanonischen Gleichungen sind
    $$\dot q = \frac{\partial H}{\partial p}$$ und $$-\dot p = \frac{\partial H}{\partial q},$$ wobei $p$ durch $$p = \frac{\partial L}{\partial \dot q} = m \dot q$$ gegeben ist. Wir können $H$ damit auch schreiben als $H = \frac{p^2}{2m} + \frac{m}{2}\omega^2 q^2$. 
    Nun gilt:
    $$\dot y = \begin{pmatrix}
        \dot q\\ \dot p 
    \end{pmatrix} = \begin{pmatrix}
        \frac{\partial H}{\partial p}\\ -\frac{\partial H}{\partial q}
    \end{pmatrix} = \begin{pmatrix}
        \frac{p}{m} \\ -m \omega^2 q
    \end{pmatrix} = \begin{pmatrix}
        0 & \frac{1}{m}\\
        -m \omega^2 & 0
    \end{pmatrix}\cdot \begin{pmatrix}
        q\\ p
    \end{pmatrix} = \begin{pmatrix}
        0 & \frac{1}{m}\\
        -m \omega^2 & 0
    \end{pmatrix}\cdot y$$
    Diese Gleichung wird gelöst durch $\vec y(t) = e^{At}\vec y_0$ mit $A = m\begin{pmatrix}
        0 & \frac{1}{m^2}\\
        -\omega^2 & 0
    \end{pmatrix}$. Wir stellen fest 
    $$A^2 = m\begin{pmatrix}
        0 & \frac{1}{m^2}\\
        -\omega^2 & 0
    \end{pmatrix} \cdot m\begin{pmatrix}
        0 & \frac{1}{m^2}\\
        -\omega^2 & 0
    \end{pmatrix} = m^2 \cdot \begin{pmatrix}
        -\frac{\omega^2}{m^2} & 0\\ 0 & -\frac{\omega^2}{m^2}
    \end{pmatrix} = - \omega^2 \cdot E_2.$$
    Es gilt also 
    \begin{align*}
        e^{At} &= \sum_{n = 0}^{\infty}\frac{(At)^n}{n!}\\
        &= \sum_{n = 0}^{\infty} \frac{(At)^{2n+1}}{(2n+1)!} + \sum_{n = 0}^{\infty} \frac{(At)^{2n}}{(2n)!}\\
        &= A \cdot \sum_{n = 0}^{\infty}  (- \omega^2 \cdot E_2)^n \frac{t^{2n+1}}{(2n+1)!} + \sum_{n = 0}^{\infty} (- \omega^2 \cdot E_2)^n \frac{t^{2n}}{(2n)!}\\
        &= \frac{1}{\omega} A \cdot \sum_{n = 0}^{\infty}  (-1)^n  \frac{(\omega t)^{2n+1}}{(2n+1)!} + E_2 \cdot \sum_{n = 0}^{\infty} (- 1)^n \frac{(\omega t)^{2n}}{(2n)!}\\
        &= \frac{1}{\omega} A \cdot \sin(\omega t) + E_2 \cos(\omega t)\\
        &= \begin{pmatrix}
            \cos(\omega t) & \frac{1}{m\omega} \sin(\omega t)\\
            m \omega \cos(\omega t) & \cos(\omega t)
        \end{pmatrix}
    \end{align*}
\end{enumerate}
\section*{Aufgabe 3}
\begin{enumerate}[(a)]
    \item Es gilt für ebene Polarkoordinaten
    \begin{align*}
        \frac{\partial^2}{\partial t^2} R(r)T(t) &= v^2\left(\frac{1}{r}\frac{\partial}{\partial r}\left(r\frac{\partial}{\partial r}\right) + \frac{1}{r^2}\frac{\partial^2}{\partial \phi^2}\right) R(r)T(t)\\
        \frac{1}{T(t)}\frac{\partial^2}{\partial t^2} T(t) &= v^2\frac{1}{rR(r)}\frac{\partial}{\partial r}\left(r\frac{\partial}{\partial r}\right)R(r)
        \intertext{Da die linke Seite nur von $t$ und die rechte Seite nur von $r$ abhängt, muss jede Seite für sich konstant sein,}
        \frac{1}{T(t)}\frac{\partial^2}{\partial t^2} T(t) &= -c = v^2\frac{1}{rR(r)}\frac{\partial}{\partial r}\left(r\frac{\partial}{\partial r}\right)R(r)
        \intertext{Daraus erhalten wir}
        \frac{\partial^2}{\partial t^2} T(t) + cT(t) &= 0
        \intertext{und}
        v^2\frac{\partial}{\partial r}\left(r\frac{\partial}{\partial r}R(r)\right) + crR(r) &= 0\\
        v^2\frac{\partial}{\partial r}R(r) + v^2r \frac{\partial^2}{\partial r^2}R(r) + crR(r)  &= 0\\
        \frac{\partial^2}{\partial r^2}R(r) + v^2\frac{1}{r} \frac{\partial}{\partial r}R(r)+ cR(r)&= 0
    \end{align*}
    und für Kugelkoordinaten (die anderen Terme des Laplace-Operators werden eh 0 also schreib ich sie gar nicht auf)
    \begin{align*}
        \frac{\partial^2}{\partial t^2} R(r)T(t) &= v^2\left(\frac{1}{r^2}\frac{\partial}{\partial r}\left(r^2\frac{\partial}{\partial r}\right)\right) R(r)T(t)\\
        \frac{1}{T(t)}\frac{\partial^2}{\partial t^2} T(t) &= v^2\frac{1}{r^2R(r)}\frac{\partial}{\partial r}\left(r^2\frac{\partial}{\partial r}\right)R(r)
        \intertext{Da die linke Seite nur von $t$ und die rechte Seite nur von $r$ abhängt, muss jede Seite für sich konstant sein,}
        \frac{1}{T(t)}\frac{\partial^2}{\partial t^2} T(t) &= -c =v^2\frac{1}{r^2R(r)}\frac{\partial}{\partial r}\left(r^2\frac{\partial}{\partial r}\right)R(r)
        \intertext{Daraus erhalten wir}
        \frac{\partial^2}{\partial t^2} T(t) + cT(t) &= 0
        \intertext{und}
        v^2\frac{1}{r^2}\frac{\partial}{\partial r}\left(r^2\frac{\partial}{\partial r}R(r)\right) + cR(r) &= 0\\
        v^2r\frac{\partial^2}{\partial r^2}R(r) + 2v^2\frac{\partial}{\partial r}R(r) + crR(r) &= 0
    \end{align*}
    \item Nun setzen wir $\tilde{R}(r) = rR(r)$ und erhalten
    $$\frac{\partial^2}{\partial r^2} \tilde{R}(r) = \frac{\partial}{\partial r}\left(r \frac{\partial}{\partial r}R(r) + R(r)\right) = r \frac{\partial^2}{\partial r^2} R(r) + 2 \frac{\partial}{\partial r}R(r)$$
    Das können wir nun in unserem Ergebnis aus der (a) einsetzen und erhalten
    \begin{align*}
        v^2r\frac{\partial^2}{\partial r^2}R(r) + 2v^2\frac{\partial}{\partial r}R(r) + crR(r) &= 0\\
        \frac{\partial^2}{\partial r^2} \tilde{R}(r) + \frac{c}{v^2}\tilde{R}(r) &= 0
    \end{align*}
    Das führt auf die Lösung (unter Berücksichtigung der Stetigkeit an der Stelle $r = 0$)
    $$R(r) = \frac{A}{r}\sin(\frac{\sqrt{c}}{v} r)$$ und $$T(t) = C \sin(\sqrt{c} t) + D \cos(\sqrt{c} t)$$
\end{enumerate}
\end{document}