\documentclass{article}

\usepackage[utf8]{inputenc}
\usepackage[T1]{fontenc}
\usepackage[ngerman]{babel}
\usepackage{amsmath, amsfonts, amsthm, mathtools, amssymb}
\usepackage[arrowdel]{physics} % derivatives
\usepackage{stmaryrd}
\usepackage{enumerate}
\usepackage{cases}
\usepackage{fancyhdr}
\usepackage{comment}
%\usepackage{xcolor}
\usepackage{tikz}
\usepackage{pgf}
\usepackage{pgfplots}
\pgfplotsset{compat=1.16}
\usepackage{cases}
\usepackage{listings}
\usepackage{siunitx}
\usepackage[left = 3cm]{geometry}
\usepackage[hidelinks]{hyperref}
\usepackage{subcaption}
\usepackage{gauss}
\newtheorem{satz}{Satz}[section]
\newtheorem{lemma}[satz]{Lemma}
\newtheorem{korollar}[satz]{Korollar}
\newtheorem{proposition}[satz]{Proposition}
\theoremstyle{definition}
\newtheorem{definition}[satz]{Def.}
\newtheorem{axiom}[satz]{Axiom}
\newtheorem{bsp}[satz]{Bsp.}
\newtheorem*{anmerkung}{Anm}
\newtheorem{bemerkung}[satz]{Bem}
\newtheorem*{notatio}{Notation}
\newcommand{\obda}{O.B.d.A. }
\newcommand{\equals}{\Longleftrightarrow}
\newcommand{\N}{\mathbb{N}}
\newcommand{\Q}{\mathbb{Q}}
\newcommand{\R}{\mathbb{R}}
\newcommand{\Z}{\mathbb{Z}}
\newcommand{\C}{\mathbb{C}}
\newcommand{\intd}{\mathrm{d}}
\newcommand{\Pot}{\operatorname{Pot}}
\newcommand{\mychar}{\operatorname{char}}
\newcommand{\myker}{\operatorname{ker}}
\newcommand{\induktion}[3]
{\begin{proof}\ \\
	\noindent\textbf{Induktionsanfang:}\ #1\\
	\noindent\textbf{Induktionsvoraussetzung:}\ #2\\
	\noindent\textbf{Induktionsschluss:}\ #3
\end{proof}}

\newcommand{\rg}{\operatorname{rg}}
\newcommand{\im}{\operatorname{im}}
\newcommand{\End}{\operatorname{End}}
\newcommand{\abb}{\operatorname{Abb}}
\newcommand{\re}{\operatorname{Re}}
\newcommand{\Ima}{\operatorname{Im}}
\newcommand{\Lagrange}[1]{\frac{\d }{\d t}\frac{\partial L }{\partial \dot #1} - \frac{\partial L}{\partial #1}}
\let\oldstackrel\stackrel
\renewcommand{\stackrel}[2]{%
    \oldstackrel{\mathclap{#1}}{#2}
}%


\newcommand{\ipilayout}[1]
{	
	\pagestyle{fancy}
	\fancyhead[L]{Einführung in die praktische Informatik, Blatt #1}
	\fancyhead[R]{Josua Kugler, Jan Metzger, David Wesner}
	\fancypagestyle{firstpage}{%
		\fancyhf{}
		\lhead{Professor: Peter Bastian\\
			Tutor: Frederick Schenk}
		\rhead{Einführung in die praktische Informatik, Übungsblatt #1\\ David, Jan, Josua}
		\cfoot{\thepage}
	}
\thispagestyle{firstpage}
}

% integral d sign
\makeatletter \renewcommand\d[1]{\ensuremath{%
		\;\mathrm{d}#1\@ifnextchar\d{\!}{}}}
\makeatother

\newcommand{\analayout}[1]
{	
	\pagestyle{fancy}
	\fancyhead[L]{Analysis 1, Blatt #1}
	\fancyhead[R]{Alexander Bryant, Josua Kugler}
	\fancypagestyle{firstpage}{%
		\fancyhf{}
		\lhead{Professor: Ekaterina Kostina\\
			Tutor: Philipp Elja Müller}
		\rhead{Analysis 1, Übungsblatt #1\\ Alexander Bryant, Josua Kugler}
		\cfoot{\thepage}
	}
	\thispagestyle{firstpage}
}
\newcommand{\lalayout}[1]
{	
	\pagestyle{fancy}
	\fancyhead[L]{Lineare Algebra 1, Blatt #1}
	\fancyhead[R]{David Wesner, Josua Kugler}
	\fancypagestyle{firstpage}{%
		\fancyhf{}
		\lhead{Professor: Denis Vogel\\
			Tutor: Marina Savarino}
		\rhead{Lineare Algebra 2, Übungsblatt #1\\ David Wesner, Josua Kugler}
		\cfoot{\thepage}
	}
	\thispagestyle{firstpage}
}

\lstset{
    frame=tb, % draw a frame at the top and bottom of the code block
    tabsize=4, % tab space width
    showstringspaces=false, % don't mark spaces in strings
    numbers=left, % display line numbers on the left
    commentstyle=\color{green}, % comment color
    keywordstyle=\color{blue}, % keyword color
    stringstyle=\color{red} % string color
}
\setlength{\headheight}{25pt}

\renewcommand{\phi}{\varphi}
\renewcommand{\theta}{\vartheta}
\begin{document}
\section*{Aufgabe 2}
\begin{enumerate}[(a)]
	\item Es gibt $N!$ Permutationen, allerdings ist die Reihenfolge der ersten $N_2$ Teilchen egal, genauso wie die Reihenfolge der letzten $N_1$ Teilchen.
	\item \begin{align*}
		\frac{1}{k_B}S &= \ln(\Omega)\\
		&= \ln N! - \ln N_2! - \ln N_1 !\\
		&\approx N\ln N - N - N_2\ln N_2 + N_2 - N_1\ln N_1 + N_1\\
		&= N\ln N - (N- N_1) \ln N_2 - N_1\ln N_1\\
		&= N\ln\left(\frac{N}{N_2}\right) - N_1\ln\left(\frac{N_1}{N_2}\right)
	\end{align*}
	Für die Temperatur gilt
	\begin{align*}
		\beta &= \pdv{\ln \Omega}{E}\\
		&= \pdv{N\ln\left(\frac{N}{N_2}\right) - N_1\ln\left(\frac{N_1}{N_2}\right)}{E}\\
		&= N \frac{N_2}{N} \pdv{\frac{N(\epsilon_2-\epsilon_1)}{E-N\epsilon_1}}{E}- \pdv{N_1}{E}\ln\left(\frac{N_1}{N_2}\right) + N_1\frac{N_2}{N_1}\pdv{\frac{N\epsilon_2-E}{E-N\epsilon_1}}{E}\\
		&= -N_2N\frac{\epsilon_2-\epsilon_1}{(E-N\epsilon_1)^2} +\frac{1}{\epsilon_2-\epsilon_1}\ln\left(\frac{N_1}{N_2}\right) + N_2 \frac{-(E - N\epsilon_1) - (N\epsilon_2 - E)}{(E-N\epsilon_1)^2}\\
		&= \frac{1}{\epsilon_2-\epsilon_1}\ln\left(\frac{N_1}{N_2}\right)
	\end{align*}
	Nun berechnen wir daraus \[T = \frac{1}{k_B \beta} = \frac{\epsilon_2-\epsilon_1}{k_B\ln\left(\frac{N_1}{N_2}\right)}.\] 
	\item Sowohl für $N_2 \to 0$ als auch für $N_2 \to \infty$ geht die Temperatur also gegen 0. Für $N_2 > \flatfrac{N}{2}$ wird der Logarithmus und damit auch die Temperatur negativ.
	\item Es gilt
	\[
		\frac{1}{k_B} S = \ln(\Omega) = (2N-1)\ln(E) + \ln(\delta E) - \ln((2N-1)!) + N\ln\left(\frac{4\pi^3R^2m}{h_0^3\omega}\right)
	\]
	Daraus berechnen wir
	\[
		\beta = \pdv{\ln\Omega}{E}= \frac{2N-1}{E}
	\] und 
	\[
		T = \frac{1}{k_B\beta} = \frac{E}{k_B(2N-1)}	
	\]
\end{enumerate}
\end{document}