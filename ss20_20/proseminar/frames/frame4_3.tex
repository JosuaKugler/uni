\begin{frame}[t]
\frametitle{Grundlegende Eigenschaften eines Baums}
\only<5>{%
\begin{lemma}%
In einem Baum gibt es genau einen Pfad zwischen zwei Knoten.%
\end{lemma}%
\begin{lemma}%
Ein Baum ist minimal zusammenhängend und maximal kreisfrei.%
\end{lemma}%
}%
\begin{lemma}
        Ein Baum mit $n$ Knoten besitzt $n-1$ Kanten.
\end{lemma}
\only<2-4>{
\begin{proof}
}%
\begin{columns}
    \begin{column}{0.58\textwidth}
            \begin{itemize}
                \item<2-4> Induktionsanfang:\\ 1 Knoten $\implies$ 0 Kanten
                \item<3-4> Induktionsschritt:\\
                $k \to k +1$ Knoten $\implies + 1$ Kante, sonst Kreis.
            \end{itemize} 
    \end{column}
    \begin{column}{0.25\textwidth}
        \only<2>{
            \begin{figure}
                \begin{tikzpicture}[scale=1, auto,swap]
                    \node[vertex] (a) at (0,0) {$a$};
                \end{tikzpicture}
        \end{figure}
        }
        \only<3>{\vspace*{-0.4cm}
            \begin{figure}
                \begin{tikzpicture}[scale=1, auto,swap]
            % Draw a 7,11 network
            % First we draw the vertices
            \foreach \pos/\name in {{(0,2)/a}, {(0,0)/b}, {(1,-1)/c},
                                    {(2,0)/d}, {(2,2)/e}}
                \node[vertex] (\name) at \pos {$\name$};
            % Connect vertices with edges and draw weights
            \foreach \source/ \dest /\weight in {b/a/1, c/b/2,
                                                c/d/3}
                \path[edge] (\source) -- node[font=\small] {} (\dest);
            
                \end{tikzpicture}
        \end{figure}
        }
        \only<4>{\vspace*{-0.4cm}
            \begin{figure}
                \begin{tikzpicture}[scale=1, auto,swap]
            % Draw a 7,11 network
            % First we draw the vertices
            \foreach \pos/\name in {{(0,2)/a}, {(0,0)/b}, {(1,-1)/c},
                                    {(2,0)/d}, {(2,2)/e}}
                \node[vertex] (\name) at \pos {$\name$};
            % Connect vertices with edges and draw weights
            \foreach \source/ \dest /\weight in {b/a/1, c/b/2,
                                                c/d/3, d/e/5}
                \path[edge] (\source) -- node[font=\small] {} (\dest);
            
                \end{tikzpicture}
        \end{figure}
        }
    \end{column}
    \begin{column}{0.13\textwidth}
        \hspace*{2cm}
    \end{column}
\end{columns}
\only<2-4>{
\end{proof}
}
\end{frame}