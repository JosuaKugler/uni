\begin{frame}[t]
\frametitle{Grundlegende Eigenschaften eines Baums}
\begin{lemma}
        Ein Baum ist minimal zusammenhängend und maximal kreisfrei.
\end{lemma}
\only<2->{
\begin{proof}
    \vspace*{-0.6cm}
}%
\begin{columns}
    \begin{column}{0.58\textwidth}
        \begin{itemize}
            \item<2-> Kante entfernen 
            \begin{itemize}
                \item[$\implies$]<3-> Pfad unterbrochen 
                \item[$\overset{\mathclap{\text{Lemma 1}}}{\implies}$]<4-> nicht mehr zusammenhängend
            \end{itemize}
            \item<5-> Kante hinzufügen 
            \begin{itemize}
                \item[$\implies$]<6-> neuer Weg 
                \item[$\overset{\mathclap{\text{Lemma 1}}}{\implies}$]<7-> Kreis
            \end{itemize}
        \end{itemize}
    \end{column}
    \begin{column}{0.25\textwidth}
        \only<2> {
        \begin{figure}
                \begin{tikzpicture}[scale=1, auto,swap]
            % Draw a 7,11 network
            % First we draw the vertices
            \foreach \pos/\name in {{(0,2)/a}, {(0,0)/b}, {(1,-1)/c},
                                    {(2,0)/d}, {(2,2)/e}}
                \node[vertex] (\name) at \pos {$\name$};
            % Connect vertices with edges and draw weights
            \foreach \source/ \dest /\weight in {b/a/1, c/b/2,
                                                c/d/3, d/e/5}
                \path[edge] (\source) -- node[font=\small] {} (\dest);
            
                \end{tikzpicture}
        \end{figure}
        }

        \only<3-4> {
        \begin{figure}
                \begin{tikzpicture}[scale=1, auto,swap]
            % Draw a 7,11 network
            % First we draw the vertices
            \foreach \pos/\name in {{(0,2)/a}, {(0,0)/b}, {(1,-1)/c},
                                    {(2,0)/d}, {(2,2)/e}}
                \node[vertex] (\name) at \pos {$\name$};
            % Connect vertices with edges and draw weights
            \foreach \source/ \dest /\weight in {b/a/1, c/b/2,
                                                d/e/5}
                \path[edge] (\source) -- node[font=\small] {} (\dest);
            
                \end{tikzpicture}
        \end{figure}
        }

        \only<5> {
        \begin{figure}
                \begin{tikzpicture}[scale=1, auto,swap]
            % Draw a 7,11 network
            % First we draw the vertices
            \foreach \pos/\name in {{(0,2)/a}, {(0,0)/b}, {(1,-1)/c},
                                    {(2,0)/d}, {(2,2)/e}}
                \node[vertex] (\name) at \pos {$\name$};
            % Connect vertices with edges and draw weights
            \foreach \source/ \dest /\weight in {b/a/1, c/b/2,
                                                c/d/3, d/e/5}
                \path[edge] (\source) -- node[font=\small] {} (\dest);
            
                \end{tikzpicture}
        \end{figure}
        }
        \only<6> {
        \begin{figure}
                \begin{tikzpicture}[scale=1, auto,swap]
            % Draw a 7,11 network
            % First we draw the vertices
            \foreach \pos/\name in {{(0,2)/a}, {(0,0)/b}, {(1,-1)/c},
                                    {(2,0)/d}, {(2,2)/e}}
                \node[vertex] (\name) at \pos {$\name$};
            % Connect vertices with edges and draw weights
            \foreach \source/ \dest /\weight in {b/a/1, c/b/2,
                                                c/d/3, d/e/5,a/e/1}
                \path[edge] (\source) -- node[font=\small] {} (\dest);
            
                \end{tikzpicture}
        \end{figure}
        }
        \only<7> {
        \begin{figure}
                \begin{tikzpicture}[scale=1, auto,swap]
            % Draw a 7,11 network
            % First we draw the vertices
            \foreach \pos/\name in {{(0,2)/a}, {(0,0)/b}, {(1,-1)/c},
                                    {(2,0)/d}, {(2,2)/e}}
                \node[vertex] (\name) at \pos {$\name$};
            % Connect vertices with edges and draw weights
            \foreach \source/ \dest /\weight in {b/a/1, c/b/2,
                                                c/d/3, d/e/5, e/a/6}
                \path[edge] (\source) -- node[font=\small] {} (\dest);
            
            \begin{pgfonlayer}{background}
                \foreach \source / \dest in {b/a,c/b,d/c,d/e,e/a}
                    \path[selected edge] (\source.center) -- (\dest.center);
                    
            \end{pgfonlayer}
                \end{tikzpicture}
        \end{figure}
        }
    \end{column}
\begin{column}{0.13\textwidth}
    \hspace*{2cm}
\end{column}
\end{columns}
\only<2->{
\end{proof}
}
\end{frame}