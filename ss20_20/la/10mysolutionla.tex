\documentclass{article}
\usepackage[utf8]{inputenc}
\usepackage[T1]{fontenc}
\usepackage[ngerman]{babel}
\usepackage{amsmath, amsfonts, amsthm, mathtools, amssymb}
\usepackage{stmaryrd}
\usepackage{enumerate}
\usepackage{cases}
\usepackage{fancyhdr}
\usepackage{comment}
%\usepackage{xcolor}
\usepackage{tikz}
\usetikzlibrary{calc,intersections,through,backgrounds, babel}
\usepackage{tikz-cd}
\usepackage{cases}
\usepackage{listings}
\usepackage{siunitx}
\usepackage[left = 2cm, right = 2cm, top=2.5cm, bottom=2.5cm]{geometry}
\usepackage[hidelinks]{hyperref}
\usepackage{subcaption}
\usepackage{gauss}
\usepackage{environ}
\newtheorem{satz}{Satz}[section]
\newtheorem{lemma}[satz]{Lemma}
\newtheorem{korollar}[satz]{Korollar}
\newtheorem{proposition}[satz]{Proposition}
\theoremstyle{definition}
\newtheorem{definition}[satz]{Def.}
\newtheorem{axiom}[satz]{Axiom}
\newtheorem{bsp}[satz]{Bsp.}
\newtheorem*{anmerkung}{Anm}
\newtheorem{bemerkung}[satz]{Bem}
\newtheorem*{notatio}{Notation}
\newcommand{\obda}{O.B.d.A. }
\newcommand{\equals}{\Longleftrightarrow}
\newcommand{\N}{\mathbb{N}}
\newcommand{\Q}{\mathbb{Q}}
\newcommand{\R}{\mathbb{R}}
\newcommand{\Z}{\mathbb{Z}}
\newcommand{\C}{\mathbb{C}}
\newcommand{\intd}{\mathrm{d}}
\newcommand{\Pot}{\operatorname{Pot}}
\newcommand{\mychar}{\operatorname{char}}
\newcommand{\myker}{\operatorname{ker}}
\newcommand{\induktion}[3]
{\begin{proof}\ \\
	\noindent\textbf{Induktionsanfang:}\ #1\\
	\noindent\textbf{Induktionsvoraussetzung:}\ #2\\
	\noindent\textbf{Induktionsschluss:}\ #3
\end{proof}}

\newcommand{\rg}{\operatorname{rg}}
\newcommand{\im}{\operatorname{im}}
\newcommand{\End}{\operatorname{End}}
\newcommand{\abb}{\operatorname{Abb}}
\newcommand{\re}{\operatorname{Re}}
\newcommand{\Ima}{\operatorname{Im}}
\newcommand{\Fit}{\operatorname{Fit}}
\newcommand{\ggt}{\operatorname{GGT}}
\newcommand{\id}{\operatorname{id}}

\let\oldstackrel\stackrel
\renewcommand{\stackrel}[2]{%
    \oldstackrel{\mathclap{#1}}{#2}
}%
\ExplSyntaxOn

% S-tackrelcompatible ALIGN environment
% some might also call it the S-uper ALIGN environment
% uses regular expressions to calculate the widest stackrel
% to put additional padding on both sides of relation symbols
\NewEnviron{salign}
{
    \begin{align}
        \lec_insert_padding:V \BODY
    \end{align}
}
% starred version that does no equation numbering
\NewEnviron{salign*}
{
    \begin{align*}
        \lec_insert_padding:V \BODY
    \end{align*}
}

% some helper variables
\tl_new:N \l__lec_text_tl
\seq_new:N \l_lec_stackrels_seq
\int_new:N \l_stackrel_count_int
\int_new:N \l_idx_int
\box_new:N \l_tmp_box
\dim_new:N \l_tmp_dim_a
\dim_new:N \l_tmp_dim_b
\dim_new:N \l_tmp_dim_needed

% function to insert padding according to widest stackrel
\cs_new_protected:Nn \lec_insert_padding:n
 {
  \tl_set:Nn \l__lec_text_tl { #1 }
  % get all stackrels in this align environment
  \regex_extract_all:nnN { \c{stackrel}{(.*?)}{(.*?)} } { #1 } \l_lec_stackrels_seq
  % get number of stackrels
  \int_set:Nn \l_stackrel_count_int { \seq_count:N \l_lec_stackrels_seq }
  \int_set:Nn \l_idx_int { 1 }
  \dim_set:Nn \l_tmp_dim_needed { 0pt }
  % iterate over stackrels
  \int_while_do:nn { \l_idx_int <= \l_stackrel_count_int }
  {
      % calculate width of text
      \hbox_set:Nn \l_tmp_box {$\seq_item:Nn \l_lec_stackrels_seq { \l_idx_int + 1 }$}
      \dim_set:Nn \l_tmp_dim_a {\box_wd:N \l_tmp_box}
      % calculate width of relation symbol
      \hbox_set:Nn \l_tmp_box {$\seq_item:Nn \l_lec_stackrels_seq { \l_idx_int + 2 }$}
      \dim_set:Nn \l_tmp_dim_b {\box_wd:N \l_tmp_box}
      % check if 0.5*(a-b) > minimum padding, if yes updated minimum padding
      \dim_compare:nNnTF
        { 1pt * \dim_ratio:nn { \l_tmp_dim_a - \l_tmp_dim_b } { 2pt } } > { \l_tmp_dim_needed }
        { \dim_set:Nn \l_tmp_dim_needed { 1pt * \dim_ratio:nn { \l_tmp_dim_a - \l_tmp_dim_b } { 2pt } } }
        { }
      \quad
      % increment list index by three, as every stackrel produces three list entries
      \int_incr:N \l_idx_int
      \int_incr:N \l_idx_int
      \int_incr:N \l_idx_int
  }
  % replace all relations with align characters (&) and add the needed padding
  \regex_replace_all:nnN
      { (\c{approx}&|&\c{approx}|\c{equiv}&|&\c{equiv}|=&|&=|\c{le}&|&\c{le}|\c{ge}&|&\c{ge}|&\c{stackrel}{.*?}{.*?}|\c{stackrel}{.*?}{.*?}&|&\c{neq}|\c{neq}&) }
      { \c{kern} \u{l_tmp_dim_needed} \1 \c{kern} \u{l_tmp_dim_needed} }
      \l__lec_text_tl
  \l__lec_text_tl
 }
\cs_generate_variant:Nn \lec_insert_padding:n { V }
\ExplSyntaxOff



\newcommand{\lalayout}[1]
{	
	\pagestyle{fancy}
	\fancyhead[L]{Lineare Algebra 1, Blatt #1}
	\fancyhead[R]{David Wesner, Josua Kugler}
	\fancypagestyle{firstpage}{%
		\fancyhf{}
		\lhead{Dozent: Denis Vogel\\
			Tutor: Marina Savarino}
		\rhead{Lineare Algebra 2, Übungsblatt #1\\ David Wesner, Josua Kugler}
		\cfoot{\thepage}
	}
	\thispagestyle{firstpage}
}

\lstset{
    frame=tb, % draw a frame at the top and bottom of the code block
    tabsize=4, % tab space width
    showstringspaces=false, % don't mark spaces in strings
    numbers=left, % display line numbers on the left
    commentstyle=\color{green}, % comment color
    keywordstyle=\color{blue}, % keyword color
    stringstyle=\color{red} % string color
}
\setlength{\headheight}{25pt}
\begin{document}
\lalayout{10}
\section*{Aufgabe 36}
Es gilt
\begin{align*}
    \textstyle{\bigwedge^2} f_A(e_1\wedge e_2) &= f_A(e_1)\wedge f_A(e_2) = \begin{pmatrix}
        0\\1\\0
    \end{pmatrix} \wedge \begin{pmatrix}
        2\\1\\3
    \end{pmatrix}
    = e_2 \wedge 2e_1 + e_2\wedge e_2 + e_2\wedge 3e_3 = -2 (e_1\wedge e_2) + 3 (e_2\wedge e_3)\\
    \textstyle{\bigwedge^2}f_A(e_1\wedge e_3) &= f_A(e_1)\wedge f_A(e_3) = \begin{pmatrix}
        0\\1\\0
    \end{pmatrix} \wedge \begin{pmatrix}
        0\\1\\2
    \end{pmatrix}
    = e_2\wedge e_2 + e_2\wedge 2e_3 = 2(e_2\wedge e_3)\\
    \textstyle{\bigwedge^2}f_A(e_2\wedge e_3) &= f_A(e_2)\wedge f_A(e_3) = \begin{pmatrix}
        2\\1\\3
    \end{pmatrix} \wedge \begin{pmatrix}
        0\\1\\2
    \end{pmatrix}\\
    &= 2e_1\wedge e_2 + 2e_1\wedge 2e_3 + e_2\wedge e_2 + e_2\wedge 2e_3 + 3e_3\wedge e_2 + 3e_3\wedge 2e_3\\
    &= 2e_1\wedge e_2 + 4e_1\wedge e_3 - e_2\wedge e_3
\end{align*}
Für die gesuchte Darstellungsmatrix erhalten wir also
\[
    \begin{pmatrix}
        -2& 0 & 2\\
        0 & 0 & 4\\
        3 & 2 &-1
    \end{pmatrix}.
\]
\section*{Aufgabe 37}
\begin{enumerate}[(a)]
    \item Sei $n\in N$, $m\in M$ und $(m_1,\dots,m_n$ eine Basis von $M$. Ist $\varphi(n)\otimes m = 0$, so ist $\id_N\otimes\Phi^{-1}(\varphi(n)\otimes m) =  \id_L\otimes\Phi^{-1}(\varphi(n)\otimes \sum_{i = 1}^{n}m_i) = \varphi(n)\otimes (\alpha_1,\dots,\alpha_n)$ ebenfalls $0$ und mit dem Isomorphismus aus 8.14 auch $0 = (\alpha_1 \varphi(n), \dots, \alpha_n \varphi(n)) = (\varphi(\alpha_1 n), \dots, \varphi(\alpha_n n))$. Aufgrund der Injektivität von $\varphi$ muss also bereits $(\alpha_1 n, \dots, \alpha_n n) = 0$ sein. Wir können nun erneut 8.14 benutzen, um zu folgern, dass $n \otimes (\alpha_1,\dots,\alpha_n) = 0$ ist. Unter der Abbildung $\id_N\otimes\Phi$ erhalten wir $n\otimes m = 0$. Daraus folgt, dass $\ker \varphi \otimes \id_M = \{0\}$ sein muss, da sich die Argumentation auf Summen von reinen Tensoren überträgt.
    \item Da $M$ flach ist, folgt aus der Injektivität von $\varphi$ sofort die Injektivität von $(\varphi \otimes \id_M)\colon M\otimes M \to N\otimes M$ nach Definition von flachen Moduln. Da $N$ ebenfalls flach ist, folgt analog die Injektivität von $(\id_N \otimes \varphi)\colon N\otimes M \to N\otimes N$. Die Komposition zweier injektiver Abbildungen ist wieder injektiv, sodass $\varphi \otimes \varphi = (\id_N \otimes \varphi) \circ (\varphi \otimes \id_M)$ injektiv ist.
    \item Wähle $R = \Z$ und $M= \Z /2\Z$. Dann ist nach Beispiel 8.12 $M$ nicht flach.
\end{enumerate}
\section*{Aufgabe 38}
\begin{enumerate}[(a)]
    \item Seien $f\colon \textstyle{\bigwedge^2}M \to M \otimes M$ und $g\colon \textstyle{\bigwedge^2}N \to N \otimes N$ die entsprechenden eindeutigen $R$-Modulhomomorphismen aus Aufgabe 35. Da $M$ und $N$ beide freie Moduln sind, sind $f$ und $g$ nach Aufgabe 35b beide injektiv, genau wie $(\varphi \otimes \varphi)\colon M\otimes M\to N\otimes N$ nach Aufgabe 37b.\\
    \textbf{Behauptung:} Das Diagramm 
    \[\begin{tikzcd}
        M\otimes M \ar[r,hook,"\varphi \otimes \varphi"] & N\otimes N\\
        \bigwedge^2M \ar[u, hook,"f"] \ar[r,"\bigwedge^2 \varphi"] & \bigwedge^2 N \ar[u,hook,"g"]
    \end{tikzcd}\] kommutiert.
    \begin{proof}
        Wir definieren die offensichtlich bilineare Abbildung
        \[
            \beta\colon M\times M\to N\otimes N,\; (a,b) \mapsto \varphi(a)\otimes \varphi(b) - \varphi(b)\otimes\varphi(a).
        \]
        Wegen der universellen Eigenschaft UA gibt es daher einen eindeutigen $R$-Modulhomomorphismus
        \[
            F\colon \textstyle{\bigwedge^2}M \to N\otimes N,\; a\wedge b \mapsto  \varphi(a)\otimes \varphi(b) - \varphi(b)\otimes\varphi(a).
        \]
        Man sieht leicht, dass $((\varphi \otimes \varphi) \circ f)(a,b) = F(a,b)$ und $(g\circ(\textstyle{\bigwedge^2} \varphi))(a,b) = F(a,b)$. Da $F$ aber eindeutig bestimmt ist, sind beide Abbildungen gleich.
    \end{proof}
    Daher ist auch der Kern beider Abbildungen gleich. 
    \[
        \ker ((\varphi \otimes \varphi) \circ f) = 0 = \ker (f\circ(\textstyle{\bigwedge^2} \varphi)).
    \]
    Ist die Komposition zweier Abbildungen injektiv, so auch die zwei Komponenten. Also muss $\textstyle{\bigwedge^2} \varphi$ injektiv sein.
    \item \ \begin{itemize}
        \item[(i)$\implies$(ii)] Es gilt $\ker \psi = \{(a,b)\in R^2 \mid \psi(a,b) = 0\} = \{(a,b)\in R^2\mid am_1 + bm_2 = 0\} \overset{(m_1,m_2)\text{ l.u.}}{=} \{0\}$. Daher ist $\psi$ injektiv. Nach Satz 9.9 lässt sich jedes $x\in \textstyle{\bigwedge^2}R^2$ auf eindeutige Weise durch $r\cdot \left(e_1\wedge e_2\right)$ mit $r\in R$ schreiben. Also ist \[(\textstyle{\bigwedge^2} \psi)(x) = (\textstyle{\bigwedge^2} \psi)(r\cdot (e_1\wedge e_2)) = r \cdot (\psi(e_1)\wedge \psi(e_2)) = r\cdot (m_1\wedge m_2).\]
        Ist nun $r\cdot (m_1\wedge m_2) = 0$, dann ist $r\cdot (e_1\wedge e_2)\in \ker \psi$. Da $\psi$ injektiv ist, muss also schon $r = 0$ sein.
        \item[(ii)$\implies$(i)] Wir zeigen die Aussage per Kontraposition. Seien $0 \neq a,b\in R$ mit $am_1+ bm_2 = 0$ gegeben. Dann gilt für $r = b$.
        \[b(m_1\wedge m_2) = (m_1\wedge bm_2) = (m_1\wedge -am_1) = -a(m_1\wedge m_1) = 0.\]
    \end{itemize}
        \item Sei $(m_1,m_2)$ eine Basis von $M$. Da $m_1\wedge m_2$ ein Erzeugendensystem ist, können wir jedes Element aus $\textstyle{\bigwedge^2} M$ schreiben als $r (m_1\wedge m_2)$ für geeignetes $r\in R$. Es gilt \[r \cdot (m_1\wedge m_2) \in \ker (\textstyle{\bigwedge^2} \varphi) \equals 0 = (\textstyle{\bigwedge^2} \varphi)(r\cdot(m_1\wedge m_2)) = r \cdot (\textstyle{\bigwedge^2} \varphi)(m_1\wedge m_2) = r \cdot \det \varphi \cdot (m_1\wedge m_2) \equals r\cdot \det \varphi = 0.\]
        Offensichtlich ist also die Injektivität von $(\textstyle{\bigwedge^2} \varphi)$ äquivalent dazu, dass $\det \varphi$ kein Nullteiler ist.
        \begin{itemize}
            \item[(i)$\implies$(ii)] Nach Teilaufgabe a ist $\textstyle{\bigwedge^2}\varphi \colon \textstyle{\bigwedge^2} M \to \textstyle{\bigwedge^2} M$ injektiv, da $M$ endlich erzeugt und frei ist und $\varphi$ injektiv ist.
            \item[(ii)$\implies$(i)] Angenommen, $\textstyle{\bigwedge^2}\varphi \colon \textstyle{\bigwedge^2} M \to \textstyle{\bigwedge^2} M$ wäre injektiv, aber $\varphi$ wäre nicht injektiv. Dann gäbe es $0\neq a,b\in R$ bzw. ein $0\neq am_1 + bm_2\in R$, sodass $\varphi(m) = 0$. Ist nun $a \neq b$, so erhalten wir daraus $(am_1 + bm_2)\wedge (m_1 + m_2) = (a-b) (m_1\wedge m_2) \neq 0$, aber $(\textstyle{\bigwedge^2}\varphi)((am_1 + bm_2)\wedge (m_1 + m_2)) = \varphi(am_1 + bm_2) \wedge \varphi(bm_1 + am_2) = 0 \wedge \varphi(bm_1 + am_2) = 0$. Ist stattdessen $a = b$, so ist $(am_1 + bm_1)\wedge(m_1-m_2) = 2a(m_1\wedge m_2)$, aber $(\textstyle{\bigwedge^2}\varphi)((am_1 + bm_2)\wedge (m_1 - m_2)) = \varphi(am_1 + bm_2) \wedge \varphi(m_1 - m_2) = 0 \wedge \varphi(m_1 - m_2) = 0$. Damit erhalten wir einen Widerspruch zur Injektivität von $\textstyle{\bigwedge^2}\varphi$. Also muss  $\varphi$ injektiv sein.
        \end{itemize}
\end{enumerate}
\section*{Aufgabe 39}
\begin{enumerate}[(a)]
    \item Offensichtlich ist $f$ injektiv. Es gilt $g(2n,0) = (\overline{2n},\overline{0},\dots) = 0$. Daher ist $\im f \subset \ker g$. Sei nun $(i,(j_1,j_2,\dots)) \in \ker g$. Daraus folgt sofort, dass $(j_1,j_2,\dots) = 0$ sein muss. $i$ muss gerade sein, da genau dann $\overline{i} = 0$ ist. Also ist $(i,(j_1,j_2,\dots)) = (2k,0)$ für ein geeignetes $k\in \N$. Die kanonische Projektion ist trivialerweise surjektiv, also insbesondere auch die komponentenweise kanonische Projektion bzw. Identität. Insgesamt folgt, dass es sich um eine kurze exakte Folge von $\Z$-Moduln handelt.
    \item Sei ein Untermodul $T$ von $N\otimes M$ gegeben, der die Eigenschaften aus Bemerkung 10.6 erfüllt. Dann existiert ein eindeutig bestimmtes Urbild $(a,(0,\dots))$ von $(1,0,\dots)$ unter $g|_{T}$. Nun betrachten wir $2(a,(0,\dots))$. Es gilt $g_{T}(2(a,(0,\dots))) = 2(1,0,\dots) = 0$, aber auch $2(a,(0,\dots)) = (a,2(0,\dots)) = (a,(0,\dots))$. Aufgrund der Injektivität von $g_{T}$ muss daher schon $(a,(0,\dots)) = 0$ sein. Nach Konstruktion ist aber $g_{T}((a,(0,\dots))) \neq 0$. Daher erhalten wir einen Widerspruch und die Folge zerfällt nicht.
\end{enumerate}
\end{document}