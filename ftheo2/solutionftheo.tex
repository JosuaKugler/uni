\documentclass{article}

\usepackage{josuamathheader}

\begin{document}
  \section*{Aufgabe 32}
  \begin{enumerate}[(a)]
    \item Es gilt nach Skript $\Im M\langle z\rangle = \frac{\Im(z)}{|cz+d|^2}$. Daher gilt
    \[
        f(M\langle z\rangle) \overline{g(M\langle z\rangle)} (\Im M\langle z\rangle)^k = (cz + d)^k f(z) \overline{(cz+d)^kg(z)} \frac{y^k}{|cz+d|^{2k}} = f(z)\overline{g(z)}y^k.
    \]
    \item Sei OE $f$ die Spitzenform. Sei außerdem $\Gamma(N) \subset \Lambda$. Dann gilt
    \begin{align*}
      \langle f, g\rangle &= \int_{\mathcal{F}} f(z)\overline{g(z)}y^k \frac{\d{x}\d{y}}{y^2}\\
      &\leq \int_{\mathcal{F}} \left|f(z)\overline{g(z)}y^{k-2} \right|\d{x}\d{y}
      \intertext{$g(z)$ ist auf dem gesamten Fundamentalbereich beschränkt, $|\overline{g(z)}| \leq C$.}
      &\leq C \cdot \int_{\mathcal{F}} \left|f(z)y^{k-2} \right|\d{x}\d{y}
      \intertext{Es gilt $\mathcal{F} \subset [0,N] \times i(0, \infty)$}
      &\leq C \cdot \int_0^N \int_0^\infty \left|f(x + iy)y^{k-2} \right| \d{y}\d{x}
      \intertext{$f$ Spitzenform $\implies \lim\limits_{y \to i\infty} \left| y^kf(z) \right| = \lim\limits_{y \to i\infty} \left|y^k\sum_{n = 1}^{\infty} a_n e^{\frac{2\pi i n z}{N}}\right| \leq \sum_{n = 1}^{\infty} |a_n| \lim\limits_{y \to i\infty} y^k e^{-\frac{2\pi n}{N} y} = 0$}
      &\leq C \cdot \int_0^N \left[\int_0^D \left|f(x + iy)y^{k-2} \right| \d{y} + \int_D^\infty \frac{\d{y}}{y^2}\right]\d{x}\\
      &\leq C \int_0^N \int_0^D \left|f(x + iy)y^{k-2} \right| \d{y} \d{x} + \frac{CN}{D}
      \intertext{Das Integral über eine stetige Funktion auf einem Kompaktum ist endlich}
      &< \infty
    \end{align*}
    \item Die Sesquilinearität folgt sofort aus der Kommutativität der Multiplikation und der Linearität des Integrals.
    Wir verwenden nun den Integraltransformationssatz.
    Es gilt 
    \begin{align*}
      \int_{M\mathcal{F}} f(z) \overline{g(z)}y^k \frac{\d{x}\d{y}}{y^2} &= \int_\mathcal{F} f(M\langle z\rangle) \overline{g(M\langle z\rangle)} (\Im M\langle z\rangle)^k |\det \mathrm{D}(M\langle z\rangle)|\frac{\d{x}\d{y}}{(\Im M\langle z\rangle)^2}\\
      &= \int_\mathcal{F} f(z) \overline{g(z)} y^k \left|\frac{a(cz + d) - c(az + d)}{(cz+d)^2}\right|^2 (|cz + d|^2)^2\frac{\d{x}\d{y}}{y^2}\\
      &= \int_\mathcal{F} f(z)\overline{g(z)}y^k \frac{1}{|cz + d|^4}|cz + d|^4 \frac{\d{x}\d{y}}{y^2}\\
      &= \int_\mathcal{F} f(z) \overline{g(z)}y^k \frac{\d{x}\d{y}}{y^2}
    \end{align*}
  \end{enumerate}
  \section*{Aufgabe 34}
  \begin{enumerate}[(a)]
    \item Es gilt 
    \begin{align*}
      \sum_{n = 0}^{\infty} \left|\chi(n)n^{-s}\right| &= \sum_{n = 0}^{\infty} 1 \cdot n^{-\Re s},
    \end{align*}
    da die Werte eines Charakters im Einheitskreis liegen. Für $\alpha > 1$ gilt
    \begin{align*}
      \int_0^\infty n^{-\alpha} \d{n} &= \frac{1}{1 - \alpha} n^{1-\alpha}\bigg|_0^\infty = 0.
    \end{align*}
    Nach dem Integralkriterium ist die Dirichletreihe $L(s, \chi)$ daher für jedes $s > 1$ absolut konvergent.
    Sei $K$ ein Kompaktum in $\{s\in \C | \Re s > 1\}$. Dann ist aufgrund der Kompaktheit $\min_{s\in K} s > 1$.
    Für $\Re s < \Re s'$ gilt schließlich $n^{-\Re s} > n^{-\Re s'}$, woraus die kompakte Konvergenz von $L(s, \chi)$ folgt.
    \item Zunächst zeigen wir, dass das Produkt absolut konvergent ist.
    \begin{align*}
      \sum_{\operatorname{ggT}(p, N) = 1} \left| 1 - \frac{1}{1 - \frac{\chi(p)}{p^s}}\right| &\leq  \sum_{\operatorname{ggT}(p, N) = 1} \left| \frac{1 - \frac{\chi(p)}{p^s} - 1}{1 - \frac{\chi(p)}{p^s}}\right|\\
      &\leq  \sum_{\operatorname{ggT}(p, N) = 1}  \frac{|\chi(p)|}{\left|p^s - \chi(p)\right|}\\
      &\leq \sum_{\operatorname{ggT}(p, N) = 1}  \frac{|\chi(p)|}{|p^s|- |\chi(p)|}\\
      &= \sum_{\operatorname{ggT}(p, N) = 1}  \frac{1}{p^{\Re s} - 1}
      \intertext{$p \geq 2, \Re s > 1$}
      &\leq \sum_{\operatorname{ggT}(p, N) = 1} \frac{2}{p^{\Re s}}
      &< 2 \cdot L(s, 1) < \infty
    \end{align*}
    Nun zeigen wir noch, dass der Wert des Produkts mit $L(s, \chi)$ übereinstimmt. 
    Sei $\mathcal{M} = \{p \in \N\colon p \text{ prim, } \operatorname{ggT}(p, N) = 1\}$. 
    Sei $p_\nu$ das Element von $\mathcal M$, für das es genau $\nu - 1$ kleinere Elemente in $\mathcal M$ gibt.
    Zunächst benutzen wir hier die geometrische Reihenentwicklung
    \begin{align*}
      \prod_{\operatorname{ggT}(p, N) = 1} \frac{1}{1 - \frac{\chi(p)}{p^s}} &= \lim\limits_{M \to \infty} \prod_{\nu = 1}^M \frac{1}{1 - \frac{\chi(p_\nu)}{p_\nu^s}}\\
      &= \lim\limits_{M \to \infty} \prod_{\nu = 1}^M \sum_{n_\nu = 0}^{\infty} \left(\frac{\chi(p_\nu)}{p_\nu^s}\right)^n\\
      \intertext{Nach dem Reihenmultiplikationssatz von Cauchy gilt}
      &= \lim\limits_{M \to \infty}\sum_{n_1, \dots, n_\nu = 0}^{\infty} \prod_{\nu = 1}^M \left(\frac{\chi(p_\nu)}{p_\nu^s}\right)^{n_\nu}\\
      &= \lim\limits_{M \to \infty} \sum_{n_1, \dots, n_\nu = 0}^{\infty} \frac{\chi\left(\prod_{\nu = 1}^M p_\nu^{n_\nu}\right)}{\left(\prod_{\nu = 1}^M p_\nu^{n_\nu}\right)^s}
      \intertext{Genau die Zahlen $n$, die in ihrer Primfaktorzerlegung nur Primzahlen $p \in \mathcal M$ enthalten, erhalten wir durch $\prod_{\nu = 1}^\infty p_\nu^{n_\nu}$. Für solche $n$ gilt aber $\operatorname{ggT}(n, N) = 1$.}
      &= \sum_{\operatorname{ggT}(n, N) = 1} \frac{\chi(n)}{n^s}\\
      \intertext{Für $\operatorname{ggT}(n, N) \neq 1$ gilt aber bereits $\chi(n) = 0$, da $n$ in $\Z/N\Z$ dann nicht mehr invertierbar ist.}
      &= \sum_{n = 1}^{\infty} \chi(n)n^{-2}\\
      &= L(s, \chi)
    \end{align*}
  \end{enumerate}
\end{document}