\documentclass{article}

\usepackage{josuamathheader}


\begin{document}

  \newcommand{\norm}[1]{\lVert #1 \rVert}
  \section*{Aufgabe 47}
  \begin{enumerate}[(a)]
    \item Es gilt
    \begin{align*}
      (0,0) * (a_1, a_2) = (a_1, a_2) \forall (a_1, a_2) \in \Z^2
    \end{align*}
    Damit ist $(0,0)$ das neutrale Element.
    Weiter ist durch
    \begin{align*}
      ((-1)^{-a_2 + 1}a_1, -a_2) * (a_1, a_2) = ((-1)^{-a_2 + 1}a_1 + (-1)^{-a_2}a_1, -a_2 + a_2) = (0,0)
    \end{align*}
    das Inverse zu $(a_1, a_2)$ bestimmt.
    Die Assoziativität folgt durch
    \begin{align*}
      ((a_1, a_2) * (b_1, b_2))*(c_1, c_2) &= (a_1 + (-1)^{a_2}b_1, a_2 + b_2)* (c_1, c_2)\\
      &= (a_1 + (-1)^{a_2}b_1 + (-1)^{a_2 + b_2}c_1, (a_2 + b_2) + c_2)\\
      &= (a_1 + (-1)^{a_2}(b_1 + (-1)^{b_2}c_1), a_2 + (b_2 + c_2))\\
      &= (a_1, a_2) * (b_1 + (-1)^{b_2}c_1, b_2 + c_2)\\
      &= (a_1, a_2) * ((b_1, b_2) * (c_1, c_2))
    \end{align*}
    Offensichtlich ist jedes Produkt wieder in $\Z^2$ enthalten. Dadurch wird $(\Z^2, *)$ zu einer Gruppe. Wegen
    \begin{align*}
      (1, 2)*(2, 1) &= (1 + (-1)^{2}2, 2 + 1) = (3, 3)\\
      (2, 1)*(1, 2) &= (2 + (-1)^{1}1, 1 + 2) = (1, 3)
    \end{align*}
    ist die Gruppe nicht abelsch.
    $(0, 0) * (x_1, x_2) = (x_1, x_2)$ mit $(x_1, x_2) \in \R^2$ folgt analog zum Beweis, dass $(0,0)$ neutrales Element von $(\Z^2, *)$ ist.
    $(a_1, a_2) * ((b_1, b_2) * (x_1, x_2)) = ((a_1, a_2) * (b_1, b_2)) * (x_1, x_2)$ folgt analog zum Beweis der Assoziativität.
    Daher handelt es sich um eine Linksoperation. Diese ist wegen
    \[ 
      D [(a_1, a_2) * (b_1, b_2)] = \begin{pmatrix}
        (-1)^{a_2} & 0\\ 0 & 1
      \end{pmatrix}
    \]
    differenzierbar. Offensichtlich sind alle höheren partiellen Ableitungen $0$.
    Daher handelt es sich um eine glatte Gruppenoperation. 
    \item Wir zeigen die beiden Eigenschaften einer freien Operation.
    \begin{enumerate}[(1)]
      \item Wähle zu $x \in \R^2$ die offene Umgebung $U_{1/2}(x)$.
      Man sieht (u.a. aus Symmetriegründen) schnell ein, dass $(a_1, a_2) * U_{\epsilon}(x) = U_{\epsilon}((a_1, a_2) * x)$ gelten muss.
      Daher erhalten wir
      \begin{align*}
        (a_1, a_2) * U_{1/2}(x) \cap U_{1/2}(x) \neq \emptyset &\Leftrightarrow U_{1/2}((a_1, a_2) * x) \cap U_{1/2}(x)\neq \emptyset\\
        &\implies U_{1/2}((a_1 + (-1)^{a_2}x_1, a_2 + x_2)) \cap U_{1/2}((x_1, x_2))\neq \emptyset\\
        &\implies |a_1 + (-1)^{a_2}x_1 - x_1|^2 + |a_2 + x_2 - x_2|^2 < 1\\
        &\implies |a_1 + (-1)^{a_2}x_1 - x_1|^2 + |a_2|^2 < 1\\
        &\xRightarrow{a_2 \in \Z} |a_1 + (-1)^{a_2}x_1 - x_1|^2 < 1 \land a_2 = 0\\
        &\implies |a_1 + x_1 - x_1|^2 < 1 \land a_2 = 0\\
        &\implies |a_1|^2 < 1 \land a_2 = 0\\
        &\xRightarrow{a_1 \in \Z} a_1 = a_2 = 0
      \end{align*}
      \item Seien $(x_1, x_2) \not \sim (y_1, y_2) \in \R^2$ gegeben. 
      Wieder nutzen wir $(a_1, a_2) * U_{\epsilon}(x) = U_{\epsilon}((a_1, a_2) * x)$.
      Daher genügt es zu zeigen, dass $\norm {(a_1, a_2) * (x_1, x_2) - (y_1, y_2)} \geq 2 \epsilon$ gilt.
      Dann sind nämlich beliebig Translate der $\epsilon$-Umgebungen von $x$ und $y$ disjunkt.
      Wir unterscheiden drei Fälle
      \begin{enumerate}
        \item $x_2 - y_2 \notin \Z$. Setze $\epsilon = \frac{x_2-y_2 \mod \Z}{2}$.
        Wegen 
        \[ 
          \norm{(a_1, a_2) * (x_1, x_2) - (y_1, y_2)} \geq \sqrt{|a_2 + x_2 - y_2|^2} \geq \sqrt{(x_2 - y_2 \mod \Z)^2}
          \geq 2 \cdot \epsilon
        \]
        folgt die Aussage für diesen Fall.
        \item $x_2 - y_2 \in 2\Z$. Setze $\epsilon = \frac{x_1 - y_1 \mod \Z}{2}$.
        Insbesondere ist $\epsilon < \frac{1}{2}$. Angenommen, $\epsilon = 0$. Dann wäre
        $y_1 - x_1, y_2 - x_2)*(x_1, x_2) = (y_1 - x_1 + (-1)^{x_2 - y_2}x_1, y_2 - x_2 + x_2) = (y_1, y_2)$, Widerspruch.
        Also $0 < \epsilon < \frac{1}{2}$.
        Daher gilt für alle $(a_1, a_2)* (x_1, x_2) = (a_1 + (-1)^{a_2}x_1, a_2 + x_2)$ mit $a_2 + x_2 \neq y_2$ bereits
        \[ 
          \norm{(a_1, a_2) * (x_1, x_2) - (y_1, y_2)} \geq \sqrt{|a_2 + x_2 - y_2|^2} \geq 1
          \geq 2 \cdot \epsilon.
        \]
        Wir müssen also nur $(a_1, a_2)$ betrachten mit $a_2 + x_2 = y_2$. 
        Aufgrund der Voraussetzung gilt $y_2 - x_2 \in 2\Z$.
        Es folgt $(a_1, y_2 - x_2) * (x_1, x_2) = (a_1 + (-1)^{y_2 - x_2}x_1, y_2) = (a_1 + x_1, y_2)$.
        Schließlich erhalten wir
        \[
            \norm{(a_1, y_2 - x_2) * x - y} = |a_1 + x_1 - y_1| \geq x_1 - y_1 \mod \Z \geq 2\epsilon.
        \]
        \item $x_2 - y_2 \in 2\Z + 1$. Setze $\epsilon = \frac{x_1 + y_1 \mod \Z}{2}$.
        Insbesondere ist $\epsilon < \frac{1}{2}$. Angenommen, $\epsilon = 0$. Dann wäre
        $y_1 + x_1, y_2 - x_2)*(x_1, x_2) = (y_1 + x_1 + (-1)^{x_2 - y_2}x_1, y_2 - x_2 + x_2) = (y_1, y_2)$, Widerspruch.
        Also $0 < \epsilon < \frac{1}{2}$.
        Daher gilt für alle $(a_1, a_2)* (x_1, x_2) = (a_1 + (-1)^{a_2}x_1, a_2 + x_2)$ mit $a_2 + x_2 \neq y_2$ bereits
        \[ 
          \norm{(a_1, a_2) * (x_1, x_2) - (y_1, y_2)} \geq \sqrt{|a_2 + x_2 - y_2|^2} \geq 1
          \geq 2 \cdot \epsilon.
        \]
        Wir müssen also nur $(a_1, a_2)$ betrachten mit $a_2 + x_2 = y_2$. 
        Aufgrund der Voraussetzung gilt $y_2 - x_2 \in 2\Z$.
        Es folgt $(a_1, y_2 - x_2) * (x_1, x_2) = (a_1 + (-1)^{y_2 - x_2}x_1, y_2) = (a_1 - x_1, y_2)$.
        Schließlich erhalten wir 
        \[
            \norm{(a_1, y_2 - x_2) * x - y} = |a_1 - x_1 - y_1| \geq x_1 + y_1 \mod \Z \geq 2\epsilon.
        \]
      \end{enumerate}
    \end{enumerate}
    \item Wir haben oben bereits gesehen, dass die Gruppenoperation glatt ist wegen \[ 
      D [(a_1, a_2) * (b_1, b_2)] = \begin{pmatrix}
        (-1)^{a_2} & 0\\ 0 & 1
      \end{pmatrix}.
      \]
      Identifiziert man $\R^2 \cong \C$, so verstößt diese Jacobimatrix für ungerade $a_2$ gegen die 
      Cauchy-Riemann-Differentialgleichungen.
      Daher ist die Gruppenoperation nicht holomorph. 
      Insbesondere wird $G\backslash \C$ nicht zu einer Riemannschen Fläche.
  \end{enumerate}

  \section*{Aufgabe 48}
    Offensichtlich ist $p_1 \times p_2 \colon X_1 \times X_2 \to Y_1 \times Y_2$ surjektiv.
    Sei $(x^1, x^2) \in X_1 \times X_2$. Dann existieren nach Definition der Überlagerung Umgebungen $x^1 \in U^1, x^2 \in U^2$ mit
    \begin{align*}
      p_k^{-1}(U^k) = \biguplus_{i\in F^k}U^k_i,
    \end{align*}
    sodass alle Einschränkungen $p_k|_{U^k_i}\colon U_k^i \overset{\sim}{\to} U^k$ bistetig sind für $k \in \{1,2\}$.
    Daher gilt
    \begin{align*}
      (p_1 \times p_2)^{-1}(U^1 \times U^2) &= \{(x_1, x_2) | p(x_1) \in U^1, p(x_2) \in U^2\}\\
      &= \{(x_1, x_2) | x_1 \in \biguplus_{i\in F^1}U^1_i, x_2 \in \biguplus_{j\in F^2}U^2_j\}\\
      &= \biguplus_{i\in F^1} \{(x_1, x_2) | x_1 \in U^1_i, x_2 \in \biguplus_{j\in F^2}U^2_j\}\\
      &= \biguplus_{i\in F^1} \biguplus_{j\in F^2} \{(x_1, x_2) | x_1 \in U^1_i, x_2 \in U^2_j\}\\
      &= \biguplus_{(i,j) \in F^1\times F^2} U^1_i \times U^2_j
    \end{align*}
    Wegen $p_k|_{U^k_i}\colon U_k^i \overset{\sim}{\to} U^k$ bistetig für $K \in \{1,2\}$ folgt, dass
    \[
        p_1 \times p_2|_{U^1_i\times U^2_j} \colon U^1_i \times U^2_j \to U^1 \times U^2
    \]
    bezüglich der Produkttopologie bistetig sein muss.
    Damit handelt es sich bei $p_1 \times p_2$ ebenfalls um eine Überlagerung.

\end{document}