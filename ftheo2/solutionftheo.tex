\documentclass{article}

\usepackage{josuamathheader}

\begin{document}
  \section*{Aufgabe 40}
  \begin{enumerate}[(a)]
    \item Angenommen, eine Folge in einem separierten Raum $(X, \mathfrak{U}_X)$ hätte zwei verschiedene Grenzwerte $x \neq x'$. 
    Aufgrund der Separiertheit existieren Umgebungen $U, U' \in \mathfrak{U}_X$ mit $x \in U, x'\in U'$ und $U \cap U' = \emptyset$.
    Dann liegen alle bis auf endlich viele Folgenglieder in $U$. Insbesondere liegen höchstens endlich viele Folgenglieder in $U'$.
    Das ist aber ein Widerspruch dazu, dass $x'$ ein Grenzwert der Folge ist.
    \item Wir betrachten die Menge $\R \cup \{0'\}$ und 
    statten sie aus mit der Standardtopologie, wobei jede Umgebung der $0$ auch $0'$ enthalten soll.
    Dann besitzt die Folge $(1/n)_{n\in \N}$ die beiden Grenzwerte $0$ und $0'$. 
    Sowohl jede Umgebung von $0$ als auch jede Umgebung von $0'$ enthält nämlich alle bis auf endlich viele Folgenglieder. 
  \end{enumerate}
  \section*{Aufgabe 41}
  Wir zeigen zunächst die Implikation (a) $\implies$ (b). Es gilt
  \begin{enumerate}[(1)]
    \item Sei $x\in X$ und $g\in G\setminus\{e\}$. Da die Operation frei ist, existiert eine offene Menge $U_x$ mit $x\in U_x$ derart, 
    dass $g(U_x) \cap U_x \neq \emptyset \implies g = e$. Für $g\neq e$ erhalten wir daher
    $g(U_x) \cap U_x = \emptyset$.
    \item 
  \end{enumerate}
  Nun zeigen wir die Implikation (b) $\implies$ (a). Es gilt
  \begin{enumerate}[(1)]
    \item Z.Z.: $\forall x \in X: \exists U_x$ offen mit $x\in U_x$ und $g(U_x) \cap U_x \neq \emptyset \implies g = e$.
    \begin{proof}
      Sei $x \in X$. Wähle dann eine Umgebung $U_0$ von $x$ derart, dass ein Kompaktum $K$ mit $U_0 \subset K$ existiert.
      Sei dann
      \begin{align*}
        M \coloneqq \{g \in G\colon K\cap g(K) \neq \emptyset\}.
      \end{align*}
      Wegen Eigenschaft (2) muss diese Menge endlich sein.
      Aufgrund der Separiertheit existieren für beliebiges $g \in M$ Umgebungen $U_g, U_x^g$ mit $gx \in U_g, x\in U_x^g$ derart, 
      dass $U_g \cap U_x^g \neq \emptyset \implies g = e$. Da $M$ nur eine endliche Menge ist, handelt es sich bei 
      \begin{align*}
        U_x' \coloneqq \bigcap_{g\in M} U_x^g
      \end{align*}
      wieder um eine offene Menge.
      Es gilt nun für alle $g\in M$: 
      \begin{align*}
        U_g \cap U_x' \neq \emptyset \implies g = e.
      \end{align*}
      Definiere schließlich
      \begin{align*}
        U_x \coloneqq \bigcap_{g\in M} g^{-1}(U_g) \cap U_x' \cap U_0
      \end{align*}
      Als endlicher Schnitt offener Mengen ist auch $U_x$ offen und es gilt
      \begin{align*}
        g(U_x) \cap U_x \neq \emptyset &\implies g(K) \cap K \neq \emptyset\\
        &\implies g \in M\\
        &\implies g(U_x) = g\left(\bigcap_{g\in M} g^{-1}(U_g) \cap U_x' \cap U_0\right)\subset g(g^{-1}(U_g)) = U_g
      \end{align*}
      Außerdem gilt $U_x \subset U_x'$. Daher erhalten wir $g(U_x) \cap U_x \subset U_g \cap U_x'$.
      Wegen $\emptyset \neq g(U_x) \cap U_x$ folgt $\emptyset \neq U_g \cap U_x' \implies g = e$.
    \end{proof}
    \item Z.Z.: $x \not \sim_G y \implies \exists U_x, U_y \in \mathfrak{U}_X$ mit $x\in U_x, y\in U_y$ und $U_y \cap g(U_x) = \emptyset \forall g\in G$.
    \begin{proof}
      Seien $x, y \in X$ gegeben mit $x\not \sim_G y$. 
      Aufgrund der Separiertheit können wir Umgebungen $U_x'$ und $U_y'$ wählen mit $x \in U_x'$ und 
      $y\in U_y'$, $U_x' \cap U_y' = \emptyset$ und $U_x' \subset K_1, U_y' \subset K_2$ für Kompakta $K_1, K_2 \subset X$.
      Sei $M \coloneqq \{g \in G\colon g(K_1) \cap K_2 \neq \emptyset\}$. Dann ist $M$ wegen (2) endlich.
      Analog wie im letzten Beweis finden wir aufgrund der Separiertheit und der Endlichkeit von $M$ offene Umgebungen $U_y''$ und $U_x''$, 
      mit $U_y'' \cap g(U_x'') = \emptyset \forall g\in M$.
      Schließlich definieren wir $U_x \coloneqq U_x' \cap U_x''$ und $U_y \coloneqq U_y' \cap U_y''$.
      Dann gilt für $g \in G\setminus M$
      \begin{align*}
        U_y \cap g(U_x) \subset U_y' \cap g(U_x') \subset K_2 \cap g(K_1) = \emptyset
      \end{align*}
      und für $g\in M$
      \begin{align*}
        U_y \cap g(U_x) \subset U_y'' \cap g(U_x'') = \emptyset.
      \end{align*}
      Damit ist die Behauptung bewiesen.
    \end{proof}
  \end{enumerate}
\end{document}