\documentclass{article}

\usepackage{josuamathheader}

\begin{document}
  \section*{Aufgabe 22}
  \begin{enumerate}[(a)]
    \item Nach VL ist $G_4^a G_6^b$ mit $4a + 6b = 2k$ eine Modulform vom Gewicht $2k$. Es gilt daher
    \[
        0 \equiv \sum_{a,b \in \N_0} c_{a,b} G_4^a G_6^b = \sum_{k = 1}^{\infty} \underbrace{\sum_{4a + 6b = 2k} c_{a,b} G_4^a G_6^b}_{\text{Modulform vom Gewicht }2k}
    \]
    Da Modulformen vom Gewicht $2k$ gerade die Forderungen von Aufgabe 21 erfüllen und Polynome höchstens endlich viele Koeffizienten $\neq 0$ besitzen, lässt sich aus
    \[
        0 \equiv \sum_{k = 1}^{\infty} \sum_{4a + 6b = 2k} c_{a,b} G_4^a G_6^b
    \]
    bereits $\forall k \geq 0$
    \[
        0 \equiv \sum_{4a + 6b = 2k} c_{a,b} G_4^a G_6^b
    \]
    schließen. Für negatives $k$ gilt sowieso $M_k = 0$ und damit auch $\sum_{4a + 6b = 2k} c_{a,b}G_4^aG_6^b \equiv 0 \quad \forall k \in \Z$.
    \item Es gilt $G_4(\rho) = 0$ und $G_6(i) = 0$. Nach VL sind beides einfache Nullstellen und auch die einzigen Nullstellen von $G_4$ bzw. $G_6$ bis auf $\Gamma$-Äquivalenz.
    Wir nehmen also an, es gibt eine Linearkombination $\sum_{4a + 6b = 2k} c_{a,b} G_4^a G_6^b$, wobei $a$ als kleinsten Wert $a_1 \geq 0$ annehme. 
    Betrachtet man nun die Laurententwicklung um den Punkt $\rho$, so erhält man
    \[
      c_{a_1,2k - a_1} G_4^{a_1}(z) G_6^{2k-a_1}(z) = c_{a_1,2k - a_1} \alpha z^{a_1} + \mathcal{O}(z^{a_1 + 1})
    \]
    für ein $\alpha \neq 0$. Alle anderen Koeffizienten enthalten nun $G_4$ mindestens zur Potenz $a_1 + 1$. Daher gilt
    \[
      \sum_{4a + 6b = 2k} c_{a,b} G_4^a(z) G_6^b(z) = c_{a_1,2k - a_1} \alpha z^{a_1} + \mathcal{O}(z^{a_1 + 1}) \neq 0.
    \]
    Das ist aber ein Widerspruch, also darf es keine solche Linearkombination geben und die Koeffizienten $c_{a,b}$ müssen alle $0$ sein.
  \end{enumerate}
  \section*{Aufgabe 23}
  \begin{enumerate}[(a)]
    \item In einem euklidischen Ring $R$ liefert der euklidische Algorithmus für zwei $a, b \in R$ ein bis auf Assoziiertheit eindeutig bestimmtes Element $g$ des Rings zurück mit $(a) + (b) = (g)$,
    wobei $(x)$ das von $x\in R$ erzeugte Ideal bezeichne (siehe LA2).
    Für $R = \Z$ ist dieses Element also eindeutig bestimmt bis auf Vorzeichen und es gilt $a\Z + b\Z = g\Z$.
    \item Seien $a, b, c \in \Z$. Dann gilt $\operatorname{ggT}(a, b, c)\Z = a \Z + b\Z + c\Z = \operatorname{ggT}(a, b) \Z + c\Z = \operatorname{ggT}(\operatorname{ggT}(a,b), c) \Z$. 
    Daraus folgt bereits die Behauptung
    \item Es gilt 
    \begin{align*}
      \operatorname{ggT}(a, b) = 1 
      &\implies \exists d, -c \in \Z\colon ad - bc = 1 \\
      &\implies \begin{pmatrix}
        a & b \\ c & d
      \end{pmatrix} \in \operatorname{SL}(2 \Z) \\
      &\implies ad - bc = 1 \\
      &\implies a\Z + b \Z = \Z\\
      &\implies \operatorname{ggT}(a, b) = 1
    \end{align*}
  \end{enumerate}
\end{document}