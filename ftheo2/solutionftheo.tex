\documentclass{article}

\usepackage{josuamathheader}

\begin{document}
  \section*{Aufgabe 17}
  $q \in \Q \implies \exists a,c \in \Z$, sodass $a$ und $c$ teilerfremd sind und $q = \frac{a}{c}$.
  Laut Hinweis auf dem Aufgabenblatt $\exists b,d \in \Z$ mit $ad - bc = 1$. Daher ist 
  \[
    \det M \coloneqq \det 
    \begin{pmatrix}
    a & b\\ c & d
    \end{pmatrix} 
    = 1 \implies M \in \operatorname{SL}(2, \Z).
  \]
  Wir betrachten nun $M\langle \tau_n \rangle = \frac{a \tau_n + b}{c \tau_n + d}$.
  Zunächst gilt $\frac{b}{c \tau_n + d} \to 0$ für $n \to\infty$, da
  \begin{equation*}
    \left|\Re \frac{b}{c \tau_n + d}\right|=\left| \frac{\Re b\overline{c \tau_n + d}}{|c \tau_n + d|^2}\right| = b \frac{|\Re c \overline{\tau} + d}{|c \tau_n+ d|^2} \leq b \frac{|c \overline{\tau_n+ d|}}{|c \tau_n+ d|^2}
  \end{equation*}
  sowie analog
  \begin{equation*}
    \left|\Im \frac{b}{c \tau_n + d}\right|=\left| \frac{\Im b\overline{c \tau_n + d}}{|c \tau_n + d|^2}\right| = b \frac{|\Im c \overline{\tau} + d}{|c \tau_n+ d|^2} \leq b \frac{|c \overline{\tau_n+ d|}}{|c \tau_n+ d|^2}
  \end{equation*}
  und schließlich
  \begin{equation*}
    b \frac{|c \overline{\tau_n+ d|}}{|c \tau_n+ d|^2} = b \frac{1}{|c \tau_n +d|} \xrightarrow{n \to \infty} 0.
  \end{equation*}
  Außerdem gilt $\frac{\tau_n}{\tau_n + \frac{d}{c}} \to 1$ für $n\to \infty$, da
  \begin{equation*}
    \left| \frac{\tau_n}{\tau_n + \frac{d}{c}} - 1\right| = \left| \frac{\tau_n^2 - \tau_n (\tau_n + \frac{d}{c})}{(\tau_n + \frac{d}{c})\tau_n}\right| = \left| \frac{\frac{d}{c}}{(\tau_n + \frac{d}{c})}\right| \xrightarrow{n\to \infty},
  \end{equation*}
  wobei wir im letzten Schritt analog wie oben schließen.
  Insgesamt erhalten wir daher
  \begin{align*}
    \lim\limits_{n \to \infty} \frac{a \tau_n + b}{c \tau_n + d} &= \lim\limits_{n \to \infty} \frac{a \tau_n}{c \tau_n + d} + \frac{b}{c \tau_n + d}\\
    &= \lim\limits_{n \to \infty} \frac{a \tau_n}{c \tau_n + d}\\
    &= \frac{a}{c} \lim\limits_{n \to \infty} \frac{\tau_n}{\tau_n + \frac{d}{c}}\\
    &= \frac{a}{c}\\
    &= q.
  \end{align*}
  \section*{Aufgabe 18}
  Laut Skript gilt
  \begin{equation*}
    \lim\limits_{\Im \tau \to \infty} e_1(\tau) = -8 \sum_{k = 1}^{\infty} (-1)^kk^{-2}.
  \end{equation*}
  und
  \begin{equation*}
    \lim\limits_{\Im \tau \to \infty} e_2(\tau) = -\sum_{n \in \Z\setminus\{0\}} n^{-2}
  \end{equation*}.
  Beide Reihen konvergieren absolut (siehe Ana1). Daher dürfen wir die Terme umordnen oder umgruppieren. Es gilt daher
  \begin{align*}
    \lim\limits_{\Im \tau \to \infty} e_1(\tau) &= -8 \sum_{k = 1}^{\infty} (-1)^kk^{-2}\\
    &= -8 \sum_{k = 1}^{\infty} \left[(-1)^{2k-1}(2k-1)^{-2} + (-1)^{2k}(2k)^{-2}\right]\\
    &= -8 \sum_{k = 1}^{\infty} \underbrace{\left[- \frac{1}{(2k-1)^2} + \frac{1}{(2k)^2}\right]}_{< 0}\\
    &> 0
  \end{align*}
  Es gilt aber offensichtlich
  \begin{equation*}
    \lim\limits_{\Im \tau \to \infty} e_2(\tau) = -\sum_{n \in \Z\setminus\{0\}} \underbrace{n^{-2}}_{> 0} < 0 .
  \end{equation*}
  Also können die beiden Grenzwerte nicht gleich sein.
\end{document}