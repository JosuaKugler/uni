\documentclass{article}

\usepackage{josuamathheader}

\begin{document}
  \section*{Aufgabe 24}
  \begin{enumerate}[(a)]
    \item Es gilt $M_12 = \C \Delta + \C G_4^3$. Daher lässt sich jede Modulform in $M_{12}$ in der Form $f = \alpha \Delta + \beta G_4^3$ schreiben. Es gilt $G_4(a) = 0 \Leftrightarrow a = \rho$ und $\Delta(a) = 0 \Leftrightarrow a = i\infty$. Für $a = \rho$ wähle daher $\alpha = 0, \beta = 1$. Dann ist $f(\rho) = 0 \cdot \Delta(\rho) + G_4^3(\rho) = 0$. Sonst wähle $\beta = \Delta(a)$ und $\alpha = - G_4^3(a)$. Dann gilt $f(a) = - G_4^3(a) \cdot \Delta(a) + \Delta(a) \cdot G_4^3(a) = 0$.
    \item Besitzt $f$ keine Nullstellen in $\mathbb{H}$, so muss nach der $\frac{k}{12}$-Formel bereits $v_{i\infty}(f) = \frac{k}{12}$ gelten. Insbesondere ist also $k \in 12\Z$. Die Funktion $g = \frac{f}{\Delta^{\frac{k}{12}}}$ ist holomorph auf $\mathbb{H}$ als Quotient zweier holomorpher Funktionen, die keine Nullstellen in $\mathbb{H}$ besitzen. Die Nullstellenordung am Punkt $i\infty$ ist bei beiden Funktionen identisch, sodass $\lim\limits_{z \to i\infty} g(z) = c$ gilt. Insbesondere ist die Funktion also beschränkt auf der Fundamentalmenge.
    Wegen $g(M\langle z \rangle) = \frac{(cz + d)^{-k}}{(cz + d)^{-k}} g(z) = g(z)$ ist $g$ daher eine Modulform vom Gewicht 0 und es gilt $g \equiv c$, also $f = c \Delta^{k/12}$. 
  \end{enumerate}
  \section*{Aufgabe 25}
  \begin{enumerate}[(a)]
    \item Es gilt $T\langle z \rangle = \frac{z + 1}{1} = z+1$ und $S\langle z \rangle = \frac{-1}{z}$. Liegt ein Punkt bereits in $\overline{\mathcal{F}}$, so wählen wir $M = (ST)^3 = -E_2$.
    Ein Punkt in der komplexen oberen Halbebene kann sonst durch iteriertes Anwenden von $T$ in den Streifen $|\Re(\tau) \le 1/2|$ gebracht werden.
    Beachte
    \[
        \Im(M\langle \tau \rangle) = \frac{\Im(\tau)}{|c\tau + d|^2}.
    \]
    Völlig analog zum Beweis im Skript folgt nun, dass die Menge $\{\Im(M\langle \tau \rangle)|M\in \operatorname{SL}(2, \Z)\}$ ein Maximum besitzt. Sei o.B.d.A. $\Im(\tau)$ dieses Maximum und durch Translation sei o.B.d.A. auch $|\Re(\tau)| \leq 1/2$. Der Vergleich von $\Im(\tau) \geq \Im(S\langle \tau \rangle)$ zeigt dann $\Im(\tau) \geq \Im(\tau)/|\tau|^2$. Daher besitzt jeder $\operatorname{SL}(2,\Z)$-Orbit in $\mathbb{H}$ einen Repräsentanten im Bereich $\overline{\mathcal{F}}$, definiert durch
    \[
        |\Re(\tau)| \leq 1/2\quad, \quad |\tau| \geq 1.
    \]
    \item Wegen
    \[
        \Im(M\langle \tau \rangle) = \frac{\Im(\tau)}{|c\tau + d|^2}.
    \]
    ist $\Gamma\langle \tau \rangle \in \mathbb{H}\; \forall \tau \in \mathbb{H}$.
    Nach Teilaufgabe a existiert daher ein $M$ mit $M A \tau \in \overline{\mathcal{F}}$. 
    \item In der Vorlesung wurde gezeigt, dass $\mathcal{F}$ ein genauer Fundamentalbereich ist. Daher ist $MA\langle \tau \rangle \notin \mathcal{F}$ für ein beliebiges $\tau \in \mathcal{F}$ und $MA \notin \{\pm E_2\}$. Das ist nur dann möglich wenn $\tau' \coloneqq MA \langle \tau \rangle \in \mathcal{F}^c \cap \overline{\mathcal{F}}$, also
    \begin{enumerate}
      \item $\Re(\tau') = \frac{1}{2}$ oder
      \item $|\tau| = 1$ und $\Re(\tau) > 0$.
    \end{enumerate}
    Im ersten Fall gilt dann aber $T\langle \tau' - 1\rangle = \tau'$ mit $\tau' - 1 \in \mathcal{F}$ und weil $\mathcal{F}$ ein genauer Fundamentalbereich ist folgt daraus $\tau = \tau'-1 \implies \Re(\tau) = \frac{1}{2} \implies \tau \in \partial F$.
   Im zweiten Fall gilt $S\langle -\overline{\tau'} \rangle = \frac{1}{\overline{\tau'}} = \frac{\tau'}{|\tau'|^2} = \tau'$ und weil $\mathcal{F}$ ein genauer Fundamentalbereich ist folgt daraus $\tau = -\overline{\tau'}$. 
   Das kann aber beides nicht sein, da $\tau$ als innerer Punkt von $\mathcal{F}$ gewählt war. Also muss $MA\langle \tau \rangle = \tau$ gelten und $MA \in \{\pm E_2\}$.
   $- E_2$ ist in $\tilde \Gamma$ enthalten. Angenommen, $E_2 \notin \tilde \Gamma$. Wähle dann $A = - E_2$. Dann gilt $M \cdot -E_2 = -E_2 \implies M = E_2 \in \tilde \Gamma$. Das ist ein Widerspruch, also liegt auch $E_2$ in $\tilde \Gamma$. Insbesondere gilt daher $\forall A \in \Gamma\colon\; A^{-1} \in \tilde \Gamma$ und damit $\tilde \Gamma = \Gamma$.
  \end{enumerate}
\end{document}