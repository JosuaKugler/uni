\documentclass{article}

\usepackage{josuamathheader}

\begin{document}
  \section*{Aufgabe 35}
  \begin{enumerate}[(a)]
    \item Wir stellen im Folgenden $\wp(\alpha \omega_1 + \beta\omega_2)$ durch $(\alpha, \beta)$ dar.
    Für $\tau = \frac{\omega_1}{\omega_2}$ gilt 
    \[ 
      M\langle\tau\rangle = \frac{a\omega_1 + b\omega_2}{c\omega_1 + d\omega_2}.
    \]
    Die Wirkung von $M$ auf das Gitter ist also gegeben durch $\omega_1 \mapsto a\omega_1 + b\omega_2$ und $\omega_2 \mapsto c\omega_1 + d\omega_2$. Insbesondere erhalten wir
    \[ 
    \alpha \omega_1 + \beta \omega_2 \mapsto \alpha(a \omega_1 + b\omega_2) + \beta(c \omega_1 + d\omega_2) = (a\alpha + c\beta) \omega_1 + (b\alpha + d\beta) \omega_2.
    \]
    Es gilt daher $\phi(M) (\alpha, \beta) = (\alpha, \beta) \cdot M$.
    Aufgrund der Periodizität von $\wp$ ist 
    \[ 
    (\alpha, \beta) \sim (a + \alpha, b + \beta)\qquad  \forall a,b \in \Z.
    \]
    Wir erhalten also
    \begin{align*}
      \phi(T)e_1 &= (1/2, 0)\begin{pmatrix}
        1 & 1\\ 0&1
      \end{pmatrix}  = (1/2, 1/2) &&= e_3\\
      \phi(T)e_2 &= (0,1/2)\begin{pmatrix}
        1 & 1\\ 0&1
      \end{pmatrix} = (0,1/2) &&= e_2\\
      \phi(T)e_3 &= (1/2, 1/2)\begin{pmatrix}
        1 & 1\\ 0&1
      \end{pmatrix} = (1/2, 1) \overset{\mod \text{Gitter}}{\equiv} (1/2,0) &&= e_1\\
      \phi(S)e_1 &= (1/2,0)\begin{pmatrix}
        0 & 1\\
        -1 & 0
      \end{pmatrix} = (0,1/2) &&= e_2\\
      \phi(S)e_2 &= (0,1/2)\begin{pmatrix}
        0 & 1\\
        -1 & 0
      \end{pmatrix} = (-1/2, 0) \overset{\mod \text{Gitter}}{\equiv}  (1/2, 0) &&= e_1\\
      \phi(S)e_3 &= (1/2, 1/2)\begin{pmatrix}
        0 & 1\\
        -1 & 0
      \end{pmatrix} = (-1/2, 1/2) \overset{\mod \text{Gitter}}{\equiv}  (1/2, 1/2) &&= e_3\\
    \end{align*}
    Also gilt $\phi(T) = (13)$ und $\phi(S) = (12)$ (Permutationen können auf kanonische Weise als Element von $\operatorname{Bij}(\{e_1, e_2, e_3\}))$ aufgefasst werden).
    \item Es gilt 
    \begin{align*}
      (132) &= (12)(13) = \phi(S)\phi(T) = \phi(ST)\\
      (123) &= (13)(12) = \phi(T)\phi(S) = \phi(TS)\\
      (23)  &= (12)(123) = \phi(T)\phi(ST) = \phi(TST)\\
      () &= (12)(12) = \phi(S)\phi(S) = \phi(S^2)
    \end{align*}
    Damit sind für alle 6 Elemente von $\Gamma/\Gamma[2]$ Vertreter bestimmt.
    \item $\lambda|_0 M$ erhalten wir einfach, indem wir die $e_i$ gemäß der von $M$ induzierten Permutation auf $\{e_1, e_2, e_3\}$ vertauschen. Es genügt daher, wenn $M$ das Vertretersystem durchläuft. Dann erhalten wir
    \begin{align*}
      \lambda|_0 T &= (13)\frac{e_3 - e_2}{e_1 - e_2} = \frac{e_1 - e_2}{e_3-e_2} &&= \lambda^{-1}\\
      \lambda|_0 S &= (12)\frac{e_3 - e_2}{e_1 - e_2} = \frac{e_3 - e_1}{e_2-e_1} &&= 1- \lambda\\
      \lambda|_0 ST &= (132)\frac{e_3 - e_2}{e_1 - e_2} = \frac{e_2 - e_1}{e_3-e_1} &&= \frac{1}{1-\lambda}\\
      \lambda|_0 TS &= (123)\frac{e_3 - e_2}{e_1 - e_2} = \frac{e_1 - e_3}{e_2-e_3} &&= 1- \lambda^{-1}\\
      \lambda|_0 TST &= (23)\frac{e_3 - e_2}{e_1 - e_2} = \frac{e_2 - e_3}{e_1-e_3} &= \frac{1}{1-\lambda^{-1}} &= \frac{\lambda}{\lambda - 1}\\
      \lambda|_0 S^2 &= ()\frac{e_3 - e_2}{e_1 - e_2} = \frac{e_3 - e_2}{e_1-e_2} &&= \lambda\\
    \end{align*}
  \end{enumerate}
  \section*{Aufgabe 36}
  Alle Reihen in dieser Aufgabe sind absolut konvergent, da der Ausdruck $P(n) \cdot e^{2\pi i \tau n} \xrightarrow{n \to  \infty} \to 0$ geht für ein beliebigies Polynom $P$, solange $\Im \tau > 0$. Das ist aber nach Voraussetzung gegeben.
  \begin{enumerate}[(a)]
    \item Wir reduzieren die Aussage für $k=2$ mittels Äquivalenzumformungen auf die triviale Gleichung $0=0$.
    \begin{align*}
      \sum_{n = -\infty}^{\infty} (\tau + n)^{-2} &= (2\pi i)^2 \sum_{n = 1}^{\infty} n e^{2\pi i n \tau}
      \intertext{Wegen Aufgabe 48 folgt}
      \frac{\pi^2}{\sin^2(\pi \tau)} &= (2\pi i )^2 \sum_{n = 1}^{\infty} ne^{2\pi i n \tau}
      \intertext{Exponentialzerlegung des Sinus}
      \frac{(2\pi i)^2}{(e^{i\pi \tau} - e^{-i\pi \tau})^2} &= (2\pi i )^2 \sum_{n = 1}^{\infty} ne^{2\pi i n \tau}\\
      1 &= \sum_{n = 1}^{\infty} ne^{2\pi i n\tau}\left[e^{2\pi i \tau} - 2 + e^{-2\pi i \tau}\right]\\
      1 &= \sum_{n = 1}^{\infty} ne^{2\pi i (n+1)\tau}  + \sum_{n = 1}^{\infty} - 2ne^{2\pi i n \tau} + \sum_{n = 1}^{\infty} ne^{2\pi i (n-1) \tau}\\
      1 &= \sum_{n = 2}^{\infty} (n-1)e^{2\pi i n \tau} + \sum_{n = 1}^{\infty} -2ne^{2\pi i n\tau} + \sum_{n = 0}^{\infty} (n+1)e^{2\pi i n \tau}\\
      1 &= 1 - 2e^{2\pi i \tau} + 2e^{2\pi i \tau} + \sum_{n = 2}^{\infty} [(n-1) - 2n + (n+1)]e^{2\pi i n\tau}\\
      1 &= 1 + \sum_{n = 2}^{\infty} 0
    \end{align*}
    Die Aussage für $k\geq 3$ folgt per Induktion. Sei die Identität für festes $k$ bewiesen.
    Dann gilt
    \begin{align*}
      (-1)^k \sum_{n = -\infty}^{\infty} (\tau + n)^{-k} &= \frac{(2\pi i)^k}{(k-1)!} \sum_{n = 1}^{\infty} n^{k-1}e^{2\pi i n \tau}
      \intertext{Ableiten ergibt}
      (-1)^{k+1}\cdot k \cdot \sum_{n = -\infty}^{\infty} (\tau + n)^{-(k+1)} &= \frac{(2\pi i)^k}{(k-1)!} \sum_{n = 1}^{\infty} n^{k-1} (2\pi i n) e^{2\pi i n \tau}\\
      (-1)^{k+1} \sum_{n = -\infty}^{\infty} (\tau + n)^{-(k+1)} &= \frac{(2\pi i)^{k+1}}{k!} \sum_{n = 1}^{\infty} n^k e^{2\pi i n \tau}
    \end{align*}
    \item Es gilt
    \begin{align*}
      G_k(\tau) &= \sum_{(c,d) \in \{0\}\times \Z\setminus \{0\}} (c\tau + d)^{-k} + \sum_{(c,d) \in \Z\setminus\{0\}\times \Z} (c\tau + d)^{-k}\\
      \intertext{$k$ gerade}
      &= 2 \cdot \sum_{d = 1}^{\infty} d^{-k} + \sum_{c \in \Z\setminus \{0\}} \sum_{d = -\infty}^{\infty} (c\tau + d)^{-k}
      \intertext{Für negatives $c$ substituieren wir $d = -d$, da $d$ über ganz $\Z$ summiert wird. Wegen $k$ gerade ändert sich aber der Wert der Reihe dadurch nicht. Außerdem können wir noch $(-1)^k = 1$ einfügen}
      &= 2 \zeta(k)  + 2 \sum_{c = 1}^{\infty} (-1)^k\sum_{d = -\infty}^{\infty} (c \tau + d)^{-k}\\
      \intertext{Mit Aufgabenteil (a) folgt}
      &= 2\zeta(k) + 2 \sum_{c = 1}^{\infty} \frac{(2\pi i)^k}{(k-1)!}\sum_{n = 1}^{\infty} n^{k-1}e^{2\pi i n (c\tau)}\\
      &= 2\zeta(k) + \frac{2(2\pi i)^k}{(k-1)!}\sum_{n, c = 1}^{\infty} n^{k-1}e^{2\pi i \tau (nc)}
      \intertext{Wir betrachten nun alle Summanden mit $(nc) = N$ und summieren dann über $N$.}
      &= 2\zeta(k) + \frac{2(2\pi i)^k}{(k-1)!} \sum_{N = 1}^{\infty} \sum_{nc = N, n, c\in \N} n^{k-1}e^{2\pi i N\tau}\\
      &= 2\zeta(k) + \frac{2(2\pi i)^k}{(k-1)!} \sum_{N = 1}^{\infty} \sum_{n | N} n^{k-1}e^{2\pi i N\tau}\\
      &= 2\zeta(k) + \frac{2(2\pi i)^k}{(k-1)!} \sum_{N = 1}^{\infty} \sigma_{k-1}(N) e^{2\pi i N\tau}
    \end{align*}
  \end{enumerate}
\end{document}