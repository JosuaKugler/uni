\documentclass{article}

\usepackage{josuamathheader}
%\usetikzlibrary{quotes}
\usetikzlibrary{babel}
\usepackage{tikz-cd}


\begin{document}

\section*{Aufgabe 51}
\begin{enumerate}[(a)]
  \item Die Verknüpfung auf $Y$ ist eine stetige Abbildung $+ \colon Y \times Y \to Y$. 
  Dabei ist $Y \times Y$ als Mannigfaltigkeit isomorph zu $Y$ via der offensichtlich bistetigen Abbildung 
  $Y \times Y \xrightarrow{\sim} Y, (x,x) \mapsto x$.
  Analog ist auch $X\times X$ eine Mannigfaltigkeit. Nach Aufgabe 48 erhalten wir eine Überlagerung 
  $X \times X \xrightarrow{p\times p}Y\times Y$.
  Unser Ziel ist nun eine stetige Abbildung $\tilde{+}\colon X\times X \to X$ mit $p(\tilde{+}(x_1, x_2)) = +(p(x_1), p(x_2))$, d.h.
  eine stetige Abbildung, sodass das folgende Diagramm kommutiert 
  \[
      \begin{tikzcd}
         & & X \arrow[d, "p"]\\
         X \times X\arrow[urr, "\tilde{+}", bend left, dashed] \arrow[r, "(p\times p)"] & Y \times Y\arrow[r, "+"]  & Y
      \end{tikzcd}
  \]
  Völlig analog zum Beweis des Liftungssatzes folgern wir die Existenz einer stetigen Abbildung 
  $\tilde{\tilde{+}} \colon X \times X \to \tilde X$ mit $\tilde X$ der universellen Überlagerung von $X$.
  \[
      \begin{tikzcd}
         & & \tilde X \arrow[dd, bend left, "\tilde p"]\\
         & & X \arrow[d, "p"]\\
         X \times X\arrow[uurr, bend left, "\tilde{\tilde{+}}"] \arrow[urr, "\tilde{+}", bend left, dashed] \arrow[r, "(p\times p)"] & Y \times Y\arrow[r, "+"]  & Y
      \end{tikzcd}
  \]
  Aufgrund der universellen Eigenschaft der universellen Überlagerung existiert eine Überlagerung 
  $\tilde{\tilde{p}} \colon \tilde{X} \to X$ derart, dass das folgende Diagramm kommutiert.
  \[
      \begin{tikzcd}
         & & \tilde X \arrow[d, "\tilde{\tilde{p}}"]\arrow[dd, bend left=50, "\tilde p"]\\
         & & X \arrow[d, "p"]\\
         X \times X\arrow[uurr, bend left, "\tilde{\tilde{+}}"] \arrow[urr, "\tilde{+}", bend left, dashed] \arrow[r, "(p\times p)"] & Y \times Y\arrow[r, "+"]  & Y
      \end{tikzcd}
  \]
  Wir erhalten durch $\tilde{+} \coloneqq \tilde{\tilde{p}} \circ \tilde{\tilde{+}}$ die gesuchte stetige Abbildung.
  \item Völlig analog zum Nachweis der Stetigkeit im Beweis des Liftungssatzes argumentieren wir, dass es sich bei 
  Differenzierbarkeit um eine lokale Eigenschaft handelt. Da die Werte von $\tilde{\tilde{+}}$ einer genügend kleinen Umgebung
  aus Stetigkeitsgründen in einem Blatt liegen müssen und unter der Überlagerung $\tilde{\tilde{p}}$ auch wieder in einem Blatt
  landen. Unter der Voraussetzung, dass $Y$ und $X$ differenzierbare Mannigfaltigkeiten sind, ist dann $\tilde{+}$ als 
  Komposition differenzierbarer Abbildungen differenzierbar.
\end{enumerate}
\section*{Aufgabe 54}
Es gilt $X^3 - 1 = (X - 1)(X - \zeta_2)(X- \zeta_3)$ wobei $1, \zeta_2, \zeta_3$ die dritten Einheitswurzeln seien.
Diese Identität werden wir für $X = \frac{f}{-g}$ verwenden.
\begin{align*}
  f^3 + g^3 &= 1\\
  \frac{f^3}{(-g)^3} - 1 &= \frac{1}{(-g)^3}\\
  (\frac{f}{-g} - 1)(\frac{f}{-g} - \zeta_2)(\frac{f}{-g}- \zeta_3) &= \frac{1}{(-g)^3}\\
  (\frac{f}{g} + 1)(\frac{f}{g} + \zeta_2)(\frac{f}{g} + \zeta_3) &= \frac{1}{g^3}
\end{align*}
Wegen $g \in \mathcal{O}(\C)$ ist $\frac{1}{g^3}\neq 0$. Daraus folgt aber sofort 
\begin{align*}
  \frac{f}{g} \notin \{-1, - \zeta_2,-\zeta_3\}
\end{align*}
Als Quotient zweier holomorpher Funktionen ist $\frac{f}{g}$ meromorph. Nach Aufgabe 52 muss $\frac{f}{g}$ konstant sein.
Also gilt $f = g$ und wir folgern
\begin{align*}
  f^3 + g^3 &= 1\\
  f^3 &= \frac{1}{2}\\
\end{align*}
Es gibt nun drei mögliche Werte, die $f$ annehmen kann. Angenommen, $f$ wäre nicht konstant.
Dann müsste $f$ als holomorphe Funktion nach dem Satz von Picard alle Werte in $\C$ bis auf 2 annehmen.
Da $\C$ mehr als 5 Elemente enthält, ist dies ein Widerspruch. Folglich ist $f$ und damit auch $g$ konstant.
\end{document}