\documentclass{article}
\usepackage{josuamathheader}
\newcommand{\SL}{\operatorname{SL}}

\begin{document}
\begin{itemize}
    \item[92] elliptische Funktion
    \item[94] Liouville,  Abelsches Theorem
    \item[96] Weierstraß $\wp$: Laurententwicklung, Differentialgleichung, Nullstellen $\wp'$
    \item[97] Funktionenkörper: $\C[\Gamma] = \C(\wp) + \wp'(z) \cdot \C(\wp)$, $\C(\wp) = \C(\Gamma)^+$.
    \item[98] elliptische Differentiale 
    \item[100] $e_1 + e_2 + e_3 = 0$, $P(x) = 4x^3 - g_2x - g_3$ mit $g_2 = 60\sum' \gamma^{-4}, g_3 = 140 \sum' \gamma^{-6}$. 
    \item[101] $\Delta = 16 \prod (e_i-e_j)^2 = g_2^3 - 27g_3^2$, $j = \frac{g_2^3}{g_2^3 - 27g_3^2}$.
    \item[102] $j(\tau) = j(M\langle \tau \rangle)$, Fundamentalbereich $|\Re \tau | \leq 1/2, |\tau| \geq 1$.
    \item[103] genauer Fundamentalbereich, Weierstraß $\wp$ als Reihe 
    \item[104] Limiten $e_i, g_2, \Delta$, $\lim\limits_{\Im \tau \to \infty} |j(\tau)| = \infty$
    \item[105] $e_{(a,b)}(\tau) = \wp(\omega_{a,b})$ sind Modulformen in $[\Gamma(N), 2]$, Limiten
    \item[106] $\Gamma, \Gamma(N)$ operieren auf $S_N = \{(a,b) \mod N\}$, $\Gamma/\Gamma(N) \cong \SL(2, \Z/N\Z)$.
    \item[107] elementare Zahlentheorie, Kardinalitäten: $\# \SL(2, \Z/N\Z) = N^3\prod_{p|N} (1-p^{-2})$.
    \item[108] $\# S_N = N^2 \prod_{p|N} (1-p^{-2})$. Dimensionsabschätzung $F^\chi$
    \item[112] Definition Modulform, Fourierentwicklung
    \item[113] Spitzenform, $k/12$-Formel
    \item[115] Ring der Modulformen, $M_0 = \C, M_2 = 0, M_4 = \C G_4, \dots, M_{12} = \C\Delta + \C G_4^3$.
    \item[116] Struktursatz $S_k = \Delta M_{k-12}$, $\bigoplus_k M_k \cong \C[G_4, G_6]$
    \item[117] $j \colon \mathbb{H} /\Gamma \cup \{i\infty\} \cong \hat C$, meromorphe Modulform, $\C(\Gamma) = \C(j)$
    \item[118] Kongruenzgruppen, Spitzen
    \item[119] Spitzen $(\Gamma_\infty \cap \Gamma(N)) \backslash \overline{\Gamma}/\Gamma(N), s = \frac{N^2}{2} \prod_{p|N} (1-p^{-2})$
    \item[120] Modulformen zu Kongruenzgruppen, Petersson Operator
    \item[121] verallgemeinerte $k/12$-Formel, Spitzenformen
    \item[122] $[\Gamma(N), k] = [\Gamma(N), k]_0 \oplus [\Gamma(N), k]_{\mathrm{Eis}}$, $\operatorname{\dim}[\Gamma(N), k]_{\mathrm{Eis}} = s/s-1$ für $k = 2/\geq 3$.
    \item[123] Spezialfälle für $\Gamma(2)$.
    \item[125] $\lambda = \frac{e_3 - e_2}{e_1 - e_2}\colon \mathbb{H}/\Gamma(2) \cong \C\setminus\{0,1\}$, $C(\Gamma(2)) = \C(\lambda)$.
    \item[126] $j = \frac{4(1-\lambda + \lambda^2)^3}{27\lambda^2(1-\lambda)^2}$, $\lambda\colon \mathbb{H}/\Gamma(2) \cup \{0,1,\infty\} \to \hat C$  
\end{itemize}
\newpage
\begin{itemize}
    \item[127] Topologie: offene Mengen, Teilraum-/Produkt-/Qoutiententopologie, Basis, separiert, stetig
    \item[128] abgeschlossen, Grenzwert, quasikompakt, \textbf{Mannigfaltigkeit}
    \item[129] glatte Abbildungen auf Mannigfaltigkeiten
    \item[130] Operation, freie Operation, frei $ \implies p \colon X \to X/\Gamma$ Überlagerung
    \item[132] Elliptische Kurven, Modulkurven $X_N$, $\hat \C$ sind RF
    \item[133] \textbf{Überlagerung}
    \item[134] Liftungslemma
    \item[135] universelle Überlagerung $\tilde X$
    \item[136] Fundamentalgruppe, $\tilde X$ ist RF
    \item[139] Eigenschaften $\tilde X$ 
    \item[142] Satz von Picard 
\end{itemize}
\end{document}