\documentclass{article}
\usepackage{josuamathheader}
\begin{document}
    \section*{Aufgabe 3}
    Wir zeigen per Induktion für $n \geq 1$ die Identität
    $\begin{pmatrix}
        i\pi & 1\\
        0 & i\pi
    \end{pmatrix}^n = \begin{pmatrix}
        (i\pi)^n & n (i\pi)^{n-1}\\
        0 & (i\pi)^n
    \end{pmatrix}$.
    Der Induktionsanfang ist offensichtlich.
    Gelte die Behauptung für $n$. Dann ist
    \begin{align*}
        \begin{pmatrix}
            i\pi & 1\\
            0 & i\pi
        \end{pmatrix}^{n+1} &= \begin{pmatrix}
            i\pi & 1\\
            0 & i\pi
        \end{pmatrix}^{n} \cdot \begin{pmatrix}
            i\pi & 1\\
            0 & i\pi
        \end{pmatrix}\\
        &= \begin{pmatrix}
            (i\pi)^n & n (i\pi)^{n-1}\\
            0 & (i\pi)^n
        \end{pmatrix} \cdot \begin{pmatrix}
            i\pi & 1\\
            0 & i\pi
        \end{pmatrix}\\
        &= \begin{pmatrix}
            (i\pi)^{n+1} & (i\pi)^n + n(i\pi)^n\\
            0 & (i\pi)^{n+1}
        \end{pmatrix}\\
        &= \begin{pmatrix}
            (i\pi)^{n+1} & (n+1)(i\pi)^n\\
            0 & (i\pi)^{n+1}
        \end{pmatrix}\\
    \end{align*}
    Es folgt
    \begin{align*}
        e^A &= \sum_{k = 0}^{\infty} \frac{A^k}{k!}\\
        &= \sum_{k = 0}^{\infty} \frac{1}{k!} \cdot \begin{pmatrix}
            (i\pi)^k & k (i\pi)^{k-1}\\
            0 & (i\pi)^k
        \end{pmatrix}\\
        &= \begin{pmatrix}
            \sum_{k = 0}^{\infty} \frac{(i\pi)^k}{k!} & \sum_{k = 0}^{\infty} \frac{k(i\pi)^{k-1}}{k!}\\
            0 & \sum_{k = 0}^{\infty} \frac{(i\pi)^k}{k!}
        \end{pmatrix}\\
        &= \begin{pmatrix}
            e^{i\pi} & \sum_{k = 1}^{\infty} \frac{(i\pi)^{k-1}}{(k-1)!}\\
            0 & e^{i\pi}
        \end{pmatrix}\\
        &= \begin{pmatrix}
            -1 & \sum_{k = 0}^{\infty} \frac{(i\pi)^k}{k!}\\
            0 & -1
        \end{pmatrix}\\
        &= \begin{pmatrix}
            -1 & -1\\
            0 & -1
        \end{pmatrix}
    \end{align*}
    Es gilt $B^2 = I_2$.
    Daraus folgt
    \begin{align*}
        e^B &= \sum_{k = 0}^{\infty} \frac{B^k}{k!}\\
        &= I_2 \sum_{k = 0}^{\infty} \frac{1}{(2k)!} + B \cdot \sum_{k = 0}^{\infty} \frac{1}{(2k+1)!}\\
        &= I_2 \cosh(1) + B \sinh(1)\\
        &= \frac{1}{2} \begin{pmatrix}
            e + e^{-1} & e - e^{-1}\\
            e - e^{-1} & e + e^{-1}
        \end{pmatrix}
    \end{align*}
\section*{Aufgabe 4}
\begin{enumerate}[(a)]
    \item Es gilt (wie durch elementares Nachrechnen leicht zu verifizieren ist) $A = SJS^{-1}$ mit
    \[
        S = \begin{pmatrix}
            0 & 0 & 1 & 0\\
            1 & 0 & 0 & 1\\
            1 & 0 & 0 & 0\\
            0 & 1 & 0 & 0\\
        \end{pmatrix}  
    \]
    und 
    \[
        J = \begin{pmatrix}
            1 & 1 & 0 & 0\\
            0 & 1 & 0 & 0\\
            0 & 0 & 1 & 1\\
            0 & 0 & 0 & 1
        \end{pmatrix}.
    \]
    \item Die allgemeine Lösung ist gegeben durch $Y = C \cdot e^{At}$.
    Gelte 
    \[
        \begin{pmatrix}
            1 & 1\\0 & 1
        \end{pmatrix}^n = \begin{pmatrix}
            1 & n\\0 & 1
        \end{pmatrix}.
    \]
    Dann erhalten wir 
    \[
        \begin{pmatrix}
            1 & 1\\0 & 1
        \end{pmatrix}^{n+1} = \begin{pmatrix}
             1 & n\\ 0 & 1
        \end{pmatrix} \cdot \begin{pmatrix}
            1 & 1\\0 & 1
        \end{pmatrix} = \begin{pmatrix}
            1 & n+1\\ 0 & 1
       \end{pmatrix},
    \]
    womit wir den Induktionsschritt gezeigt haben. Der Induktionsanfang ist trivial.
    Durch Zusammensetzung zweier solcher Matrizen und unter Ausnutzung elementarer Eigenschaften für Blockmatrizen erhalten wir
    \[
        J^n = \begin{pmatrix}
            1 & n & 0 & 0\\
            0 & 1 & 0 & 0\\
            0 & 0 & 1 & n\\
            0 & 0 & 0 & 1
        \end{pmatrix}.
    \]
    Weiter berechnen wir
    \[
        \sum_{k = 0}^{\infty} \frac{kt^k}{k!} = \sum_{k = 1}^{\infty} \frac{t^k}{(k-1)!} = t \cdot \sum_{k = 0}^{\infty} \frac{1}{k!} = t e^t
    \]
    Es folgt 
    \begin{align*}
        e^{At} &= S e^{Jt} S^{-1}\\
        &= S \sum_{k = 0}^{\infty} \frac{J^kt}{k!} S^{-1}\\
        &= S \begin{pmatrix}
            e^t & \sum_{k = 0}^{\infty} \frac{k}{k!} & 0 & 0\\
            0 & e^t & 0 & 0\\
            0 & 0 & e^t & \sum_{k = 0}^{\infty} \frac{k}{k!} & 0 & 0\\
            0 & 0 & 0 & e^t
        \end{pmatrix} S^{-1}\\
        &= e^t S \begin{pmatrix}
            1 & t & 0 & 0\\
            0 & 1 & 0 & 0\\
            0 & 0 & 1 & t\\
            0 & 0 & 0 & 1
        \end{pmatrix} S^{-1} = e^t \begin{pmatrix}
            1 & t & -t & 0\\
            0 & 1 & 0 & t\\
            0 & 0 & 1 & t\\
            0 & 0 & 0 & 1
        \end{pmatrix}
    \end{align*}
\end{enumerate}
\end{document}