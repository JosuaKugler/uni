%\documentclass{../../theo-lecture/lecture}
\documentclass{article}
\usepackage{josuamathheader}

\newcommand{\ggT}{\operatorname{ggT}}
\begin{document}
\alglayout{5}
\def\headheight{25pt}
    \section*{Aufgabe 1}
    \begin{enumerate}[(a)]
        \item Es genügt zu zeigen, dass 
        \begin{enumerate}[(i)]
            \item alle Nullstellen von $X^4 - 2 = (X - \sqrt[4]{2})(X + \sqrt[4]{2})(X - i\sqrt[4]{2})(X + i\sqrt[4]{2})$ in $L = \Q(\sqrt[4]{2}, i)$ liegen. Das ist allerdings aus der Produktdarstellung von $X^4 - 2$ sofort offensichtlich.
            \item $L = \Q(\sqrt[4]{2}, i)$ wird von den Nullstellen von $X^4 + 2$ erzeugt. Wegen $i = \frac{i \sqrt[4]{2}}{\sqrt[4]{2}} \in \Q(\sqrt[4]{2}, i\sqrt[4]{2})$ ist $\Q(\sqrt[4]{2}, i) \subset \Q(\sqrt[4]{2}, i\sqrt[4]{2})$ und wird damit von den Nullstellen von $X^4 - 2$ erzeugt. 
        \end{enumerate}
        \item Sei $\sigma$ ein $\Q$-Automorphismus von $L$. Dann gilt $\sigma|_\Q = \operatorname{id}_\Q$. 
        Nach Lemma 3.40 ist die Anzahl der verschiedenen $\Q$-Homomorphismen $\sigma\colon \Q(\sqrt[4]{2}) \to L$ gleich 
        \[|\{\alpha \in L | f^\sigma(\alpha) = f(\alpha)= 0\}| = 4,\]
        wobei $f = X^4 - 2 \in \Q[X]$ das Minimalpolynom zu $\sqrt[4]{2}$ über $\Q$ bezeichne (siehe letzter Zettel) und $f^\sigma = f$ wegen $\sigma|_\Q = \operatorname{id}_\Q$. 
        Die einzelnen Fortsetzungen sind nach Lemma 3.40 (ii) eindeutig bestimmt durch ihren Wert auf $\alpha = \sqrt[4]{2}$.
        Da $g = X^2 + 1$ das Minimalpolynom zu $i$ über $\Q(\sqrt[4]{2})$ darstellt, ist die Anzahl der verschiedenen Fortsetzungen auf ganz $L = \Q(\sqrt[4]{2})(i)$ nach Lemma 3.40 gleich 
        \[
            |\{\alpha \in L | g^\sigma(\alpha) = 0\}| = |\{\alpha \in L | \sigma(1)X^2 + \sigma(1) = 0\}| = |\{i, -i\}| = 2.   
        \]
        Die einzelnen Fortsetzungen sind nach Lemma 3.40 (ii) eindeutig bestimmt durch ihren Wert auf $\alpha = i$.
        Daher können wir jeden der 4 $\Q$-Homomorphismen auf zwei verschiedene Weisen zu einem $L$-Automorphismus fortsetzen, 
        sodass wir insgesamt $8$ $\Q$-Automorphismen erhalten, die eindeutig durch ihre Werte auf $\sqrt[4]{2}$ und $i$ gegeben sind.
        \begin{enumerate}[1.]
            \item $\sqrt[4]{2} \mapsto \sqrt[4]{2}, i \mapsto i$
            \item $\sqrt[4]{2} \mapsto \sqrt[4]{2}, i \mapsto -i$
            \item $\sqrt[4]{2} \mapsto -\sqrt[4]{2}, i \mapsto i$
            \item $\sqrt[4]{2} \mapsto -\sqrt[4]{2}, i \mapsto -i$
            \item $\sqrt[4]{2} \mapsto i\sqrt[4]{2}, i \mapsto i$
            \item $\sqrt[4]{2} \mapsto i\sqrt[4]{2}, i \mapsto -i$
            \item $\sqrt[4]{2} \mapsto -i\sqrt[4]{2}, i \mapsto i$
            \item $\sqrt[4]{2} \mapsto -i\sqrt[4]{2}, i \mapsto -i$
        \end{enumerate}
        \item Sei $f = X^2 - 2\sqrt{2}X + 3 \in \Q(\sqrt{2})$. Dann gilt $f(\sqrt{2} +i) = 1 + 2\sqrt{2}i - 4 - 2\sqrt{2}i + 3 = 0$. Wäre $f$ reduzibel, so gäbe es eine Zerlegung in zwei Linearfaktoren über $\Q(\sqrt{2})$. Dann müsste mindestens einer der beiden Linearfaktoren $X - (\sqrt{2} + i)$ sein. Dann wäre aber $\sqrt{2} + i \in \Q(\sqrt{2})$. Das ist aber nicht der Fall, also muss $f$ irreduzibel und damit das Minimalpolynom von $\sqrt{2} + i$ sein. Daher ist aber $[\Q(\sqrt{2}, \sqrt{2} + i): \Q(\sqrt{2})] = 2$ und nach dem Gradsatz $\Q(\sqrt{2}, \sqrt{2} + i)\colon \Q] = 4$. Wegen $\sqrt{2} = \frac{1}{6}(5(\sqrt{2} + i) - (\sqrt{2} + i)^3)$ ist aber $\sqrt{2} \in \Q(\sqrt{2} + i)$ bereits enthalten. Also ist $\Q(\sqrt{2} + i, \sqrt{2}) = \Q(\sqrt{2} + i)$.
        Offensichtlich ist $\sqrt{2} + i \in \Q(\sqrt{2}, i)$ und damit $\Q(\sqrt{2} + i) \subset \Q(\sqrt{2},i)$. Wegen $\dim_\Q \Q(\sqrt{2} + i) = \dim_Q \Q(\sqrt{2} + i, \sqrt{2}) = 4 = \dim_\Q \Q(\sqrt{2}, i)$ folgern wir mit LA1, dass dann $\Q(\sqrt{2} + i) = \Q(\sqrt{2}, i)$ gelten muss.
    \end{enumerate}
    \section*{Aufgabe 2}
    \begin{enumerate}
        \item Es gilt $X^4 + 4 = (X - \sqrt{2}e^{i\frac{\pi}{4}})(X - \sqrt{2}e^{i\frac{3\pi}{4}})(X - \sqrt{2}e^{i\frac{5\pi}{4}})(X -\sqrt{7}e^{i\pi})$.
        Der Zerfällungskörper von $X^4 + 4$ ist daher durch $\Q(\sqrt{2}e^{i\frac{\pi}{4}}, \sqrt{2}e^{i\frac{3\pi}{4}}, \sqrt{2}e^{i\frac{5\pi}{4}}, \sqrt{2}e^{i\frac{7\pi}{4}})$ gegeben.
        Wegen $\sqrt{2}e^{i\frac{3\pi}{4}} = \frac{1}{2}\left(\sqrt{2}e^{i\frac{\pi}{4}}\right)^3$, $\sqrt{2}e^{i\frac{5\pi}{4}} = \frac{1}{4}\left(\sqrt{2}e^{i\frac{\pi}{4}}\right)^5$ und
        $\sqrt{2}e^{i\frac{7\pi}{4}} = \frac{1}{8}\left(\sqrt{2}e^{i\frac{\pi}{4}}\right)^7$ wird dieser Körper bereits von $\sqrt{2}e^{i\frac{\pi}{4}}$ erzeugt. 
        Da keine der Nullstellen von $X^4 + 4$ in $\Q$ liegt, ist das Polynom irreduzibel und damit das Minimalpolynom zu $\sqrt{2}e^{i\frac{\pi}{4}}$. Also hat die Erweiterung $L \colon \Q$ Grad 4.
        \item Es gilt $X^8 - 1 = (X - e^{i\frac{\pi}{4}})(X - e^{i\frac{\pi}{2}})(X - e^{i\frac{3\pi}{4}})(X - e^{i\pi})(X - e^{i\frac{5\pi}{4}})(X - e^{i\frac{3\pi}{2}})(X - e^{i\frac{7\pi}{4}})(X - e^{i2\pi})$.
        Analog zum Polynom $X^4 + 4$ lässt sich hier jede Nullstelle als Potenz von $e^{i\frac{\pi}{4}}$ schreiben. Daher ist der Zerfällungskörper von $X^8 -1$ einfach $\Q(e^{i\frac{\pi}{4}})$. 
        Wegen $\left(e^{i\frac{\pi}{4}}\right)^4 + 1 = 0$ ist $X^4 + 1$ das Minimalpolynom zu $e^{i\frac{\pi}{4}}$ (Irreduzibilität folgt aus der Bonusaufgabe auf dem letzten Zettel).
        Insbesondere hat also $\Q(e^{i\frac{\pi}{4}}) / \Q$ Grad 4.
        \item Es gilt 
        \begin{align*}
            X^4 + 2X^2 - 2 &= (X^2 + 1 + \sqrt{3})(X^2 + 1 - \sqrt{3})\\
            &= (X + \sqrt{1 + \sqrt{3}})(X - \sqrt{1 + \sqrt{3}})(X + \sqrt{1 - \sqrt{3}})(X + \sqrt{1 - \sqrt{3}})
        \end{align*}
        Der Zerfällungskörper von $X^4 + 2X^2 - 2$ ist daher $\Q(\sqrt{1 + \sqrt{3}}, \sqrt{1 - \sqrt{3}})$. 
        Die Erweiterung $\Q(\sqrt{1 + \sqrt{3}})/\Q$ hat Grad 4, da $X^4 + 2X^2 -2$ nach Eisenstein irreduzibel ist und damit Minimalpolynom zu $\sqrt{1 + \sqrt{3}}$. 
        Da $\sqrt{1 - \sqrt{3}}$ einen nicht verschwindenden Imaginärteil hat, kann es nicht in $\Q(\sqrt{1 + \sqrt{3}}) \subset \R$ enthalten sein.
        Da das Polynom $X^2 + \sqrt{1 + \sqrt{3}}^2 - 2$ die beiden Nullstellen $\pm \sqrt{1 - \sqrt{3}}$ besitzt, ist es folglich irreduzibel über $\Q(\sqrt{1 + \sqrt{3}})$.
        Also hat die Erweiterung $\Q(\sqrt{1 + \sqrt{3}}, \sqrt{1 - \sqrt{3}}) / \Q(\sqrt{1 + \sqrt{3}})$ Grad 2. Insgesamt hat die Erweiterung daher Grad $8 = 4 \cdot 2$.
    \end{enumerate}
    
    \section*{Aufgabe 3}
    \begin{enumerate}[(a)]
        \item Sei $f \in K[X]$ das Minimalpolynom zu $\alpha$. Sei  
        \[
            M \coloneqq \{x \in L\colon f(x) = 0\}.
        \] die Menge der Nullstellen von $f$, wobei die Koeffizienten von $f$ gemäß der Körpererweiterung $L/K$ als Elemente von $L$ auffassen. Da $\sigma$ ein $K$-Automorphismus ist, gilt $0 = f^{\sigma_i}(\sigma_i(x)) = f(\sigma_i(x)) \forall x \in M$. Also ist $\sigma_i(\alpha) \in M$. $M$ enthält also die $n$ verschiedenen Elemente $\sigma_i(\alpha) \forall 1\le i \le n$. 
        Damit hat $f$ mindestens Grad $n$. Allerdings hat $f$ auch höchstens Grad $n$, da $[L\colon K] = n$ ist. Daher ist $\deg f = n$. Damit ist $[K(\alpha)\colon K] = n$. Insbesondere ist also $\dim_K K(\alpha) = \dim_K L$, $K(\alpha) \subset L$ und nach LA1 also $K(\alpha) = L$.
        \item Da Körperhomomorphismen stets injektiv sind, genügt es zu zeigen, dass jedes $\alpha \in L$ ein Urbild $a \in L$ unter $\sigma\colon L \to L$ besitzt, wobei es sich bei $\sigma$ um einen $K$-Homomorphismus handelt, d.h. \begin{align*}
            \sigma_K \colon K &\to L\\
            k \mapsto k.
        \end{align*}
        Sei also $\alpha \in L$. Da die Erweiterung algebraisch ist, existiert ein Minimalpolynom $f \in K[X]$ mit $f(\alpha) = 0$, wobei wir hier und in der nächsten Definition die Koeffizienten von $f$ gemäß der Körpererweiterung $L/K$ als Elemente von $L$ auffassen. Sei also
        \[
            M \coloneqq \{x \in L\colon f(x) = 0\}.
        \] die Menge der Nullstellen von $f$. 
        Nach Lemma 3.40 ist dann auch $0 = f^\sigma(\sigma(x)) = f(\sigma(x)) \forall x \in M$, wobei die letzte Gleichheit gilt, weil $\sigma_K = \operatorname{id}_K$ ist. Daraus folgt $\sigma(M) \subset M$. Da $M$ eine endliche Menge und $f$ injektiv ist, muss aber bereits $f^\sigma(M) = M$ gelten und wegen $\alpha \in M$ existiert ein Urbild $a \in M$ mit $\sigma(a) = \alpha$.
        Ist die Körpererweiterung nicht algebraisch, so gilt die Aussage nicht. Für die Körpererweiterung $K(t)/K$ ist
        \begin{align*}
            \sigma \colon K(t) &\to K(t)
            k &\mapsto k \forall k \in K\\
            t &\mapsto t^2
        \end{align*}
        ein $K$-Homomorphismus, der nicht surjektiv ist, weil $t$ kein Urbild besitzt.
    \end{enumerate}
    \section*{Aufgabe 4}
    \begin{enumerate}[(a)]
        \item Sei $0 \neq \alpha \in R$. Da $R$ ein Ring ist, gilt $K[\alpha] \subset R$. Dann gilt nach Satz 3.20 $K[\alpha] = K(\alpha)$. Insbesondere ist also auch $\alpha^{-1} \in R$. Daher ist $R^\times = R\setminus\{0\}$ und folglich ist $R$ ein Körper.
        \item Es gilt $K \subset E, F \subset L$, wobei $K$ und $L$ Körper sind. Daher genügt es zu zeigen, dass $M = \left\{\sum_{i = 1}^{n} a_ib_i | n \in \N, a_i \in E, b_i \in F\right\}$ bezüglich Addition und Multiplikation abgeschlossen ist.
        Es gilt
        \[
            \sum_{i = 1}^{n} a_ib_i + \sum_{i = 1}^{m} a_i'b_i' \overset{\text{Umnummerierung}}{=} \sum_{i = 1}^{n+m} a_ib_i \in M.
        \]
        Außerdem gilt
        \[
            \left(\sum_{i = 1}^{n} a_ib_i\right)\left(\sum_{j = 1}^{m} a_j'b_j'\right) = \sum_{(i,j)\in \{1,\dots,n\} \times \{1,\dots,m\}} \underbrace{a_ia_j'}_{\alpha_k}\underbrace{b_ib_j'}_{\beta_k} = \sum_{k = 1}^{n\cdot m} \alpha_k\beta_k \in M.
        \]
        Nach Aufgabe (a) muss $M$ also ein Körper sein. Offensichtlich muss jedes Element von $M$ in $EF$ enthalten sein. Daher ist $M$ gerade der kleinste Teilkörper, der $E$ und $F$ enthält.
        \item Sind $[E:K]$ und $[F:K]$ endlich, so gilt $E = K(\alpha_1,\dots, \alpha_n)$ und $F = K(\beta_1, \dots, \beta_m)$. Dann ist $EF \subset K(\alpha_1,\dots, \alpha_n, \beta_1, \dots, \beta_m)$, da $K(\alpha_1,\dots, \alpha_n, \beta_1, \dots, \beta_m)$ die beiden Körper $E$ und $F$ enthält. Jeder Körper, der $E$ und $F$ enthält, enthält auch sofort $\alpha_1, \dots, \alpha_n$ und $\beta_1, \dots, \beta_m$. Also ist auch $K(\alpha_1,\dots, \alpha_n, \beta_1, \dots, \beta_m)\subset EF$.
        \[
            \dim_K EF = \dim_K K(\alpha_1,\dots, \alpha_n, \beta_1, \dots, \beta_m).
        \]
        Sei $f_i$ das Minimalpolynom von $\alpha_i$ über $K(\alpha_1,\dots, \alpha_{i-1})$ und analog $g_i$ das Minimalpolynom von $\beta_i$ über $K(\beta_1, \dots, \beta_{i-1})$. 
        \[
            [E:K] = \prod_{i=1}^n \deg f_i \qquad [F:K] = \prod_{i=1}^m \deg g_i
        \]
        Sei außerdem $h_i$ das Minimalpolynom von $\beta_i$ über $K(\alpha_1, \dots, \alpha_n, \beta_1,\dots, \beta_{i-1})$. Dann gilt $\deg h_i \leq \deg g_i$ und
        \[
            [K(\alpha_1,\dots, \alpha_n, \beta_1, \dots, \beta_m) \colon K(\alpha_1,\dots, \alpha_n)] = \prod_{i=1}^n \deg h_i
        \]
        \begin{align*}
            [EF \colon K] &= [K(\alpha_1,\dots, \alpha_n, \beta_1, \dots, \beta_m) \colon K]\\
            &= [K(\alpha_1,\dots, \alpha_n, \beta_1, \dots, \beta_m) \colon K(\alpha_1,\dots, \alpha_n)] \cdot [K(\alpha_1,\dots, \alpha_n):K]\\
            &=  \prod_{i=1}^n \deg h_i \cdot \prod_{i=1}^n \deg f_i\\
            &\leq \prod_{i=1}^n \deg g_i \cdot \prod_{i=1}^n \deg f_i\\
            &= [F:K] \cdot [E:K]
        \end{align*}
        \item Sind $[F:K]$ und $[E:K]$ teilerfremd, so gibt es keine Darstellung $E = K(\alpha_1, \dots, \alpha_n)$ und $F = K(\beta_1, \dots, \beta_m)$, sodass es $\alpha_i, \beta_j$ und zugehörige Minimalpolynome $f_i, g_j$ gibt mit $\deg f_i = \deg g_j$. Also kann es kein Element $a$ in $F \setminus K$ geben, dass auch in $E \setminus K$ liegt. Sonst könnte man o.B.d.A. $\alpha_1 = a$ und $\beta_1 = a$ wählen und erhielte zwei identische Minimalpolynome mit insbesondere gleichem Grad.
        Daher ist $[E(\beta_1, \dots, \beta_i) : E(\beta_1, \dots, \beta_{i-1})] = [K(\beta_1, \dots, \beta_i): K(\beta_1, \dots, \beta_{i-1})]$. Insbesondere ist also stets $\deg h_i = \deg g_i$. Damit wird die Abschätzung in Aufgabe (c) zu einer Gleichheit.
    \end{enumerate}
\end{document}