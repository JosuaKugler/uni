\documentclass{article}
\usepackage{josuamathheader}

\newcommand{\ggT}{\operatorname{ggT}}
\newcommand{\id}{\operatorname{id}}
\newcommand{\ord}{\operatorname{ord}}
\newcommand{\mychar}{\operatorname{char}}

\begin{document}
\alglayout{11}
\def\headheight{25pt}
\section*{Aufgabe 1}
\begin{enumerate}[(a)]
    \item Weil $L$ Zerfällungskörper von $f$ über $K$ ist, muss $L /K$ normal sein.
          Für endliche Körper oder $\mychar K = 0$ ist $L/K$ separabel. Wegen $(n, \mychar K) = 1$ gilt $f' = n X^{n-1} \neq 0$
          und daher ist $L/K$ separabel. Für beliebige $K$ ist also $L/K$ normal und separabel und damit galoissch.
          Sei $b$ eine Nullstelle von $f$.
          Es gilt $X^n - 1 = \prod_{\zeta \in \mu_n} (X- \zeta)$ und daher
          \[
              X^n - a = X^n - b^n = b^n ((Xb^{-1})^n - 1) = b^n \prod_{\zeta \in \mu_n} (Xb^{-1} - \zeta ) = \prod_{\zeta \in \mu_n} (X - \zeta b).
          \]
          Insbesondere gilt $\zeta b \in K(b) \forall \zeta \in \mu_n$, sodass $f$ über $K(b)$ vollständig in Linearfaktoren zerfällt.
          $b$ liegt als Nullstelle von $f$ notwendigerweise in $L$.
          Insgesamt folgern wir $L = K(b)$.
    \item Jedes $\sigma \in \Gal(L/K)$ ist wegen Teilaufgabe (a) eindeutig gegeben durch $\sigma(b)$.
          Die Menge der Nullstellen hatten wir in (a) bereits bestimmt als $M = \{b\zeta \colon \zeta \in \mu_n\}$.
          Daher gilt $\Gal(L/K) = \{\sigma_\zeta \colon \zeta \in \mu_n\}$ mit $\sigma_\zeta(b) = \zeta b$.
          Insbesondere gilt $\psi(\sigma_\zeta) = \frac{\sigma_\zeta(b)}{b} = \zeta \in \mu_n$.
          Daher hängt $\psi$ nur von der Wahl von $\sigma \in \Gal(L/K)$ ab.
          Offensichtlich besitzt jedes $\zeta \in \mu_n$ ein Urbild unter $\psi$, sodass wir
          $\operatorname{im} \psi = \mu_n$ folgern können.
          Weiterhin gilt
          \[
              \psi(\sigma_\zeta \sigma_{\zeta'})
              = \frac{\sigma_\zeta(\sigma_{\zeta'}(b))}{b}
              = \frac{\sigma_\zeta(\zeta'b)}{b}
              = \frac{\zeta'\sigma_\zeta(b)}{b}
              = \zeta'\zeta.
          \]
          Es handelt sich also um einen Gruppenhomomorphismus.
          Für die Injektivität genügt es zu zeigen, dass $\ker \psi = \{\id\}$.
          Das folgt aber sofort aus $\psi(\sigma) = 1 \Leftrightarrow \sigma(b) = b \Leftrightarrow \sigma = \id$.
          Wir erhalten daher einen Gruppenisomorphismus $\psi\colon \Gal(L/K) \overset{\sim}{\to} \mu_n$.
          Da $\mu_n$ zyklisch ist, muss auch $\Gal(L/K)$ zyklisch sein.
    \item Für ein Gegenbeispiel siehe Aufgabe 2 auf Zettel 8. Dort gilt
          $K = \Q$, $n = 4$, $f = X^4 - 2$,  $\mu_n = \{1, -1, i, -i\} \subsetneq \Q$ und $\Gal(L/K) \cong D_4$.
          $D_4$ ist aber nicht zyklisch.
\end{enumerate}
\section*{Aufgabe 2}
\begin{enumerate}[(a)]
    \item Wir zeigen zunächst, dass jedes $\begin{pmatrix}
                  e \\f
              \end{pmatrix} \in V$ eine Darstellung $g \cdot m$ mit $g\in G, m \in M$ besitzt.
          Dazu unterscheiden wir drei Fälle
          \begin{enumerate}[(1)]
              \item $f \neq 0$. Dann gilt
                    \[
                        \begin{pmatrix}
                            e \\f
                        \end{pmatrix}  = \begin{pmatrix}
                            1 & e \\
                            0 & f
                        \end{pmatrix} \cdot \begin{pmatrix}
                            0 \\1
                        \end{pmatrix}.
                    \]
              \item $f = 0, e\neq 0$. Dann gilt
                    \[
                        \begin{pmatrix}
                            e \\0
                        \end{pmatrix}  = \begin{pmatrix}
                            e & 0 \\
                            0 & 1
                        \end{pmatrix} \cdot \begin{pmatrix}
                            1 \\0
                        \end{pmatrix}.
                    \]
              \item $f = e = 0$. Dann gilt
                    \[
                        \begin{pmatrix}
                            0 \\0
                        \end{pmatrix}  = \begin{pmatrix}
                            1 & 0 \\
                            0 & 1
                        \end{pmatrix} \cdot \begin{pmatrix}
                            0 \\0
                        \end{pmatrix}.
                    \]
          \end{enumerate}
          Nun müssen wir zeigen, dass die von $m, n \in M$ erzeugten Bahnen für $n \neq m$ disjunkt sind.
          Es gilt aufgrund der Linearität
          \[
              G\begin{pmatrix}
                  0 \\0
              \end{pmatrix} = \left\{ g \begin{pmatrix}
                  0 \\0
              \end{pmatrix} \colon g\in G\right\}
              = \left\{\begin{pmatrix}
                  0 \\0
              \end{pmatrix}\right\}.
          \]
          Da alle Matrizen aus $G \subset \operatorname{GL}_2(\mathbb{F}_p)$ invertierbar sind, gilt außerdem
          \[
              \begin{pmatrix}
                  0 \\0
              \end{pmatrix}  \notin G\begin{pmatrix}
                  0 \\1
              \end{pmatrix}\qquad \text{und}\qquad \begin{pmatrix}
                  0 \\0
              \end{pmatrix}  \notin G\begin{pmatrix}
                  1 \\0
              \end{pmatrix}.
          \]
          Es verbleibt zu zeigen, dass $G\begin{pmatrix}
                  0 \\1
              \end{pmatrix} \cap G\begin{pmatrix}
                  1 \\0
              \end{pmatrix} = \emptyset$.
          Nehmen wir an $\exists g \in G$ mit
          \[
              \begin{pmatrix}
                  1 \\0
              \end{pmatrix} = g\begin{pmatrix}
                  0 \\1
              \end{pmatrix} \implies \exists a,b,d\in \mathbb{F}_p, a\neq 0, d\neq 0\colon\; \begin{pmatrix}
                  1 \\0
              \end{pmatrix} = \begin{pmatrix}
                  a & b \\0 & d
              \end{pmatrix} \begin{pmatrix}
                  0 \\1
              \end{pmatrix} = \begin{pmatrix}
                  b \\d
              \end{pmatrix}
          \]
          so erhalten wir durch Komponentenvergleich $d = 0$, Widerspruch.
          Nehmen wir stattdessen an $\exists g \in G$ mit
          \[
              \begin{pmatrix}
                  0 \\1
              \end{pmatrix} = g\begin{pmatrix}
                  1 \\0
              \end{pmatrix} \implies \exists a,b,d\in \mathbb{F}_p, a\neq 0, d\neq 0\colon\; \begin{pmatrix}
                  0 \\1
              \end{pmatrix} = \begin{pmatrix}
                  a & b \\0 & d
              \end{pmatrix} \begin{pmatrix}
                  1 \\0
              \end{pmatrix} = \begin{pmatrix}
                  a \\0
              \end{pmatrix},
          \]
          so erhalten wir durch Komponentenvergleich $a = 0$, Widerspruch.
          Daher zerfällt $V$ in die disjunkte Vereinigung der durch Gruppenoperation von $G$ aus $m \in M$ erzeugten Teilmengen,
          $M$ ist also eine Repräsentantensystem der Bahnen.
    \item Sei $x = \begin{pmatrix}
                  e \\f
              \end{pmatrix} \in V$. Wir unterscheiden drei Fälle
          \begin{enumerate}[(1)]
              \item $f \neq 0$. Für $\begin{pmatrix}
                            a & b \\0&d
                        \end{pmatrix} \in G_x$ gilt dann
                    \[
                        \begin{pmatrix}
                            e \\f
                        \end{pmatrix} = \begin{pmatrix}
                            a & b \\ 0 & d
                        \end{pmatrix}\begin{pmatrix}
                            e \\f
                        \end{pmatrix} = \begin{pmatrix}
                            ae + bf \\df
                        \end{pmatrix} \Leftrightarrow d = 1 \land ae+bf = e \Leftrightarrow d = 1\land b = (1-a)e\cdot f^{-1}
                    \]
                    Daraus folgt
                    \[
                        G_x = \left\{\begin{pmatrix}
                            a & (1-a)e \cdot f^{-1} \\
                            0 & 1
                        \end{pmatrix} \in \operatorname{GL}_2(\mathbb{F}_p), a\neq 0\right\}.
                    \]
                    Insbesondere gilt $\# G_x = p-1$, da ein Vertreter von $G_x$ durch $a\in \mathbb{F}_p^\times$ bereits eindeutig bestimmt ist.
              \item $f = 0, e\neq 0$. Für $\begin{pmatrix}
                            a & b \\0&d
                        \end{pmatrix} \in G_x$ gilt dann
                    \[
                        \begin{pmatrix}
                            e \\0
                        \end{pmatrix} = \begin{pmatrix}
                            a & b \\ 0 & d
                        \end{pmatrix}\begin{pmatrix}
                            e \\0
                        \end{pmatrix} = \begin{pmatrix}
                            ae \\0
                        \end{pmatrix} \Leftrightarrow e = ae \Leftrightarrow a = 1.
                    \]
                    Wir erhalten daher
                    \[
                        G_x = \left\{\begin{pmatrix}
                            1 & b \\0&d
                        \end{pmatrix}\in \operatorname{GL}_2(\mathbb{F}_p), d \neq 0\right\}.
                    \]
                    Insbesondere ist $\# G_x = p \cdot (p-1)$.
              \item Für $x = \begin{pmatrix}
                            0 \\0
                        \end{pmatrix}$ ist die Isotropiegruppe wegen $G \begin{pmatrix}
                            0 \\0
                        \end{pmatrix} = \left\{\begin{pmatrix}
                            0 \\0
                        \end{pmatrix}\right\}$ durch ganz $G$ gegeben.
          \end{enumerate}
    \item Es gilt $\# G = (p-1) \cdot p \cdot (p-1)$,
          da es jeweils $p-1$ Möglichkeiten für $a$ und $d$ und $p$ Möglichkeiten für $b$ gibt.
          Sei $x = \begin{pmatrix}
                  e \\f
              \end{pmatrix} \in V$. Wir unterscheiden wieder drei Fälle
          \begin{enumerate}[(1)]
              \item $f \neq 0$.
                    Wegen
                    \[
                        \begin{pmatrix}
                            a & b \\
                            0 & d
                        \end{pmatrix} \cdot \begin{pmatrix}
                            e \\f
                        \end{pmatrix} = \begin{pmatrix}
                            ae + bf \\df
                        \end{pmatrix}
                    \]
                    und $df \neq 0$ gilt $Gx \subset \mathbb{F}_p \times \mathbb{F}_p^\times$. ($df \neq 0$ folgt wegen $d \neq 0$, $f\neq 0$).

                    Da $\mathbb{F}_p$ ein Körper ist, gilt außerdem für beliebige $a\in \mathbb{F}_p, b\in \mathbb{F}_p^\times$
                    \[
                        \begin{pmatrix}
                            a \\b
                        \end{pmatrix} = \begin{pmatrix}
                            1 & (a-e)f^{-1} \\
                            0 & bf^{-1}
                        \end{pmatrix} \begin{pmatrix}
                            e \\f
                        \end{pmatrix} = \begin{pmatrix}
                            e + (a-e)f^{-1}f \\
                            bf^{-1}f
                        \end{pmatrix} \in Gx.
                    \]
                    Daraus folgt sofort die Gleichheit $Gx = \mathbb{F}_p \times \mathbb{F}_p^\times$.
                    Insbesondere gilt $\# Gx = p \cdot (p-1)$. Damit erhalten wir
                    \[
                        \# Gx \cdot \# G_x = p\cdot (p-1) \cdot (p-1) = \# G.
                    \]
              \item $e\neq 0, f = 0$.
                    Wegen
                    \[
                        \begin{pmatrix}
                            a & b \\
                            0 & d
                        \end{pmatrix}  \cdot \begin{pmatrix}
                            e \\0
                        \end{pmatrix} = \begin{pmatrix}
                            ae \\0
                        \end{pmatrix}
                    \]
                    und $ae \neq 0$ gilt $Gx \subset \mathbb{F}_p^\times \times \{0\}$. ($ae\neq 0$ folgt wegen $a \neq 0$, $e\neq 0$).

                    Da $\mathbb{F}_p$ ein Körper ist, gilt außerdem für beliebiges $a\in \mathbb{F}_p$
                    \[
                        \begin{pmatrix}
                            a \\0
                        \end{pmatrix} = \begin{pmatrix}
                            ae^{-1} & b \\0&d
                        \end{pmatrix}\begin{pmatrix}
                            e \\0
                        \end{pmatrix} \in Gx.
                    \]
                    Daraus schließen wir die Gleichheit $Gx = \mathbb{F}_p^\times \times \{0\}$.
                    Insbesondere gilt $\# Gx = p-1$. Damit erhalten wir
                    $\# Gx \cdot \# G_x = (p-1) \cdot (p-1)\cdot p = \# G$.
              \item $e = f = 0$. Wie oben gezeigt ist dann $Gx = \{x\}$ und $G_x = G$.
                    Es gilt also $\# G_x \cdot \# (Gx) = \# G \cdot \# \{x\} = \# G$.
          \end{enumerate}
    \item Nach Lemma 5.10 gilt $(G \colon G_x) = \# Gx$. Wir betrachten die drei Elemente von $M$.
          \begin{enumerate}[(1)]
              \item $x = \begin{pmatrix}
                            0 \\1
                        \end{pmatrix}$. Es gilt $(G \colon G_x) = \# Gx = p\cdot (p-1)$ (siehe (c), Fall 1).
              \item $x = \begin{pmatrix}
                            1 \\0
                        \end{pmatrix}$. Es gilt $(G \colon G_x) = \# Gx = (p-1)$ (siehe (c), Fall 2).
              \item $x = \begin{pmatrix}
                            0 \\0
                        \end{pmatrix}$. Es gilt $(G \colon G_x) = \# Gx = 1$ (siehe (c), Fall 3).
          \end{enumerate}
          Insgesamt erhalten wir
          \[
              \sum_{x\in M} (G\colon G_x) = p \cdot (p-1) + p-1 + 1 = p^2 = \# \mathbb{F}_p^2 = \# V.
          \]
\end{enumerate}
\section*{Aufgabe 3}
\begin{enumerate}[(a)]
    \item Die Anzahl $s$ der 101-Sylowgruppen teilt $2020 = 2^2 \cdot 5 \cdot 101$. Außerdem gilt $s \equiv 1 \mod 101$.
    Allerdings gilt für jeden Teiler $d$ von $2020$ mit $d > 20$ sofort $101 | d$, also $d \equiv 0 \mod 101$.
    Daraus folgt $s = 1$. 
    Wir bezeichnen die eindeutig bestimmte $101$-Sylowgruppe mit $S$. Nach Bemerkung 5.30 und wegen $2020 = 20 \cdot 101$
    mit $(20, 101) = 1$ folgt $\# S = 101$.
    $G$ operiert durch Konjugation auf seinen Untergruppen.
    $gSg^{-1}$ ist daher eine Untergruppe von $G$ und wegen $\# gSg^{-1} = \# S = 101$ ist $gSg^{-1}$ eine $101$-Gruppe.
    Nach Satz 5.29 existiert dann eine $101$-Sylowgruppe $S'$ mit $gSg^{-1} \subset S'$.
    Es gibt aber nur eine $101$-Sylowgruppe, $S = S'$.
    Insbesondere ist also $S \in \operatorname{Fix}_G(\{H \text{ Untergruppe in } G\}) \Leftrightarrow S \triangleleft G$ 
    wegen Lemma 5.18. Da $\# S$ prim ist, muss $S$ kommutativ sein.
    Also ist $S$ der gesuchte kommutative nicht-triviale Normalteiler.
    \item Es gilt $43 < 47$ und $43 \not | 47-1$. Nach Korollar 5.34 ist jede Gruppe der Ordnung $2021 = 43\cdot 47$ zyklisch 
    und insbesondere abelsch. Nach dem Hauptsatz für endliche abelsche Gruppen ist daher jede Gruppe der Ordnung 2021
    isomorph zu $\Z/2021\Z$.
    \item Die Anzahl $s$ der $3$-Sylowgruppen teilt $36 = 2^2\cdot 3^2$. Außerdem gilt $s \equiv 1 \mod 4$.
    Daher gilt $3 \not | s$. Wir erhalten die zwei Möglichkeiten $s = 1$ oder $s = 4$. Für $s = 1$ argumentieren wir analog wie
    in Teilaufgabe (a): Wir bezeichnen die eindeutig bestimmte $3$-Sylowgruppe mit $S$, dann gilt
    $S \in \operatorname{Fix}_G(\{H \text{ Untergruppe in } G\}) \Leftrightarrow S \triangleleft G$.
    $S$ hat die Ordnung $9$, da $36 = 4\cdot 9$ mit $(4,9) = 1$. In diesem Fall existiert also ein 
    nicht-trivialer Normalteiler.
    Ist nun $s = 4$, so gibt es vier verschiedene $3$-Sylowgruppen. 
    Wie oben bewiesen ist für eine $p$-Sylowgruppe $S$ auch $gSg^{-1}$ eine $p$-Sylowgruppe.
    Daher operiert $G$ vermöge der Konjugation auf der Menge ihrer $3$-Sylowgruppen.
    \textbf{noch nicht vollständig}
\end{enumerate}

\section*{Aufgabe 4}
\begin{definition}
    Zwei Transpositionen $(a,b)$ und $(c,d)$ heißen disjunkt, wenn $\{a,b\} \cap \{c,d\} = 0$ gilt.
\end{definition}
\begin{enumerate}[(a)]
    \item Sei $n \in \{1, \dots, n\}$. Wir unterscheiden zwei Fälle.
    \begin{enumerate}[(1)]
        \item $n = \sigma(x_i)$ für ein $i \in \{1, \dots, r\}$. Dann gilt
        \[
            \sigma(x_1, \dots, x_r) \sigma^{-1}(n) = \sigma(x_1, \dots, x_r)(x_i) = \sigma(x_{i+1}).  
        \]
        \item $n \neq \sigma(x_i) \forall i \in \{1, \dots, r\}$. Dann gilt
        \[
            \sigma(x_1, \dots, x_r) \sigma^{-1}(n) = \sigma(\sigma^{-1}(n)) = n.
        \]
    \end{enumerate}
    Insgesamt erhalten wir daher
    \[
        \sigma(x_1, \dots, x_r) \sigma^{-1} = (\sigma(x_1), \dots, \sigma(x_n)).
    \]
    \item Für $\tau \in \mathfrak{V}_4$ gilt $\tau = \tau_1 \tau_2$ für zwei disjunkte Transpositionen $\tau_1$ und $\tau_2$.
    Wir folgern $\sigma \tau \sigma^{-1} = \sigma \tau_1 \sigma^{-1}\sigma \tau_2 \sigma^{-1} = \tau_1' \tau_2'$ für zwei Zyklen $\tau_1'$ und $\tau_2'$.
    Permutationen erhalten Disjunktheit von Mengen, also insbesondere auch Disjunktheit von Transpositionen.
    Daher sind $\tau_1'$ und $\tau_2'$ ebenfalls disjunkt.
    $\mathfrak{V}_4$ enthält aber bereits alle möglichen Kompositionen von zwei disjunkten Zyklen in $\mathfrak{S}_4$.
    Daher gilt $\sigma \tau \sigma^{-1} \in \mathfrak{V}_4 \forall \tau \in \mathfrak{V}_4$.
    Folglich ist $\mathfrak{V}_4$ ein Normalteiler in $\mathfrak{S}_4$.
    \item $1 \triangleleft \mathfrak{V}_4$ ist trivial. $\mathfrak{V}_4$ ist Normalteiler in $\mathfrak{S}_4$,
    also insbesondere auch in $\mathfrak{A}_4$.
    $\mathfrak{A}_4$ ist als Kern des Gruppenhomomorphismus $\operatorname{sgn} \colon \mathfrak{S}_4 \to \{1, -1\}$ Normalteiler in $\mathfrak{S}_4$.
    $1 \triangleleft \mathfrak{V}_4 \triangleleft \mathfrak{A}_4 \triangleleft \mathfrak{S}_4$ bildet daher eine Normalreihe.
    Es gilt $(\mathfrak{S}_4 : \mathfrak{A}_4) = 2$. Jede Gruppe der Ordnung $2$ ist abelsch, da ein Element das neutrale Element ist.
    Außerdem gilt $\# \mathfrak{A}_4 = 12$, $\# \mathfrak{V}_4 = 4$ und damit $(\mathfrak{A}_4 \colon \mathfrak{V}_4) = 3$.
    Eine Gruppe der Ordnung drei besitzt die Elemente $e, a$ und $b$. $e$ kommutiert mit allen Gruppenelementen.
    Da jedes Element ein Inverses besitzen muss und wegen $a \neq e \implies a^{-1} \neq e$ gilt weiter $ab = e = ba$.
    Folglich ist $\mathfrak{A}_4 /\mathfrak{V}_4$ abelsch.
    Schließlich müssen wir noch nachweisen, dass $\mathfrak{V}_4$ abelsch ist. 
    Da Transpositionen kommutieren, ist dies aber sofort klar.
    Per Definition ist $\mathfrak{S}_4$ daher auflösbar.
\end{enumerate}
\end{document}