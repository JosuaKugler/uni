\documentclass{article}

\usepackage{josuamathheader}

\begin{document}
\alglayout{1}
\def\headheight{25pt}
    \section*{Aufgabe 1}
    Offensichtlich gilt $e \in HK$. Um die Abgeschlossenheit bezüglich Multiplikation zu zeigen, wählen wir beliebige Elemente $h\cdot k, h'\cdot k' \in HK$ und betrachten das Produkt $h\cdot k \cdot h' \cdot k'$. Da die Gruppen $HK$ und $KH$ identisch sind und $k\cdot h' \in KH$, gibt es $h'' \cdot k''\in HK$ mit $k\cdot h' = h''\cdot k''$. Setzen wir dies ein, so erhalten wir 
    \[h\cdot k \cdot h' \cdot k' = \underbrace{h\cdot h''}_{\in H}\cdot \underbrace{k'' \cdot k'}_{\in K} \in HK.\]
    Schließlich zeigen wir noch die Abgeschlossenheit bezüglich Inversenbildung. Sei also erneut $h \cdot k$ ein beliebiges Element in $HK$. Das Inverse $k^{-1}h^{-1}$ liegt in $KH$. Wegen $KH = HK$ existiert also ein Element $h'\cdot k' \in HK$ mit \[(h\cdot k)^{-1} = k^{-1}h^{-1} = h'k' \in HK.\]
    \section*{Aufgabe 2}
    \begin{enumerate}[(a)]
        \item Es gilt $\forall g \in G: g^{-1} \in G$. Sofern also $g \neq g^{-1}$ ist, kürzen sich beim Produkt über alle Gruppenelemente $g \cdot g^{-1} = e$. Daher ist also
        \[
            a = \prod_{g\in G} g = \prod_{\substack{g\in G\\g= g^{-1}}} g = \prod_{\substack{g\in G\\g^2 = e}} g.
        \]
        Insbesondere erhalten wir also
        \[
            a^2 = \bigg(\prod_{\substack{g\in G\\g^2 = e}} g\bigg)^2 = \prod_{\substack{g\in G\\g^2 = e}} g^2 = \prod_{\substack{g\in G\\g^2 = e}} e = e.
        \]
        \item Zunächst bestimmen wir alle natürlichen Zahlen $0 < z < p$, für die $\exists k \in \N_0$ mit 
        \begin{align*}
            z^2 &= k \cdot p + 1\\
            z^2 -1 &= k \cdot p\\
            (z+1) \cdot (z-1) &= k \cdot p.
        \end{align*}
        Da $p$ eine Primzahl ist und $z < p$ gilt, muss entweder $k = 0 \Leftrightarrow z = 1$ oder $p = z+1 \Leftrightarrow z = p-1$ gelten. Nach Teilaufgabe (a) erhalten wir also
        \[
            (p-1)! = 1 \cdot (p-1) = p-1 \equiv -1 \mod p  
        \]
    \end{enumerate}
    \section*{Aufgabe 3}
    \begin{enumerate}[(a)]
        \item Offensichtlich ist $e\in Z(G)$. Ist $Z(G) = \{e\}$, so handelt es sich um einen Normalteiler. Seien ansonsten $g, g' \in Z(G)$. Dann ist $h\cdot g\cdot g' = g \cdot h \cdot g' = g \cdot g'\cdot h$ und damit $gg' \in Z(G)$. Ist $g \in Z(G)$, so gilt wegen
        \[
            g^{-1} \cdot h = g^{-1} \cdot h \cdot g\cdot g^{-1} \overset{g \in Z(G)}{=} g^{-1} \cdot g \cdot h\cdot g^{-1} = h\cdot g^{-1}
        \] auch $g^{-1}\in Z(G)$. $Z(G)$ ist also eine Untergruppe von $G$. Sei nun $g\in Z(G)$ und $a \in G$. Dann gilt 
        \[
            a\cdot g \cdot a^{-1} = a \cdot a^{-1} \cdot g = g \in Z(G)
        \]
         und damit also $aZ(G)a \subset Z(G) \quad \forall a \in G$. Nach Lemma 1.24 ist daher $Z(G)$ ein Normalteiler.
        \item Sei $G/Z(G)$ zyklisch. Dann $\exists g \in G$ mit $G/Z(G) = \langle g Z(G)\rangle$. Nun gilt 
        \[G = \bigcup\limits_{n\in \N} \left(g Z(G)\right)^n = \bigcup\limits_{n\in \N} g^n Z(G).\]
        Jedes Element von $G$ lässt sich also in der Form $g^n\cdot h$ mit $n\in \N$ und $h\in Z(G)$ schreiben. Betrachten wir ein Produkt zweier solcher Elemente, so erhalten wir, da $g$ mit sich selbst kommutiert,
        \[
            g^nh \cdot g^kh' \overset{h\in Z(G)}{=} g^ng^kh\cdot h' = g^kg^n h'\cdot h \overset{h'\in Z(G)}{=} g^k h'\cdot g^n h.
        \]
        Die Gruppe ist also abelsch.
    \end{enumerate}
    \section*{Aufgabe 4}
    \begin{enumerate}[(a)]
        \item Offensichtlich ist $\begin{pmatrix}
            1 & 0\\ 0 & 1
        \end{pmatrix} \in D_4$. Wendet man zwei lineare Abbildungen nacheinander an, die beide das Quadrat invariant lassen, so bleibt das Quadrat auch unter der Hintereinanderausführung invariant. Da die Hintereinanderausführung zweier linearen Abbildungen äquivalent zur Ausführung des Produkts der beiden zugehörigen Matrizen ist, liegt also das Produkt zweier Matrizen aus der $D_4$ wieder in der $D_4$. Genauso lässt auch die Umkehrabbildung einer Matrix bzw. der zugehörigen Abbildung aus der $D_4$ das Quadrat invariant und liegt damit selbst wieder in der $D_4$.
        \item Sei $R \in \overline{PQ}$, $R$ ist also ein Punkt auf der Strecke von $P$ nach $Q$. Dann lässt sich $R$ als Konvexkombination von $P$ und $Q$ schreiben, $R = \lambda P + (1-\lambda) Q$. Unter einer linearen Abbildung $A$ wird $R$ auf den Punkt $A(R) = A(\lambda P + (1-\lambda) Q) = \lambda A(P) + (1-\lambda) A(Q)$ abgebildet, der nun auf der Strecke $\overline{A(P)A(Q)}$ liegt. Ist nun $R$ ein Endpunkt von $\overline{PQ}$, so ist $\lambda\in \{0,1\}$. Dann ist $A(R)$ auch ein Endpunkt der Strecke $\overline{A(P)A(Q)}$. Da $E_1 = (1,1)^T$ und $E_2 = (-1,1)^T$ eine Basis des $\R^2$ bilden, sind lineare Abbildungen bereits durch ihre Werte auf $E_1$ und $E_2$ eindeutig bestimmt. Da $E_1$ und $E_2$ aber beides Endpunkte von Strecken sind, müssen sie unter einer linearen Abbildung, die das Quadrat erhält, wieder auf einen Endpunkt von Strecken abgebildet werden, also auf einen der 4 Eckpunkte. Für $E_1$ gibt es vier Eckpunkte zur Auswahl, für $E_2$ dann nur noch drei. Insgesamt erhalten wir also 12 Möglichkeiten.
        \item Da jede der Möglichkeiten bereits durch die Werte für $E_1$ und $E_2$ bestimmt ist und die Werte nur in $\{E_1, E_2, E_3 \coloneqq (-1,-1)^T, E_4 \coloneqq (1,-1)^T\}$ liegen dürfen, genügt es, Tupel $(i,j)$ anzugeben, wobei dann $E_1$ auf $E_i$ und $E_2$ auf $E_j$ abgebildet werde. Damit erhalten wir die Möglichkeiten
        \[(1,2), (1,3), (1,4), (2,1), (2,3), (2,4), (3,1), (3,2), (3,4).\]
    \end{enumerate} 
\end{document}