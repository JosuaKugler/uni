\documentclass{article}
\usepackage{josuamathheader}

\begin{document}

\pagestyle{fancy}
\fancyhead[L]{Algebraische Zahlentheorie 1, Blatt 9}
\fancyhead[R]{Filip Zivanov, Josua Kugler}
\fancypagestyle{firstpage}{%
    \fancyhf{}
    \lhead{Professor: Alexander Schmidt\\
        Tutor: Tim Holzschuh}
    \rhead{Algebraische Zahlentheorie 1, Übungsblatt 9\\ Filip, Josua}
    \cfoot{\thepage}
}
\thispagestyle{firstpage}
\section*{Aufgabe 1}
\begin{enumerate}[(a)]
    \item Es gilt 
    \[
        \operatorname{Gal}(\Q(\zeta_8)|\Q) \cong (\Z/8\Z)^\times \xrightarrow{\phi} \Z/2\Z \times \Z/2\Z
    \]
    mit
    \begin{align*}
        \phi \colon (\Z/8\Z)^\times = \{1,3,5,7\} &\to \Z/2\Z \times \Z/2\Z\\
        1 &\mapsto (0,0)\\
        3 &\mapsto (1,0)\\
        5 &\mapsto (0,1)\\
        7 &\mapsto (1,1)
    \end{align*}
    Nachrechnen ergibt, dass es sich tatsächlich um einen Homomorphismus handelt, dieser ist offensichtlich surjektiv und auf endlichen Gruppen definiert, also ein Isomorphismus.
    Weiter gilt $$\zeta_8 = e^{i \frac{\pi}{4}} = \frac{\sqrt{2}}{2} + i \frac{\sqrt{2}}{2} = \frac{\sqrt{2}}{2} (1+i)$$.
    Insbesondere erhalten wir $i = (\zeta_8)^2 = \frac{1}{2} (1 + 2i - 1)$, 
    $\sqrt{2} = \zeta_8 + \zeta_8^{-1} = \frac{\sqrt{2}}{2}(1+i) + \frac{\sqrt{2}}{2}(1 - i) = \frac{\sqrt{2}}{2} \cdot 2$
    und $\sqrt{-2} = \zeta_8 - \zeta_8^{-1} = \frac{\sqrt{2}}{2}(1+i) - \frac{\sqrt{2}}{2}(1 - i) = \frac{\sqrt{2}}{2} \cdot 2i$.
    Es gilt $[K : \Q] = \phi(8) = \phi(2^3) = 4$. Wegen $[\Q(i):\Q] = [\Q(\sqrt{2}):\Q] = [\Q(\sqrt{-2}):\Q] = 2$ sind alle drei quadratische Unterkörper von $K$.
    \item Nach Korollar 6.11 (i) ist $p$ unverzweigt, da $8 \not \equiv 0 \mod p$ ist. Da die Erweiterung $K|\Q$ endlich galoissch ist, folgt mit Korollar 5.36, dass die Zerlegungsgruppe $Z_p$ zyklisch sein muss und durch den Frobeniusautomorphismus $\operatorname{Frob}_p\colon a \mapsto a^{\#k(\mathfrak{p} \cap \mathcal{O}_\Q)} = a^{\#k(p\Z)} = a^p$ erzeugt wird (gilt für ein beliebiges Primideal $\mathfrak{p}$ über $p$).
    \begin{itemize}
        \item[$p \equiv 1 \mod 8$] Dann gilt $\operatorname{Frob}_p = (\zeta_8 \mapsto \zeta_8^p = \zeta_8) = \operatorname{id}$. Als Untergruppe von $(\Z/8\Z)^\times$ erhalten wir $Z_p = \{1\}$.
        \item[$p \equiv 3 \mod 8$] Dann gilt $\operatorname{Frob}_p = (\zeta_8 \mapsto \zeta_8^p = \zeta_8^3)$. Als Untergruppe von $(\Z/8\Z)^\times$ wird $Z_p$ also von $3$ erzeugt, wegen $3^2 \equiv 1 \mod 8$ folgt $Z_p = \{1, 3\}$.
        Der Zerfällungskörper $K^{Z_p}$ ist wegen $\# Z_p = 2$ ein quadratischer Unterkörper von $K$. Wegen $\operatorname{Frob}_p(\zeta_8 + \zeta_8^3) = \zeta_8^3 + \zeta_8^9 = \zeta_8 + \zeta_8^3$ muss 
        $$\zeta_8 + \zeta_8^3 = \frac{\sqrt{2}}{2}(1 + i) + \frac{\sqrt{2}}{2}(-1 + i) = \frac{\sqrt{2}}{2} \cdot 2i = \sqrt{-2}$$
        in $K^{Z_p}$ enthalten sein. Nun ist aber $\Q(\sqrt{-2})$ bereits ein quadratischer Unterkörper von $K$, es folgt $K^{Z_p} = \Q(\sqrt{-2})$.
        \item[$p \equiv 5 \mod 8$] Dann gilt $\operatorname{Frob}_p = (\zeta_8 \mapsto \zeta_8^p = \zeta_8^5)$. Als Untergruppe von $(\Z/8\Z)^\times$ wird $Z_p$ also von $5$ erzeugt, wegen $5^2 \equiv 1 \mod 8$ folgt $Z_p = \{1, 5\}$.
        Der Zerfällungskörper $K^{Z_p}$ ist wegen $\# Z_p = 2$ ein quadratischer Unterkörper von $K$. Wegen $\operatorname{Frob}_p(\zeta_5^2) = \zeta_5^{10} = \zeta_8^2 = i$ muss $i$ in $K^{Z_p}$ enthalten sein. Nun ist aber $\Q(i)$ bereits ein quadratischer Unterkörper von $K$, es folgt $K^{Z_p} = \Q(i)$.
        \item[$p \equiv 7 \mod 8$] Dann gilt $\operatorname{Frob}_p = (\zeta_7 \mapsto \zeta_8^p = \zeta_8^7)$. Als Untergruppe von $(\Z/8\Z)^\times$ wird $Z_p$ also von $7$ erzeugt, wegen $7^2 \equiv 1 \mod 8$ folgt $Z_p = \{1, 7\}$.
        Der Zerfällungskörper $K^{Z_p}$ ist wegen $\# Z_p = 2$ ein quadratischer Unterkörper von $K$. Wegen $\operatorname{Frob}_p(\zeta_8 + \zeta_8^7) = \zeta_8^7 + \zeta_8^{49} = \zeta_8 + \zeta_8^7$ muss 
        $$\zeta_8 + \zeta_8^7 = \frac{\sqrt{2}}{2}(1 + i) + \frac{\sqrt{2}}{2}(1 - i) = \frac{\sqrt{2}}{2} \cdot 2 = \sqrt{2}$$
        in $K^{Z_p}$ enthalten sein. Nun ist aber $\Q(\sqrt{2})$ bereits ein quadratischer Unterkörper von $K$, es folgt $K^{Z_p} = \Q(\sqrt{2})$.
    \end{itemize}
\section*{Aufgabe 2}
Es gilt nach VL $$\mathcal{O}_{\Q(\zeta_n + \zeta_n^{-1})} = \mathcal{O}_{\Q(\zeta_n)} \cap \Q(\zeta_n + \zeta_n^{-1}) = \Z[\zeta_n] \cap \Q(\zeta_n + \zeta_n^{-1}).$$ Daraus folgt sofort
 $$\Z[\zeta_n + \zeta_n^{-1}] \subset \mathcal{O}_{\Q(\zeta_n + \zeta_n^{-1})} \subset \Z[\zeta_n].$$
 Der schwierige Teil des Beweises fehlt
\section*{Aufgabe 3}
    Z.Z.: $$\left(\frac{2}{p}\right) = (-1)^{\frac{p^2-1}{2}}$$
    \begin{proof}
        Den größten Teil der Äquivalenzen aus dem Beweis von Korollar 6.12 können wir sofort für $q = 2$ übernehmen.
        Es gilt $\left(\frac{2}{p}\right) = 1  \Leftrightarrow 2 $ zerfällt in $K \coloneqq \Q(\sqrt{(-1)^{\frac{p-1}{2}}}p)$.
        Dass 2 in diesem quadratischen Zahlkörper zerfällt, ist nach Satz 5.23(ii) äquivalent dazu, dass
        $(-1)^{\frac{p-1}{2}}\cdot p = d_K \equiv 1 \mod 8$ ist.
        Das ist der Fall, wenn $p \equiv \pm 1 \mod 8$ ist, $(-1)^{2k} \cdot p = p \equiv 1$ oder $(-1)^{2k-1} \cdot p = -1 \cdot p \equiv -1 \cdot -1 \equiv 1 \mod 8$.
        Ist $p \equiv \pm 3 \mod 8$, so erhalten wir $(-1)^{\frac{p-1}{2}}p = p \equiv -3 \mod 8$ oder $(-1)^{\frac{p-1}{2}}p = -p \equiv -3 \mod 8$.
        Es gilt also $\left(\frac{2}{p}\right) = 1 \Leftrightarrow p \equiv \pm 1 \mod 8$.
        Laut der Umformulierung aus Theorem 2.11 ist das bereits die zu zeigende Aussage.
    \end{proof}
\section*{Aufgabe 4}
    Z.Z.: Satz 6.14
    \begin{proof}
        Angenommen, es gibt nur endlich viele Primzahlen $p \equiv 1 \mod n$. Sei $P$ ihr Produkt.
        Betrachte nun $\Phi_n(xnP) \in \N$ für beliebiges $x\in \N$. Angenommen, $\Phi_n(xnP) > 1$. 
        Dann existiert eine Primzahl $p$ mit $p | \Phi_n(xnP)$. Nach Lemma 6.13 gilt dann $p \equiv 1 \mod n$, also $p | P$.
        Folglich gilt $xnP \equiv 0 \mod p$ und $\Phi_n(xnP) \equiv 0 \mod p$. Wir erhalten $\Phi_n(0) = 0 \mod p$, d.h.
        $0$ ist eine Nullstelle von $\Phi_n(X)$ in $\Z/p\Z[X]$. Das kann wegen $\Phi_n(X) | X^n -1$ nicht sein.
        Aus Lemma 6.13 geht hervor, dass es sich bei $\Phi_n(xnP)$ um eine natürliche Zahl handelt.
        Wäre $\Phi_n(xnP) = 0$ für ein $x$, so hätte $\Phi_n$ eine Nullstelle in $\Z$, im Widerspruch zur Irreduzibilität.
        Also folgt $\Phi_n(xnP) = 1 \forall x\in \N$. Damit hätte $\Phi_n(xnP) - 1$ unendlich viele Nullstellen in $\Z[X]$.
        Das ist ein Widerspruch und damit folgt die Behauptung.
    \end{proof}
\end{enumerate}
\end{document}