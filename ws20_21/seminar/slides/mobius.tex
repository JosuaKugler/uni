\begin{frame}

    \begin{definition}
        \[
            \mu(n) = \begin{cases}
                (-1)^k & n \text{ ist quadratfrei und hat $k$ Primfaktoren}\\
                0 & \text{sonst}
            \end{cases}  
        \]
    \end{definition}
    \begin{block}{Beispiel.}
        \begin{align*}
            1 &= \mu(1) = \mu(6) = \mu(10) = \mu(14) = \mu(15)\\ 
            -1 &= \mu(2) = \mu(3) = \mu(5) = \mu(30 = 2 \cdot 3 \cdot 5)\\
            0 &= \mu(4) = \mu(8) = \mu(9) = \mu(12 = 2^2 \cdot 3)
        \end{align*}
    \end{block}
\end{frame}
\begin{frame}
    \begin{lemma}
        Es gilt \[
            \frac{1}{\zeta(s)} = \sum_{n = 1}^{\infty} \frac{\mu(n)}{n^s}  
            \] 
        für $\Re s > 1$.
    \end{lemma}
    \begin{proof}
        \[
            \frac{1}{\zeta(s)} = \prod_{k=1}^\infty \left(1 - \frac{1}{p_k^s}\right) = \left(1 - \frac{1}{2^s}\right)\left(1 - \frac{1}{3^s}\right) \dots = \sum_{n = 1}^{\infty} \frac{\mu(n)}{n^s} 
        \]
    \end{proof}
\end{frame}
\begin{frame}
    \begin{theorem}
        Die Gleichung \[
            \frac{1}{\zeta(s)} = \sum_{n = 1}^{\infty} \frac{\mu(n)}{n^s}
        \] für $\Re s > \frac{1}{2}$ ist äquivalent zur Riemannschen Hypothese.
    \end{theorem}
\end{frame}