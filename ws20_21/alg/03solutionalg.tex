%\documentclass{../../theo-lecture/lecture}
\documentclass{article}
\usepackage{josuamathheader}

\newcommand{\ggT}{\operatorname{ggT}}
\begin{document}
\alglayout{1}
\def\headheight{25pt}
    \section*{Aufgabe 1}
    \begin{enumerate}[(a)]
        \item Es gilt $X^2 - 4 = (X - 2)(X + 2)$. Da $\Q[X]$ ein faktorieller Ring ist, gibt es eine eindeutige Zerlegung in Primfaktoren. Wäre also $\ggT(f, g) \neq 1$, so müsste $X^3-3$ ein Vielfaches von einem der beiden Faktoren sein. Wegen $2^3 - 3\neq 0$ und $-2^3 -3 \neq 0$, ist dies aber nicht der Fall. Also ist $\ggT(f,g) = 1$ in $\Q[X]$.
        \item Es gilt $X^2 - 4 = (X-2)(X+2)$ und $X^3 - 3 = (X-2)(X^2 + 2X + 4)$. Wegen $(-2)^2 + 2(-2) + 4 = 4 \neq 0$ ist $(X^2 + 2X + 4)$ kein Vielfaches von $(X + 2)$, da sonst $-2$ eine Nullstelle sein müsste. Da $\mathbb{F}_5[X]$ ein faktorieller Ring ist, gibt es eine eindeutige Zerlegung in Primfaktoren. Daher ist $\ggT(f,g) = X-2$ in $\mathbb{F}_5[X]$.
        \item Sei $M$ die Menge aller irreduziblen, normierten Polynome. Angenommen, $M$ ist endlich. Betrachte nun
        \[
            p = \left(\prod_{f\in M} f \right) + 1.
        \]
        Nach Vl gilt $\deg p = \sum_{f \in M} \deg f > \max_{f\in M} \deg f$. Also ist $p \notin M$. Allerdings ist $p$ als Produkt von normierten Polynomen wieder normiert. Daher muss $p$ eine Zerlegung in irreduzible Polynome besitzen. Sei also $p = h\cdot g$ mit $g\in M$. Sei dann $\pi\colon K[X] \to K[X]/(g)$ die kanonische Projektion. Es gilt zunächst $\pi(q) = \pi(h \cdot g) = \pi(h)\cdot \pi(g) = 0$. Allerdings gilt auch
        \[
          \pi\left(\prod_{f\in M} f + 1\right) = \pi\left(\prod_{f\in M} f \right) + \pi(1) = \pi(g) \cdot \pi\left(\prod_{g \neq f\in M} f \right) + \pi(1) = 0 + 1 = 1\neq 0
        \]
        Das ist ein Widerspruch. Also kann $M$ nicht endlich sein.
        \item Seien $h,h' \in \overline{g}$. Da $K[X]$ euklidisch ist, existieren $q, q', \tilde g, \tilde g'\in K[X]$ mit $\deg \tilde g, \tilde g' < \deg f$ und $h = q\cdot f + \tilde g,\; h' = q'\cdot f + \tilde g'$. Es gilt $0 = \overline{h - h'} = \overline{q\cdot f + \tilde g - q'\cdot f - \tilde g'} = \overline{q}\overline{f} + \overline{\tilde g} - \overline{q'}\overline{f} - \overline{\tilde g'} = \overline{\tilde g - \tilde g'}$. Daraus folgt $\tilde g - \tilde g' = c\cdot f$ mit $c\in K[X]$. Wegen $\deg \tilde g, \deg \tilde g' < \deg f \implies \deg (\tilde g - \tilde g') < \deg f $ und $\deg(cf) = \deg c + \deg f$ muss $\deg c < 0 \Leftrightarrow c = 0$ gelten. Also gilt $\tilde g = \tilde g' \eqqcolon g$. Also besitzt jedes Polynom $h\in \overline{g}$ eine eindeutige Darstellung $h = q\cdot f + g$ mit $\deg g < \deg f$. Es gibt also eine injektive Abbildung $\varphi\colon L \to M\coloneqq \{g \in K| \deg g < \deg f\}$. Da durch $\overline{g}$ mit $g \in M$ aber auch eine Äquivalenzklasse gegeben ist, deren eindeutiger Repräsentant genau $g$ sein muss, ist $\varphi$ sogar eine Bijektion. Da die Polynome $\underline{m} = (1, X, \dots, X^{\deg f - 1})$ über $K$ linear unabhängig sind und jedes Polynom in $M$ per Definition in $\operatorname{Lin}(\underline{m})$ liegt, ist $\underline{m}$ eine Basis von $M$, Es gilt also $\dim L = \dim M = \# \underline{m} = \deg f$.
        Sei $f(X) = a_n X^n + \dots a_1 X + a_0 \in K[X]$. Dann gilt in $L[X]$
        \[
            f(\overline{X}) = \overline{a_n} \overline{X}^n + \dots + \overline{a_1}\overline{X} + \overline{a_0} = \overline{a_n X^n + \dots a_1 X + a_0} = \overline{f} = 0.
        \]
    \end{enumerate}
    \section*{Aufgabe 2}
    \begin{enumerate}[(a)]
        \item Sei $\varphi \colon \Z[X] \to \Z/3\Z[X]$ die kanonische Projektion. Dann gilt $\varphi(f) = X^3 + 2X^2 - 2$. Dieses Polynom ist nach dem Eisensteinkriterium mit $p = 2$ irreduzibel. Nach dem Satz von Gauss ist somit auch $f\in \Q[X]$ irreduzibel.
        \item Sei $f(X) = X^6 + X^3 + 1$. Dann gilt 
        \[
            f(X + 1) = (X + 1)^6 + (X + 1)^3 + 1 = X^6 + 6X^5 + 15X^4 + 21X^3 + 18X^2 + 9 X + 3  
        \]
        Dieses Polynom ist nach dem Eisensteinkriterium mit $p = 3$ irreduzibel. Somit ist auch $f(X)$ irreduzibel. Nach dem Satz von Gauss ist $f(X)$ also auch als Element von $\Q[X]$ irreduzibel.
        \item Wir können $f(X,Y) \in X^7 + 2X^5 Y + 3XY^3 + 4Y^3 + 5XY + 6X \in \C[X,Y]$ auch auffassen als Elemente von $\C[X][Y]$. Dann gilt $f(Y) = Y^3(4 + 3X) + Y(2X^5 + 5X) + X^7 + 6X$. Nach dem Eisensteinkriterium mit $p = X$ ($X$ ist ein Primelement in $\C[X]$) ist $f$ irreduzibel.
    \end{enumerate}
    \section*{Aufgabe 3}
    \begin{enumerate}[(a)]
        \item Sei $(p_i)_{i\in I}$ ein Vertretersystem von Primelementen in $R$. Dann gilt
        \[
            a = \prod_{i\in I} p_i^{v_{p_i}(a)},\qquad b \cdot c = \prod_{i\in I} p_i^{v_{p_i}(b) + v_{p_i}(c)}  
        \]
        Gilt $a | x$, so existiert ein $d\in R$ mit $ad = x$. Daher gilt
        \[
          a | bc \qquad \Leftrightarrow \qquad v_{p_i}(a) \leq v_{p_i}(a) + v_{p_i}(d) = v_{p_i}(x) \quad \forall i\in I.  
        \]
        Wir wissen also wegen $a | bc$ sofort
        \[
          v_{p_i}(a) \leq v_{p_i}(b) + v_{p_i}(c) \quad \forall i\in I.  
        \]
        Ist $\ggT(a,b) = 1$, so gilt $\forall i \in I$
        \[
            \min(v_{p_i}(a), v_{p_i}(b)) = 0.
        \]
        Gälte diese Identität nicht für ein beliebiges $i\in I$, so wäre $p_i$ ein Teiler von $a$ und von $b$ und somit $p_i | \ggT(a,b)$.
        Wir wählen also ein $i\in I$ und unterscheiden zwei Fälle
        \begin{enumerate}[(1.)]
            \item $v_{p_i}(a) = 0$. Dann gilt offensichtlich $v_{p_i}(a) \leq v_{p_i}(c)$.
            \item $v_{p_i}(b) = 0$. Aus $v_{p_i}(a) \leq v_{p_i}(b) + v_{p_i}(c)$ folgt dann $v_{p_i}(a) \leq 0 + v_{p_i}(c) = v_{p_i}(c)$.
        \end{enumerate}
        Insgesamt gilt also
        \[
          v_{p_i}(a) \leq v_{p_i}(c) \quad \forall i\in I.
        \]
        Das ist äquivalent zu $a | c$.
        \item Ist $f(\alpha) = 0$, so besitzt $f$ die Zerlegung $f(x) = \underbrace{(x - \alpha)}_{\eqqcolon g} \cdot \underbrace{(X^{n-1} + b_{n-2}X^{n-2} + \dots + \beta)}_{\eqqcolon h}$. Da $f, g$ und $h$ normiert sind folgt aus $f \in R[X]$ nach VL $g, h \in R[X]$. Insbesondere gilt also $\alpha, \beta\in R$ und, wie man aus der Produktdarstellung von $f$ leicht sieht, $\alpha \cdot \beta = a_0$. Daher gilt $\alpha | a_0$ in $R$.
        \item Hier können wir sofort Aufgabe $b$ anwenden. Ist $f \coloneqq X^3 + aX^2 + bX + 1$ reduzibel, so besitzt es eine Produktdarstellung, wobei mindestens einer der Faktoren Grad $1$ besitzen, d.h. von der Form $(X - \alpha)$ sein muss. Also muss $f$ eine Nullstelle besitzen. Diese muss nach $b$ aber ein Teiler von $1$ in $\Z$ sein, also $\alpha = \pm 1$. Es gilt aber
        \[
            0 = f(1) = 1 + a + b + 1 \qquad \Leftrightarrow \qquad a + b = -2
        \] und
        \[
            0 = f(-1) = -1 + a - b + 1 \qquad \Leftrightarrow \qquad a - b = 0 \Leftrightarrow a = b.    
        \]
        Daher ist $f$ genau dann irreduzibel, wenn $a + b = -2$ oder $a = b$ gilt.
    \end{enumerate}
    \section*{Aufgabe 4}
    Sei $(p_i)_{i\in I}$ ein Vertretersystem von Primelementen in $R$.
    \begin{enumerate}[(a)]
        \item Sei $f = a_nX^n + \dots a_1X^1 + a_0$
        Wähle $a = \prod_{i\in I} p_i^{- \min(v_{p_i}(f), 0)}$. Dann gilt nämlich 
        \[
            v_{p_i}(a\cdot f) = v_{p_i}(a) + v_{p_i}(f) = - \min(v_{p_i}(f), 0) + v_{p_i}(f).
        \]
        Ist nun für ein $i\in I$ $v_{p_i}(f) < 0$, so gilt $v_{p_i}(a \cdot f) = - \min(v_{p_i}(f), 0) + v_{p_i}(f) = -v_{p_i}(f) +v_{p_i}(f) = 0$. Gilt  $v_{p_i}(f) \geq 0$, so ändert sich nichts. Es existiert also ein solches $a$.
        Für beliebiges $a$ mit $a\cdot f \in R$ gilt
        \begin{align*}
            I(f) &= a^{-1} \cdot \ggT(a \cdot a_0, \dots, a \cdot a_n)\\
            &= \prod_{i\in I} p_i^{-v_{p_i}(a)} \cdot \prod_{i\in I} p_i^{\min(v_{p_i}(a\cdot a_0), \dots, v_{p_i}(a \cdot a_n))}\\
            &= \prod_{i\in I} p_i^{-v_{p_i}(a)} \cdot \prod_{i\in I} p_i^{v_{p_i}(a) + \min(v_{p_i}(a_0), \dots, v_{p_i}(a_n))}\\
            &= \prod_{i\in I} p_i^{-v_{p_i}(a)} \cdot \prod_{i\in I} p_i^{v_{p_i}(a) + v_{p_i}(f)}\\
            &= \prod_{i\in I} p_i^{v_{p_i}(f)}
        \end{align*}
       Damit ist $I(f)$ unabhängig von der Wahl von $a$. Für $f = \frac{3}{7} X^3 + X - 5$ gilt 
       \[I(f) = \frac{1}{7} \ggT(3, 7, -35) = \frac{1}{7}.\]
        \item Es gilt 
        \begin{salign*}
            I(f\cdot g) &\stackrel{\text{Teilaufgabe (a)}}{=} \prod_{i\in I} p_i^{v_{p_i}(f \cdot g)}\\
            &\stackrel{\text{Vorlesung}}{=} \prod_{i\in I} p_i^{v_{p_i}(f) + v_{p_i}(g)}\\
            &= \prod_{i\in I} p_i^{v_{p_i}(f)} \cdot \prod_{i\in I} p_i^{v_{p_i}(g)}\\
            &\stackrel{\text{Teilaufgabe (a)}}{=} I(f) \cdot I(g)
        \end{salign*}
        \item Es gilt allgemein:
        \[
            d | a_0 \land d | a_1 \Leftrightarrow d | a_1 + \lambda a_0.
        \]
        Daraus folgt
        \[
            \ggT(a_0, a_1) = \ggT(a_0, a_0 + \lambda a_1).
        \]
        Per Induktion folgt daraus
        \[
            \ggT(a_n, a_{n-1} + \lambda a_n, \dots, a_0 + \sum_{i = 1}^{n} \lambda_i a_i) = \ggT(a_n, \dots, a_0). 
        \]
        Sei also $f(X) = \sum_{i = 0}^{n} a_i X^i$. Dann gilt 
        \[
            h(X) = \sum_{i = 0}^{n} a_i (X + r)^i = \sum_{i = 0}^{n} a_i \sum_{j = 0}^{i} \binom{i}{j} r^{i-j} X^j
        \]
        $X^j$ taucht 
        Für den Vorfaktor $b_j$ von $X^j$ in $h$ gilt nun $b_j = \sum_{i = j}^{n} a_i\binom{i}{j} r^{i-j} = a_j + \sum_{i = j+1}^{n} a_i \binom{i}{j} r^{i-j}$. Es gilt daher
        \[
            I(h) = \ggT(b_n, \dots, b_0) = \ggT\left(a_n, \dots, a_0 + \sum_{i = 1}^{n} a_i \binom{i}{j} r^{i-j}\right)  = \ggT(a_n, \dots, a_0) = I(f).
        \] 
    \end{enumerate}
    \section*{Bonusaufgabe 5}
    \begin{enumerate}[(a)]
        \item Wir zeigen $u$ Einheit $\implies N(u) = 1 \implies u\in \{\pm 1, pm i\} \implies u$ Einheit und damit die Äquivalenz dieser Aussagen. Sei also $u$ eine Einheit. Dann $\exists v$ mit $u\cdot v = 1 \implies N(u\cdot v) = 1 = N(u)\cdot N(v)$. Wegen $N(x) > 0 \forall x \in \Z[i]$ muss $N(u) = N(v) = 1$ gelten. Sei nun $N(u) = 1$. $u$ besitzt eine Darstellung durch $u = a + b\cdot i$. Wegen $N(u) = 1$ muss also $a^2 + b^2 = 1$ gelten. Also ist $u \in \{\pm 1, \pm i\}$. Da $\{\pm 1, \pm i\}$ offensichtlich Einheiten sind, folgt die Behauptung.
        \item Behauptung: $(4n+1) | [(2n)!^2 - (4n)!]$.
        \begin{proof}
            Es gilt $(2n)!^2 - (4n)! = (2n)! \cdot \left((2n)! - \prod_{i=2n + 1}^{4n} i\right)$. Da $(4n+1) \not | (2n)!$, konzentrieren wir uns auf den zweiten Faktor.
            \begin{align*}
                (2n)! - \prod_{i=2n + 1}^{4n} i &= (2n)! - \prod_{i=1}^{2n} ((4n + 1) - i)
                \intertext{Wenden wir  das Distributivgesetz auf dieses Produkt an, so erhalten wir für ein irrelevantes $\alpha$}
                &= (2n)! - \alpha (4n + 1) - \prod_{i=1}^{2n} (-i)\\
                &= (2n)! - (-1)^{2n} (2n)! - \alpha (4n +1)\\
                &= - \alpha (4n + 1)
            \end{align*}
            Daraus folgt bereits die Behauptung.
        \end{proof}
        Da $4n + 1$ eine Primzahl ist, gilt nach dem Satz von Wilson $(4n)! \equiv -1 \mod (4n + 1) \Leftrightarrow (4n+1) | (4n!) + 1$. 
        Es gilt $(2n)!^2 + 1 = ((2n)!^2 + 1) - (4n)! + 1) + (4n)! + 1 = [(2n)!^2 - (4n)!] + (4n)! + 1$ und wegen $c | [(2n)!^2 - (4n)!], c | (4n)! + 1$ folgt daraus $c | (2n)!^2 + 1$.

        Angenommen, $\exists c + di \in \Z[i]$ mit $(c + di) \cdot (4n +1) = (2n)! \pm i$. Dann folgt aus separater Betrachtung von Real- und Imaginärteil $d \cdot (4n + 1) = \pm 1$ mit $d, n\in \Z$. Das ist offensichtlich ein Widerspruch. Daher kann $p$ kein Primelement in $Z[i]$ sein.
        \item Wegen $\pi | p$ gilt auch $N(\pi) | N(p) = p^2$. 
        Da $p$ eine Primzahl ist und $\pi$ als Primelement keine Einheit sein darf, muss also entweder $N(\pi) = p$ oder $N(\pi) = p^2$ gelten. 
        Angenommen, $N(\pi) = p^2 = N(p)$. 
        Wegen $\pi | p$ existiert ein $c \in \Z[i]$ mit $p = c \cdot \pi$. 
        Daraus folgt $N(p) = N(c) \cdot N(\pi) = N(c) \cdot N(p) \implies N(c) = 1$. 
        Wegen Teilaufgabe a muss also $c$ eine Einheit sein, es folgt also $\pi \hat{=} p$. 
        Dann folgt aber aus Teilaufgabe $b$ völlig analog, dass $\pi$ kein Primelement sein kann. 
        Die einzige verbleibende Möglichkeit ist $N(\pi) = p$. Wegen $\pi | p | (2n)!^2 + 1$ und $\pi$ prim, muss $\pi$ auch eine der beiden Zahlen $(2n)! \pm i$ teilen. Angenommen, $\Im \pi = 0$. Dann folgt die Gleichung $\pi \cdot ( a + bi) = (2n)! \pm i$ und, bei separater Betrachtung von Real- und Imaginärteil erhalten wir $\pi \cdot b = \pm 1, \pi, b \in \Z$. Wegen $\pi \neq 1$ ist das ein Widerspruch. Also ist $\Im \pi \neq 0$. Angenommen, $\Re \pi = 0$. Dann folgt die Gleichung $\pi \cdot (a + bi) = (2n)! \pm i$ und, bei separater Betrachtung von Real- und Imaginärteil erhalten wir $\pi \cdot a = \pm 1, \pi, a \in \Z$. Wegen $\pi \neq 1$ ist auch das ein Widerspruch. Daher ist $\pi = a + bi$ mit $a, b \in \Z\setminus\{0\}$. Insbesondere ist daher $p = N(\pi) = a^2 + b^2$ mit $a, b \neq 0$. Damit ist der Zwei-Quadrate-Satz bewiesen.
    \end{enumerate}
\end{document}