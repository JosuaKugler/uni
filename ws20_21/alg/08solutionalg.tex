%\documentclass{../../theo-lecture/lecture}
\documentclass{article}
\usepackage{josuamathheader}

\newcommand{\ggT}{\operatorname{ggT}}
\begin{document}
\alglayout{8}
\def\headheight{25pt}
    \section*{Aufgabe 1}
    \begin{enumerate}[(a)]
        \item Wir führen einen Induktionsbeweis. 
        Sei $L/K$ eine Körpererweiterung, sodass $L$ Zerfällungskörper zu $f\in K[X]$ mit $\deg f = 1$ ist.
        Dann ist $[L\colon K] = 1 \leq 1!$.
        Sei die Behauptung also für alle $L/K$ bewiesen, wobei $L$ Zerfällungskörper zu einem $f\in K[X]$ mit $\deg f \leq n-1$ sind.
        Wir betrachten nun eine Körpererweiterung $L/K$, wobei $L$ Zerfällungskörper zu einem $f\in K[X]$ mit $\deg f = n$ ist.
        Seien $\alpha_1,\dots,\alpha_n \in L$ die Nullstellen von $f$.
        Dann ist $f(\alpha_1) = 0$ mit $f \in K[X]$, also ist $f$ ein Vielfaches des Minimalpolynoms von $\alpha_1$ in $K$.
        Insbesondere ist $[K(\alpha_1) \colon K] \leq \deg f = n$. Daraus folgt
        \[
            [L\colon K] = [K(\alpha_1, \dots, \alpha_n)\colon K(\alpha_1)] \cdot [K(\alpha_1) \colon K] \leq (n-1)! \cdot n = n!.
        \]
        Damit ist die Aussage bereits bewiesen.
        \item Offensichtlich ist $\bigcap_{i\in I} L_i / K$ eine Teilerweiterung von $L_i/K$ für beliebiges $i$ und damit ebenfalls algebraisch.
        Sei $\alpha \in \bigcap_{i\in I} L_i$ und $f \in K[X]$ mit $f(\alpha) = 0$. 
        Da $L_i$ eine normale Erweiterung ist, zerfällt $f$ in Linearfaktoren über $L_i$. 
        Insbesondere liegen alle Nullstellen von $f$ in $L_i$.
        Da dies für beliebiges $i$ gilt, liegen alle Nullstellen von $f$ in $\bigcap_{i\in I} L_i$. 
        Daher zerfällt $f$ auch über $\bigcap_{i\in I} L_i$ in Linearfaktoren.
        Folglich ist $\bigcap_{i\in I}$ ebenfalls algebraisch und normal.
    \end{enumerate}
    \section*{Aufgabe 2}
    \renewcommand{\epsilon}{\varepsilon}
    \begin{enumerate}[(a)]
        \item Es gilt $N = \{\sqrt[4]{2}, -\sqrt[4]{2}, i\sqrt[4]{2}, -i \sqrt[4]{2}\}$.
        Wir definieren $\sigma_{\zeta, \epsilon} \in \Gal(L/\Q)$ durch 
        $\sigma_{\zeta, \epsilon}(\sqrt[4]{2}) = \zeta \sqrt[4]{2}$ und $\sigma_{\zeta, \epsilon}(i) = \epsilon i$.
        Wir erhalten daher folgende Zuordnung:
        \begin{align*}
            \sigma_{1, 1} &\mapsto \begin{pmatrix}
                  \sqrt[4]{2}& i\sqrt[4]{2}&- \sqrt[4]{2}&-i\sqrt[4]{2}\\
                  \sqrt[4]{2}& i\sqrt[4]{2}&- \sqrt[4]{2}&-i\sqrt[4]{2}
            \end{pmatrix} \cong \begin{pmatrix}
                1 & 2 & 3 & 4\\
                1 & 2 & 3 & 4
            \end{pmatrix}\\
            \sigma_{1, -1} &\mapsto \begin{pmatrix}
                  \sqrt[4]{2}& i\sqrt[4]{2}&- \sqrt[4]{2}&-i\sqrt[4]{2}\\
                  \sqrt[4]{2}&-i\sqrt[4]{2}&- \sqrt[4]{2}& i\sqrt[4]{2}
            \end{pmatrix} \cong \begin{pmatrix}
                1 & 2 & 3 & 4\\
                1 & 4 & 3 & 2
            \end{pmatrix}\\
            \sigma_{i, 1} &\mapsto \begin{pmatrix}
                  \sqrt[4]{2}& i\sqrt[4]{2}&- \sqrt[4]{2}&-i\sqrt[4]{2}\\
                 i\sqrt[4]{2}&- \sqrt[4]{2}&-i\sqrt[4]{2}&  \sqrt[4]{2}
            \end{pmatrix} \cong \begin{pmatrix}
                1 & 2 & 3 & 4\\
                2 & 3 & 4 & 1
            \end{pmatrix}\\
            \sigma_{i, -1} &\mapsto \begin{pmatrix}
                \sqrt[4]{2}& i\sqrt[4]{2}&- \sqrt[4]{2}&-i\sqrt[4]{2}\\
               i\sqrt[4]{2}&  \sqrt[4]{2}&-i\sqrt[4]{2}&- \sqrt[4]{2}
            \end{pmatrix} \cong \begin{pmatrix}
                1 & 2 & 3 & 4\\
                2 & 1 & 4 & 3
            \end{pmatrix}\\
            \sigma_{-1, 1} &\mapsto \begin{pmatrix}
                  \sqrt[4]{2}& i\sqrt[4]{2}&- \sqrt[4]{2}&-i\sqrt[4]{2}\\
                - \sqrt[4]{2}&-i\sqrt[4]{2}&  \sqrt[4]{2}& i\sqrt[4]{2}
            \end{pmatrix} \cong \begin{pmatrix}
                1 & 2 & 3 & 4\\
                3 & 4 & 1 & 2
            \end{pmatrix}\\
            \sigma_{-1, -1} &\mapsto \begin{pmatrix}
                  \sqrt[4]{2}& i\sqrt[4]{2}&- \sqrt[4]{2}&-i\sqrt[4]{2}\\
                - \sqrt[4]{2}& i\sqrt[4]{2}&  \sqrt[4]{2}&-i\sqrt[4]{2}
            \end{pmatrix} \cong \begin{pmatrix}
                1 & 2 & 3 & 4\\
                3 & 2 & 1 & 4
            \end{pmatrix}\\
            \sigma_{-i, 1} &\mapsto \begin{pmatrix}
                  \sqrt[4]{2}& i\sqrt[4]{2}&- \sqrt[4]{2}&-i\sqrt[4]{2}\\
                -i\sqrt[4]{2}&  \sqrt[4]{2}& i\sqrt[4]{2}&- \sqrt[4]{2}
            \end{pmatrix} \cong \begin{pmatrix}
                1 & 2 & 3 & 4\\
                4 & 1 & 2 & 3
            \end{pmatrix}\\
            \sigma_{-i, -1} &\mapsto \begin{pmatrix}
                \sqrt[4]{2}& i\sqrt[4]{2}&- \sqrt[4]{2}&-i\sqrt[4]{2}\\
              -i\sqrt[4]{2}&- \sqrt[4]{2}& i\sqrt[4]{2}&  \sqrt[4]{2}
            \end{pmatrix} \cong \begin{pmatrix}
                1 & 2 & 3 & 4\\
                4 & 3 & 2 & 1
            \end{pmatrix}\\
        \end{align*}
        Dabei erhält man die Permutation, indem man $\sqrt[4]{2}$ mit $1$, $i\sqrt[4]{2}$ mit $2$, $-\sqrt[4]{2}$ mit $3$ und $-i\sqrt[4]{2}$ mit $4$ identifiziert.
        \item Wie auf Blatt 1 bewiesen, sind Elemente der $D_4$ eindeutig durch die Wirkung auf den Eckpunkten bestimmt.
        Daher können sie als Permutation der Eckpunkte $1,2,3,4$ geschrieben werden. In dieser Schreibweise gilt
        \begin{multline*}
            D_4 = \left\{\begin{pmatrix}
                1 & 2 & 3 & 4\\
                1 & 2 & 3 & 4
            \end{pmatrix}, \begin{pmatrix}
                1 & 2 & 3 & 4\\
                1 & 4 & 3 & 2
            \end{pmatrix},\begin{pmatrix}
                1 & 2 & 3 & 4\\
                2 & 3 & 4 & 1
            \end{pmatrix},\begin{pmatrix}
                1 & 2 & 3 & 4\\
                2 & 1 & 4 & 3
            \end{pmatrix},\right.
            \\
            \left.\begin{pmatrix}
                1 & 2 & 3 & 4\\
                3 & 4 & 1 & 2
            \end{pmatrix},\begin{pmatrix}
                1 & 2 & 3 & 4\\
                3 & 2 & 1 & 4
            \end{pmatrix},\begin{pmatrix}
                1 & 2 & 3 & 4\\
                4 & 1 & 2 & 3
            \end{pmatrix},\begin{pmatrix}
                1 & 2 & 3 & 4\\
                4 & 3 & 2 & 1
            \end{pmatrix}\right\}  
        \end{multline*}
        Insbesondere erhalten wir einen Isomorphismus zwischen $\Gal(L/\Q)$ und $D_4$ durch die Zuordnung von Permutationen der Nullstellen und Permutationen der Eckpunkte.
        \item Ist $M/\Q$ quadratisch, so gilt $[M\colon \Q] = 2.$ Daher erhalten wir $8 = [L\colon \Q] = [L\colon M] \cdot [M\colon \Q] \implies 4 = [L\colon M] = \# \Gal(L/M)$. Wir bestimmen daher die Untergruppen $\Gal(L/M)$ von $\Gal(L/\Q)$ mit Ordnung 4.
        \begin{enumerate}[1.]
            \item Zunächst bestimmen wir alle Untergruppen, die $\sigma_{i,1}$ enthalten. Es gilt $\sigma_{-1,1} = \sigma_{i,1}^2$, $\sigma_{-i,1} = \sigma_{i,1}^3$ und $\sigma_{1,1} = \sigma_{i,1}^4$. Daher ist $\operatorname{ord}(\sigma_{i,1}) = 4 = \operatorname{ord} \langle \sigma_{i,1}$. Folglich ist $\{\sigma_{1,1}, \sigma_{i,1}, \sigma_{-1, 1}, \sigma_{-i,1}\}$ die einzige echte Untergruppe, die $\sigma_{i,1}$ enthält. Völlig analoge Argumente zeigen, dass $\sigma_{-i,1}$ dieselbe Untergruppe erzeugt und diese die einzige echte Untergruppe ist, die $\sigma_{-i,1}$ enthält. 
            %Diese Untergruppe entspricht genau den Drehungen in $D_4$.
            \item Nun bestimmen wir alle restlichen Untergruppen der Ordnung 4, die $\sigma_{-1,1}$ enthalten. 
            \begin{enumerate}
                \item Angenommen, $\sigma_{\pm i, -1}$ liegt in der Untergruppe. Wegen $\sigma_{-1,1} \circ \sigma_{i,-1} = \sigma_{-i, -1}$ und $\sigma_{-1,1} \circ \sigma_{-i, -1} = \sigma_{i, -1}$ muss dann auch $\sigma_{\mp i, -1}$ in der Untergruppe liegen.
                Wegen $\sigma_{i, -1} \circ \sigma_{i,-1}(i) = i$ und $\sigma_{i, -1} \circ \sigma_{i,-1}(\sqrt[4]{2}) = \sigma_{i,-1}(i\sqrt[4]{2}) = -i \cdot i \sqrt[4]{2} = \sqrt[4]{2}$ ist $\sigma_{i,-1}^2 = \sigma_{1,1}$.
                Wegen $\sigma_{-i, -1}^2(i) = i$ und $\sigma_{-i, -1}^2(\sqrt[4]{2}) = \sigma_{-i,-1}(-i\sqrt[4]{2}) = i \cdot -i \sqrt[4]{2} = \sqrt[4]{2}$ ist $\sigma_{-i, -1}^2 = \sigma_{1,1}$.
                Also ist die Menge $\{\sigma_{1,1}, \sigma_{-1,1}, \sigma_{i,-1}, \sigma_{-i,-1}\}$ abgeschlossen bezüglich Multiplikation und Inversion und damit eine Untergruppe von $\Gal(L/\Q)$.
                \item Angenommen, $\sigma_{\pm 1, -1}$ liegt in der Untergruppe. Wegen $\sigma_{-1,1} \circ \sigma_{1,-1} = \sigma_{-1, -1}$ und $\sigma_{-1,1} \circ \sigma_{-1, -1} = \sigma_{1, -1}$ muss dann auch $\sigma_{\mp 1, -1}$ in der Untergruppe liegen.
                Es gilt $\sigma_{\pm 1, -1}^2 = \sigma_{1,1}$.
                Also ist die Menge $\{\sigma_{1,1}, \sigma_{-1,1}, \sigma_{1,-1}, \sigma_{-1,-1}\}$ abgeschlossen bezüglich Multiplikation und Inversion und damit eine Untergruppe von $\Gal(L/\Q)$.
            \end{enumerate}
            Damit sind alle Elemente von $\Gal(L/\Q)$ erschöpft. Wir untersuchen also im Folgenden Untergruppen, die weder $\sigma_{\pm i,1}$ noch $\sigma_{-1, 1}$ enthalten. 
            \item Es gilt wie oben bewiesen stets $\sigma_{\zeta, -1}^2 = \sigma_{1,1}$. Sein nun $\zeta \neq \zeta'$. Dann gilt $\sigma_{\zeta, -1} \circ \sigma_{\zeta', -1} = \sigma_{\zeta'', 1}$ mit $\zeta'' \neq 1$, sonst wäre das Inverse nicht eindeutig. Also gilt $\sigma_{\zeta'', 1} \in \{\sigma_{\pm i,1}, \sigma_{-1, 1}\}$ und damit kann keine Untergruppe der Ordnung 4 geben, die weder $\sigma_{\pm i,1}$ noch $\sigma_{-1, 1}$ enthält.
        \end{enumerate}
        Damit erhalten wir drei Untergruppen der Ordnung 4: 
        \begin{enumerate}[1.]
            \item $H_1 \coloneqq \{\sigma_{1,1}, \sigma_{ i,1}, \sigma_{-1,1}, \sigma_{-i, 1}\}$
            \item $H_2 \coloneqq \{\sigma_{1,1}, \sigma_{-1,1}, \sigma_{i,-1}, \sigma_{-i,-1}\}$
            \item $H_3 \coloneqq \{\sigma_{1,1}, \sigma_{-1,1}, \sigma_{1,-1}, \sigma_{-1,-1}\}$
        \end{enumerate}
        Diese Untergruppen korrespondieren zu drei Zwischenkörpern $M_1 \coloneqq L^{H_1}, M_2 \coloneqq L^{H_2}$ und $M_3 \coloneqq L^{H_3}$. Es gilt
        \begin{enumerate}[1.]
            \item $M_1 = \{a \in L\coloneqq \sigma(a) = a \forall \sigma \in H_1\}$. Offensichtlich ist $\sigma(a) = a\forall a \in K$. Darüberhinaus sieht man leicht, dass $\sigma(i) = i \forall \sigma\in H_1$ und damit $K(i) \subset M_1$. Wegen $[K(i)\colon K] = 2$ folgt $M_1 = K(i)$.
            \item $M_2 = \{a \in L\coloneqq \sigma(a) = a \forall \sigma \in H_2\}$. Offensichtlich ist $\sigma(a) = a\forall a \in K$. 
            Wir betrachten nun $i \sqrt{2}\in L$. Es gilt offensichtlich $\sigma_{1,1}(i\sqrt{2}) = i\sqrt{2}$.
            \begin{align*}
                \sigma_{-1, 1}(i\sqrt{2}) &= \sigma_{-1, 1}(i) \cdot \sigma_{-1, 1}(\sqrt[4]{2})^2 = i \cdot (-\sqrt[4]{2})^2 = i\sqrt{2}\\
                \sigma_{ i,-1}(i\sqrt{2}) &= \sigma_{ i,-1}(i) \cdot \sigma_{ i,-1}(\sqrt[4]{2})^2 =-i \cdot (i\sqrt[4]{2})^2 = i\sqrt{2}\\
                \sigma_{-i,-1}(i\sqrt{2}) &= \sigma_{-i,-1}(i) \cdot \sigma_{-i,-1}(\sqrt[4]{2})^2 =-i \cdot(-i\sqrt[4]{2})^2 = i\sqrt{2}
            \end{align*}
            Daher ist $K(i\sqrt{2}) \subset M_2$. Das Minimalpolynom von $i\sqrt{2}$ über $K$ ist $x^2 + 2$, also ist $K(i\sqrt{2})$ bereits eine quadratische Erweiterung und damit $M_2 = K(i\sqrt{2})$.
            \item $M_3 = \{a \in L\coloneqq \sigma(a) = a \forall \sigma \in H_3\}$. Offensichtlich ist $\sigma(a) = a\forall a \in K$. 
            Wir betrachten nun $\sqrt{2}\in L$. Es gilt offensichtlich $\sigma_{1,1}(\sqrt{2}) = \sqrt{2}$.
            \begin{align*}
                \sigma_{-1, 1}(\sqrt{2}) &= \sigma_{-1, 1}(\sqrt[4]{2})^2 = (-\sqrt[4]{2})^2 = \sqrt{2}\\
                \sigma_{ 1,-1}(\sqrt{2}) &= \sigma_{ 1,-1}(\sqrt[4]{2})^2 = ( \sqrt[4]{2})^2 = \sqrt{2}\\
                \sigma_{-1,-1}(\sqrt{2}) &= \sigma_{-1,-1}(\sqrt[4]{2})^2 = (-\sqrt[4]{2})^2 = \sqrt{2}
            \end{align*}
            Daher ist $K(\sqrt{2}) \subset M_3$. Das Minimalpolynom von $\sqrt{2}$ über $K$ ist $x^2 - 2$, also ist $K(\sqrt{2})$ bereits eine quadratische Erweiterung und damit $M_3 = K(\sqrt{2})$.
        \end{enumerate}
    \end{enumerate}
    \section*{Aufgabe 3}
    \begin{enumerate}[(a)]
        \item Bei $\mathbb{F}_9^\times$ handelt es sich um eine zyklische Gruppe der Ordnung $8$. Es existiert also ein $a\in \mathbb{F}_9^\times$ mit $\mathbb{F}_9^\times = \langle a\rangle$. Sei $a$ eine Nullstelle des Polynoms $X^4 + 1$. Dann gilt $a^4 \neq 1$. Angenommen, $a$ wäre kein Erzeuger der Einheitengruppe. Dann gäbe es ein kleinstes $k < 8$ mit $a^k = 1$. Für $\nu \in \N$ ist dann auch $a^{\nu k} = 1$. Nun ist aber $a^{4} \neq 1$. Angenommen, $k > 4$. Dann ist $1 = a^8 = a^k \cdot a^{8-k} = a^{8-k}$. Dann hätte man aber statt $k$ auch $8-k < k$ wählen können, Widerspruch. Also muss $k < 4$ sein. $k$ kann auch nicht 1 oder 2 sein, da dann $a^4 = 1$ wäre. Also muss $k = 3$ sein. Daraus folgt aber $a = a^9 = (a^3)^3 = 1$. Aus diesem Widerspruch folgt, dass $a$ ein Erzeuger der Einheitengruppe sein muss. Wegen $X^9 -X = X (X^4 + 1)(X^4 -1)$ und weil $\mathbb{F}_9$ gerade aus den 9 Nullstellen von $X^9 - 9$ besteht, muss es vier Elemente $a$ geben, die die Bedingung $a^4 + 1 = 0$ erfüllen. Eines dieser $a$ ist dann das gesuchte primitive Element.
        
        Es gilt $[\mathbb{F}_9\colon \mathbb{F}_3] = 2$. Daher handelt es sich um eine normale Erweiterung. Außerdem sind endliche Körper vollkommen, sodass die Erweiterung auch separabel und damit galoissch ist.
        Wie in Beispiel 4.2 erläutert ist $\Gal(\mathbb{F}_9/\mathbb{F}_3)$ zyklisch von der Ordnung $2$ und wird erzeugt vom Frobenius-Automorphismus $\sigma\colon \mathbb{F}_9 \to \mathbb{F}_9,\; a \mapsto a^3$. Es gilt $\sigma^2 = (a\mapsto a^9 = a) = \operatorname{id}_{\mathbb{F}_9}$. Daher ist $\Gal(\mathbb{F}_9/\mathbb{F}_3) = \{\sigma, \operatorname{id}_{\mathbb{F}_9}\}$.
        \item Es gilt $(\sqrt{3} + \sqrt{5})^3 = 18\sqrt{3} + 14 \sqrt{5}$. 
        Daher gilt $$\sqrt{3} = \frac{(\sqrt{3} + \sqrt{5})^3 - 14(\sqrt{3} + \sqrt{5})}{4} \in \Q(\sqrt{3} + \sqrt{5}).$$ 
        Insbesondere ist auch $$\sqrt{5} = \sqrt{3} + \sqrt{5} - \sqrt{3} \in \Q(\sqrt{3} + \sqrt{5}).$$ Daraus folgt die Inklusion $\Q(\sqrt{3}, \sqrt{5}) \subset \Q(\sqrt{3} + \sqrt{5})$. Die andere Inklusion ist trivial, also ist $\sqrt{3} + \sqrt{5}$ ein primitives Element.
        Wegen $\operatorname{char} \Q = 0$ ist $\Q(\sqrt{3}, \sqrt{5})/\Q$ separabel. 
        Die Familie $(X^2 -3, X^2 - 5)$ von Polynomen aus $\Q[X]$ zerfällt in Linearfaktoren über $\Q(\sqrt{3}, \sqrt{5})$. 
        Die Nullstellen der Polynome in der Familie sind gerade $\pm \sqrt{3}, \pm \sqrt{5}$. Also ist $\Q(\sqrt{3}, \sqrt{5})$ der Zerfällungskörper dieser Familie. 
        Insbesondere ist die Erweiterung $\Q(\sqrt{3}, \sqrt{5})/\Q$ normal und aufgrund der Separabilität bereits galoissch.
        Das Minimalpolynom $f \in \Q[X]$ von $\sqrt{3} + \sqrt{5}$ ist gegeben durch $X^4 - 16X^2 + 4$ (Reduktionskriterium für $p = 3$).
        Es gilt
        \[
            X^4 - 16X^2 + 4 = (X - (\sqrt{3} + \sqrt{5}))(X + (\sqrt{3} + \sqrt{5}))(X - (\sqrt{3} - \sqrt{5}))(X + (\sqrt{3} - \sqrt{5}))
        \]
        Die Menge aller Nullstellen ist daher
        \[
            N = \{\sqrt{3} + \sqrt{5}, -\sqrt{3} -\sqrt{5}, \sqrt{3} - \sqrt{5}, - \sqrt{3} + \sqrt{5}\}.
        \]
        Wegen Lemma 3.40 ist ein $\Q$-Automorphismus $\sigma\colon \Q(\sqrt{3} + \sqrt{5}) \to \Q(\sqrt{3} + \sqrt{5})$ eindeutig bestimmt durch $\sigma(\sqrt{3+5}) \in N$.
        Die Galoisgruppe ist daher gegeben durch
        \[
            \Gal(\Q(\sqrt{3} + \sqrt{5})/\Q) = \operatorname{Aut}_{\Q}(\Q(\sqrt{3}, \sqrt{5})) = \{\sigma_\zeta \colon \zeta \in N\},
        \] wobei $\sigma_\zeta \in \operatorname{Aut}_{\Q}(\Q(\sqrt{3}, \sqrt{5}))$ eindeutig bestimmt ist durch $\sigma_\zeta(\sqrt{3} + \sqrt{5}) = \zeta$.
    \end{enumerate}
    \section*{Aufgabe 4}
    \begin{enumerate}[(a)]
        \item Sei $\mathscr{M}$ die Menge aller Zwischenkörper von $K$ und $L$. Sei nun $f_M \in M[X]$ gegeben. Definiere dann
        \[
            A \coloneqq \bigcap_{\substack{M' \in \mathscr{M}\\ f_M \in M'}} M'.
        \]
        Daher ist $f_M \in A[X]$ und daher $M \subset A$. Allerdings ist auch $f_M \in M[X]$ und daher $A =\bigcap_{\substack{M' \in \mathscr{M}\\ f_M \in M'}} M' \subset M$. Daraus folgt bereits die Gleichheit und wir haben $M$ eindeutig bestimmt. 
        Wir können $f_K$ als Element von $M[X]$ auffassen. Auch dann gilt noch $f_K(\alpha) = 0$. 
        Per Definition des Minimalpolynoms ist dann $f_M$ ein Teiler von $f_K$ in $M[X]$ und daher insbesondere auch in $L[X]$.
        
        Aufgrund der Eindeutigkeit der Primfaktorzerlegung, weil es nur endlich viele Primfaktoren gibt und jeder Teiler von $f_K$ sich als Produkt einer Teilmenge der Primfaktoren schreiben lässt, kann es nur endlich viele Teiler von $f_K$ geben. 
        
        Gäbe es unendlich viele verschiedene Zwischenkörper $M$, so gäbe es auch unendlich viele verschiedene dazugehörige Minimalpolynome $f_M$ und damit auch unendlich viele verschiedene Teiler von $f_K$ in $L$, Widerspruch. Also gibt es nur endlich viele Zwischenkörper.
        \item Wir zeigen zunächst den Fall $L = K(\alpha, \beta)$. Angenommen, es gäbe keine $c \neq c'\in K$ mit $K(\alpha + c\beta) = K(\alpha + c'\beta)$. Dann gäbe es zu zwei verschiedenen Elementen $c, c' \in K$ stets verschiedene Körpererweiterungen $K(\alpha + c\beta)$ und  $K(\alpha + c'\beta)$. Wegen $\# K = \infty$ gäbe es also unendlich viele verschiedene Körpererweiterungen der Form $K(\alpha + c \beta)$. Es gilt aber $K \subset K(\alpha + c \beta) \subset K(\alpha, \beta) \forall c \in K$. Insbesondere gäbe es also unendlich viele Zwischenkörper zu der Erweiterung $L/K$. Das steht aber im Widerspruch zur Voraussetzung. 
        
        Es existieren also $c, c' \in K$ mit $c \neq c'$ und $M \coloneqq K(\alpha + c\beta) = K(\alpha + c'\beta)$. Insbesondere ist 
        \[
         \beta = \frac{\alpha + c \beta - (\alpha + c'\beta)}{c - c'} \in M
        \] und damit auch 
        \[
            \alpha = \alpha + c \beta - c \cdot \beta \in K.
        \]
        Also ist $K(\alpha, \beta) = K(\alpha + c\beta)$.

        Nun betrachten wir den allgemeinen Fall $L = K(\alpha_1, \dots, \alpha_n)$. Wir zeigen die Aussage per Induktion. Für $n=1$ ist sie offensichtlich wahr.
        Sei die Aussage für $n= k-1$ bewiesen. Dann gilt
        $K(a_1, \dots, a_k) = (K(a_1, \dots, a_{k-1}))(a_k) = K(\beta)(a_k) = K(\alpha)$. Also gilt die Aussage auch für $n = k$. Per Induktion folgt die Behauptung.
    \end{enumerate}
    \section*{Aufgabe 5}
    Zu jedem Zwischenkörper $M$ von $L/K$ existiert nach dem Hauptsatz der Galoistheorie eine Untergruppe $H$ von $\Gal(L/K)$ mit $M = L^H$.
    Da $L/K$ abelsch ist, muss $H$ als Untergruppe einer abelschen Gruppe ein Normalteiler sein, $M/K$ ist also normal.
    Insbesondere ist $M$ also Zerfällungskörper eines Polynoms in $K[X]$. Wählt man ein beliebiges irreduzibles Polynom $f \in K[X]$ mit einer Nullstelle in $L$, so erhält man durch den Zerfällungskörper dieses Polynoms einen Zwischenkörper $M$. 
    Gibt man nun eine beliebige Nullstelle $\alpha \in L$ dieses Polynoms an, so erhält man durch Adjunktion den Körper $K(\alpha)$. Offenbar gilt $K(\alpha) \subset M$, da $f$ über $M$ in Linearfaktoren zerfällt. Da $K(\alpha)/K$ aber ebenfalls normal ist, liegen alle anderen Nullstellen von $f$ auch in $K(\alpha)$. Damit ist $K(\alpha)$ bereits Zerfällungskörper von $f$. Da zwei Zerfällungskörper stets isomorph sind und $K(\alpha) \subset M$ folgt $K(\alpha) = M$.
    Der Weihnachtsmann wurde also nicht betrogen.
\end{document}