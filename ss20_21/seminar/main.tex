\documentclass{beamer}
\usetheme{Berlin}
\usepackage[utf8]{inputenc}
\usepackage[T1]{fontenc}
\usepackage[ngerman]{babel}
\usepackage{amsmath, amsfonts, amsthm, mathtools, amssymb, mathrsfs}
\usepackage{tikz}
\usepackage{marvosym}

\newcommand{\Q}{\mathbb{Q}}
\newcommand{\R}{\mathbb{R}}
\newcommand{\Z}{\mathbb{Z}}
\newcommand{\C}{\mathbb{C}}
\newcommand{\E}{\mathbb{E}}

\renewcommand{\Re}{\operatorname{Re}}
\renewcommand{\Im}{\operatorname{Im}}

\title{Uniformisierungstheorie}
\author{Josua Kugler}
\date{03.11.2020}

\begin{document}
\titlepage
\section{Ziel}
    \begin{frame}
        \begin{theorem}[Uniformisierungssatz]
            Jede einfach zusammenhängende Riemann'sche Fläche ist biholomorph äquivalent zur Einheitskreisscheibe $\E$ oder zur Zahlebene $\C$ oder zur Zahlkugel $\overline{\C}$.
        \end{theorem}
    \end{frame}
    \section{Grundlagen}
    \begin{frame}
        \begin{definition}
            $u(z)$ logarithmisch singulär bei $a \colon\!\!\!\!\Leftrightarrow u(z) + \log |z-a|$ harmonisch. %entsprechend für RF mit Karten, rein technische Sache
        \end{definition}
        \begin{definition}
            \[\mathcal{M}_a(X) \coloneqq \{u \colon X \setminus \{a\} \to \R|\; u \geq 0, u \text{ logarithmisch singulär bei } a\}\]
        \end{definition}
        \begin{definition}[Greensche Funktion]
            $\mathcal{M}_a \neq \emptyset \implies$ es existiert ein minimales Element $G_a$ (nicht trivial)
            $G_a$ heißt die Green'sche Funktion von $X$ in Bezug auf $a$.
        \end{definition}
    \end{frame}
    \begin{frame}
    \begin{definition}[positiv berandet/nullberandet]
        Eine Riemann’sche Fläche $X$ heißt positiv berandet, wenn zu jedem Punkt $a \in X$ die Greensche Funktion $G_a \colon X \to \R$ existiert. 
        Sonst heißt $X$ nullberandet.
    \end{definition}
\end{frame}
    \begin{frame}
    \begin{lemma}
        Auf nullberandeten Riemann'schen Flächen gilt der Satz von Liouville.
    \end{lemma}
    \begin{lemma}
        Auf einer Riemann'schen Fläche existiert eine harmonische Funktion $u \coloneqq u_{a,b} \colon X \setminus \{a, b\} \to \R$ mit folgenden Eigenschaften:
        \begin{itemize}
            \item $u$ ist logarithmisch singulär bei $a$.
            \item $-u$ ist logarithmisch singulär bei $b$.
            \item $u$ ist beschränkt im Unendlichen, d.h. in $X \setminus [U (a) \cup U (b)]$, wobei $U(a)$ und $U(b)$ zwei beliebige Umgebungen von $a, b$ seien.
        \end{itemize}
    \end{lemma}
\end{frame}
    \begin{frame}
    \begin{definition}
        Elementar $\colon\!\!\Leftrightarrow$ Beträge meromorpher Funktionen bilden eine Garbe, 
        d.h. aus $|f_i| = |f_j|$ auf $U_i \cap U_j \forall i, j \in I$ folgt die Existenz einer meromorphen Funktion $f \colon X \to \C$ mit $|f| = |f_i|$ auf $U_i$.
    \end{definition}
    \begin{theorem}[Monodromiesatz]
        Sei $X$ eine einfach zusammenhängende Riemann'sche Fläche und $f \colon U(a) \to \overline{\C}$ entlang jedes von $a$ ausgehenden Weges fortsetzbar. Dann existiert eine meromorphe Funktion $F\colon X \to \overline{\C}$ mit $F|_{U(a)} = f$.
    \end{theorem}
    \begin{lemma}
        Einfach zusammenhängende Riemannsche Flächen sind elementar.
    \end{lemma}
    
\end{frame}
\begin{frame}
    \begin{lemma}
        Einfach zusammenhängende Riemannsche Flächen sind elementar.
    \end{lemma}
    \begin{proof}
        \begin{tikzpicture}
            \path (-5,0) -- (5,0);
            \draw [black,thick] plot [smooth,samples=200, tension=1] coordinates { 
                (-2,0.2) (-.5,0) (1.2,-.3) (2.5,0) (4,-.5)};
            %\draw [blue,thick] plot [smooth,samples=200, tension=1] coordinates { 
            %    (-.5,.1) (1.2,-.2) (2.5,.1) (4,-.4)};
            %\draw [red,thick] plot [smooth,samples=200, tension=1] coordinates { 
            %    (-2,0.1) (-.5,-.1) (1.2,-.4) (2.2, -.1) (2.5,-.1)};
            \draw[gray] (0,0) circle (2.5);
            \draw[gray] (2,0) circle (2.5);
            \node[black] (alpha) at (-1.3,.4) {$\alpha$};
            %\node[red] (fi) at (-2,-.2) {$f_i$};
            %\node[blue] (fj) at (4,-.1) {$f_j$};
            \node[gray] (Ui) at (-2.5,1.3) {$U_i$};
            \node[gray] (Uj) at (4.4,1.3) {$U_j$};
          \end{tikzpicture}
    \end{proof}
\end{frame}
\begin{frame}
    \begin{lemma}
        Einfach zusammenhängende Riemannsche Flächen sind elementar.
    \end{lemma}
    \begin{proof}
        \begin{itemize}
            \item $|f_i/f_j| = 1$ auf $U_i \cap U_j \implies f_i/f_j = c_{ij}$
            \item Setze $f_i$ fort durch $c_{ij} \cdot f_j$
            \item Erhalte $f$ mit $f/f_k = \mathrm{const}$ auf $U_k$ mit $|f/f_k| = 1$. 
        \end{itemize}
    \end{proof}
\end{frame}

%\begin{proof}
    %    Sei $X = \bigcup U_i$ eine offene Überdeckung von $X$ und sei $f_i\colon U_i \to \overline{\C}$ eine Schar 
    %    invertierbarer (\textbf{Warum?}) meromorpher Funktionen mit der Eigenschaft $|f_i/f_j| = 1$ auf $U_i \cap U_j$.
    %    Man kann nun für ein $a \in U_i$ analytische Fortsetzungen für $f_i$ entlang von Wegen auf $U_i$ konstruieren, indem man auf $U_j$ die Funktion $f_j$ benutzt und diese mit dem konstanten (Maximumprinzip) Faktor $c_{ij} = |f_i/f_j|$ multipliziert. So kann man $f_i$ auf $U_j$ analytisch fortsetzen.
    %    Da $X$ einfach zusammenhängend ist, erhält man nach dem Monodromiesatz eine meromorphe Fortsetzungen von $f_i$ auf ganz $X$. Nach Konstruktion ist auf $U_k$ dann $f/f_k$ konstant und vom Betrag 1.
    %\end{proof}
    \section{Fall 1 (positiv berandet)}
    \begin{frame}
    \frametitle{Vorgehen}
    \begin{itemize}
        \item Es existiert eine holomorphe Funktion $F_a\colon X \to \C$ mit $|F_a(x)| = e^{-G_a(x)}$ für $x \neq a$.
        \item $F_a$ ist injektiv.
        \item Wir erhalten eine bijektive holomorphe (und damit direkt biholomorphe nach Funktheo 1) Abbildung von $X$ auf $F_a(X)$.
        \item $F_a(X)$ ist beschränkt ($|F_a(x)| < 1$) und einfach zusammenhängend.
        \item Mit dem Riemann’schen Abbildungssatz folgt $X \cong \mathbb{E}$.
    \end{itemize}
\end{frame}
    \begin{frame}
        \begin{lemma}
            Es existiert eine holomorphe Funktion $F_a\colon X \to \C$ mit $|F_a(x)| = e^{-G_a(x)}$ für $x \neq a$, $G_a\colon X \setminus \{a\} \to \R$ Greensche Funktion.
        \end{lemma}
        \begin{itemize}
            \item Greensche Funktion existiert stets.
            \item Es genügt, zu jedem Punkt $b$ mit Umgebung $U(b)$ eine holomorphe Funktion $F$ mit $|F(x)| = e^{-G_a(x)} \forall x \in U(b), x \neq a$ anzugeben. Nach Garbenaxiom 2 kann man diese zusammenkleben
        \end{itemize}
    \end{frame}
    \begin{frame}
        \begin{lemma}
            Es existiert eine holomorphe Funktion $F_a\colon U(b) \to \C$ mit $|F_a(x)| = e^{-G_a(x)}$ für $a \neq x \in U(b) \forall b \in X$.
        \end{lemma}
        \begin{itemize} 
            \item Fall 1:$b \neq a$.
            \begin{itemize}
                \item[$\implies$] OE $U(b)$ Elementargebiet
                \item[$\implies$] $\exists f$ mit $G_a = \Re f$
                \item[$\implies$] Wähle $F_a \coloneqq e^{-f}$
            \end{itemize}
            \item Fall 2: $b = a$. 
            \begin{itemize}
                \item[$\implies$] OE $U(b) = \E$
                \item[$\implies$] $G_a(z) = -\log|z|$
                \item[$\implies$] Wähle $F_a \coloneqq z$
            \end{itemize}
            %(kleine Umgebung um $a$, die sich konform auf eine Kreisscheibe abbildet). 
        \end{itemize}
    \end{frame}
    \begin{frame}
        \begin{lemma}
            Es existiert eine holomorphe Funktion $F_a\colon X \to \C$ mit $|F_a(x)| = e^{-G_a(x)}$ für $x \neq a$, $G_a\colon X \setminus \{a\} \to \R$ Greensche Funktion.
        \end{lemma}
        Insbesondere:
        \begin{itemize}
            \item $\lim\limits_{x \to a} |F_a(x)| = \lim\limits_{x \to a} e^{-G_a(x)} = 0$, also $F_a(a) = 0$
            \item $G_a(x) > 0 \implies |F_a(x)| < 1$.
        \end{itemize}
    \end{frame}
    \begin{frame}
    \frametitle{Vorgehen}
    \begin{itemize}
        \item Es existiert eine holomorphe Funktion $F_a\colon X \to \C$ mit $|F_a(x)| = e^{-G_a(x)}$ für $x \neq a$.
        \item $F_a$ ist injektiv.
        \item Wir erhalten eine bijektive holomorphe (und damit direkt biholomorphe nach Funktheo 1) Abbildung von $X$ auf $F_a(X)$.
        \item $F_a(X)$ ist beschränkt ($|F_a(x)| < 1$) und einfach zusammenhängend.
        \item Mit dem Riemann’schen Abbildungssatz folgt $X \cong \mathbb{E}$.
    \end{itemize}
\end{frame}
    \begin{frame}
        \begin{lemma}
            $F_a$ ist injektiv.
        \end{lemma}
        Betrachte $$F_{a,b}(x) \coloneqq \frac{F_a(x) - F_a(b)}{1 - \overline{F_a(b)}F_a(x)}.$$ Diese Funktion erfüllt folgende Eigenschaften.
        \begin{itemize}
            \item $|F_{a,b}| < 1$. (Rechnung)
            \item $F_{a,b}$ ist als Quotient analytischer Funktionen meromorph. Aufgrund der Beschränktheit muss $F_{a,b}$ aber sogar analytisch in $X$ sein.
            \item $|F_a(b)|^2 < 1 \implies F_{a,b}(b) = 0$, Ordnung $k$.
            \item $F_a(a) = 0 \implies F_{a,b}(a) = -F_a(b)$.
        \end{itemize}
    \end{frame}
    \begin{frame}
        \textbf{Behauptung.} $|F_{a,b}(x)| = |F_b(x)| \forall x\in X$.
        \begin{proof}
            \begin{itemize}
                \item $u(x) \coloneqq - \frac{1}{k} \log|F_{a,b}(x)|$ ist $\geq 0$ und harmonisch auf $X \setminus \{b\}$ mit einer logarithmischen Singularität bei $x = b$.
                \item Greensche Funktion: $G_b(x) \leq u(x)$.
                \item $e^{G_b(x)} \leq e^{u(x)}$. Umformen ergibt $\frac{|F_{a,b}(x)|}{|F_b(x)|} \leq 1$.
                \item Für $x = a$ folgt $|F_a(b)| \leq |F_b(a)|$. Symmetrie $\implies$ $\frac{|F_{a,b}(x)|}{|F_b(x)|}$ nimmt an einer Stelle ein Maximum an, nach dem Maximumprinzip erhalten wir die Behauptung.
            \end{itemize}
        \end{proof}
    \end{frame}
    \begin{frame}
        \begin{lemma}
            $F_a$ ist injektiv.
        \end{lemma}
        \begin{proof}
            Betrachte $$F_{a,b}(x) \coloneqq \frac{F_a(x) - F_a(b)}{1 - \overline{F_a(b)}F_a(x)}.$$
            Es gilt $|F_{a,b}(x)| = |F_b(x)| \forall x\in X$.
            Daraus folgt $F_{a,b} \neq 0$ für $x \neq b$, also $F_a(x) \neq F_a(b)$ für $x\neq b$. $b$ war beliebig $\implies F_a$ injektiv.
        \end{proof}
    \end{frame}
    \begin{frame}
    \frametitle{Vorgehen}
    \begin{itemize}
        \item Es existiert eine holomorphe Funktion $F_a\colon X \to \C$ mit $|F_a(x)| = e^{-G_a(x)}$ für $x \neq a$.
        \item $F_a$ ist injektiv.
        \item Wir erhalten eine bijektive holomorphe (und damit direkt biholomorphe nach Funktheo 1) Abbildung von $X$ auf $F_a(X)$.
        \item $F_a(X)$ ist beschränkt ($|F_a(x)| < 1$) und einfach zusammenhängend.
        \item Mit dem Riemann’schen Abbildungssatz folgt $X \cong \mathbb{E}$.
    \end{itemize}
\end{frame}
    \section{Zusammenfassung}
    \begin{frame}
        \begin{theorem}[Uniformisierungssatz]
            Jede einfach zusammenhängende Riemann'sche Fläche ist biholomorph äquivalent zur Einheitskreisscheibe $\E$ oder zur Zahlebene $\C$ oder zur Zahlkugel $\overline{\C}$.
        \end{theorem}
        Wir haben gezeigt:
        \begin{lemma}
            Jede positiv berandete einfach zusammenhängende Riemann'sche Fläche ist biholomorph äquivalent zur Einheitskreisscheibe $\E$.
        \end{lemma}
    \end{frame}

%%%%%%%%%%%%%%%%%%%%%%%%%%%%%%%%%%%%%%% PART II %%%%%%%%%%%%%%%%%%%%%%%%%%%%%%%%%%%%%%%%%%%%%%%%%%%
    \section{Wiederholung}
    \begin{frame}
    \begin{definition}[positiv berandet/nullberandet]
        Eine Riemann’sche Fläche $X$ heißt positiv berandet, wenn zu jedem Punkt $a \in X$ die Greensche Funktion $G_a \colon X \to \R$ existiert. 
        Sonst heißt $X$ nullberandet.
    \end{definition}
\end{frame}
    \begin{frame}
    \begin{lemma}
        Auf nullberandeten Riemann'schen Flächen gilt der Satz von Liouville.
    \end{lemma}
    \begin{lemma}
        Auf einer Riemann'schen Fläche existiert eine harmonische Funktion $u \coloneqq u_{a,b} \colon X \setminus \{a, b\} \to \R$ mit folgenden Eigenschaften:
        \begin{itemize}
            \item $u$ ist logarithmisch singulär bei $a$.
            \item $-u$ ist logarithmisch singulär bei $b$.
            \item $u$ ist beschränkt im Unendlichen, d.h. in $X \setminus [U (a) \cup U (b)]$, wobei $U(a)$ und $U(b)$ zwei beliebige Umgebungen von $a, b$ seien.
        \end{itemize}
    \end{lemma}
\end{frame}
    \section{Fall 2 (nullberandet)}
    \begin{frame}
        \begin{lemma}
            Wähle $a\neq b \in X$. Dann existiert eine holomorphe Funktion 
            $$f_{a,b} \colon X \setminus \{a,b\} \to \C$$ mit folgenden Eigenschaften
            \begin{enumerate}
                \item $f_{a,b}$ hat in $a$ bzw. $b$ eine Null- bzw. Polstelle 1. Ordnung
                \item $U(a), U(b)$ Umgebungen. $\exists C$ mit $$C^{-1} \leq |f_{a,b}(x)| \leq C$$ für $x \neq U(a) \cup U(b)$, d.h. $f_{a,b}$ hat außer $a$ und $b$ weder Pole noch Nullstellen.
            \end{enumerate}
        \end{lemma}
    \end{frame}
    \begin{frame}
        \begin{lemma}
            Auf einer beliebigen Riemann'schen Fläche existiert eine harmonische Funktion $u \coloneqq u_{a,b} \colon X \setminus \{a, b\} \to \C$ mit folgenden Eigenschaften:
            \begin{itemize}
                \item $u$ ist logarithmisch singulär bei $a$.
                \item $-u$ ist logarithmisch singulär bei $b$.
                \item $u$ ist beschränkt im Unendlichen, d.h. in $X \setminus [U (a) \cup U (b)]$, wobei $U(a)$ und $U(b)$ zwei beliebige Umgebungen von $a, b$ seien.
            \end{itemize}
        \end{lemma}
    \end{frame}
    \begin{frame}
        \begin{itemize}
            \item Lokal ist $u_{a,b}$ Realteil einer analytischen Funktion $f$. 
            \item Wähle also $f_{a,b} = e^f$ für eine Umgebung $U(c)$ mit $c \notin \{a,b\}$.
            \item $X$ elementar, also $f_{a,b} \colon X\setminus \{a,b\} \to \C$ analytisch.
            \item $u_{a,b}$ ist beschränkt auf $X \setminus [U (a) \cup U (b)]$. Folglich gilt 
            $e^{-C} \leq f_{a,b} \leq e^C$ auf $X \setminus [U (a) \cup U (b)]$.
            \item $f_{a,b}$ hat in $a$ eine Nullstelle und in $b$ eine Polstelle (jeweils 1. Ordnung), sonst aber werde Pol- noch Nullstellen.
        \end{itemize}
    \end{frame}
    \begin{frame}
        \begin{lemma}
            $f_{a,b}$ ist injektiv.
        \end{lemma}
        \begin{itemize}
            \item Als Quotient analytischer Funktionen ist $$g(z) \coloneqq \frac{f_{a,b}(z)-f_{a,b}(c)}{f_{c,b}(z)}.$$ meromorph und beschränkt außerhalb einer gewissen Umgebung um $a,b,c$. 
            \item Wegen $\lim\limits_{z \to c} g(z) = \lim\limits_{z \to c} \frac{f_{a,b}(z) - f_{a,b}(c)}{f_{c,b}(z)} = \mathrm{const}$ ist $g$ analytisch und beschränkt auf ganz $X$ und damit nach dem Satz von Liouville für nullberandete RF konstant.
            \item $f_{a,b}(z) - f_{a,b}(c) = \lambda f_{c,b}(z)$. Insbesondere hat $f_{a,b}(z) - f_{a,b}(c)$ genau eine Nullstelle bei $z = c$, d.h. $f_{a,b}$ ist injektiv.
        \end{itemize}
    \end{frame}
    \begin{frame}
        \begin{itemize}
            \item Wir erhalten eine bijektive holomorphe (und damit direkt biholomorphe nach Funktheo 1) Abbildung von $X$ auf $f_{a,b}(X) \subset \overline{\C}$.
            \item $f_{a,b}(X)$ nicht kompakt $\implies f_{a,b}(X) \neq \overline{\C}$  OE $f_{a,b}(X) \subset \C$. Riemann'scher Abbildungssatz $\implies$  $X \cong \C$ oder $X \cong \E$
            \item $X \cong \E \implies X$ positiv berandet \Lightning, weil $G_0$ existiert und die konformen Selbstabbildungen von $\E$ transitiv operieren.
        \end{itemize}
    \end{frame}
    \section{Klassifikation}
    \begin{frame}
        \begin{itemize}
            \item Einfach zusammenhängende Flächen $\checkmark$
            \item allgemeine Flächen: $X \cong \tilde{X}/\Gamma$, $\tilde{X}$ einfach zshgd.
            \item $\Gamma \subset \operatorname{Bihol}(\tilde{X}, \tilde{X})$ operiert frei auf $\tilde{X}$
        \end{itemize}
    \end{frame}
    \begin{frame}
        \frametitle{Universelle Überlagerung $\overline{\C}$}
        \begin{itemize}
            \item konforme Selbstabbildungen: Möbiustransformationen
            \item Möbiustrafos haben stets Fixpunkt auf $\overline{\C}$
            \item[$\implies$] Gruppen von Möbiustrafos operieren nicht frei, außer die triviale Gruppe
            \item[$\implies$] $X \cong \overline{\C}$  
        \end{itemize}
    \end{frame}
    \begin{frame}
            \frametitle{Universelle Überlagerung $\C$}
            \begin{itemize}
                \item konforme Selbstabbildungen: $z \mapsto az + b$ (Funktionentheorie I VL) 
                \item Besitzen für $a \neq 1$ einen Fixpunkt, also $a = 1$. 
                \item[$\implies$] Es gibt drei Möglichkeiten für eine frei operierende Gruppe.
            \begin{itemize}
                \item $\{0\}$, $X \cong \C$.
                \item zyklische Untergruppen $L = \{z \mapsto z + \tilde{b}, \tilde{b} \in \Z b\}$. Dann ist $\C/L \xrightarrow{z \mapsto e^{2\pi i z/b}} \C^*$ eine konforme Äquivalenz.
                \item $L$ ist ein Gitter, d.h. $L$ wird von den Abbildungen $z \mapsto z + 1$ und $z \mapsto z + \tau$ erzeugt $\implies$ $\C/L$ ist ein Torus.
                \item Zwei Tori sind äquivalent gdw $j(\tau)$ gleich ist
            \end{itemize}
        \end{itemize}
    \end{frame}
    \begin{frame}
        \frametitle{Universelle Überlagerung $\E \cong \mathbb{H}$}
        \begin{itemize}
            \item konforme Selbstabbildungen: $\operatorname{SL}(2, \R)/{\pm E}$
            \item freie Operation $\Leftrightarrow$ $\Gamma$ diskret und fixpunktfrei
            \item $\mathbb{H}/\Gamma \cong \mathbb{H}/\Gamma' \Leftrightarrow \Gamma = L\Gamma'L^{-1}$ mit $L \in \operatorname{SL}(2,\R)$. 
        \end{itemize}
    \end{frame}
\end{document}