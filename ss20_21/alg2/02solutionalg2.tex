\documentclass{article}

\usepackage{josuamathheader}
\newcommand{\ann}{\operatorname{Ann}}

\begin{document}
\section*{Aufgabe 2}
Gilt $A^m \cong A^n$, so auch $A^m \otimes_A A/\mathfrak{m} \cong A^n \otimes_A A/\mathfrak{m}$.
Nach Korollar 3.8 folgt $(A/\mathfrak{m})^m \cong (A/\mathfrak{m})^n$.
Da $\mathfrak{m}$ ein maximales Ideal ist, handelt es sich bei $A/\mathfrak{m}$ um einen Körper.
Wir zeigen, dass jeder $A$-Modulhomomorphismus $\phi \colon (A/\mathfrak{m})^m \to (A/\mathfrak{m})^n$ ein $A/\mathfrak{m}$-Vektorraumhomomorphismus ist.
Offensichtlich bleibt $\phi$ ein Gruppenhomomorphismus $\phi\colon (A/\mathfrak{m})^m \to (A/\mathfrak{m})^n$, wenn man die Modulstruktur vergisst.
Weiter folgt aus $f(ax) = af(x) \forall a\in A, x\in (A/\mathfrak{m})^m$ auch $f(ax) = af(x) \forall a\in A/\mathfrak{m}, x \in (A/\mathfrak{m})^m$.
Ein Modulisomorphismus liegt genau dann vor, wenn es zwei zueinander inverse Modulhomomorphismus gibt.
Diese sind dann beide auch Vektorraumhomomorphismen, sodass wir einen Vektorraumisomorphismus erhalten.
Insbesondere ist also der Modulisomorphismus $(A/\mathfrak{m})^m \cong (A/\mathfrak{m})^n$ auch ein $A/\mathfrak{m}$-Vektorraumisomorphismus. Nach LA1 folgt daraus $m = n$.
\section*{Aufgabe 4}
\begin{enumerate}[(a)]
    \item Sei $M \subset \mathfrak{p}$ für ein Primideal $\mathfrak{p}$. 
    Das von $M$ erzeugte Ideal $\mathfrak{a}$ ist das kleinste Ideal, das $M$ enthält und daher gilt $\mathfrak{a} \subset \mathfrak{p}$.
    Insbesondere ist also $V(M) \subset V(\mathfrak{a})$. 
    Die andere Richtung, also $V(\mathfrak{a}) \subset V(M)$, ist klar, da jedes Primideal, das $\mathfrak{a}$ enthält, sofort auch $M$ enthalten muss.
    Sei nun $\mathfrak{p}$ ein Primideal mit $\mathfrak{a}\subset \mathfrak{p}$.
    Wir zeigen, dass dann auch $r(\mathfrak{a}) \subset \mathfrak{p}$ gilt.

    Sei $x \in r(\mathfrak{a})$. Dann $\exists n \in\N$ mit $x^n \in \mathfrak{a} \subset \mathfrak{p}$.
    Nun gilt $x^n \in \mathfrak{p} \xRightarrow{\mathfrak{p} \text{ prim}} x \in \mathfrak{p}$.
    Insgesamt erhalten wir $r(\mathfrak{a}) \subset \mathfrak{p}$. 
    Es folgt
    \[
        \mathfrak{p} \in V(\mathfrak{a}) \implies \mathfrak{a} \subset \mathfrak{p} \implies r(\mathfrak{a}) \subset \mathfrak{p} \implies \mathfrak{p} \in V(r(\mathfrak{a})),
    \]
    also $V(\mathfrak{a}) \subset V(r(\mathfrak{a}))$.
    Die andere Richtung, also $V(r(\mathfrak{a})) \subset V(\mathfrak{a})$, ist klar, da jedes Primideal, das $r(\mathfrak{a})$ enthält, sofort auch $\mathfrak{a}$ enthalten muss.
    \item Für ein beliebiges Primideal $\mathfrak{p}$ gilt per Definition $0 \subset \mathfrak{p}$.
    Also ist $V(0) = \operatorname{Spec}(A)$. Wegen $\mathfrak{p} \neq A$ für ein Primideal $\mathfrak{p}$, aber $1 \in \mathfrak{p} \implies \mathfrak{p} = A$ folgt $V(1) = \emptyset$.
    \item Es gilt
    \[
        V\left( \bigcup_{i \in I} M_i\right) = \left\{\mathfrak{p} \colon \bigcup_{i\in I} M_i \subset \mathfrak{p}\right\} = \{\mathfrak{p} \colon \forall i \colon M_i \subset \mathfrak{p}\} = \bigcap_{i\in I} \{\mathfrak{p} \colon  M_i\subset \mathfrak{p}\} = \bigcap_{i\in I} V(M_i)
    \]
    \item Wir zeigen zunächst $\mathfrak{a}\cap \mathfrak{b} \subset \mathfrak{p} \Leftrightarrow \mathfrak{ab} \subset \mathfrak{p}$.
    Nach VL gilt $\mathfrak{ab} \subset \mathfrak{a}\cap \mathfrak{b}$, also ist "$\Rightarrow$" bereits klar.
    Sei nun $x \in \mathfrak{a}\cap \mathfrak{b}$. Dann gilt $x^2 \in \mathfrak{a}\mathfrak{b} \subset \mathfrak{p} \implies x \in \mathfrak{p}$.
    Damit ist auch "$\Leftarrow$" bewiesen.
    Wir schließen sofort $V(\mathfrak{a} \cap\mathfrak{b}) = V(\mathfrak{a}\mathfrak{b})$.
    Die Aussage $V(\mathfrak{a} \cap\mathfrak{b}) = V(\mathfrak{a}) \cup V(\mathfrak{b})$ folgt aus Aufgabe (c).
\end{enumerate}
\end{document}