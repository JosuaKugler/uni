% Explain surgery in general and cite the relevant lemmas from mil61

\subsection{Smooth surgery}
For the details of how to realize surgery in a smooth way, consult \cite[paragraph 1]{Milnor61}.
Surgery is a quite general operation, in fact \cite[Theorem 1]{Milnor61} states that two manifolds can be transferred 
into one another by a sequence of surgeries if and only if they belong to the same cobordism class.
In particular, the Stiefel-Whitney numbers (and hence orientability) are preserved under surgery.

Let $W^n$ be a manifold, $\lambda \in \pi_p(W)$ a homotopy class and $(f_0: S^p \to W)\in \lambda$. When can $\lambda$ be killed?
According to \cite[Lemma 3]{Milnor61}, a homotopy group can be killed if $n \geq 2p+1$ and the induced $S^p$-bundle $f_0^*(TW)$ is trivial.
In this case, all homotopy groups $\pi_i(W)$ for $i < p$ stay unchanged, but $\pi_p(W)$ changes to $\pi_p(W)/G$ where $G$ is a subgroup containing $\lambda$.

Pick any base point $x \in M \times S^1$ and then embed $x \times S^1 \hookrightarrow M \times T^2$.
As the normal bundle to $S^1 \hookrightarrow T^2$ is trivial, it follows that also the normal bundle to $S^1 \hookrightarrow M \times T^2$ is trivial.
This yields the desired embedding and so one can kill the respective homotopy class 
(which in this case is the same as a homology class because $\pi_1 = \mathbb Z^2$ is already an abelian group).
It turns out that the second generator in $H_1$ can be killed like that, too.
$H_2$ is then isomorphic to $\pi_2$ by the Hurewicz theorem.
After proving that the respective 2-surgery is possible, one has created a manifold whose $H_1$ and $H_2$ homology groups are zero.
In fact, one can prove that all homology groups are zero after the described surgery operations.
Finally, the following two lemmata conclude the proof that the resulting manifold is diffeomorphic to a sphere.

\begin{lemma}
    A simply connected homology sphere is homeomorphic to the sphere.
\end{lemma}
\begin{proof}
    Let $M^n$ be a simply connected CW-complex that is a homology sphere, i.e. $H^0(M) = \mathbb Z$ and $H^n(M) = \mathbb Z$. 
    As $M$ is simply connected, i.e. $\pi_1(M) = 0$, the Hurewicz theorem states that
    \[
    \pi_k(M) = 0 \; \forall 1 < k < n\text{ and }\pi_n(M) = H^n(M) = \mathbb Z.
    \]
    As a result, $M$ is a homotopy sphere.
    Consider a generator $f: S^n \to M$ of $\pi_n(M)$.
    On $\pi_0$ level, it maps one connected component to one connected component, so here the induced map is obviously bijective.
    On $\pi_k$ level with $0 < k < n$, it just maps $0$ to $0$ which is an isomorphism.
    On $\pi_n$ level, one needs to show that 
    \[
        f_*: \mathbb Z = \pi_n(S^n) \to \pi_n(M) = \mathbb Z 
    \]
    is an isomorphism. The identity map is a generator for $\pi_n(S^n)$. Now $f_*(\mathrm{id}) = f$, so one obtains a generator of $\pi_n(M)$. 
    A map from $\mathbb Z \to \mathbb Z$ that sends $1$ to $1$ is a group isomorphism. These considerations show that $f$ is a weak homotopy equivalence.
    As a smooth manifold, $M$ is a CW-complex. By Whiteheads theorem it follows that $f$ is a homotopy equivalence, so the generalized Poincar\'e conjecture 
    shows that $M$ is homeomorphic to the sphere.
\end{proof}

\begin{lemma}
    A topological sphere that bounds a homology ball is diffeomorphic to the standard smooth sphere.
\end{lemma}
\begin{proof}
    As $W$ is a simply connected homology ball, all homotopy groups $\pi_{\ge 1}$ are 0 by the Hurewicz theorem.
    Taking any constant map on $W$ therefore is a homotopy equivalence by Whiteheads theorem, i.e. $W$ is contractible. 
    Hence, cut out a ball inside $W$ and obtain a cobordism $W'$ from a sphere to $M$.
    One can prove, then, that $W'$ is an $h$-cobordism.
    Thus, by the $h$-cobordism theorem \cite[Theorem 1.9]{Ranicki02}, $M$ is diffeomorphic to a sphere.
\end{proof}