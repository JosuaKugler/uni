In the previous chapter, a contact structure $(S^{2n+1}, \xi_\text{new})$ has been constructed by applying the Bourgeois construction
to a contact manifold coming from a specific Milnor open book and then adjusting the topology with subcritical contact surgeries.
In the end it was proved that $\xi_\text{new}$ is homotopically standard.
The next step is to show that it is tight. 

In general, for a contact structure to be tight means that it's \textit{not overtwisted}, i.e. in dimension 3 it doesn't contain an overtwisted disk.
In higher dimensions, the notion of overtwistedness is a little more complicated. 
The seminal paper \cite{BEM15} provides a thorough understanding of overtwistedness in higher dimensions.
Finally, in \cite{CMP19}, the equivalence of many previous criteria related to overtwistedness was established.
For overtwistedness one usually has to find some object (like e.g. a \textit{Plastikstufe}) in the manifold.
This generally seems to be easier to prove than to show that it is tight, because then the existence of such an object has to be ruled out.

Usually, tightness is proved via holomorphic curves. Gromov and Eliashberg were the first to use that technique \cite{Gromov85,Eliashberg91}
when they proved that fillable manifolds are tight. They assumed the existence of an overtwisted disk and used that assumption to construct
a moduli space of holomorphic disks that in the end led to a contradiction.

Since then, holomorphic curves are the standard tool to show tightness.
Recent approaches like contact homology are also based on holomorphic curves techniques.
As an example, consider the proof in this situation. It uses holomorphic curves hidden in the framework of contact homology.

For a given contact manifold, one can define a partial order by the following relation
\[
    \alpha \geq \beta \Leftrightarrow \exists \text{ smooth function } f: M \to \mathbb R_{\geq 1} \text{ s.t. } \alpha = f \cdot \beta.
\]
\begin{definition}[k-ADC]cf.~\cite{Zhou21b,Lazarev20}
    A contact structure $(M, \xi)$ is called asymptotically dynamically convex, if there is an ordered sequence of contact forms 
    $\alpha_1 \geq \alpha_2 \geq \dots \geq \alpha_i \geq \dots$ and a sequence of real numbers $D_i \to \infty$
    with the following property:
    All contractible periodic orbits of the Reeb vector field $R_{\alpha_i}$ with $\alpha_i$-action $\leq D_i$ are non-degenerate
    and have degree $> k$.
\end{definition}

Let $n \geq 4$ (for $n = 3$, a similar but more complicated argument gives the same result).
According to \cite{vK08}, the Brieskorn sphere $\Sigma_{n-1}$ is index-positive, i.e. for a given contact form $\alpha$, all contractible periodic orbits
are non-degenerate and have positive degree.
In particular, by choosing $\alpha_i = \alpha$ and $D_i$ to be any ascending sequence, $\Sigma_{n-1}$ is 0-ADC.
Then, by the following lemma, the Bourgeois manifold $S^{2n-1} \times T^2$ is 1-ADC (in this case it actually shows that it's 2-ADC, but one only needs 1-ADC).

\begin{lemma} cf.~\cite[Lemma 2.8]{BGMZ22}\label{binding_adc}
    If the binding of an open book decomposition is $k$-ADC, then the corresponding Bourgeois manifold is $(k+2)$-ADC.
\end{lemma}
The proof uses the specific Reeb dynamics as computed in \cref{reeb_dynamics}, but it is omitted here.

According to \cite[Proposition 3.2 5 (c)]{BGMZ22}, 1-ADC contact manifolds are algebraically tight.
A manifold is algebraically tight if the contact homology doesn't vanish, i.e. this is the step where the holomorphic curves are hidden.

As algebraic tightness is preserved under contact surgery, the contact structure on $S^{2n+1}$ is algebraically tight, too.
Finally, algebraic tightness implies tightness.