% Explain why all surgery can be realized in a contact way

\subsection{What is an $h$-principle?}
$h$-principle stands for homotopy-principle. The term often appears in the following setting: There is an underlying set of topological objects (whatever that may be) $T$
and among them a subset $G$ of geometric (more special) objects. An $h$-principle would state that every object $x \in T$ is homotopic to an object $x \in G$.
For example, there is an $h$-principle for overtwisted contact structures: Any manifold that admits an almost contact structure will also admit an overtwisted
contact structure in the same homotopy class, i.e. every homotopy class of almost contact structures contains an overtwisted contact structure.


\subsection{The $h$-principle for this specific case}
% This work only deals with the subcritical case
% Is the following correct?
In this case, one needs an $h$-principle guaranteeing that the necessary surgeries can be realized as contact surgeries.
As prerequisite for any surgery, an embedding of the neighborhood of a sphere is required. 
A subcritical contact surgery is posssible if and only if there exists an isotropic embedding of such a sphere.
Now by the $h$-principle for isotropic embeddings \cite[section 12.4]{EM02}, any embedding of a sphere can be realized in an isotropic way,
i.e. all of the surgeries can be realized as contact surgeries.

