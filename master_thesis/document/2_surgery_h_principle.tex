% Explain why all surgery can be realized in a contact way

\subsection{What is an $h$-principle?}\label{sec:h_principle}
$h$-principle stands for homotopy-principle. The term often appears in the following setting: There is an underlying set of topological objects (whatever that may be) $T$
and among them a subset $G$ of geometric (more special) objects. An $h$-principle would state that every object $x \in T$ is homotopic to an object $x \in G$.

\subsubsection*{Examples:}
\begin{itemize}
    \item A car in the plane: A car can only move in the direction of its wheels. Such paths are called holonomic. Any non-holonomic path (e.g. sliding the car
    to the right or left) can be arbitrarily close approximated by a holonomic one. In particular, it is homotopic to a holonomic path.
    The existence of an arbitrarily close approximation is often expressed by calling it a \textbf{dense} $h$-principle.
    \item Nash-Kuiper-Theorem: Any smooth embedding of $M^{n-1}$ into $\mathbb R^n$ can be arbitrarily well approximated by an isometric $C^1$-embedding. 
    In particular, any such smooth embedding is homotopic to an isometric $C^1$-embedding. This also is a dense $h$-principle.
    \item The $h$-principle for overtwisted contact structures: Any manifold that admits an almost contact structure will also admit an overtwisted
    contact structure in the same homotopy class, i.e. every homotopy class of almost contact structures contains an overtwisted contact structure.
\end{itemize}


\subsection{The $h$-principle for subcritical isotropic embeddings}
Any operation performed on the Bourgeois manifold $M\times T^2$ needs to be compatible with the contact structure
in order to be useful for the remainder of the proof.
In this case, one needs that the abovementioned surgeries can be realized as contact surgeries.
This can be guaranteed by an $h$-principle:
As prerequisite for any surgery, an embedding of the neighborhood of a sphere is required. 
A contact surgery is possible if and only if there exists an isotropic embedding of such a sphere.
There is an $h$-principle that assures that given an embedding of a sphere, it can be homotoped to an isotropic embedding of that sphere.
However, it only applies for \textit{subcritical} surgery.
The term subcritical is often used to say that the involved dimensions are less then half of the ambient dimension.
For example, contact manifolds are always odd-dimensional, so for a contact manifold $M^{2n+1}$, any dimension $m \leq n$ would be subcritical.
Now, as the involved surgeries are $1$- and $2$-surgeries, they are subcritical as long as $\dim M\times T^2 \geq 5$, i.e. $\dim M \geq 3$.
In this case one can apply the $h$-principle for subcritical isotropic embeddings \cite[12.4.1]{EM02}.
Thus, in all cases considered in this thesis, any embedding of a $1$- or $2$-sphere is homotopic to an isotropic embedding of that sphere.
Therefore, all of the surgeries can be realized as contact surgeries.