The whole construction starts with an open book.

\begin{definition}\label{def:open_book_1}
    An open book decomposition of a manifold $M$ is a pair $(B,p)$ where the binding $B$ is a closed codim-2-submanifold of $M$ and the map $p$ is a smooth, locally trivial fibration
    \[
        p: M\setminus B \to S^1.
    \]
    The fibres $p^{-1}(\varphi), \varphi \in S^1$ are called the pages.
    Moreover, it is required that the binding $B$ has a trivial tubular neighborhood $B\times D^2$ in which $p$ is given by the angular coordinate in the $D^2$-factor.
    In this thesis, open books will also have Weinstein pages unless stated otherwise.
\end{definition}

In general, many open books would work, but for concreteness the following Milnor open book is chosen.
Define $f\colon \mathbb C^n \to \mathbb C$ by
\[
    (z_0, \dots, z_{n-1}) \mapsto z_0^k + z_1^2 + \dots z_{n-1}^2
\]
and consider the sphere $S^{2n-1} \subset \mathbb C^n$.
The intersection $f^{-1}(0) \cap S^{2n-1}$ is the so called Brieskorn sphere $B = \Sigma_{n-1}(k,2,\dots,2)$.
On the complement $S^{2n-1} \setminus B$, the map
\[
    \pi_f\colon S^{2n-1}\setminus B \to S^1\colon (z_0, \dots, z_{n-1}) \mapsto \frac{f(z_0, \dots, z_{n-1})}{|f(z_0, \dots, z_{n-1})|}
\]
is a fibration over $S^1$, the Milnor fibration.
According to \cite[Lemma 6.1]{Milnor69}, the fibers are smooth $2n-2$-dimensional manifolds with boundary $B$.
It is well-known that such a fibration is precisely an open book decomposition (of $S^{2n-1}$).

For $\dim B \geq 3$, the Brieskorn sphere is (simply) connected \cite[Lemma 2]{Brieskorn66}.

According to \cite[Thm 6.5]{Milnor69}, the pages of this open book (i.e. the Milnor fibers) have the homotopy type of a bouquet of spheres 
$S^{n-1} \vee \dots \vee S^{n-1}$ (it can be proved that the pages are Weinstein).
It follows from the Hurewicz theorem that 
\[
    H_0 = \mathbb Z, \qquad H_i = 0, \; 0 < i < n \qquad \text{and } H_n = \pi_n.
\]
\cite[Theorem 9.1]{Milnor69} implies that $H_n = \pi_n = \mathbb Z^{k-1}$.
This will be useful to find specific homology classes later in the thesis.