\begin{definition}[Convex hypersurface]cf.~\cite[Definition 1.1.4]{HH19}
    A hypersurface $\Sigma \subset (M, \xi)$ is convex if there exists a contact
    vector field $v$, i.e., a vector field whose flow preserves $\xi$, which is transverse to $\Sigma$
    everywhere.
\end{definition}

\begin{definition}[Dividing set]cf.~\cite[Definition 1.2.1]{HH19}
    With notation as above, define the dividing set $\Gamma(\Sigma) \coloneqq \{\alpha(v) = 0\}$ and $R_\pm(\Sigma) \coloneqq \{\pm \alpha(v) > 0\}$ 
    as subsets of $\Sigma$.
\end{definition}
It turns out that this is a codimension 2 submanifold and $R_\pm(\Sigma)$ are Liouville manifolds. 



\begin{definition}[ideal Liouville domain] \cite[Definition 1]{Giroux20}
An ideal Liouville domain
$(F, \omega)$ is a domain $F$ endowed with an ideal Liouville structure $\omega$. This
ideal Liouville structure is an exact symplectic form on $\operatorname{int} F$
admitting a primitive $\lambda$ such that: for some (and then any) function 
$u \colon F \to \mathbb R_{\geq 0}$  with regular level set $\partial F = \{u = 0\}$,
the product $u\lambda$ extends to a smooth 1-form on $F$ which 
induces a contact form on $\partial F$.
\end{definition}

\begin{definition}[corresponding Giroux domain]\cite[Section 5.3]{MNW13}
    \label{def:giroux_domain}
    Given an ideal Liouville domain $(F, \omega)$ with primitive $\lambda$
    and function $u \colon F \to \mathbb R_{\geq 0}$ as above,
    the corresponding Giroux domain is given by
    \[
        F \times S^1_\theta
    \]
    endowed with contact structure
    \[
        \ker(u \d \theta + u \lambda)
    \]
\end{definition}
$\partial_\theta$ is a contact vector field, as it doesn't change the contact structure. 
Also, it is transversal to $F$. Hence, $F$ is a convex surface and there always is a convex decomposition.

Start with a Bourgeois manifold $\operatorname{BO}(\Sigma,\dots)$.
Smoothly,
\[
    \operatorname{BO}(\Sigma,\dots) = \operatorname{OB}(\Sigma, \dots) \times T^2 
    = \big[\operatorname{OB}(\Sigma, \dots) \times S^1\big] \times S^1 
    \eqqcolon V \times S^1.
\]
In the convex decomposition of the first factor
\[
    V = V_+ \bigcup_\Gamma \overline{V}_-,
\]
where $\Gamma$ is the dividing set, $V_\pm$ turn out to be ideal Liouville domains according to \cite[Section 6]{DG12}.
In \cite[Section 5.3]{DG12}, the authors explicitly compute $\Gamma$ and 
$V_\pm$ for the Bourgeois construction and obtain %maybe make a picture here?
\[
    \Gamma = \{y = 0\} = p^{-1}(\{0\}) \cup_B p^{-1}(\{\pi\})
\]
and
\[
    V_+ = p^{-1}([0, \pi]) \times S^1, \qquad V_- = p^{-1}([\pi, 2\pi]) \times S^1,
\]
i.e. topologically $V_\pm = \Sigma \times D^*S_1$.
If $\alpha + x \d \phi + y \d \theta$ is the contact structure on $\operatorname{OB}(\Sigma, \dots)$,
then as explained in \cite[Section 5.3]{DG12}, $\alpha + x \d \phi$ is a 
$S^1_\phi$-invariant contact form on $\Gamma$,
\[
    \omega_\pm = \pm \d \left(\frac{\alpha}{y} + \frac{x}{y}\d \phi\right) 
\]
is an $S^1_\phi$-invariant symplectic form on $V_\pm$ and
$y$ is a function with zero level set $\pm \Gamma = \partial V_\pm$.
Hence, $(V_\pm, \omega_\pm)$ is an ideal Liouville domain with Liouville form
\[
    \beta_\pm = \pm \left(\frac{\alpha}{y} + \frac{x}{y}\d \phi\right).
\]
According to \cref{def:giroux_domain}, $V_\pm \times S^1_\theta$
endowed with the contact structure 
\[
    \ker(y \d \theta + y \beta_\pm) = \alpha + x \d \phi + y \d \theta
\]
is the corresponding Giroux domain. Clearly, this is just the restriction
of the open book contact structure. Hence, the whole procedure actually yields a
splitting into two Giroux domains
\[
    \operatorname{OB}(\Sigma, \dots) = V_+ \times S^1_\theta 
    \bigcup_{\Gamma \times S^1_\theta} V_- \times S^1_\theta
\]