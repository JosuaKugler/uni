\section*{Abstract}
Contact geometry is the study of odd-dimensional smooth manifolds equipped with contact structures, i.e.\ hyperplane distributions $\xi = \ker \alpha$ satisfying the contact condition
\[
    \alpha \wedge (\d \alpha)^n \neq 0.
\]
While they originally arise in the study of ODEs and in classical mechanics, the topological study of contact manifolds is a more recent and very active field of research.

A manifold can have multiple different contact structures, which can be either rigid (in which case one speaks of a ``tight" manifold) or flexible (in the sense that they satisfy an h-principle). The latter contact manifolds are then called overtwisted. A foundational result of Eliashberg \cite{Eliashberg89} and Borman--Eliashberg--Murphy \cite{BEM15}, roughly speaking, states that overtwisted contact manifolds exist in abundance, namely whenever the manifold admits the topological version of a contact structure (an \emph{almost} contact structure), which is a first obvious obstruction. In dimension three, an almost contact structure is simply an oriented $2$-plane field. 


To illustrate this dichotomy, consider the sphere $S^3$. By a result of Eliashberg \cite{Eliashberg92}, it has precisely one tight contact structure. On the other hand, it has infinitely many overtwisted contact structures, corresponding to the infinitely many homotopy classes of 2-plane fields on the 3-sphere. There are other examples where there are infinitely many or no tight contact structures on a contact manifold.

A further interesting property of contact manifolds comes from the fact that contact geometry is the odd-dimensional counterpart to symplectic geometry. Often, it is possible to view a contact manifold as the boundary of a symplectic manifold. Manifolds that are in this sense ``fillable" are always tight \cite{Gromov85,Eliashberg91}. The contrary, however, doesn't need to hold and one can ask the question under which conditions such tight, but non-fillable manifolds exist. The first examples of tight and non-fillable contact manifolds were constructed by Etnyre--Honda \cite{EH02} in dimension three, and by Massot--Niederkrueger--Wendl \cite{MNW13} in higher dimensions.


More recently, Bowden--Gironella--Moreno--Zhou \cite{BGMZ22} have shown that there exist homotopically standard, non-fillable but tight contact structures on all spheres $S^{2n+1}$ with $n >= 2$. 

Starting with a specific open book decomposition of $S^{2n-1}$, one can construct a contact form on this manifold using a well-known construction by Thurston--Winkelnkemper. Then, according to Bourgeois, this contact structure can be extended to a tight contact structure on $S^{2n-1}\times T^2$.
Applying subcritical surgery (preserving the tightness), one can kill the topology of the $T^2$-factor and obtain a tight contact structure on $S^{2n+1}$. Because of the special way of constructing it, one can show that it is non-fillable, but still homotopically standard.

The goal of my master thesis is to give a streamlined explanation of 
the results of Bowden--Gironella--Moreno--Zhou, including the necessary background needed to understand the main ideas.