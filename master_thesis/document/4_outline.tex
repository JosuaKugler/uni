So far, a tight and homotopically standard contact structure $(S^{2n+1}, \xi_\text{new})$ has been constructed on all odd-dimensional spheres.
This would be not very interesting by itself, as even the standard contact structure satisfies these properties.
This chapter, however, proves that $\xi_\text{new}$ is non-fillable and thereby finishes the proof.

The idea is to first assume that $S^{2n+1}$ is fillable and then derive a contradiction.
The proof essentially is a clever application of holomorphic curves.
As the dimension is potentially very large, one needs a good way to leverage holomorphic curves techniques.
This takes quite some geometric construction work: 
Convex decomposition is a method to split a convex hypersurface into two Liouville domains separated by the dividing set $\Gamma$.
In particular, this can be used to split the Bourgeois manifold into two Giroux domains (which basically are Liouville domains times $S^1$).
Specifically for such Giroux domains, there is a certain type of surgery, called "blow down operation" that removes the
interior of the domain and blows down its boundary.
Applying this blow down operation to both sides of the decomposed Bourgeois manifold, one obtains a
capping cobordism from the Bourgeois manifold to $\Gamma \times S^2$. 
If the Bourgeois manifold was fillable, this then gives a filling of $\Gamma \times S^2$.

Technically, the correct assumption would be that the \textit{surgered} Bourgeois manifold is fillable. 
However, it is similar and easier to directly assume that the Bourgeois manifold itself is fillable, 
so this thesis considers only the simpler case.
The argument can be modified to work in the more difficult case, too.

In any case, this $S^2$-factor is precisely what was needed, 
because it can be used to define a suitable moduli space of holomorphic curves. 
Given some regularity properties, this moduli space is the foundation for the proof of the homological obstruction theorem.
\begin{theorem}[Homological obstruction theorem]
    Any homology class in the dividing set $\Gamma$ that dies in the filling $W$, will also die in $V_\pm$.
\end{theorem}
\Cref{sec:proof_hom_obstr} makes this theorem more precise and provides a proof sketch, assuming that the moduli space and
related objects are smooth manifolds.
Finally, in \cref{sec:homology_class}, a homology class in $\Gamma$ is constructed that survives in $V_\pm$, but assuming
the existence of a filling, would die in the filling. This is the desired contradiction. 