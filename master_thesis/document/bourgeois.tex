\subsection*{The Bourgeois construction}

\begin{definition}\label{def:open_book_1}
    An open book decomposition of a manifold $M$ is a pair $(B,p)$ where the binding $B$ is a closed codim-2-submanifold of $M$ and the map $p$ is a smooth, locally trivial fibration \underline{i.e. a fiber bundle?}
    \[
        p: M\setminus B \to S^1.
    \]
    The fibres $p^{-1}(\varphi), \varphi \in S^1$ are called the pages.
    Moreover, it is required that the binding $B$ have a trivial tubular neighborhood $B\times D^2$ in which $p$ is given by the angular coordinate in the $D^2$-factor.
\end{definition}

\begin{definition}[another definition of open book]\label{def:open_book_2}
    An open book decomposition of a manifold $M$ is a pair $(B, p)$, together with a defining map $\Phi:M \to \mathbb R^2$ so that each $z \in \operatorname{int}(D^2)$ is a regular value.
    Here, $B \subset M$ is a closed codimension-2 submanifold, $p: M\setminus B \to S^1$ is a fiber bundle, and $\Phi$ is such that $\Phi^{-1}(0) = B$ and $p =  \Phi / |\Phi|$.
\end{definition}

\begin{lemma}
    \Cref{def:open_book_1} and \cref{def:open_book_2} are equivalent.
\end{lemma}
\begin{proof}
\ 
    \begin{itemize}
        \item["$\implies$"] First, we need to construct a defining map $\Phi$. On the trivial tubular neighborhood take $\Phi$ to be the $D^2$-component. Outside this neighborhood, just set $\Phi = p$.
        Then, $\Phi^{-1}(0) = B, p = \Phi/|\Phi|$ and for all $z \in \operatorname{int}(D^2)$, there is always a small neighborhood s.t. $\Phi$ is just a projection map and hence, $z$ is a regular value.
        \item["$\impliedby$"] We need to show the existence of the trivial tubular neighborhood of $B$.
        $\Phi^{-1}(D^2) \to D^2$ is a smooth submersion, as all points are regular points. 
        According to the rank theorem, we can choose local coordinates for a neighborhood of any $x \in B$ s.t. $\Phi$  is just a projection to the $D^2$-component. Due to the compactness of $B$, finitely many such neighborhoods suffice to cover $B$.
        Moreover, the coordinate maps can be chosen in a compatible way and glued together.
        In total, we find a chart $\psi$ of a neighborhood $U$ of $B$ s.t.
        \[
            \psi(U) \subset \psi(B) \times D^2.
        \]
        As there are only finitely many neighborhoods involved, we find an $\epsilon > 0$ s.t. $\Phi^{-1}(D_\epsilon^2) \subset \psi(U)$.
        Therefore, $U$ contains a neighborhood of the form $B \times D^2$.
        This is the desired trivial tubular neighborhood.
    \end{itemize}
\end{proof}

\begin{remark}
    Let $M$ be an oriented manifold with an open book decomposition $(B,p)$.
    Choose a parametrization of $S^1$ (i.e. the direction of $\partial_\varphi$).
    Then, there are two equivalent ways of defining the orientation of $B$ and the pages.
    \begin{itemize}
        \item Via the orientation of the pages: The orientation of $M$ is the same as the orientation of the page together with $\partial_\varphi$.
        \item Via the orientation of the binding: $B \times D^2$ is an embedded submanifold of the same dimension as $M$. Therefore, it inherits the orientation. As $B \times D^2$ carries the product orientation and the orientation of $D^2$ is given by $r \d r \wedge \d \varphi$, we can deduce the orientation of $B$ from the orientation of $M$.
    \end{itemize}
    In both cases, the orientation of the pages and the binding have to agree in the sense that the induced orientation on the boundary of (the closure) of each page coincides with the orientation of $B$.
    To see that both options are in fact equivalent, choose a positive basis $\mathfrak{b}$ of the binding $B$. Then, $\mathfrak{b}, \partial_r, \partial_\varphi$ is a positive basis for $M$ at that point according to the first option.
    As $- \partial_r$ is the outer normal vector of the page at this point, $-\partial_r, \mathfrak{b}$ is a positive basis of the page and so we get $-\partial_r, \mathfrak{b}, \partial_\varphi$ as a positive basis for $M$, which agrees with option 1 because $B$ is odd-dimensional.
\end{remark}

\begin{definition}\label{def:support}
    Let $(B,p)$ be an oriented open book decomposition of the oriented manifold $M$.
    A contact structure $\xi = \ker \alpha$ on $M$ is said to be \textbf{supported} by the open book decomposition $(B,p)$ of $M$ if
    \begin{enumerate}[(i)]
        \item the contact form $\alpha$ induces the positive orientation of $M$ ($\alpha \wedge (\d \alpha)^n > 0$).
        \item the 2-form $\d \alpha$ induces a symplectic form on each page, defining its positive orientation
        \item the 1-form $\alpha$ induces a positive contact form on $B$, i.e. 
        \[ 
            \alpha|_{TB} \wedge (\d \alpha|_{TB})^{(n-2)} > 0.
        \]
    \end{enumerate}
\end{definition}

\begin{theorem}\label{thm:bourgeois}
    Let $(M, \xi = \ker \alpha )$ be a closed contact manifold of dimension $2n - 1, n \geq 2$. One con find an open book decomposition $(B,p)$ of $M$ supporting $\xi$. According to Bourgeois, (\cite{Bourgeois02}) there is a contact structure $\tilde{\xi}$ on $M \times T^2$ (where $\tilde \xi$ massively depends on the choice of open book).
\end{theorem}
\begin{proof}
    We follow the proof of \cite[Thm 7.3.6]{Geiges08}.
    Wlog let $M$ be connected.
    The existence of an open book decomposition for $M$ is the theorem of Giroux-Mohsen as in \cite[Thm 7.3.5]{Geiges08}.
    By definition of an open book, there exists a tubular neighborhood $B \times D^2$ with polar coordinates $(r, p)$ on the $D^2$-part of the binding $B$, i.e. the angular coordinate is simply given by $p: M \setminus B \to S^1$ in that neighborhood.
    Now, we want to define smooth functions $x_1, x_2$ on $M$ that coincide with the cartesian coordinate functions on $D^2$ close to the binding $B$. In order to do that, choose a smooth function $\rho(r)$ on $B \times D^2$, s.t. 
    \begin{itemize}
        \item $\rho = r$ near the binding $B$,
        \item $\rho'(r) \geq 0$
        \item $\rho \equiv 1$ near $B \times \partial D^2$.
    \end{itemize}
    We extend this function to a smooth function $\rho: M \to [0,1]$
    by setting $\rho \equiv 1$ outside $B\times D^2$.
    Now, $x_1 \coloneqq \rho \cos p$ and $x_2 \coloneqq \rho \sin p$ are the desired smooth functions on $M$ that coincide with the Cartesian coordinate functions on the $D^2$-factor near $B$.
    We compute
    \begin{align*}
        x_1 \d x_2 - x_2 \d x_1 &= \rho^2 \cos^2 p \d p + \rho \cos p \sin p \d \rho + \rho^2 \sin^2 p \d p - \rho \cos p \sin p \d \rho\\
        &= \rho^2 (\cos^2 p + \sin^2 p) \d p\\
        &= \rho^2 \d p
    \end{align*}
    and, analogously,
    \begin{align*}
        \d x_1 \wedge \d x_2 = \rho \d \rho \wedge \d p.
    \end{align*}
    On $M\times T^2$, choose coordinates $(\theta_1, \theta_2)$ on the torus part of the manifold.
    Now we have all ingredients together to construct our contact form.
    Let
    \[
        \tilde \alpha \coloneqq x_1 \d \theta_1 - x_2 \d \theta_2 + \alpha.
    \]
    This is a well-defined 1-form on $M \times T^2$ ($\alpha$ is extended to $M\times T^2$ as the pullback $\pi_1^*\alpha$) and we can compute the derivative
    \[
        \d \tilde \alpha = \d x_1 \wedge \d \theta_1 - \d x_2 \wedge \d \theta_2 + \d \alpha,
    \]
    hence
    \begin{align*}
        (\d \tilde \alpha)^n =& (n-1)(\d \alpha)^{n-1}\wedge (\d x_1 \wedge \d \theta_1 - \d x_2 \wedge \d \theta_2)\\
        &- n(n-1)(\d \alpha)^{n-2}\wedge \d x_1 \wedge \d \theta_1 \wedge \d x_2 \wedge \d \theta_2.
    \end{align*}
    In order to verify the contact condition, we compute
    \begin{align*}
        \tilde \alpha \wedge (\d \tilde \alpha)^n =& (x_1 \d \theta_1 - x_2 \d \theta_2 + \alpha) \wedge (n-1)(\d \alpha)^{n-1}\!\wedge (\d x_1 \wedge \d \theta_1 - \d x_2 \wedge \d \theta_2)\\
        &- (x_1 \d \theta_1 - x_2 \d \theta_2 + \alpha) \wedge n(n-1)(\d \alpha)^{n-2}\wedge \d x_1 \wedge \d \theta_1 \wedge \d x_2 \wedge \d \theta_2\\
        =& (n-1)(\d \alpha)^{n-1}\wedge(x_1\d x_2 - x_2 \d x_1)\wedge \d \theta_1 \wedge \d \theta_2\\
        &+\underbrace{\alpha\wedge(n-1)(\d \alpha)^{n-1}\!\wedge \d x_1}_{2n-\text{form on } M} \wedge \d \theta_1 - \underbrace{\alpha \wedge (n-1)(\d \alpha)^{n-1}\!\wedge\d x_2}_{2n-\text{form on } M} \wedge \d \theta_2\\
        & + n(n-1)\alpha\wedge(\d \alpha)^{n-2}\wedge \d x_1 \wedge \d \theta_1 \wedge \d x_2 \wedge \d \theta_2\\
        \intertext{$M$ has dimension $2n-1$, i.e. the middle term is 0}
        =& (n-1)(\d \alpha)^{n-1}\wedge \rho^2 \d\phi \wedge \d \theta_1 \wedge \d \theta_2\\
        & + n(n-1)\alpha \wedge (\d \alpha)^{n-2}\wedge \rho \d\rho \wedge \d \phi \wedge \d \theta_1 \wedge \d \theta_2\\
    \end{align*}
    By condition (ii) of \cref{def:support}, $(\d \alpha)^{n-1}$ must be a positive volume form on the pages. As explained in that definition, the orientation on $M$ is given by $\partial_\phi$ and the orientation of the page. In particular, $(\d \alpha)^{n-1} \wedge \rho \d \phi$ is a positive volume form on $M$. Multiplied with a second $\rho$-factor, it vanishes along $B$. As $\theta_1 \wedge \theta_2$ is a positive volume form on $T^2$, the first term is non-negative everywhere and positive away from 
    \[
        \underbrace{B \times 0}_{\subset B \times D^2 \subset M} \times T^2.
    \]
    Let $\mathfrak{b}$ be a basis of the binding $B$ that is positively ordered. Then, $- \partial_r, \mathfrak{b}$ and (because the binding is odd-dimensional) $\mathfrak{b}, \partial_r$ are positive bases of the page. Clearly, then, 
    \[ 
        \mathfrak{a} \coloneqq \mathfrak{b}, \partial_r, \partial_\phi, \partial_{\theta_1}, \partial_{\theta_2}
    \] 
    is an ordered basis of $M\times T^2$.
    Using $\rho'(r) \geq 0$ everywhere, we deduce that $\d \rho(\partial_r)$ is non-negative.
    Hence, plugging $\mathfrak{a}$ into the second term, we conclude
    \begin{align*}
        &\left(n(n-1)\alpha \wedge (\d \alpha)^{n-2}\wedge \rho \d\rho \wedge \d \phi \wedge \d \theta_1 \wedge \d \theta_2\right)(\mathfrak{a})\\
        =& n(n-1) \rho \cdot (\alpha \wedge (\d \alpha)^{n-2})(\mathfrak{b}) \cdot d\rho(\partial_r) \cdot \d \phi(\partial_\phi) \cdot \d \theta_1(\partial_{\theta_1}) \cdot \d \theta_2(\partial_{\theta_2})\\
        \geq& 0.
    \end{align*}
    By condition (iii) of \cref{def:support}, $\alpha \wedge (\d \alpha)^{n-2}$ is positive on $B$. Therefore, the second term is positive on $B \times 0 \times T^2$ (hence also on a neighborhood) and non-negative everywhere else.
    In total, we have checked the contact condition and $\tilde \alpha$ is indeed a contact form on $M\times T^2$.
\end{proof}
\begin{definition}
    An almost contact structure is a cooriented hyperplane field $\eta$ (with an oriented trivial line bundle complementary to $\eta$ defining the coorientation) and a complex bundle structure $J$ on $\eta$.
    According to \cite[Prop 2.4.5]{Geiges08}, the space of complex bundle structures compatible to a symplectic form on $\eta$ is non-empty and contractible. Thus, the almost contact structure can (up to homotopy) be defined by equipping $\eta$ with a symplectic form.

    \underline{If we don't care about the coorientation}, we can therefore talk of the almost contact structure $(\eta, \omega)$ where $\omega$ is a symplectic form on $\eta$.
\end{definition}
\begin{remark}
    Let $\epsilon > 0$. The, $\tilde \alpha_\epsilon \coloneqq \epsilon x_1 \d \theta_1 - \epsilon x_2 \d \theta_2 + \alpha$ is a contact form.
    By Gray stability, all these contact structures are isotopic, so the underlying almost contact structures are homotopic, too. 
    For $\epsilon = 0$, we have the almost contact structure $\xi \oplus TT^2, \d \alpha \wedge \Omega$ where $\Omega$ is a volume form on $T^2$.
    This is homotopic to the almost contact structures for $\epsilon > 0$.
    Thus, up to homotopy, the almost contact structure of the Bourgeois form is given by 
    \[
        \xi \oplus TT^2, \d \alpha \wedge \Omega.
    \]
\end{remark}