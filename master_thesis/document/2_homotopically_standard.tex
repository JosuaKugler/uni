\begin{definition}
    An almost contact structure is a cooriented hyperplane field $\eta$ (with an oriented trivial line bundle complementary to $\eta$ defining the coorientation) 
    and a complex bundle structure $J$ on $\eta$.
    According to \cite[Prop 2.4.5]{Geiges08}, the space of complex bundle structures compatible with a symplectic form on $\eta$ is non-empty and contractible. 
    Hence, it suffices to choose a symplectic form $\omega$ on $\eta$ to determine the almost contact structure up to homotopy 
    (as the space of trivial line bundles complementary to $\eta$ is non-empty and contractible, too).
\end{definition}

An almost contact structure on $S^{2n+1}$ induces an almost complex structure on the hyperplane distribution.
It is possible to extend this almost complex structure to an almost complex structure on the ball $B^{2n+2}$.

For a given hypersurface $S$ in an almost complex manifold $(M, J)$, one can compute the field of complex tangencies by intersecting
the tangent bundle of the surface $TS$ with $J(TS)$. 
This gives a hyperplane distribution on $S$ and it is known that this induces an almost contact structure on $S$.
In particular, there is an almost contact structure on $S^{2n+1}$ for every almost complex structure on $B^{2n+2}$.

Constructing an almost complex structure $J$ on the ball from an almost contact structure $\xi$ on the sphere and then 
extending this almost complex structure to find an almost complex structure $\xi'$ on the sphere again results in a
homotopic almost complex structure, $\xi' \simeq \xi$.

As any two almost complex structure on the ball are homotopic, the preceding explanations show
that two almost complex structures on the sphere must be homotopic, too.