\begin{definition}
    An almost contact structure is a cooriented hyperplane field $\eta$ (with an oriented trivial line bundle complementary to $\eta$ defining the coorientation) 
    and a complex bundle structure $J$ on $\eta$.
    According to \cite[Prop 2.4.5]{Geiges08}, the space of complex bundle structures compatible with a symplectic form on $\eta$ is non-empty and contractible. 
    Hence, it suffices to choose a symplectic form $\omega$ on $\eta$ to determine the almost contact structure up to homotopy 
    (as the space of trivial line bundles complementary to $\eta$ is non-empty and contractible, too).
\end{definition}

\begin{comment} % not needed, the argument is a different one
Let $\epsilon > 0$. The, $\tilde \alpha_\epsilon \coloneqq \epsilon x_1 \d \theta_1 - \epsilon x_2 \d \theta_2 + \alpha$ is a contact form.
By Gray stability, all these contact structures are isotopic, so the underlying almost contact structures are homotopic, too. 
For $\epsilon = 0$, the kernel of this expression is the hyperplane field $\xi \oplus TT^2$.

Consider the form $\d \alpha \oplus \Omega$ where $\Omega$ is a volume form on $T^2$. It is a symplectic form on $\xi \oplus TT^2$.

The trivial line bundle for the coorientation can just be pulled back from $M$ to $M \times T^2$. 
As a result, $(\xi \oplus TT^2, \alpha \oplus \Omega)$ is an almost contact structure that is homotopic to the almost contact structures for $\epsilon > 0$.
Thus, up to homotopy, the almost contact structure of the Bourgeois form is given by 
\[
    \xi \oplus TT^2, \d \alpha \wedge \Omega.
\]
As a result, all of our constructions just return the 
\end{comment}

% fill in after writing the surgery part