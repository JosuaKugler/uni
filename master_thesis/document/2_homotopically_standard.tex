\begin{definition}
    An almost contact structure is a cooriented hyperplane field $\eta$ (with an oriented trivial line bundle complementary to $\eta$ defining the coorientation) 
    and a complex bundle structure $J$ on $\eta$.
    According to \cite[Prop 2.4.5]{Geiges08}, the space of complex bundle structures compatible with a symplectic form on $\eta$ is non-empty and contractible. 
    Hence, it suffices to choose a symplectic form $\omega$ on $\eta$ to determine the almost contact structure up to homotopy 
    (as the space of trivial line bundles complementary to $\eta$ is non-empty and contractible, too).
\end{definition}

An almost contact structure on $S^{2n+1}$ induces an almost complex structure on the hyperplane distribution.
In this particular situation, it is possible to extend this almost complex structure to an almost complex structure on the ball $B^{2n+2}$.
Then, as any two almost complex structures on the ball are homotopic, the resulting almost complex structure is homotopic to the
standard almost complex structure on the ball.

Now, one can extend this homotopy to almost complex structures o the boundary as follows: 
For a given hypersurface $S$ in an almost complex manifold $(M, J)$, one can compute the field of complex tangencies by intersecting
the tangent bundle of the surface $TS$ with $J(TS)$. 
This gives a hyperplane distribution on $S$, and it is known that this induces an almost contact structure on $S$.
Applying this to $M = B^{2n+2}, S = S^{2n+1}$ shows that the homotopy of almost complex structures on $B^{2n+2}$ 
induces a homotopy of almost complex structures on $S^{2n+1}$, hence proving that the almost complex structure is homotopically standard.