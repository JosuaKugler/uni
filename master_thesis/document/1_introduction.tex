Contact topology is the study of contact manifolds up to isotopy. 
Early in the development of the field, Eliashberg discovered the "overtwisted disk".
If a threedimensional contact manifold admits an embedding of such a special disk, it is called overtwisted.
As mentioned in the abstract, such contact structures satisfy an $h$-principle.
The term $h$-principle describes a situation where every element of an underlying set is homotopic to an
element in a subset of more special, "geometric" objects (see \cref{sec:h_principle}).
In this case, the more general set are almost contact structures, where an almost contact structure
is simply the underlying topological structure of a contact structure.
Any such almost contact structure is homotopic to an overtwisted contact structure.
This is why overtwisted contact structures are comparably uninteresting: They exist
in abundance and can be classified by purely topological methods.

Tightness, another property of contact structures, was soon proved to be equivalent to "not overtwisted".
Tight contact structures, however, are far more difficult to classify or even find than overtwisted ones.
Understanding tightness has since been a big motivating question in contact geometry.

The first systematic step in the direction of understanding tightness was the seminal work
by Eliashberg--Gromov \cite{Gromov85, Eliashberg91}. In order to understand their work,
one needs to consider a second important aspect of contact geometry:

While contact manifolds are always odd-dimensional, there is a corresponding idea in even dimensions.
Manifolds that are equipped with a closed nondegenerate 2-form are called symplectic. They are related
to contact manifolds in many ways.
For example, a hypersurface in a contact manifold with a transversal contact vector field
induces a convex splitting of this hypersurface: There is the so-called dividing set (a contact submanifold)
which divides the hypersurface into Liouville domains, i.e. exact symplectic manifolds (cf. \cref{sec:convex_decomposition}).

The main connection between symplectic and contact manifolds, however, is the concept of fillability.
A contact manifold is (strongly) fillable if it is the smooth boundary of a symplectic manifold and
near the boundary, the symplectic form $\omega$ is equal to $\d \alpha$.
For example, the standard contact structure on the sphere $(S^3, \xi)$ is fillable
by the standard symplectic structure on the ball $(B^4, \omega)$.

Now, as the word "strongly" already suggests, there are several variants of fillability:
A contact manifold can be weakly fillable, (strongly) fillable, Liouville fillable or (Wein-)Stein fillable.
From left to right, there are more and more conditions to be satisfied, i.e. we have a sequence
of implications from the right to the left.

Now, all components to understand the work of Eliashberg--Gromov are introduced:
They showed that any kind of fillability implies tightness.
This was the first general way to approach tightness.
For the proof, they used the technique of holomorphic curves. Until today, holomorphic curves techniques are
the only general approach to proving tightness.

In three dimensions, these inclusions were all proven to be strict, see \cref{fig:fillability}.
\begin{figure}
    \begin{align*}
        \{\text{Weinstein fillable}\} \stackrel{\text{\cite{Bowden12}}}{\subsetneq} 
        \{\text{Liouville fillable}\} \stackrel{\text{\cite{Ghiggini05}}}{\subsetneq} 
        \{\text{Strongly  fillable}\}\\ \stackrel{\text{\cite{Eliashberg96}}}{\subsetneq}
        \{\text{Weakly    fillable}\} \stackrel{\text{\cite{EH02}}}{\subsetneq}
        \{\text{Tight}\}
    \end{align*}
    \caption{Hierarchy of fillability in three dimensions}
    \label{fig:fillability}
\end{figure}

Basically all of the concepts explained before have been generalized to higher dimensions.
One example where this was not straightforward is the generalization of overtwistedness: 
After a long process, multiple potential definitions were shown to be equivalent in arbitrary dimensions in \cite{BEM15}.

In particular, there is a similar hierarchy of fillabilities in higher dimensions.

%create similar figure or find another good solution 
%Explain where Agustins papers come in
%Explain what I will do
%- open books and giroux correspondence
%- rough sketch of how the proof works -> shorten proof sketch in abstract