Contact topology is the study of contact manifolds up to isotopy. 
Early in the development of the field, Eliashberg discovered the "overtwisted disk".
If a threedimensional contact manifold admits an embedding of such a special disk, it is called overtwisted.
As mentioned in the abstract, such contact structures satisfy an $h$-principle.
The term $h$-principle describes a situation where every element of an underlying set is homotopic to an
element in a subset of more special, "geometric" objects (see \cref{sec:h_principle}).
In this case, the more general set are almost contact structures, where an almost contact structure
is simply the underlying topological structure of a contact structure.
Any such almost contact structure is homotopic to an overtwisted contact structure.
This is why overtwisted contact structures are comparably uninteresting: They exist
in abundance and can be classified by purely topological methods.

Tightness, another property of contact structures, was soon proved to be equivalent to "not overtwisted".
Tight contact structures, however, are far more difficult to classify or even find than overtwisted ones.
Understanding tightness has since been a big motivating question in contact geometry.

The first systematic step in the direction of understanding tightness was the seminal work
by Eliashberg--Gromov \cite{Gromov85, Eliashberg91}. In order to understand their work,
one needs to consider a second important aspect of contact geometry:

While contact manifolds are always odd-dimensional, there is a corresponding idea in even dimensions.
Manifolds that are equipped with a closed nondegenerate 2-form are called symplectic. They are related
to contact manifolds in many ways.
For example, a hypersurface in a contact manifold with a transversal contact vector field
induces a convex splitting of this hypersurface: There is the so-called dividing set (a contact submanifold)
which divides the hypersurface into Liouville domains, i.e. exact symplectic manifolds (cf. \cref{sec:convex_decomposition}).

The main connection between symplectic and contact manifolds, however, is the concept of fillability.
A contact manifold is (strongly) fillable if it is the smooth boundary of a symplectic manifold and
near the boundary, the symplectic form $\omega$ is equal to $\d \alpha$.
For example, the standard contact structure on the sphere $(S^3, \xi)$ is fillable
by the standard symplectic structure on the ball $(B^4, \omega)$.

Now, as the word "strongly" already suggests, there are several variants of fillability:
A contact manifold can be weakly fillable, (strongly) fillable, Liouville fillable or (Wein-)Stein fillable.
From left to right, there are more and more conditions to be satisfied, i.e. we have a sequence
of implications from the right to the left.

Now, all components to understand the work of Eliashberg--Gromov are introduced:
They showed that any kind of fillability implies tightness.
This was the first general way to approach tightness.
For the proof, they used the technique of holomorphic curves. Until today, holomorphic curves techniques are
the only general approach to proving tightness.

As a result of their work, there is a chain of inclusions from Weinstein- and Liouville- over strongly and weakly fillable to tight.
In three dimensions, these inclusions were all proven to be strict, see \cref{fig:fillability}.
\begin{figure}
    \begin{align*}
        \{\text{Weinstein fillable}\} \stackrel{\text{\cite{Bowden12}, \cite{BCS14}}}{\subsetneq} 
        \{\text{Liouville fillable}\} \stackrel{\text{\cite{Ghiggini05}, \cite{Zhou21}}}{\subsetneq}\\
        \{\text{Strongly  fillable}\} \stackrel{\text{\cite{Eliashberg96}, \cite{BGM22}}}{\subsetneq}
        \{\text{Weakly    fillable}\} \stackrel{\text{\cite{EH02}, \cite{MNW13}}}{\subsetneq}
        \{\text{Tight}\}
    \end{align*}
    \caption{Hierarchy of fillability in three dimensions (first citation) and in higher dimensions (second citation)}
    \label{fig:fillability}
\end{figure}

Basically all of the concepts explained before have been generalized to higher dimensions.
One example where this was not straightforward is the generalization of overtwistedness: 
After a long process, multiple potential definitions were shown to be equivalent in arbitrary dimensions in \cite{BEM15}.

Another example is that there is a similar hierarchy of fillabilities in higher dimensions. 
Most definitions generalize to higher dimensions in a straightforward way. Tightness can just be defined as \textit{not overtwisted}.
It turns out that the same chain of inclusions is true in higher dimensions, too.
Proving the strictness of these inclusions took another few years, but now all inclusions are known to be strict also in higher dimensions, 
see \cref{fig:fillability}.

Often, however, these counterexamples are manifolds where the underlying smooth manifold is very specific.
This leads to the question whether the strictness of these inclusions is due to specific geometric properties of the counterexamples.
In order to answer this question, Bowden--Gironella--Moreno--Zhou \cite{BGMZ22} have shown that tight but
non-fillable contact structures exist in every almost contact class that admits a tight structure (in $\dim = 5$, there is a small technical assumption).
This proves that there really is a inherently contact geometric difference 
that can't be attributed to specific properties of the underlying smooth topology.
They prove that result by first showing that there are homotopically standard, tight, but non-fillable contact structures on the sphere.
Then, by so called "contact connected sum" (a connected sum operation that respects the contact structure),
the more general result follows.

Also, they prove that any sphere in $\dim \geq 7$ admits homotopically standard Liouville- but not Weinstein-fillable contact structures
and, analogously, deduce a more general statement from this.
As weak and strong fillability are equivalent on the sphere, the remaining question along this line of research is
whether there is a Liouville- but not Weinstein-fillable contact structure on $S^5$ and if
a higherdimensional sphere admits strongly, but not Liouville-fillable contact structures.

This thesis focuses on the first theorem of \cite{BGMZ22}: $S^{2n+1}$ admits a homotopically standard, tight, but non-fillable contact structure
for any $n \geq 2$. In fact, as the case for $n = 2$ is similar, but technically more difficult, it is omitted.

The chapters of this thesis correspond to the three main steps of the proof. First, one constructs a homotopically standard contact structure on $S^{2n+1}$.
This contact structure is then, secondly, shown to be tight.
Thirdly, non-fillability is deduced.

The starting point of \cref{chap:construction} is the famous Giroux-correspondence.
Giroux disvovered that contact structures are closely related with open book decomposition.
An open book decomposition really is a very intuitive concept: In a manifold $M^n$, find a codimension two submanifold $B^{n-2}$
s.t. there is a fibration $p\colon M\setminus B \to S^1$. $B$ is then called "binding" and the fibers $p^{-1}(\theta)$ are the pages of the open book
(see the figure that I will create next week).

The Giroux correspondence states that in 3 dimensions, there is a (more or less) a bijection between contact structures
and open book decompositions.
In higher dimensions, similar statements hold, in particular any open book decomposition leads to a corresponding supported
contact structure via the Thurston-Winkelnkemper-construction.
This is used in the proof in order to construct a contact structure that is hopefully different from the standard contact structure on $S^{2n+1}$.

Constructing a tight contact structure is not so easy: The standard way would be to take the boundary of some symplectic manifold,
but then it is of course fillable, so this approach can't be used here.
Even though there are many procedures that make a contact structure overtwisted (e.g. adding a Lutz twist), tightening procedures are very rare.
Only the Bourgeois construction was known to often yield tight contact structures. 
A recent preprint by Avdek and Zhou now proves that, in fact, Bourgeois manifolds are always tight \cite{AZ24}.
Therefore, the next step in the proof is to apply the Bourgeois construction, resulting in a Bourgeois manifold $S^{2n-1} \times T^2$.
Clearly, this has the wrong topology, so the final step in the construction is to kill the $T^2$-factor with
subcricital contact surgeries, establishing a homotopically standard contact structure on the sphere $(S^{2n+1},\xi_\text{new}$.

In \cref{chap:tightness}, one uses Reeb dynamics and holomorphic curves hidden in the framework of contact homology to prove that $\xi_\text{new}$
is really tight.

Finally, \cref{chap:nonfillable} finds a contradiction if the manifold was fillable. So first, assume it is fillable.
The largest part then is devoted to proving that fillability implies a certain property for specific homology classes
(homological obstruction theorem).
Then, in the final section, such a homology class is constructed.
However, it doesn't satisfy the required property, establishing the desired contradiction.
The proof for the homological obstruction theorem is only sketched here.
Also as it is easier to prove for $S^{2n-1} \times T^2$ instead of the surgered manifold $S^{2n+1}$,
this thesis covers only the proof for $S^{2n-1} \times T^2$ according to \cite{BGM22}. 
The proof in the general case is similar.