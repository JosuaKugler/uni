\begin{comment}
In \cite[Section 6]{MNW13}, where the clean cut-out is first introduced,
it is shown that it corresponds to a symplectic cobordism.
The setting there is actually more general: The authors consider a Giroux
domain where already some of the boundary components have been blown down.
In this case, the situation is simpler: The Giroux domain 
$V_- \times S^1 \subset \operatorname{BO}(\Sigma, \dots)$
is directly obtained from the corresponding ideal Liouville domain $V_-$ by 
round contactization.
Its boundary is given by 
\[
    \partial V_- \times S^1 = \Gamma \times S^1.
\]
We want a cobordism from $\operatorname{BO}(\Sigma, \dots)$ to
the same manifold after clean cut-out of $V_- \times S^1$.
Topologically, the cobordism
\[
    W \coloneqq [0,1] \times \operatorname{BO}(\Sigma, \dots) \bigcup_{\{1\} \times \left(V_- \times S^1\right)} V_- \times D^2.
\]
fulfills the requirements: One boundary is simply 
$\operatorname{BO}(\Sigma, \dots)$.
The other boundary is 
\[
    \left[\operatorname{BO}(\Sigma, \dots) \setminus \left(V_- \times S^1\right)\right] 
    \cup_{\Gamma \times S^1} \Gamma \times D^2,
\]
which is homeomorphic to 
$\operatorname{BO}(\Sigma, \dots) \setminus \left(V_- \times S^1\right)$
with blown down boundary, which is exactly the desired clean cut-out.

Next, compute the contact structure on $\operatorname{BO}(\Sigma, \dots) \setminus \left(V_- \times S^1\right)$
with blown down boundary $\partial V_- \times S^1 = \Gamma \times S^1$.

This should be the contact boundary of $V_+ \times D^2$.
Topologically, that makes sense,
\[
    \partial (V_+ \times D^2) = V_+ \times S^1 \cup \partial V_+ \times D^2 = V_+ \times S^1 \cup_{\Gamma \times S^1} \Gamma \times D^2.
\]
However, it is unclear to me why $V_+ \times D^2$ should carry a \underline{symplectic structure}.

It should also be supported by the open book $\operatorname{OB}(V_+, \operatorname{id})$,
which, again, topologically, makes perfect sense, but I don't understand the
\underline{contact side of things}.
\end{comment}

For now, consider the easier case where no additional surgery is carried out on the Bourgeois manifold.
That makes the proof for the homological obstruction theorem easier to understand.
It can be generalized to the case with surgery later.
The idea is to cap off both sides (i.e. $V_+$ and $V_-$) of the convex decomposition
using the surgery procedure introduced in the last section.
Topologically, this will result in $\Gamma \times S^2$ and 
it will turn out to carry a stable Hamiltonian structure.

In \cite[Section 6]{MNW13}, where the clean cut-out is first introduced,
it is shown that it corresponds to a symplectic cobordism.
The setting there is actually more general: The authors consider a Giroux
domain where already some of the boundary components have been blown down.
In this case, the situation is simpler: The Giroux domain 
$V_\pm \times S^1 \subset \operatorname{BO}(\Sigma, \dots)$
is directly obtained from the corresponding ideal Liouville domain $V_\pm$ by 
round contactization.
Its boundary is given by 
\[
    \partial V_\pm \times S^1 = \Gamma \times S^1.
\]

Now, considering the manifold $(V_+ \cup_\Gamma V_-)\times S^1$, perform a clean cut-out
on both ends, i.e. remove the interior of $V_\pm \times S^1$ and blow down the boundary $\Gamma \times S^1$.
In order to do this properly, it is necessary to consider a neighborhood 
$\Gamma \times (-\delta, \delta)$ around the dividing set and 
instead of $V_\pm$ take $V_\pm' \coloneqq V_\pm \setminus \Gamma \times [0, \pm \delta]$.
Topologically, blowing down $\Gamma \times S^1$ yields $\Gamma \times D^2$.
As this is done on both sides, the result is 
\[
    \Gamma \times D^2 \cup_{\Gamma \times S^1 \times -\delta} \Gamma \times (-\delta,\delta) 
    \cup_{\Gamma \times S^1 \times +\delta} \Gamma \times D^2 \cong \Gamma \times S^2.
\]

%add picture here from drawing_capping_cobordism.xournal

This whole surgery procedure can be realized as applying a certain symplectic cobordism.
This is made precise in the following
\begin{lemma}(follows from \cite[Theorem 6.1]{MNW13}, cf. \cite[Lemma 6.1]{BGM22})
    There is a symplectic cobordism $(W, \omega)$ with negative (i.e. concave) 
    contact boundary $\partial_-W = V \times S^1$ and weakly convex positive boundary
    $\partial_+ W = \Gamma \times S^2$ where $\Gamma = \partial V$. 
    Moreover, there is a tubular neighborhood $(-\delta, 0] \times \partial_+ W$
    such that $\omega$ is of the form $d(e^t \alpha_\Gamma) + \omega_S$, 
    where $t \in (-\delta, 0]$ and 
    \begin{itemize}
        \item $\omega_S$ is an area form on $S^2$,
        \item $\alpha_\Gamma$ is a contact form on $\Gamma$.
    \end{itemize}
    Lastly, there are symplectic submanifolds $W_\pm$ , diffeomorphic to $V_\pm$, 
    such that $W\setminus W_\pm$ deformation retracts onto its negative boundary, 
    and such that $W_\pm$ intersect transversely, positively and in exactly one point, 
    each symplectic sphere in the previously described neighborhood 
    of the positive boundary $\partial_+ W$.
\end{lemma}

The submanifolds $W_\pm$ are basically $V_\pm \times 0 \subset V_\pm \times D^2$.
If these submanifolds are taken out, the effect of the capping is topologically trivial.
Therefore, the remaning cobordism is topologically just $\partial_- W\times [0,1]$.
This clearly deformation retracts to the negative boundary.

Now there is a moduli space of holomorphic curves
