
$W$, like any symplectic manifold, admits a compatible almost complex structure $J$.
Now, the maps
\begin{align*}
    u_{(t,q)} \colon S^2 &\to W,\\
    y &\mapsto (t, q, y) \in (-\delta, 0] \times \underbrace{\Gamma \times S^2}_{= \partial_+ W}.
\end{align*}
are $J$-holomorphic curves.
%Why are they holomorphic?

Consider the connected component $\mathcal{M}$ of the moduli space of $J$-holomorphic spheres containing these $u_{t,q}$
and denote with $\mathcal{M}_*$ the corresponding marked moduli space with evaluation map
\[
    \operatorname{ev}\colon \mathcal{M}_* \to W.
\]
Define the respective Gromov compactifications $\overline{\mathcal{M}}$ and $\overline{\mathcal{M}_*}$.
Gromov proved that any sequence of holomorphic curves with finite energy bound converges to a
holomorphic curve that possibly has nodes or bubbles.
For now, just think of the Gromov compactification as adding all such limits of sequences in our moduli space to the space.
The finite energy bound follows from the fact that the homology class of the curve doesn't change.
\[
    E(u_k) = \int_{S^2} u_k^*(\omega) = \langle [u_k], [\omega] \rangle = \langle [S^2], [\omega]\rangle = \operatorname{const}.
\]
%What is a Gromov compactification precisely?

The following lemma describes the compactified moduli space close to the boundary.
\begin{lemma}(\cite[Lemma 6.3]{BGM22}, cf. \cite[page 334]{MNW13})\label{lem:local_uniqueness}
    Any curve in $\overline{\mathcal M_*}$ that intersects a small enough collar neighborhood
    \[
        (-\epsilon, 0] \times \Gamma \times S^2
    \]
    is already a reparametrization of a sphere $u_{t,q}$ (i.e. equivalent in the moduli space).
\end{lemma}

Now consider the evaluation map $\operatorname{ev}\colon \overline{\mathcal M_*} \to W$ close to the boundary (in the neighborhood of \cref{lem:local_uniqueness}).
There, the map is a diffeomorphism: Take any point $p  = (t, q, y) \in (-\epsilon, 0] \times \Gamma \times S^2$. 
Then, by construction there exists at least one holomorphic sphere ($u_{t,q}$) that intersects this point
and even has this point as marked point.
Now take any such curve. By the lemma, it is equivalent in the moduli space $\overline{\mathcal M_*}$ to $u_{t,q}$, hence the map is bijective.
The proof of smoothness is skipped here.

According to \cref*{lem:capping_cobordism}, any such curve intersects $W_\pm$ transversely, positively and in exactly one point.
This gives rise to an intersection map
\[
    \mathcal{I}^\pm\colon \overline{\mathcal{M}} \to W_\pm.
\]

As $\partial W_\pm \subset \partial W$, a collar neighborhood of $\partial W_\pm$ is contained in a collar neighborhood of $\partial W$.
Therefore, the uniqueness lemma applies and it turns out that this map is a diffeomorphism close to the boundary $\partial \overline{\mathcal M}$.
We skip the smoothness and show that it's bijective.
\begin{itemize}
    \item Surjectivity: For $x \in W_\pm \cap (-\epsilon, 0] \times \Gamma \times S^2$, consider the coordinate representation $x = (t, q, y)$.
    Clearly, a preimage is given by $u_{t,q} \in \overline{\mathcal M_*}$ with marked point $x$. 
    \item Injectivity: The preimage is unique due to the uniqueness lemma.
\end{itemize}