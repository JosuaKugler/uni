\section*{Bourgeois \& Thurston--Winkelnkemper}
Applying the Thurston--Winkelnkemper construction yields a contact form $\alpha$ on
\[
    M = \left(\Sigma_{\overline{\varphi}}(\phi)\; \dot\cup\; \mathcal{N}\right)/\sim.
\]
Now, we apply the Bourgeois construction to it and obtain a contact form
\[
    \tilde \alpha = \alpha + x_1 \d \theta_1 - x_2 \d\theta_2
\]
on $M \times T^2$ where $\theta_1, \theta_2$ are coordinates on $T^2$
and $x_1, x_2$ are coordinates as described in the section on the Bourgeois construction, i.e. there is a function $\rho: M \to [0,1]$ that agrees with $r$ near the binding $B$ and we define
\[
    x_1 \coloneqq \rho \cos(p);\qquad x_2 \coloneqq \rho \sin(p).
\]

Inside $\mathcal{N}$, our projection map $p$ is given by the angular coordinate $\varphi$. Therefore,
\[
    \alpha|_\mathcal{N} = \alpha_\text{ext} = h_1(r)\beta_\partial + h_2(r)\d \varphi = h_1(r)\beta_\partial + h_2(r)\d p.
\]
In total, we obtain for the contact form on $\mathcal{U} \coloneqq \mathcal{N}\times T^2$
\[
    \tilde \alpha = h_1(r)\beta_\partial + h_2(r)\d p + \rho(r)(\cos(p) \d \theta_1 - \sin(p) \d \theta_2)
\]
Outside $\mathcal{U}$, the form is given by
\[
    \tilde \alpha = \beta + \d p + \cos(p) \d\theta_1 - \sin(p) \d \theta_2,
\]
as $\rho(r) = 1\; \forall r \geq 1$.
Collecting all the conditions on $h_1$ and $h_2$ from the last section, we require that
\begin{itemize}
    \item $h_1(r) = e^s = e^{1-r}, h_2(r) = 1$ in the gluing area ($ 1 \le r \le 2$).
    \item Smoothness around the binding and contact condition on the binding: $h_1(r) = 2 - r^2$ and $h_2(r) = r^2$ around $r = 0$.
    The $-r^2$-part is only important to have $h_1'(r) < 0$ around $0$ to satisfy the symplectic condition on the pages.
    \item Contact condition on the tubular neighborhood: \[
        h_1(r)^{n-1}\det \begin{pmatrix}
            h_1(r) & h_2(r)/r\\
            h_1'(r) & h_2'(r)/r
        \end{pmatrix} > 0 \qquad \forall r \in [0,2].
    \]
\end{itemize}
We now define two functions $\mu, \nu: M \to \mathbb R$ as follows:
\[
    \mu = \begin{cases}
        \frac{\rho'}{\rho' h_1 - \rho h_1'} &\text{inside } \mathcal{U}\\
        0 &\text{outside } \mathcal{U} 
    \end{cases}  
    \quad \text{and} \quad 
    \nu = \begin{cases}
        \frac{-h_1'}{\rho' h_1 - \rho h_1'} &\text{inside } \mathcal{U}\\
        1 &\text{outside } \mathcal{U}
    \end{cases}.
\]
\underline{They are smooth etc.}
\begin{lemma}
    The Reeb vector field of the contact form $\tilde \alpha$ is given by
    \[
        R = \mu(r)R_B + \nu(r)[\cos(p)\partial_{\theta_1} - \sin(p)\partial_{\theta_2}],
    \]
    where $R_B$ is the Reeb vector field of $\beta_\partial$, the contact form on $\partial \Sigma$.
\end{lemma}
\begin{proof}
    Inside $\mathcal{U}$, we compute
    \begin{align*}
        \tilde\alpha(R) &= h_1(r)\beta_\partial(R) + h_2(r)\d p(R) + \rho(r)(\cos(p) \d \theta_1 - \sin(p) \d \theta_2)(R)\\
        \intertext{$\beta_\partial$ and $\d p$ are 0 on $\partial_{\theta_i}$, $p$ and $\theta_i$ are constant on $R_B$}
        &= h_1\mu\beta_\partial(R_B) + \rho\nu(\cos(p) \d \theta_1 - \sin(p) \d \theta_2)(\cos(p)\partial_{\theta_1} - \sin(p)\partial_{\theta_2})\\
        &= h_1 \mu + \rho\nu[\cos(p)\cos(p) + \sin(p)\sin(p)]\\
        &= \frac{h_1\rho'}{\rho'h_1 - \rho h_1'} + \frac{- \rho h_1'}{\rho'h_1 - \rho h_1'}\\
        &= 1.
    \end{align*}
    Computing $\d \tilde\alpha$, we immediately drop the $\d r$-terms, as they evaluate to $0$ on $R$.
    \begin{align*}
        \d\tilde \alpha(R, \cdot) &= h_1(r)\d \beta_\partial(R, \cdot) - \rho(r)[\sin(p)\d p \wedge \d \theta_1 + \cos(p) \d p\wedge \d \theta_2](R, \cdot)
        \intertext{$\beta_\partial$ and $\d p$ are 0 on $\partial_{\theta_i}$, $\d \beta_\partial(R_B) = 0$, $p$ is 0 on $R_B$}
        &= \rho(r)\left[\sin(p)\d\theta_1(R)\d p + \cos(p)\theta_2(R)\d p\right]
        \intertext{$\d \theta_i$ is 0 on $R_B$}
        &= \rho(r)\nu(r)\left[\sin(p)\cos(p) - \cos(p)\sin(p)\right]\d p\\
        &= 0
    \end{align*}
    Outside $\mathcal{U}$, we have
    \[
        R = \cos(p)\partial_{\theta_1} - \sin(p)\partial_{\theta_2}.  
    \]
    Therefore,
    \begin{align*}
        \tilde\alpha(R) &= \beta(R) + \d p(R) + \cos(p) \d\theta_1(R) - \sin(p) \d \theta_2(R)\\
        \intertext{$\beta$ and $\d p$ are 0 on $\partial_{\theta_i}$}
        &= [\cos(p) \d \theta_1 - \sin(p) \d \theta_2](\cos(p)\partial_{\theta_1} - \sin(p)\partial_{\theta_2})\\
        &= \cos(p)\cos(p) + \sin(p)\sin(p)\\
        &= 1
    \end{align*}
    Also,
    \begin{align*}
        \d \tilde\alpha(R) &= \d \beta(R, \cdot) - \sin(p)\d p \wedge \d\theta_1(R, \cdot) - \cos(p)\d p \wedge \d \theta_2(R, \cdot)\\
        \intertext{$\beta$ and $\d p$ are 0 on $\partial_{\theta_i}$}
        &= \sin(p)\d \theta_1(R) \d p + \cos(p)\d\theta_2(R)\d p\\
        &= \sin(p)\cos(p) \d p - \cos(p)\sin(p)\d p\\
        &= 0
    \end{align*}
\end{proof}
A computation in \cite{Bourgeois02} \underline{(look up the computation)}
shows that, in order for $\tilde \alpha$ to be a contact form in the neighborhood $\mathcal{U}$ of $B \times T^2$ where 
$\alpha = h_1(r) \beta_\partial + h_2(r) \d p$, it is in fact enough that $h_1(h_1 - h_1') > 0$, a condition which only depends on $h_1$.

We need that $(\d \alpha)^n$ is a positive volume form on the pages and
$\alpha \wedge (\d \alpha)^{n-1}$ must be a positive volume form on $B$.
We have seen that
\[
    \d \alpha = h_1'(r)\d r\wedge \beta_\partial + h_1(r) \d \beta_\partial + h_2'(r) \d r \wedge \d\varphi.
\]
and 
\[
    (\d \alpha_\text{ext})^n = n\cdot \d r\wedge (h_1'(r) \beta_\partial + h_2'(r) \d\varphi) \cdot h_1(r)^{n-1}(\d \beta_\partial)^{n-1}
\]
In order for the latter to be a positive volume form on the pages, the $\phi$ coordinate doesn't matter. As a result, it is enough that $h_1'(r) \cdot h_1(r)^{n-1}$ is positive.
Further, we compute
\[
    (\d \alpha_\text{ext})^{n-1} = .
\]

In particular, one can homotope the pair $(h_1 , h_2)$ in a compactly supported way \underline{Why compactly supported?} among pairs of functions satisfying this condition, 
and this will result in a homotopy of contact forms (hence isotopy by Gray's stability) on $M \times T^2$, 
independently of the fact that the resulting $\alpha$ on M might not be adapted to the open book or even a contact form. 
Moreover, the explicit formula in the last lemma still holds for the homotoped contact form, as the explicit computation does not use any specific property of the pair of functions $(h_1, h_2)$ listed above.
In particular, up to homotopy one can achieve the following form, which
will be useful below: \underline{Where?}
\begin{lemma}
    For $\delta > 0$ sufficiently small, up to a deformation among contact structures on $M\times T^2$ supported in the neighbourhood of radius $2\delta$ of $B \times T^2$ , one can assume that
    \begin{itemize}
        \item $h_1(r) = 1$ for $r \le \delta$
        \item $h_1(h_1 - h_1') > 0$ everywhere
        \item $h_2(r) = 0$ for $r \le 3\delta/2$ 
    \end{itemize}
    \underline{weird diffeo with cotangent bundle here}
    Our description of the Reeb vector field also holds for the deformed contact form; in particular, it coincides with $R_B$ for $r\le \delta$.
\end{lemma}