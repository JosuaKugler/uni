\begin{definition}
    Let $(\Sigma, j)$ be a Riemann surface and $(M, J)$ an almost-complex manifold. Then a smooth map
    \[
        u \colon \Sigma \to M
    \]
    is called $J$-holomorphic (or pseudoholomorphic) if its differential at every point is complex-linear, i.e.
    \[
        Tu \circ j = J \circ Tu.
    \]
\end{definition}

Pseudoholomorphic curves are used because holomorphic curves (where $M$ has to be a complex manifold) are to specific: 
A lot of manifolds don't even have a complex structure, whereas all symplectic manifolds admit an almost-complex structure.

%section 2.2.: Tameness and compatibility of almost complex structures on symplectic manifolds. Non-emptiness and contractibility of the total space
%
%4: We investigate the moduli space of holomorphic curves. The point is that parametrization shouldn't matter to a holomorphic curve.
%Therefore, one takes the space of all curves modulo reparametrization diffeos.
%Then, one tries to understand the resulting spaces. They are in some sense finite-dimensional (if Fredholm-regularity holds).
%This can be ensured in various ways. The goal, however, is always to understand the structure of these moduli spaces

