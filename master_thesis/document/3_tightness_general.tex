In general, for a contact structure to be tight means that it's \textit{not overtwisted}, i.e. in dimension 3 it doesn't contain an overtwisted disk.
In higher dimensions, the notion of overtwistedness is a little more complicated. In the important paper \cite{CMP19}, the authors prove the equivalence 
of many criteria for overtwistedness and thereby provide a good understanding. In any case, it seems easier to prove that a manifold is overtwisted than
to prove that it is not.
Usually, tightness is proved via holomorphic curves. Gromov and Eliashberg were the first to use that technique \cite{Gromov85,Eliashberg91}
when they proved that fillable manifolds are tight. This has since then often been used as a criterion to show tightness.
Recent approaches like contact homology are also based on holomorphic curves techniques.

\underline{Maybe I can write something on holomorphic curves here?}