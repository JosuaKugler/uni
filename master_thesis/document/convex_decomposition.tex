\section*{Convex decomposition}

Start with a Bourgeois manifold $\operatorname{BO}(V,\dots)$.
Smoothly, this is $\operatorname{OB}(\Sigma, \dots) \times T^2$.
Viewing this as $(\operatorname{OB}(\Sigma, \dots) \times S^1) \times S^1$,
we obtain a decomposition $V_+ \times S^1 \bigcup \overline{V}_- \times S^1$,
where $V_\pm$ are ideal Liouville domains.
The details of the proof of 3.16 in DG12 reveal that $V_\pm = \Sigma \times D^*S^1$.
Also, $\operatorname{OB}(\Sigma, \dots) \times S^1 = V_+\bigcup \overline{V}_-$
is a convex decomposition (why is that necessary/interesting?).

Where does the decomposition come from? What is the effect of the blowdown?
The blowdown is applied to the boundary components of the Giroux domain.



\begin{itemize}
    \item $S^1$-invariant contact structure on $V^{2n} \times S^1$ induces a decomposition
    of the $V$-part into ideal Liouville domains $V = V_+ \cup V_-$. \underline{Details}
    \item The products $V_\pm \times S^1$ are Giroux domains, 
    as they are round contactizations of ideal Liouville domains.
    My guess would be that a contactization works similar to a symplectization:
    One takes a symplectic manifold $\times \mathbb R$ and it becomes a contact manifold.
    A round contactization takes a manifold $\times S^1$.
    \item In the Bourgeois case we have $V^{2n} = \operatorname{OB}(\Sigma) \times S^1$.
    Then, $V_\pm = \Sigma \times D^*S^1$. \underline{Where does the binding go?}
    Maybe into the boundary in the middle (= the convex hypersurface)?
\end{itemize}

