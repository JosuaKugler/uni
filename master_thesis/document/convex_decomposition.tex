\section*{Convex decomposition}

\begin{definition}[ideal Liouville domain] \cite[Definition 1]{Giroux20}
An ideal Liouville domain
$(F, \omega)$ is a domain $F$ endowed with an ideal Liouville structure $\omega$. This
ideal Liouville structure is an exact symplectic form on $\operatorname{int} F$
admitting a primitive $\lambda$ such that: for some (and then any) function 
$u \colon F \to \mathbb R_{\geq 0}$  with regular level set $\partial F = \{u = 0\}$,
the product $u\lambda$ extends to a smooth 1-form on $F$ which 
induces a contact form on $\partial F$.
\end{definition}

\begin{definition}[corresponding Giroux domain]\cite[Section 5.3]{MNW13}
    \label{def:giroux_domain}
    Given an ideal Liouville domain $(F, \omega)$ with primitive $\lambda$
    and function $u \colon F \to \mathbb R_{\geq 0}$ as above,
    the corresponding Giroux domain is given by
    \[
        F \times S^1_\theta
    \]
    endowed with contact structure
    \[
        \ker(u \d \theta + u \lambda)
    \]
\end{definition}

Start with a Bourgeois manifold $\operatorname{BO}(\Sigma,\dots)$.
Smoothly, we have
\[
    \operatorname{BO}(\Sigma,\dots) = \operatorname{OB}(\Sigma, \dots) \times T^2 
    = \big[\operatorname{OB}(\Sigma, \dots) \times S^1\big] \times S^1 
    \eqqcolon V \times S^1
\]
According to \cite[Section 6]{DG12} we obtain a convex decomposition of the first factor
\[
    V = V_+ \bigcup_\Gamma \overline{V}_-,
\]
where $V_\pm$ are ideal Liouville domains and $\Gamma$ is the dividing set.
In \cite[Section 5.3]{DG12}, the authors explicitly compute $\Gamma$ and 
$V_\pm$ for the Bourgeois construction and obtain %maybe make a picture here?
\[
    \Gamma = \{y = 0\} = p^{-1}(\{0\}) \cup_B p^{-1}(\{\pi\})
\]
and
\[
    V_+ = p^{-1}([0, \pi]) \times S^1, \qquad V_- = p^{-1}([\pi, 2\pi]) \times S^1,
\]
i.e. topologically we get $V_\pm = \Sigma \times D^*S_1$.
If $\alpha + x \d \phi + y \d \theta$ is the contact structure on $\operatorname{OB}(\Sigma, \dots)$,
then as explained in \cite[Section 5.3]{DG12}, $\alpha + x \d \phi$ is a 
$S^1_\phi$-invariant contact form on $\Gamma$,
\[
    \omega_\pm = \pm \d \left(\frac{\alpha}{y} + \frac{x}{y}\d \phi\right) 
\]
is an $S^1_\phi$-invariant symplectic form on $V_\pm$ and
$y$ is a function with zero level set $\pm \Gamma = \partial V_\pm$.
Hence, $(V_\pm, \omega_\pm)$ is an ideal Liouville domain with Liouville form
\[
    \beta_\pm = \pm \left(\frac{\alpha}{y} + \frac{x}{y}\d \phi\right).
\]
According to \cref{def:giroux_domain}, $V_\pm \times S^1_\theta$
endowed with the contact structure 
\[
    \ker(y \d \theta + y \beta_\pm) = \alpha + x \d \phi + y \d \theta
\]
is the corresponding Giroux domain. Clearly, this is just the restriction
of the open book contact structure. Hence, the whole procedure actually yields a
splitting into two Giroux domains
\[
    \operatorname{OB}(\Sigma, \dots) = V_+ \times S^1_\theta 
    \bigcup_{\Gamma \times S^1_\theta} V_- \times S^1_\theta
\]

\subsection*{Surgery along embedded Giroux domains}
Given a embedded Giroux domain, this section describes a procedure
to remove its interior and "blow down" the resulting boundary.
We will refer to the procedure as "clean cut-out" of the Giroux domain.

These boundary components are always of the form $B = M \times S^1$.
Topologically, blowing down is equivalent to simply gluing in $M \times D^2$.

This operation can be performed in a way that respects the contact structure,
provided that $S^1 \times M$ has a neighborhood
of the form $[0, \epsilon)_s \times S^1_t \times M$ where $\alpha_M + s \d t$
defines a contact form.
In general this holds by \cite[Lemma 5.1]{MNW13} if the boundary components 
are $\xi$-round hypersurfaces, but it is also possible to show the existence 
of that neighborhood directly.

Let $D$ be the disk of radius $\epsilon$ in $\mathbb R^2$. The map 
\[
    \Psi \colon (re^{i\theta} , m) \mapsto (r^2, \theta, m)
\]
is a diffeomorphism from $(D \setminus \{0\}) \times M$ to 
$(0, \epsilon) \times S^1 \times M$ which pulls back $\alpha_B + s \d t$ 
to the contact form $\alpha_M + r^2 \d\theta$. 
%check pullback!
Provided that there is such a neighborhood of $M \times S^1$ as described above, 
we can glue $D \times M$ to $V \setminus B$ to get a new contact manifold 
in which $B$ has been replaced by $M$.
%add image here!

Boundary components of Giroux domains are 
$\xi$-round hypersurfaces (\cite[Section 5.3]{MNW13}),
Therefore, after removal of a Giroux domain, we can blow down
its boundary components.
These two steps together form the clean cut-out.

In \cite[Section 6]{MNW13}, where the clean cut-out is first introduced,
it is shown that it corresponds to a symplectic cobordism.
The setting there is actually more general: The authors consider a Giroux
domain where already some of the boundary components have been blown down.
In our case, the situation is simpler: The Giroux domain 
$V_- \times S^1 \subset \operatorname{BO}(\Sigma, \dots)$
is directly obtained from the corresponding ideal Liouville domain $V_-$ by 
round contactization.
Its boundary is given by 
\[
    \partial V_- \times S^1 = \Gamma \times S^1.
\]
We want a cobordism from $\operatorname{BO}(\Sigma, \dots)$ to
the same manifold after clean cut-out of $V_- \times S^1$.
Topologically, the cobordism
\[
    W \coloneqq [0,1] \times \operatorname{BO}(\Sigma, \dots) \bigcup_{\{1\} \times \left(V_- \times S^1\right)} V_- \times D^2.
\]
fulfills the requirements: One boundary is simply 
$\operatorname{BO}(\Sigma, \dots)$.
The other boundary is 
\[
    \left[\operatorname{BO}(\Sigma, \dots) \setminus \left(V_- \times S^1\right)\right] 
    \cup_{\Gamma \times S^1} \Gamma \times D^2,
\]
which is homeomorphic to 
$\operatorname{BO}(\Sigma, \dots) \setminus \left(V_- \times S^1\right)$
with blown down boundary, which is exactly the desired clean cut-out.
Using the more general \cite[Theorem 6.1]{MNW13}, one can realize 
$W$ as a strong symplectic cobordism.

\underline{Why is it strong? What would be the Liouville vector field?}
Maybe the reasoning is as follows: Start with the symplectization
of $\operatorname{BO}(\Sigma, \dots)$. This clearly is a strong cobordism.
Maybe the handle attachment can be done in a strong way?

Next, we compute the contact structure on $\operatorname{BO}(\Sigma, \dots) \setminus \left(V_- \times S^1\right)$
with blown down boundary $\partial V_- \times S^1 = \Gamma \times S^1$.

This should be the contact boundary of $V_+ \times D^2$.
Topologically of course that all makes sense,
\[
    \partial (V_+ \times D^2) = V_+ \times S^1 \cup \partial V_+ \times D^2 = V_+ \times S^1 \cup_{\Gamma \times S^1} \Gamma \times D^2.
\]
However, it is unclear to me why $V_+ \times D^2$ should carry a \underline{symplectic structure}.

It should also be supported by the open book $\operatorname{OB}(V_+, \operatorname{id})$,
which, again, topologically, makes perfect sense, but I don't understand the
\underline{contact side of things}.

\subsection*{Subcritical surgery and blowdown in one big symplectic cobordism}
Now we apply subcritical surgery in the complement of $\overline{V}_- \times S^1$ and
remove the Giroux domain $V_- \times S^1$.


% How is this cobordism (2.13) used in the rest of the paper?