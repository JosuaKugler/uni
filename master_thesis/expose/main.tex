\documentclass{amsart}
\usepackage{amsmath}
\usepackage{amsthm}
\usepackage{amssymb}
\usepackage{mathtools}
\usepackage{hyperref}
\usepackage{cleveref}
\usepackage{enumerate}
\newtheorem{theorem}{Theorem}
\newtheorem{lemma}{Lemma}
\newtheorem{remark}{Remark}
\newtheorem{definition}{Definition}
\newtheorem{question}{Question}
\newtheorem{fact}{Fact}
\renewcommand*{\d}{\mathrm{d}}



\begin{document}
\title{Classifying Contact Structures on the Sphere}
\address{}
\date{}
\keywords{Contact geometry, symplectic geometry, open books, h-principle, }

\maketitle
\section*{Introduction}
Contact geometry is the study of odd-dimensional smooth manifolds equipped with contact structures, i.e.\ hyperplane distributions $\xi = \ker \alpha$ satisfying the contact condition
\[
    \alpha \wedge (\d \alpha)^n \neq 0.
\]
While they originally arise in the study of ODEs and in classical mechanics, the topological study of contact manifolds is a more recent and very active field of research.

A manifold can have multiple different contact structures, which can be either rigid (in which case one speaks of a "tight" manifold) or flexible (in the sense that they satisfy an h-principle). The latter contact manifolds are then called overtwisted. A foundational result of Eliashberg \cite{Eliashberg89} and Borman--Eliashberg--Murphy \cite{BEM15}, roughly speaking, states that overtwisted contact manifolds exist in abundance, namely whenever the manifold admits the topological version of a contact structure (an \emph{almost} contact structure), which is a first obvious obstruction. In dimension three, an almost contact structure is simply an oriented $2$-plane field.

To illustrate this dichotomy, consider the sphere $S^3$. By a result of Eliashberg \cite{Eliashberg92}, it has precisely one tight contact structure. On the other hand, it has infinitely many overtwisted contact structures, corresponding to the infinitely many homotopy classes of 2-plane fields on the 3-sphere. There are other examples where there are infinitely many or no tight contact structures on a contact manifold.
For example, Ustilovsky \cite{Ustilovsky99} showed that, for $n \geq 2$, the sphere 
$S^{2n+1}$ admits infinitely many non-isomorphic tight contact structures in each almost 
contact class (Uebele \cite{Uebele16} extended this result to $S^7, S^{11}$ and $S^{15}$)

A further interesting property of contact manifolds comes from the fact that contact geometry is the odd-dimensional counterpart to symplectic geometry. Often, it is possible to view a contact manifold as the boundary of a symplectic manifold. Manifolds that are in this sense "fillable" are always tight \cite{Gromov85,Eliashberg91}.
The contrary, however, doesn't need to hold and one can ask the question under which conditions such tight, but non-fillable manifolds exist. The first examples of tight and non-fillable contact manifolds were constructed by Etnyre--Honda \cite{EH02} in dimension three, and by Massot--Niederkrueger--Wendl \cite{MNW13} in higher dimensions.


In the following, I will start with a brief overview of the results and methods of my Master's thesis (for more details see the summary of the thesis).
Then, I give a brief overview of the current research status in the area and
finally state and explain some open questions that I plan to work on in the dissertation.

\section*{Open books and the Giroux correspondence}
\begin{definition}
    An open book decomposition of a manifold $M$ is a pair $(B, p)$, together with a defining map $\Phi:M \to \mathbb R^2$ so that each $z \in \operatorname{int}(D^2)$ is a regular value.
    Here, $B \subset M$ is a closed codimension-2 submanifold, $p: M\setminus B \to S^1$ is a fiber bundle, and $\Phi$ is such that $\Phi^{-1}(0) = B$ and $p =  \Phi / |\Phi|$.
\end{definition}
We will typically assume that the pages are Weinstein, i.e. there exists a Morse-function on the page that is gradient-like for the radial direction.


\begin{figure}
    \includegraphics*[width=\textwidth]{../document/images/abstract_open_book.pdf}
    \caption{An abstract open book in $\dim = 3$}
    \label{fig:abstract_open_book}
\end{figure}

Also, we will often use the notion of an abstract open book $(\Sigma, \phi)$
where $\Sigma$ denotes the page and $\phi: \Sigma \to \Sigma$ the monodromy.
Then, by taking the binding to be $\partial \Sigma$ we can glue together the 
mapping torus $\Sigma \times [0,1] / (x, 0) \sim (\phi(x), 1)$
with $\partial \Sigma \times D^2$ (see \cref{fig:abstract_open_book})

Positive (resp. negative) \textit{stabilization of open books} is an important procedure where
one attaches a handle to a page and then composes the monodromy with 
a positive (resp. negative) Dehn-Seidel twist along the \underline{co-core} of the handle.

\begin{fact}\cite{Giroux02}
There is a one to one correspondence between oriented contact structures up to isotopy
and open book decompositions up to positive stabilization.
\end{fact}

One direction of this correspondence can (partially) be established by means of the 
Thurston-Winkelnkemper construction.
We need a few additional assumptions on the open book (see \cite[Theorem 7.3.3]{Geiges08}), but then following the construction in Geiges we can construct a contact form first
on the mapping torus and the binding. Finally we can glue it together on the $\partial \Sigma \times D^2$-part.

\section*{Tight non-fillable contact structures on the sphere}

Recently, Bowden--Gironella--Moreno--Zhou \cite{BGMZ22} have proved the existence of homotopically standard, non-fillable but tight contact structures on all spheres $S^{2n+1}$ with $n >= 2$. 
One fundamental ingredient of the work is the Giroux correspondence \cite{Giroux02}.
Starting with a specific open book decomposition of $S^{2n-1}$, one can construct a contact form on this manifold using the construction by Thurston--Winkelnkemper. 
Then, according to Bourgeois, this contact structure can be extended to a contact structure on $S^{2n-1}\times T^2$ that is $T^2$-invariant.
More recent results by \cite{BGM22} and \cite{AZ24} show that this contact structure is tight.
Applying subcritical surgery (preserving the tightness), one can kill the topology of the $T^2$-factor and obtain a tight contact structure on $S^{2n+1}$.
The symplectic fillability fails for homological reasons as can be shown via holomorphic curves.


\section*{Fillability hierarchy}
As explained above, the notion of fillability is crucial in understanding tight contact structures. However, there are actually a lot of different notions of fillability in symplectic geometry.

\begin{definition}[weak and strong filling]
    Let $(M, \xi)$ be a contact manifold and $(W, \omega)$ a symplectic manifold s.t. $\partial W = M$ (as oriented manifolds).
    \begin{itemize}
        \item $(W, \omega)$ is a weak filling of $(M, \xi)$ iff $\omega|_\xi > 0$
        \item $(W, \omega)$ is a strong filling of $(M, \xi)$ iff $\omega = \d \alpha$ near the boundary.
    \end{itemize}
\end{definition}


\begin{definition}[Liouville filling]
    A Liouville cobordism is a compact oriented symplectic manifold where the globally defined Liouville vector field $X$ is negatively transverse  along $\partial W_-$ (concave side) and positively transverse along $\partial W_+$ (convex side).
    If $\partial W_- = \emptyset$, we call it a Liouville domain.
    A Liouville filling of a contact manifold $(M, \xi = ker \alpha)$ is a Liouville domain $(W, \lambda)$ such that $(\partial W, ker \lambda|\partial W)$ is contactomorphic to $(M, ker \alpha)$.
\end{definition}

\begin{definition}[Weinstein filling]
    A Weinstein filling is a Liouville filling such that the Liouville vector field $X$ is gradient-like for a Morse function which is locally constant on the boundary.
\end{definition}

In general, we have the following sequence of strict inclusions (we abbreviate e.g. "Weinstein" for "Weinstein fillable contact manifold")
\[
      \{\text{Weinstein}\} \overset{1}{\subsetneq} \{\text{Liouville}\} \overset{2}{\subsetneq} \{\text{strong}\} \overset{3}{\subsetneq} \{\text{weak}\} \overset{4}{\subsetneq} \{\text{tight}\}
\]
In dimension $\ge 5$, 1 was proved by\cite[Theorem 1.5]{BCS14}, 2 by \cite{Zhou21}, 3 by \cite{BGM22} and 4 by \cite{MNW13} (in dimension 3, they have been known for a while).
All of the above examples where such an inclusion is strict rely on special geometric
constructions. 
Now, due to \cite{BGMZ22}, the last inclusion is known to be strict for all contact manifolds (of course only if there are tight contact structures at all).
In the same spirit, they proof that for any $n \geq 3$, there exist homotopically standard 
Liouville fillable contact structures on $S^{2n+1}$ that are not Weinstein fillable.
For $S^3$, this is not the case as the only homotopically standard contact structure is
Stein fillable, hence Weinstein fillable \underline{cite Cieliebak-Eliashberg}.
The only remaining case, therefore, is $S^5$.

As strong and weak fillability are equivalent on the sphere \underline{cite someone},
the next question would be to ask if there are strongly fillable contact structures on the sphere that are not Liouville fillable.



\section*{Open questions}
\begin{question}
    For being able to further classify contact structures on the sphere, it is certainly interesting to better understand the underlying almost contact structures.
    Using the fact that $S^{2n+1}$ is stably parallelizable, we can show that an almost contact structure on $S^{2n+1}$ corresponds to a map that associates an almost complex structure on $R^{2n+2}$
    to every point of the sphere.
    Almost complex structures on $R^{2n+2}$ are given by $\operatorname{SO}(2n + 2)/\operatorname{U}(n + 1)$, so we have the following correspondence
    \[
        \operatorname{Acont}\left(S^{2n+1}\right) = \pi_{2n+1}(\operatorname{SO}(2n + 2)/\operatorname{U}(n + 1)).
    \]
    This fundamental group has been computed by Harris \cite{Harris63} and we obtain
    \[
        \operatorname{Acont}\left(S^{2n+1}\right) = \pi_{2n+1}(\operatorname{SO}(2n + 2)/\operatorname{U}(n + 1)) = \begin{cases}
            \mathbb Z_{n!} &n = 0 \operatorname{mod} 4\\
            \mathbb Z &n = 1 \operatorname{mod} 4\\
            \mathbb Z_{n!/2} &n = 2 \operatorname{mod} 4\\
            \mathbb Z \oplus \mathbb Z_2 &n = 3 \operatorname{mod} 4
        \end{cases}.
    \]
    It is possible to choose the first identification in such a way that $\xi_{\mathrm{std}}$ corresponds to $0$.
    Taking the negative stabilization of that open book, we obtain a contact structure $\xi_\mathrm{neg}$. Which element does $\xi_\mathrm{neg}$ correspond to?

\end{question}
\begin{question}    
Is there a Liouville but not Weinstein fillable contact structure on $S^5$? 
\end{question}
\begin{question}
Is there a strong but not Liouville fillable structure on $S^{2n+1}$, $n \ge 2$?
\end{question}
\newpage
\nocite{*}
\bibliographystyle{alpha}
\bibliography{../document/bibliography}
\end{document}