\documentclass{amsart}
\usepackage{amsmath}
\usepackage{amsthm}
\usepackage{amssymb}
\usepackage{mathtools}
\usepackage{hyperref}
\usepackage{cleveref}
\usepackage{enumerate}
\newtheorem{theorem}{Theorem}
\newtheorem{lemma}{Lemma}
\newtheorem{remark}{Remark}
\newtheorem{definition}{Definition}
\newtheorem{question}{Question}
\newtheorem{fact}{Fact}
\renewcommand*{\d}{\mathrm{d}}



\begin{document}
\title{Classifying Contact Structures on the Sphere}
\address{}
\date{}
\keywords{Contact geometry, symplectic geometry, open books, h-principle, }

\maketitle
\section*{Introduction}
Contact geometry is the study of odd-dimensional smooth manifolds equipped with contact structures, i.e.\ hyperplane distributions $\xi = \ker \alpha$ satisfying the contact condition
\[
    \alpha \wedge (\d \alpha)^n \neq 0.
\]
While they originally arise in the study of ODEs and in classical mechanics, the topological study of contact manifolds is a more recent and very active field of research.

A manifold can have multiple different contact structures, which can be either rigid (in which case one speaks of a "tight" manifold) or flexible (in the sense that they satisfy an h-principle). The latter contact manifolds are then called overtwisted. A foundational result of Eliashberg \cite{Eliashberg89} and Borman--Eliashberg--Murphy \cite{BEM15}, roughly speaking, states that overtwisted contact manifolds exist in abundance, namely whenever the manifold admits the topological version of a contact structure.
This topological version of a contact structure is called an \emph{almost} contact structure, and it is a first obvious obstruction to the existence of a (geometric) contact structure. 
In dimension three, an almost contact structure is simply an oriented $2$-plane field,
in general, it is a hyperplane distribution with an almost complex structure
and an oriented complementary line bundle.
Due to the topological nature of almost contact structures, they can be classified
by methods of algebraic topology and are hence comparably well-understood.

To illustrate the above-mentioned dichotomy between tight and overtwisted contact structures, consider the sphere $S^3$. By a result of Eliashberg \cite{Eliashberg92}, it has precisely one tight contact structure. On the other hand, it has infinitely many overtwisted contact structures, corresponding to the infinitely many homotopy classes of 2-plane fields on the 3-sphere. There are other examples where there are infinitely many or no tight contact structures on a contact manifold.
For example, Ustilovsky \cite{Ustilovsky99} showed that, for $n \geq 2$, the sphere 
$S^{2n+1}$ admits infinitely many non-isomorphic tight contact structures in each almost 
contact class (Uebele \cite{Uebele16} extended this result to $S^7, S^{11}$ and $S^{15}$)

A further interesting property of contact manifolds comes from the fact that contact geometry is the odd-dimensional counterpart to symplectic geometry. Often, it is possible to view a contact manifold as the boundary of a symplectic manifold. Manifolds that are in this sense "fillable" are always tight \cite{Gromov85,Eliashberg91}.
The contrary, however, doesn't need to hold and one can ask the question under which conditions such tight, but non-fillable manifolds exist. The first examples of tight and non-fillable contact manifolds were constructed by Etnyre--Honda \cite{EH02} in dimension three, and by Massot--Niederkrueger--Wendl \cite{MNW13} in higher dimensions.


In the following, I will give some of the necessary background.
Then, I give a brief overview of the current research status in the area, which leads
to an explanation of the results in \cite{BGMZ22}.
In particular, I sketch the proof of the part that I explain in my master's thesis. 
Finally, I state and explain some open questions that I plan to work on in the dissertation.

\section*{Open books and the Giroux correspondence}
\begin{definition}
    An open book decomposition of a manifold $M$ is a pair $(B, p)$, together with a defining map $\Phi:M \to \mathbb R^2$ so that each $z \in \operatorname{int}(D^2)$ is a regular value.
    Here, $B \subset M$ is a closed codimension-2 submanifold, $p: M\setminus B \to S^1$ is a fiber bundle, and $\Phi$ is such that $\Phi^{-1}(0) = B$ and $p =  \Phi / |\Phi|$.
\end{definition}
We will typically view a page as a Liouville domain that is Weinstein, i.e. the Liouville vector field is gradient-like for a Morse function on the page (\cite[Def 1.1.2]{BHH23}).


\begin{figure}
    \includegraphics*[width=\textwidth]{../document/images/abstract_open_book.pdf}
    \caption{An abstract open book in $\dim = 3$}
    \label{fig:abstract_open_book}
\end{figure}

Also, we will often use the notion of an abstract open book $(\Sigma, \phi)$
where $\Sigma$ denotes the page and $\phi: \Sigma \to \Sigma$ the monodromy.
Then, by taking the binding to be $\partial \Sigma$ we can glue together the 
mapping torus $\Sigma \times [0,1] / (x, 0) \sim (\phi(x), 1)$
with $\partial \Sigma \times D^2$ (see \cref{fig:abstract_open_book})

As a first step of describing the connection between open book decompositions and 
contact structures on a certain manifold, we define what it means for a contact structure to be
supported by an open book decomposition.
\begin{definition}\label{def:support}\cite{Geiges08}
    Let $(B,p)$ be an oriented open book decomposition of the oriented manifold $M$.
    A contact structure $\xi = \ker \alpha$ on $M$ is said to be \textbf{supported} by the open book decomposition $(B,p)$ of $M$ if
    \begin{enumerate}[(i)]
        \item the contact form $\alpha$ induces the positive orientation of $M$ ($\alpha \wedge (\d \alpha)^n > 0$).
        \item the 2-form $\d \alpha$ induces a symplectic form on each page, defining its positive orientation
        \item the 1-form $\alpha$ induces a positive contact form on $B$, i.e. 
        \[ 
            \alpha|_{TB} \wedge (\d \alpha|_{TB})^{(n-2)} > 0.
        \]
    \end{enumerate}
\end{definition}

A further ingredient of the Giroux correspondence is the following operation on open books.
\begin{definition}\cite{Koert17}
    Positive (resp. negative) \textit{stabilization of open books} is the procedure where
    one attaches a handle to a page $W$ along the boundary $\partial L$ of a Lagrangian disk $L\subset W$ and then composes the monodromy with
    a positive (resp. negative) Dehn-Seidel twist along the Lagrangian sphere formed by $L$ 
    and the core of the handle.
    This produces a new open book decomposition of the same manifold.
\end{definition}

\begin{fact}\cite{Giroux02}
For 3-manifolds, there is a one-to-one correspondence between oriented contact structures up to isotopy
and open book decompositions up to positive stabilization.
\end{fact}

Part of the Giroux correspondence in higher dimensions can be established by means 
of the generalized Thurston-Winkelnkemper construction.
We need a few additional assumptions on the open book (see \cite[Theorem 7.3.3]{Geiges08}), but then we can construct a contact form on the manifold that is supported by the open book.
Following the construction in Geiges, we first define it on the mapping torus and the binding, then we can glue it together on the $\partial \Sigma \times D^2$-part and finally check
the orientation conditions.

\section*{Fillability hierarchy}
As explained above, the notion of fillability is crucial in understanding tight contact structures. However, there are actually a lot of different notions of fillability in symplectic geometry.

\begin{definition}[weak and strong filling]
    Let $(M, \xi)$ be a contact manifold and $(W, \omega)$ a symplectic manifold s.t. $\partial W = M$ (as oriented manifolds).
    \begin{itemize}
        \item $(W, \omega)$ is a weak filling of $(M, \xi)$ iff $\omega|_\xi > 0$
        \item $(W, \omega)$ is a strong filling of $(M, \xi)$ iff $\omega = \d \alpha$ near the boundary.
    \end{itemize}
\end{definition}


\begin{definition}[Liouville filling]
    A Liouville cobordism is a compact oriented symplectic manifold where the globally defined Liouville vector field $X$ is negatively transverse  along $\partial W_-$ (concave side) and positively transverse along $\partial W_+$ (convex side).
    If $\partial W_- = \emptyset$, we call it a Liouville domain.
    A Liouville filling of a contact manifold $(M, \xi = ker \alpha)$ is a Liouville domain $(W, \lambda)$ such that $(\partial W, ker \lambda|\partial W)$ is contactomorphic to $(M, ker \alpha)$.
\end{definition}

\begin{definition}[Weinstein filling]
    A Weinstein filling is a Liouville filling such that the Liouville vector field $X$ is gradient-like for a Morse function which is locally constant on the boundary.
\end{definition}

We have the following sequence of strict inclusions (we abbreviate e.g. "Weinstein" for "Weinstein fillable contact manifold")
\[
      \{\text{Weinstein}\} \overset{1}{\subsetneq} \{\text{Liouville}\} \overset{2}{\subsetneq} \{\text{strong}\} \overset{3}{\subsetneq} \{\text{weak}\} \overset{4}{\subsetneq} \{\text{tight}\}
\]
In dimension $\ge 5$, 1 was proved by \cite[Theorem 1.5]{BCS14}, 2 by \cite{Zhou21}, 3 by \cite{BGM22} and 4 by \cite{MNW13}.



\section*{Tight non-fillable contact structures on the sphere}

All of the above examples where an inclusion is strict rely on special geometric
constructions, that is, special manifolds have been constructed where a specific
contact structure is fillable in one sense, but not in the other.
One might wonder if the strictness in these inclusions is hence limited
to specific manifolds. 
Recently, Bowden--Gironella--Moreno--Zhou \cite{BGMZ22} have given a negative answer to this question by proving the existence of homotopically standard, non-fillable but tight contact structures on all spheres $S^{2n+1}$ with $n >= 2$.
By taking the connected sum of a manifold with the sphere, it can be shown that the last inclusion (4) is known to be strict for all contact manifolds (of course only if there are tight contact structures at all).

In my master's thesis, I explain the proof of the existence of such tight non-fillable contact structures on the sphere.
Start with a specific open book decomposition of $S^{2n-1}$.
Then construct a contact form on this manifold using the construction by Thurston--Winkelnkemper. 
Next, according to Bourgeois, this contact structure can be extended to a contact structure on $S^{2n-1}\times T^2$ that is $T^2$-invariant.
Recent results by \cite{BGM22} and \cite{AZ24} show that this contact structure is tight.
Applying subcritical surgery (preserving the tightness), one can kill the topology of the $T^2$-factor and obtain a tight contact structure on $S^{2n+1}$.
This uses standard surgery techniques and the $h$-cobordism-theorem.
The symplectic fillability fails for homological reasons as can be shown via holomorphic curves. 

\section*{Open questions}
In order to further classify contact structures on the sphere, it is certainly interesting to gain understanding of the underlying almost contact structures.
Using the fact that $S^{2n+1}$ is stably parallelizable, we can show that an almost contact structure on $S^{2n+1}$ corresponds to a map that associates an almost complex structure on $R^{2n+2}$
to every point of the sphere.
Almost complex structures on $R^{2n+2}$ are given by $\operatorname{SO}(2n + 2)/\operatorname{U}(n + 1)$, so we have the following correspondence
\[
    \operatorname{Acont}\left(S^{2n+1}\right) = \pi_{2n+1}(\operatorname{SO}(2n + 2)/\operatorname{U}(n + 1)).
\]
This homotopy group has been computed by Harris \cite{Harris63} and we obtain
\[
    \operatorname{Acont}\left(S^{2n+1}\right) = \pi_{2n+1}(\operatorname{SO}(2n + 2)/\operatorname{U}(n + 1)) = \begin{cases}
        \mathbb Z_{n!} &n = 0 \operatorname{mod} 4\\
        \mathbb Z &n = 1 \operatorname{mod} 4\\
        \mathbb Z_{n!/2} &n = 2 \operatorname{mod} 4\\
        \mathbb Z \oplus \mathbb Z_2 &n = 3 \operatorname{mod} 4
    \end{cases}.
\]
It is possible to choose the first identification in such a way that $\xi_{\mathrm{std}}$ corresponds to $0$.
According to \cite{Giroux02}, positive stabilization results in a contactomorphic
contact structure. In particular, it does not change the almost contact class.
Taking the negative stabilization of that open book, we obtain a contact structure $\xi_\mathrm{neg}$ that might be different (i.e. not contactomorphic).
\begin{question}
     Which element of $\pi_{2n+1}](\operatorname{SO}(2n + 2)/\operatorname{U}(n + 1))$ does $\xi_\mathrm{neg}$ correspond to?
\end{question}

For the next question, we consider another result by \cite{BGMZ22}: For any $n \geq 3$, there exist homotopically standard Liouville fillable contact structures on $S^{2n+1}$ that are not Weinstein fillable.
For $S^3$, this is not the case as the only homotopically standard contact structure is
Stein fillable, hence Weinstein fillable \cite{CE12}.
As a result, the only remaining case in inclusion 1 is $S^5$.
\begin{question}
Is there a Liouville but not Weinstein fillable contact structure on $S^5$? 
\end{question}
Finally, as strong and weak fillability are equivalent on the sphere,
the last question of this type is whether there are strongly fillable contact structures on the sphere that are not Liouville fillable.
\begin{question}
    Is there a strong but not Liouville fillable structure on $S^{2n+1}$, $n \ge 2$?
\end{question}
\newpage
\nocite{*}
\bibliographystyle{alpha}
\bibliography{../document/bibliography}
\end{document}