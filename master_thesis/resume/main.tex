\documentclass{amsart}
\usepackage{amsmath}
\usepackage{amsthm}
\usepackage{amssymb}
\usepackage{mathtools}
\usepackage{hyperref}
\usepackage{cleveref}
\usepackage{enumerate}
\newtheorem{theorem}{Theorem}
\newtheorem{lemma}{Lemma}
\newtheorem{remark}{Remark}
\newtheorem{definition}{Definition}
\renewcommand*{\d}{\mathrm{d}}



\begin{document}
\title[Homotopically Standard Tight Non-fillable Contact Structures]{Homotopically Standard Tight Non-fillable Contact Structures on the Sphere}
\maketitle
\section*{Introduction and Results}
Contact topology is the study of contact manifolds up to isotopy. It can be viewed as the
odd-dimensional counterpart to symplectic topology.
In fact, many contact manifolds are fillable, i.e. they are the (contact) boundary of a 
corresponding symplectic  manifold. This is an important tool in constructing and 
classifying contact structures.
For the general question of existence and classification, the dichotomy between tight and 
overtwisted contact manifolds is essential \cite{Eliashberg89,BEM15}.
It turns out that overtwisted contact structures are very much topological objects:
The topological obstruction for the existence of a contact structure is the (non-)existence of
an almost contact structure (which can be described as reduction of the structure group to 
$\mathrm U(n) \times \mathrm{id}$). 
In any such almost contact class there is a overtwisted contact structure.

\underline{Are overtwisted contact structures up to isotopy the same as 
almost contact structures up to homotopy?}

However, it is much harder to understand tight manifolds.
Thanks to groundbreaking work by Gromov and Eliashberg \cite{Gromov85,Eliashberg91} 
We know that fillable contact manifolds are tight. The standard contact structure 
$\xi_\mathrm{std}$ on $S^{2n+1}$ is just the contact boundary of $B^{2n+2}, \omega_\mathrm{std}$.
On $S^3$, this is the unique tight contact structure, so there are no tight but non-fillable 
contact structures on $S^3$. However, this is not true in general:
The first examples of non-fillable tight contact structures were provided by
Etnyre-Honda ($\dim = 3$, \cite{EH02}) and Massot-Niederkrueger-Wendl ($\dim \ge 5$, \cite{MNW13}).
More recently, Bowden--Gironella--Moreno--Zhou \cite{BGMZ22} have shown that for any
tight manifold in $\dim \ge 7$, there exists a tight non-fillable contact structure in
the same almost contact class (for $\dim = 5$ if the first Chern class vanishes). 
The first step towards this result is to construct a tight non-fillable contact structure
on $S^{2n+1}$ that is homotopically standard, i.e. in the same almost contact class
as the standard contact structure $\xi_\mathrm{std}$.
The goal of my Master's thesis is to give a streamlined explanation of the proof for that.

\section*{Sketch of proof}
The proof consists of mainly 3 steps:
\begin{itemize}
    \item Construction of the contact structure $\xi$ on $S^{2n+1}$
    \item Establishing tightness of $\xi$
    \item Showing that $\xi$ is not symplectically fillable
\end{itemize}
\subsection*{Construction}
According to \cite{Giroux02}, there is a correspondence between open books up to positive
stabilization and contact structures that are supported by the open book, up to isotopy.
For concreteness, we start with a very specific open book, the Milnor open book
\[
    B = .  
\]
(Other choices would work, too).
By the Thurston-Winkelnkemper construction, we explicitly construct a contact form supported
this open book decomposition of $S^{2n-1}$.
By a construction due to Bourgeois \cite{Bourgeois02}, we use this contact form to 
build a contact form on $S^{2n-1} \times T^2$.
Next, 
- subcritical contact surgery

\subsection*{Tightness of $\xi$}

\subsection*{$\xi$ is not symplectically fillable}


\newpage
\bibliographystyle{alpha}
\bibliography{../document/bibliography}
\end{document}