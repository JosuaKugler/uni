\documentclass{article}
\usepackage{amsmath}
\usepackage{amsthm}
\usepackage{amssymb}
\usepackage{mathtools}
\usepackage{cleveref}
\usepackage{enumerate}
\newtheorem{theorem}{Theorem}
\newtheorem{lemma}{Lemma}
\newtheorem{remark}{Remark}
\newtheorem{definition}{Definition}
\renewcommand*{\d}{\mathrm{d}}


\title{Tight and non-Fillable Contact Structures on the Sphere}
\date{\vspace*{-2cm}}
\begin{document}

\maketitle
\noindent
Contact topology is the study of contact manifolds up to isotopy. 
Contact structures are divided into two classes: Overtwisted (flexible) and tight (rigid) contact structures.
While the former are fully classified by their topological properties \cite{Eliashberg89,BEM15}, the latter turn out to be very interesting. 
A first step in understanding tight contact structures was to show that fillable contact structures are tight \cite{Gromov85,Eliashberg91}. 
However, there are several known contact manifolds where the contrary is wrong \cite{EH02,MNW13}. 
In the talk, I explain recent results by Bowden--Gironella--Moreno--Zhou \cite{BGMZ22} showing that any contact manifold that admits a tight contact structure also admits a non-strongly-fillable tight contact structure in the same almost contact class.
\bibliographystyle{alpha}
\bibliography{../document/bibliography}
\end{document}