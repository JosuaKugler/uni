\documentclass{beamer}
\usepackage{marvosym}
\usepackage{amsfonts,amsmath,amssymb,amsthm, mathtools}
\usepackage[ngerman]{babel}
\usepackage[T1]{fontenc}
\usepackage[utf8]{inputenc}
\usepackage[german]{alg}
\usepackage{tikz, pgf}
\usepackage[font=scriptsize,labelfont=bf]{caption}
\usepackage{verbatim}
\usepackage{comment}
\usetikzlibrary{arrows,shapes}

\pgfdeclarelayer{background}
\pgfsetlayers{background,main}

\theoremstyle{definition}
\newtheorem{korollar}{Korollar}

\mode<presentation>
{
  \usetheme{Ilmenau}
  \usecolortheme{beaver}
  \usefonttheme{professionalfonts}
%  \setbeamercolor{palette primary}{fg=white,bg=black}
  \setbeamercovered{transparent=0}
  \setbeamertemplate{navigation symbols}{}%remove navigation symbols
}
\author{Nico Haaf und Josua Kugler}
\title{Minimal auf\textbf{spannende} Bäume}
\date{19.05.20}
\begin{document}
\tikzstyle{vertex}=[circle,fill=black!25,minimum size=20pt,inner sep=0pt, draw = none]
\tikzstyle{selected vertex} = [vertex, fill=red!24, draw = none]
\tikzstyle{edge} = [draw,thick,-]
\tikzstyle{selected edge} = [draw,-,red!50,line width=5pt]
\graphicspath{{images/}}
  \maketitle
  \begin{frame}
    \frametitle{Motivation}
    \only<1,2,4,6> {
    \begin{itemize}
    		\item<1-> Konstruktion verbundener Netzwerke: 	\begin{itemize}
    															\item<2-> 	Telekommunikation
    															\item<4-> 	Wasser- und Elektrizitätsversorgung
    														\end{itemize}
    		
    \end{itemize} }
    
   	
	
	
	\only<3> 	{
					\begin{figure}
						\includegraphics[scale=0.2]{pictures/Example-Internet-backbone-network-topology-in-Europe-6}
						\caption{[Abb4] Haupt-Internetverbindungen des Europäischen Netzwerks}
					\end{figure}
				}
				
	\only<5>		{
					\begin{figure}
						\includegraphics[scale=0.42]{pictures/Map-of-the-water-supply-network-diameter-of-pipes}
						\caption{[Abb5] Wasserversorgungsnetzwerk in einer Stadt}
					\end{figure}
				}
	
	
	
	
	
	
	
    
\end{frame}
\section*{Grundlagen}
  \input{frames/frame1}
  \input{frames/frame2}
  \input{frames/frame3}
  \input{frames/frame4_1}
  \input{frames/frame4_2}
  \input{frames/frame4_3}
  \input{frames/frame_cayley}
  \begin{frame}
\only<1>{
$(a_1, \dots, a_{n-2}) = (2,2,7,7,1,5,3,1,1,4,4,5)$\\$(b_1, \dots, b_i) = (6)$
\begin{tikzpicture}[>=latex,line join=bevel,]
%%
\begin{scope}[scale=0.5]
  \node (2) at (27.0bp,90.0bp) [draw,ellipse,vertex] {$2$};
  \node (6) at (27.0bp,18.0bp) [draw,ellipse,vertex] {$6$};
\end{scope}
  \draw [edge] (2) -- (6);
%
\end{tikzpicture}


\hfill}
\only<2>{
$(a_1, \dots, a_{n-2}) = (2,2,7,7,1,5,3,1,1,4,4,5)$\\$(b_1, \dots, b_i) = (6,8)$
\begin{tikzpicture}[>=latex,line join=bevel,scale=1.0, auto]
%%
\node (a) at (91.0bp,162.0bp) [draw,ellipse,vertex] {$a$};
  \node (b) at (55.0bp,90.0bp) [draw=red,ellipse,selected vertex] {$b$};
  \node (c) at (27.0bp,18.0bp) [draw,ellipse,vertex] {$c$};
  \node (d) at (127.0bp,90.0bp) [draw,ellipse,vertex] {$d$};
  \node (e) at (154.0bp,18.0bp) [draw=red,ellipse,selected vertex] {$e$};
  \draw [black,edge] (a) ..controls (76.625bp,133.25bp) and (69.278bp,118.56bp)  .. node {$$} (b);
  \draw [black,edge] (a) ..controls (51.186bp,143.1bp) and (29.202bp,128.4bp)  .. (19.0bp,108.0bp) .. controls (7.495bp,84.99bp) and (14.014bp,54.525bp)  .. node {$$} (c);
  \draw [black,edge] (a) ..controls (105.37bp,133.25bp) and (112.72bp,118.56bp)  .. node {$$} (d);
  \draw [black,edge] (a) ..controls (130.81bp,143.1bp) and (152.8bp,128.4bp)  .. (163.0bp,108.0bp) .. controls (174.52bp,84.966bp) and (167.58bp,54.508bp)  .. node {$$} (e);
  \draw [black,edge] (b) ..controls (43.852bp,61.334bp) and (38.189bp,46.773bp)  .. node {$$} (c);
  \draw [red,edge] (b) ..controls (91.321bp,63.585bp) and (117.79bp,44.334bp)  .. node {$$} (e);
%
\end{tikzpicture}


\hfill}
\only<3>{
$(a_1, \dots, a_{n-2}) = (2,2,7,7,1,5,3,1,1,4,4,5)$\\$(b_1, \dots, b_i) = (6,8,2)$\input{cayley_animation/graph2.tex}
\hfill}
\only<4>{
$(a_1, \dots, a_{n-2}) = (2,2,7,7,1,5,3,1,1,4,4,5)$\\$(b_1, \dots, b_i) = (6,8,2,9)$\input{cayley_animation/graph3.tex}
\hfill}
\only<5>{
$(a_1, \dots, a_{n-2}) = (2,2,7,7,1,5,3,1,1,4,4,5)$\\$(b_1, \dots, b_i) = (6,8,2,9,7)$\input{cayley_animation/graph4.tex}
\hfill}
\only<6>{
$(a_1, \dots, a_{n-2}) = (2,2,7,7,1,5,3,1,1,4,4,5)$\\$(b_1, \dots, b_i) = (6,8,2,9,7,10)$\input{cayley_animation/graph5.tex}
\hfill}
\only<7>{
$(a_1, \dots, a_{n-2}) = (2,2,7,7,1,5,3,1,1,4,4,5)$\\$(b_1, \dots, b_i) = (6,8,2,9,7,10,11)$\input{cayley_animation/graph6.tex}
\hfill}
\only<8>{
$(a_1, \dots, a_{n-2}) = (2,2,7,7,1,5,3,1,1,4,4,5)$\\$(b_1, \dots, b_i) = (6,8,2,9,7,10,11,3)$\input{cayley_animation/graph7.tex}
\hfill}
\only<9>{
$(a_1, \dots, a_{n-2}) = (2,2,7,7,1,5,3,1,1,4,4,5)$\\$(b_1, \dots, b_i) = (6,8,2,9,7,10,11,3,12)$\input{cayley_animation/graph8.tex}
\hfill}
\only<10>{
$(a_1, \dots, a_{n-2}) = (2,2,7,7,1,5,3,1,1,4,4,5)$\\$(b_1, \dots, b_i) = (6,8,2,9,7,10,11,3,12,1)$
\begin{tikzpicture}[>=latex,line join=bevel,]
%%
\begin{scope}[scale=0.5]
  \node (2) at (99.0bp,306.0bp) [draw,ellipse,vertex] {$2$};
  \node (6) at (27.0bp,234.0bp) [draw,ellipse,vertex] {$6$};
  \node (8) at (99.0bp,234.0bp) [draw,ellipse,vertex] {$8$};
  \node (7) at (171.0bp,234.0bp) [draw,ellipse,vertex] {$7$};
  \node (9) at (135.0bp,162.0bp) [draw,ellipse,vertex] {$9$};
  \node (1) at (207.0bp,162.0bp) [draw,ellipse,vertex] {$1$};
  \node (3) at (135.0bp,90.0bp) [draw,ellipse,vertex] {$3$};
  \node (12) at (207.0bp,90.0bp) [draw,ellipse,vertex] {$12$};
  \node (4) at (279.0bp,90.0bp) [draw,ellipse,vertex] {$4$};
  \node (11) at (135.0bp,18.0bp) [draw,ellipse,vertex] {$11$};
  \node (5) at (243.0bp,306.0bp) [draw,ellipse,vertex] {$5$};
  \node (10) at (243.0bp,234.0bp) [draw,ellipse,vertex] {$10$};
\end{scope}
  \draw [edge] (2) -- (6);
  \draw [edge] (2) -- (8);
  \draw [edge] (2) -- (7);
  \draw [edge] (7) -- (9);
  \draw [edge] (7) -- (1);
  \draw [edge] (1) -- (3);
  \draw [edge] (1) -- (12);
  \draw [edge] (1) -- (4);
  \draw [edge] (3) -- (11);
  \draw [edge] (5) -- (10);
%
\end{tikzpicture}


\hfill}
\only<11>{
$(a_1, \dots, a_{n-2}) = (2,2,7,7,1,5,3,1,1,4,4,5)$\\$(b_1, \dots, b_i) = (6,8,2,9,7,10,11,3,12,1,13)$\input{cayley_animation/graph10.tex}
\hfill}
\only<12>{
$(a_1, \dots, a_{n-2}) = (2,2,7,7,1,5,3,1,1,4,4,5)$\\$(b_1, \dots, b_i) = (6,8,2,9,7,10,11,3,12,1,13,4)$\input{cayley_animation/graph11.tex}
\hfill}
\only<13>{
$(a_1, \dots, a_{n-2}) = (2,2,7,7,1,5,3,1,1,4,4,5)$\\$(b_1, \dots, b_i) = (6,8,2,9,7,10,11,3,12,1,13,4,)$\\ Die letze Kante ergibt sich aus $f_i = d_i-1$, in diesem Fall: $(5,14)$\input{cayley_animation/graph12.tex}
\hfill}
\end{frame}
  \input{frames/frame5}
\section*{Tiefensuche für ungewichtete Graphen}
  \input{frames/frame6}
  \input{frames/depthsearch_algorithm_plot}
  \input{frames/frame7}
\section*{Matroide}
  \input{frames/frame201}
  \input{frames/frame_matroid_vr}
  \begin{frame}
    \frametitle{Matroide und Graphen}
    Sei $W \subseteq P(M)$ die Familie der Kantenmenge aller Wälder eines Graphen $G = (E, K)$.
\onslide<1->{
    \begin{lemma}
    		Ist $G = (E, K)$ Graph, so ist $M = (K, W)$ ein Matroid.
    \end{lemma}
}   
\only<2-3>{
    \begin{proof}
        \begin{itemize}
            \item<2-3> Axiom 1
            \item<3> Axiom 2
        \end{itemize}
    \end{proof}
}
\onslide<4-5>{
	\begin{proof}
		\begin{itemize}
			\item<4-> Wälder $W = (E, A)$, $W' = (E, B)$ mit $\# B = \# A + 1$. Komponenten $T_{1},..., T_{m}$, Eckenmengen $E_{1},...,E_{m}$, \\ Kantenmengen $A_{1},...,A_{m}$.
			\item<5-> Nun: $\# A_{i} = \# E_{i} - 1$, \ $E = E_{1} \cup ... \cup E_{m}$, \ $A = A_{1} \cup ... \cup A_{m}$. \\ $\# B > \# A \implies$ $\exists$ Kante $k \in B$, die $E_{s}$, $E_{t}$ verbindet. \\
		Dann ist $W'' = (E, A \cup {k})$ Wald.
		\end{itemize}
		
		
	\end{proof}
}
\end{frame}
  \input{frames/frame_das_matroid_vw}
  \begin{frame}
    \frametitle{Matroide und Graphen - Folgerungen}
\onslide<1->{
    \begin{korollar}
    		Basen von $M = (K, W)$ sind die aufspannenden Bäume.
    \end{korollar}
}   
\onslide<2->{
	\begin{korollar}
		Rang des Matroids ist $r(M) = \# E - t$, wobei $t$ die Anzahl der Komponenten von $G$ ist.
	\end{korollar}
}
\end{frame}
\section*{Algorithmen von Kruskal und Prim}
  \begin{frame}
    \frametitle{Algorithmus von Kruskal}
    \begin{algorithm}
        \alginout{gewichteter Graph $G = (V,E)$ mit $n$ Knoten, Funktion $w : E \rightarrow \mathbb{R}$  }{minimal Spannbaum $G'$ von $G$}
        \begin{algtab}
            \algwhile{$\# G < n - 1$}
            		betrachte Kante $e$ aus $G$ mit $w(e) = \min_{e \in E}w(e)$\\
                    \algifthenelse{$G'$ mit $e$ azyklisch}{$e$ von $G$ zu $G'$}{entferne $e$ in $G$}
        \end{algtab}
    \end{algorithm}
\end{frame}

  \input{frames/kruskal_algorithm_plot}
  \begin{frame}
    \frametitle{Algorithmus von Kruskal - Korrektheitsbeweis}
    \begin{theorem}
    		$M = (K, W)$ Matroid mit Gewichtsfunktion $w: K \rightarrow \mathbb{R}$. Algorithmus liefert minimalen Spannbaum:
    		\begin{enumerate}
    			\item Sei $A_{0} = \emptyset \in W$.
    			\item Ist $A_{1} = \{ a_{1},...,a_{i}\} \subseteq K$, so sei $X_{i} = \{ k \in S\setminus A_{i} \ | \ A_{i} \cup \{ x\} \in U \}$. Falls $X_{i} = \emptyset$, so ist $A_{i}$ gesuchte Basis. Andernfalls wähle ein $a_{i+1} \in X_{i}$ mit minimalem Gewicht, und setze $A_{i + 1} = A_{i} \cup \{ a_{i+1}\}$. \\ Iteriere (2).
    		\end{enumerate}
    \end{theorem}
\end{frame}

  \input{frames/kruskal_matroid.tex}
  \begin{frame}
    \frametitle{Algorithmus von Kruskal - Korrektheitsbeweis}
\onslide<1-> {
    \begin{proof}
    		\item<1-> Sei $A = \{ a_{1},...,a_{r}\}$ die erhaltene Menge.
    		\item<2-> Axiom 3 $\implies$ $A$ ist Basis.
    		\item<3-> Konstruktion und Axiom 2 $\implies$ $w(a_{1}) \leq ... \leq w(a_{r})$. 
    		\item<4-> Angenommen $B = \{ b_{1},...,b_{r}\}$ wäre eine Basis mit $w(B) < w(A)$
    \end{proof}
}
\end{frame}
  \input{frames/frame207}
  \input{frames/prim_algorithm_plot}
  \begin{frame}
\frametitle{Ausblick}

\only<1,4,6,9> {
	\begin{itemize}
		\item<1->	Negative Gewichte \onslide<3->{ (Kruskal) }
		\item<4-> 	Computernetzwerke und Steiner-Bäume
		\item<6->	Traveling-Salesman-Problem \onslide<8->{ (Approximation) }
	\end{itemize}
}

\only<2> {
\begin{figure}
		\begin{tikzpicture}[scale=1.6, auto,swap]
    % Draw a 7,11 network
    % First we draw the vertices
    \foreach \pos/\name in {{(0,0)/a}, {(1,1)/b}, {(3,1)/c},
                            {(0,-2)/d}, {(2,0)/e}, {(1,-1)/f}, {(3,-1)/g}}
        \node[vertex] (\name) at \pos {$\name$};
    % Connect vertices with edges and draw weights
    \foreach \source/ \dest /\weight in {b/a/1, c/b/-2,d/a/4,d/b/-9,
                                         e/b/-3, e/c/42, d/g/15,
                                         f/d/4,f/e/7,
                                         g/e/3,g/f/-5}
        \path[edge] (\source) -- node[font=\small] {$\weight$} (\dest);
    
     
    
\end{tikzpicture}
	\caption{Graph mit negativen Gewichten}
	\end{figure}
}

\only<3> {
	\begin{figure}
		\begin{tikzpicture}[scale=1.6, auto,swap]
    % Draw a 7,11 network
    % First we draw the vertices
    \foreach \pos/\name in {{(0,0)/a}, {(1,1)/b}, {(3,1)/c},
                            {(0,-2)/d}, {(2,0)/e}, {(1,-1)/f}, {(3,-1)/g}}
        \node[vertex] (\name) at \pos {$\name$};
    % Connect vertices with edges and draw weights
    \foreach \source/ \dest /\weight in {b/a/1, c/b/-2,d/a/4,d/b/-9,
                                         e/b/-3, e/c/42, d/g/15,
                                         f/d/4,f/e/7,
                                         g/e/3,g/f/-5}
        \path[edge] (\source) -- node[font=\small] {$\weight$} (\dest);
    
     \begin{pgfonlayer}{background}
        \foreach \source / \dest in {b/a,b/c,b/e,b/d,f/g,g/e}
            \path[selected edge] (\source.center) -- (\dest.center);
            
    \end{pgfonlayer}
    
\end{tikzpicture}
	\caption{Graph mit negativen Gewichten}
	\end{figure}
}




\only<5> 	{
					\begin{columns}
						\column{0.5\textwidth}
							\begin{figure}
								\includegraphics[scale=0.2]{pictures/euclidean-steiner-trees}
								\caption{[Abb2] Steiner Bäume}
							\end{figure}
						\column{0.5\textwidth}
							\begin{figure}
								\includegraphics[scale=0.12]{pictures/computer-chip}
								\caption{[Abb3] Computerchip Platine}
							\end{figure}
						\end{columns}
				}

\only<7>	{
					\begin{figure}
						\includegraphics[scale=0.3]{pictures/traveling-salesman-problem}
						\caption{[Abb6] Traveling-Salesman-Problem}
					\end{figure}
				}

\only<8>{
					\begin{figure}
						\includegraphics[scale=0.3]{pictures/tsp-msp}
						\caption{[Abb7] Traveling-Salesman-Problem}
					\end{figure}

}

\end{frame}
	\input{frames/frameBilderVerzeichnis}
  \input{frames/frame_literature}
\end{document}