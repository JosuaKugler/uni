\begin{frame}[t]
 	\frametitle{Der Satz von Cayley}
\only<1-2> {   
	Sei ein $G=(E,K)$ ein vollständiger Graph ($n$ Konten und $\binom{n}{2}$ Kanten). Sei $t(n)$ die Anzahl und $T$ die Menge der aufspannenden Bäume. Seien die Konten von $G$ von $1$ bis $n$ durchnummeriert.
}

\only<2-4>{
    \begin{theorem}
    		Es gilt: $t(n) = n^{n-2}$.
    \end{theorem}
}
\onslide<3->{
	\begin{proof}
		\begin{itemize}	
			\item<3->	Abb: $T \xrightarrow{\sim } \text{Menge der} \ (n-2) \ \text{-Tupel, Einträge aus} \ \{ 1,...,n\}$.
			\item<4->	Zuordnung durch Prüfer-Code
			\item<5-> 	\begin{enumerate}[1]
							\item 	Finde Knoten Grad $1$ mit minimaler Nummer $v$. Nachbar von $v$ ist $a_{1}$.
							\item 	Entferne $v$ und indizierte Kante. Gehe zu (1), führe $n-2$ mal aus. Gesuchtes Tupel ist $ ( a_{1},...,a_{n-2})$.
						\end{enumerate}
			
			\item<6->	\begin{enumerate}[1]
							\item 	Suche minimales $b_{1}$ nicht in $ ( a_{1},...,a_{n-2})$. Dies ergibt Kante $b_{1}a_{1}$.
							\item 	Suche das minimale $b_{2} \neq b_{1}$ nicht in $ ( a_{2},...,a_{n-2})$, usw.
						\end{enumerate}
		\end{itemize}
	\end{proof}
}

\end{frame}    