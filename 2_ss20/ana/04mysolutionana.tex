\documentclass{article}

\usepackage[utf8]{inputenc}
\usepackage[T1]{fontenc}
\usepackage[ngerman]{babel}
\usepackage{amsmath, amsfonts, amsthm, mathtools, amssymb}
\usepackage{stmaryrd}
\usepackage{enumerate}
\usepackage{cases}
\usepackage{fancyhdr}
\usepackage{comment}
%\usepackage{xcolor}
\usepackage{tikz}
\usepackage{cases}
\usepackage{listings}
\usepackage{siunitx}
\usepackage[left = 3cm]{geometry}
\usepackage[hidelinks]{hyperref}
\usepackage{subcaption}
\usepackage{gauss}
\newtheorem{satz}{Satz}[section]
\newtheorem{lemma}[satz]{Lemma}
\newtheorem{korollar}[satz]{Korollar}
\newtheorem{proposition}[satz]{Proposition}
\theoremstyle{definition}
\newtheorem{definition}[satz]{Def.}
\newtheorem{axiom}[satz]{Axiom}
\newtheorem{bsp}[satz]{Bsp.}
\newtheorem*{anmerkung}{Anm}
\newtheorem{bemerkung}[satz]{Bem}
\newtheorem*{notatio}{Notation}
\newcommand{\obda}{O.B.d.A. }
\newcommand{\equals}{\Longleftrightarrow}
\newcommand{\N}{\mathbb{N}}
\newcommand{\Q}{\mathbb{Q}}
\newcommand{\R}{\mathbb{R}}
\newcommand{\Z}{\mathbb{Z}}
\newcommand{\C}{\mathbb{C}}
\newcommand{\intd}{\mathrm{d}}
\newcommand{\Pot}{\operatorname{Pot}}
\newcommand{\mychar}{\operatorname{char}}
\newcommand{\myker}{\operatorname{ker}}
\newcommand{\induktion}[3]
{\begin{proof}\ \\
	\noindent\textbf{Induktionsanfang:}\ #1\\
	\noindent\textbf{Induktionsvoraussetzung:}\ #2\\
	\noindent\textbf{Induktionsschluss:}\ #3
\end{proof}}

\newcommand{\rg}{\operatorname{rg}}
\newcommand{\im}{\operatorname{im}}
\newcommand{\End}{\operatorname{End}}
\newcommand{\abb}{\operatorname{Abb}}
\newcommand{\re}{\operatorname{Re}}
\newcommand{\Ima}{\operatorname{Im}}
\newcommand{\norm}[1]{\left\Vert #1 \right\Vert}


\newcommand{\ipilayout}[1]
{	
	\pagestyle{fancy}
	\fancyhead[L]{Einführung in die praktische Informatik, Blatt #1}
	\fancyhead[R]{Josua Kugler, Jan Metzger, David Wesner}
	\fancypagestyle{firstpage}{%
		\fancyhf{}
		\lhead{Professor: Peter Bastian\\
			Tutor: Frederick Schenk}
		\rhead{Einführung in die praktische Informatik, Übungsblatt #1\\ David, Jan, Josua}
		\cfoot{\thepage}
	}
\thispagestyle{firstpage}
}

\newcommand{\analayout}[1]
{	
	\pagestyle{fancy}
	\fancyhead[L]{Analysis 2, Blatt #1}
	\fancyhead[R]{David Wesner, Josua Kugler}
	\fancypagestyle{firstpage}{%
		\fancyhf{}
		\lhead{Professor: Ekaterina Kostina\\
			Tutor: Julian Matthes}
		\rhead{Analysis 1, Übungsblatt #1\\ David Wesner, Josua Kugler}
		\cfoot{\thepage}
	}
	\thispagestyle{firstpage}
}
\newcommand{\lalayout}[1]
{	
	\pagestyle{fancy}
	\fancyhead[L]{Lineare Algebra 2, Blatt #1}
	\fancyhead[R]{David Wesner, Josua Kugler}
	\fancypagestyle{firstpage}{%
		\fancyhf{}
		\lhead{Professor: Denis Vogel\\
			Tutor: Marina Savarino}
		\rhead{Lineare Algebra 2, Übungsblatt #1\\ David Wesner, Josua Kugler}
		\cfoot{\thepage}
	}
	\thispagestyle{firstpage}
}

\lstset{
    frame=tb, % draw a frame at the top and bottom of the code block
    tabsize=4, % tab space width
    showstringspaces=false, % don't mark spaces in strings
    numbers=left, % display line numbers on the left
    commentstyle=\color{green}, % comment color
    keywordstyle=\color{blue}, % keyword color
    stringstyle=\color{red} % string color
}
\setlength{\headheight}{25pt}

\makeatletter \renewcommand\d{\ensuremath{%
		\;\mathrm{d}\@ifnextchar\d{\!}{}}}
\makeatother

\let\oldstackrel\stackrel
\renewcommand{\stackrel}[2]{%
    \oldstackrel{\mathclap{#1}}{#2}
}%

% maximum norm
\newcommand{\maxnorm}[1]{\left|\left|#1\right|\right|_\infty}
\renewcommand{\epsilon}{\varepsilon}

\begin{document}
\analayout{3}
\section*{Aufgabe 1}
\begin{enumerate}[(a)]
	\item Behauptung: Diese Aussage ist falsch.
	\begin{proof}
		Sei $d(\cdot,\cdot):\mathbb{K}^n\times\mathbb{K}^n\longrightarrow \R$ eine Metrik. Definiere $\varphi: \R_{\geq 0} \longrightarrow \R, \varphi(x)=x^2.$ $\varphi$ ist nach Analysis 1 eine stetige, wohldefinierte Funktion. Betrachte nun die Funktion $\rho: \mathbb{K}^n\longrightarrow \R, \rho(x)=\varphi(||x||_2)$ und gelte $d(x,y)=\rho(x-y)$. Sei nun $\alpha\in\R$. Dann gilt 
		\begin{align*}
			\rho(\alpha(x-y))=& \varphi(||\alpha(x-y)||_2)\\
			\overset{||x||_2\text{ ist eine Norm}}{=}& \varphi(\alpha ||x-y||_2)\\
			=& (\alpha||x-y||_2)^2\\
			=& \alpha^2\cdot ||x-y||_{2}^{2}\\
			=& \alpha^2 \cdot \rho(x-y)
		\end{align*}
	Somit ist $\rho$ nicht linear im ersten Argument und somit keine Norm.
	\end{proof}
	\item Behauptung: Diese Aussage ist richtig.
	\begin{proof}
		\begin{itemize}
			\item "$\Rightarrow$": Angenommen $r\in O$ sei ein Randpunkt von $O$. Dann befindet sich in jeder Umgebung von $r$ sowohl ein Element aus $O$, als auch aus der Menge $\mathbb{K}^n\backslash O$. Somit ist $O$ insbesondere nicht abgeschlossen.
			\item "$\Leftarrow$": Diese Richtung folgt direkt aus 2.26 (i)
		\end{itemize}
	\end{proof}
	\item Diese Aussage ist wahr, siehe 2.26 (iii) im Skript
	\item Behauptung: Diese Aussage ist falsch für $\emptyset\neq M\neq \mathbb{K}^n$
	\begin{proof}
		\begin{itemize}
		\item ($\emptyset\neq M\neq \mathbb{K}^n$): Nach Satz 2.26 (i) ist jede Menge $\overline{M}$ mit $M\subset \mathbb{K}^n$ abgeschlossen und $M^{\circ}$ offen. $(\overline{M})^{\circ}= \overline{M^{\circ}}$ fordert also die Gleichheit einer offenen und einer abgeschlossenen Menge, was ein Widerspruch ist.
		\item ($\emptyset= M $ oder $M= \mathbb{K}^n$): Das Innere der leeren Menge, ist wieder leer, genau so auch der Abschluss. Somit gilt die Gleichung für die leere Menge. Es gilt $(\mathbb{K}^n)^{\circ}=\emptyset$, also insbesondere auch die Gleicheit.
		\end{itemize}
	\end{proof}
	\end{enumerate}

\section*{Aufgabe 2}
\begin{enumerate}[(a)]
	\item Behauptung: $\partial M = A \coloneqq \{x \in \R^n | \norm{x}_\infty \leq 1\}$.
	\begin{proof}
		
		Da $\Q$ dicht in $\R$ liegt, gibt es in jeder Umgebung eines beliebigen Elementes aus $M$ sowohl Punkte in $\Q^n$ als auch Punkte in $\R^n$. Auch für alle $x\in \R^n$ mit $\norm{x}_\infty = 1$ liegen in jeder Umgebung sowohl Punkte in $M$ als auch in $\R^n\setminus M$. Sei $x\in \R^n$ ein Punkt mit $\norm{x}_\infty = 1 + \epsilon$ mit $\epsilon >1$. Dann ist $B_\frac{\epsilon}{2}(x) \cap M = \emptyset$. Daher ist $\partial M = \{x \in \R^n| \norm{x}_\infty \leq 1\}$. 
	\end{proof}
		Der Abschluss ergibt sich dann durch $\overline{M} = M \cup \partial M = \{x \in \R^n| \norm{x}_\infty \leq 1\}$ und für das Innere erhalten wir $M^\circ = M \setminus \partial M = \emptyset$.
		\item Behauptung: $\partial M = M$.
	\begin{proof}
		In jeder Umgebung eines Punktes aus $M$ liegen sowohl Punkte aus $M$ (z.B. der Punkt selbst), als auch Punkte aus $\R^n \setminus M$, da es immer auch Punkte mit $x_1 \neq 0$ gibt. Sei nun $x \in \R^n$ mit $x_1 = \epsilon \neq 0$. Dann ist $B_\frac{\epsilon}{2}(x) \cap M = \emptyset$, da kein Punkt mit $x_1 = 0$ in  $B_\frac{\epsilon}{2}(x)$ liegt. Sei ansonsten $x \in \R^n$ mit $\norm{x}_2 = 1 + \epsilon, \epsilon > 0$. Dann ist aufgrund der Dreiecksungleichung $B_\frac{\epsilon}{2}(x) \cap M = \emptyset$.
	\end{proof}
	Der Abschluss ergibt sich dann durch $\overline{M} = M \cup M = M$ und für das Innere erhalten wir $M^\circ = M \setminus \partial M = M \setminus M = \emptyset$.
	\item $F \coloneqq \{x\in \R^n| f(x) = 1\} = \{x \in \R^n | x \in (1, -1)^n\} = \{x \in \R^n | \norm{x}_\infty < 1\}$. Behauptung: $\partial F = A \coloneqq \{x \in \R^n | \norm{x}_\infty = 1\}$.
	\begin{proof}
		Wegen $A \cap F = \emptyset$ liegen natürlich in jeder Umgebung eines Punktes von $A$ Punkte aus $\R^n \setminus F$. Allerdings liegen in jeder Umgebung auch Punkte mit $\norm{x}_\infty < 1$, also Punkte aus $F$. Wäre jetzt $\norm{x}_\infty = 1 - \epsilon$, dann wäre $B_\frac{\epsilon}{2}(x) \subset F$. Analog wäre für $\norm{x}_\infty = 1 + \epsilon$ $B_\frac{\epsilon}{2}(x) \cap F = \emptyset$. Daher können solche Punkte nicht auf dem Rand liegen.
	\end{proof}
	Wir erhalten also $\overline{F} = F \cup\partial F = \{x \in \R^n | \norm{x}_\infty \leq 1\}$ und $F^\circ = F \setminus \partial F = F$.
	Nun ist $M$ gleich $\R^n \setminus F = \{x \in \R^n | \norm{x}_\infty > 1\}$. Daher erhalten wir $\partial M = \partial F$. Also ist $\overline{M} = M \cup \partial M = \{x \in \R^n | \norm{x}_\infty \geq 1\}$ und $M ^\circ = M \setminus \partial M = \{x \in \R^n | \norm{x}_\infty > 1\}$.
	\item Hier ist $M = F$ und wir erhalten aus der (c) sofort $\partial F = A \coloneqq \{x \in \R^n | \norm{x}_\infty = 1\}$, $\overline{F} = F \cup\partial F = \{x \in \R^n | \norm{x}_\infty \leq 1\}$ und $F^\circ = F \setminus \partial F = F$.
\end{enumerate}
\section*{Aufgabe 3}
Seien $V,\overset{\sim}{V}$ definiert, wie auf dem Übungsblatt
\begin{enumerate}[(a)]
	\item Behauptung: Auf $\overset{\sim}{V}$ ist die Abbildung $(\cdot,\cdot)$ nicht definit.
	\begin{proof}
		Gelte $(f,f)=0$ für ein $f\in \overset{\sim}{V},$ d.h. $\int\limits_{a}^{b}f'(x)^2 \d x=0$. Nun ist $\int\limits_{a}^{b}0\d x=0$, weshalb $f'$ gleich der Nullabbildung sein kann. Nun ist dann aber $f$ der Form $f(x)=c$ für alle $x\in [a,b]$, also nicht die Nullabbildung in $\overset{\sim}{V}$. Also gilt $\exists x\in \overset{\sim}{V}: (x,x)=0 \wedge x\neq 0$  
	\end{proof}
	\item Wir zeigen die Skalarprodukt eigenschaften:
	\begin{enumerate}[(S1)]
		\item (Definitheit): Nach Analysis 1 Korollar 6.22 gilt $\forall x\in V: (x,x)=0\implies x=0$.  Außerdem ist $f'(x)^2\geq 0$ für alle $f\in V$, weshalb $(x,x)\geq 0, \forall x\in V$
		\item (Symmetrie): Es gilt für $f,g\in V$:
		\begin{align*}
			(f,g)=& \int\limits_{a}^{b}f'(x)g'(x)\d x \\
			\overset{V\subset C[a,b]}{=}& \int\limits_{a}^{b}g'(x)f'(x)\d x\\
			=& (g,f)\\
		\end{align*}
		\item (Linearität im ersten Argument): Seien $\alpha, \beta\in \R$ und $f_1,f_2,g\in V$. Dann gilt:
			\begin{align*}
				(\alpha f_1+\beta f_2,g)&= \int\limits_{a}^{b}(\alpha f_1+\beta f_2)g \d x\\
				&=  \int\limits_{a}^{b}\alpha f_1g+\beta f_2 g \d x\\
				&= \int\limits_{a}^{b}\alpha f_1g\d x+\int\limits_{a}^{b}\beta f_2g\d x\\
				&= \alpha\int\limits_{a}^{b}f_1g\d x+\beta\int\limits_{a}^{b} f_2g\d x\\
				& =\alpha(f_1,g)+\beta(f_2,g)
			\end{align*}
		\end{enumerate}
\end{enumerate}
\section*{Aufgabe 4}
Wir suchen 4 Elemente $x_1, x_2, x_3, x_4 \in C([0,1])$ mit $(x_i, x_j) = \delta_{ij}$. $x_1 = 1$ ist bereits normiert. Daher bestimmen wir zunächst
$$\tilde{x_2} = t - (t, x_1) \cdot x_1 = t - \int_0^1 t \d t = t - \frac{t^2}{2}\bigg|_0^1 = t - \frac{1}{2}.$$
Nun müssen wir $\tilde{x_2}$ noch normieren.
$$(\tilde{x_2}, \tilde{x_2}) = \int_0^1 x_2^2 \d t= \int_0^1 t^2 -t + \frac{1}{4}\d t = \frac{1}{3} - \frac{1}{2} + \frac{1}{4} = \frac{1}{12}$$
Es gilt $x_2 = \frac{\tilde{x_2}}{\norm{\tilde{x_2}}} = \frac{t-\frac{1}{2}}{\sqrt{\frac{1}{12}}} = 2 \sqrt{3} \left(t - \frac{1}{2}\right)$
Als nächstes berechnen wir $\tilde{x_3}$.
\begin{align*}
	\tilde{x_3} &= t^2 - \int_0^1 t^2\cdot 1 \d t 1 - \left(\int_0^1t^2 \cdot 2 \sqrt{3} \left(t - \frac{1}{2}\right) \d t\right)\cdot 2 \sqrt{3} \left(t - \frac{1}{2}\right)\\
	&= t^2 - \frac{1}{3} - \left(t - \frac{1}{2}\right) \cdot 12 \cdot \int_0^1t^3 - \frac{1}{2}t^2 \d t\\
	&= t^2 - \frac{1}{3} - \left(12t - 6\right)  \cdot (\frac{1}{4} - \frac{1}{6})\\
	&= t^2 - \frac{1}{3} - t + \frac{1}{2}\\
	&= t^2 - t + \frac{1}{6}
\end{align*}

\begin{align*}
	\norm{\tilde{x_3}}^2 &= \int_0^1 (t^2 - t + \frac{1}{6})^2 \d t\\
	&= \int_0^1 (t^4 - 2t^3 + t^2\frac{1}{3} - t\frac{1}{3} + t^2 + \frac{1}{36})\\
	&= \frac{1}{5} - \frac{1}{2} + \frac{1}{9} - \frac{1}{6} + \frac{1}{3} + \frac{1}{36}\\
	&= \frac{1}{180}
\end{align*}
$$x_3 = \tilde{x_3} \cdot \frac{1}{\norm{\tilde{x_3}}} = (t^2 - t + \frac{1}{6}) \cdot \frac{30}{\sqrt{5}} = 6 \sqrt{5} \left(t^{2} - t + \frac{1}{6}\right)$$
\begin{align*}
	\tilde{x_4} &= t^3 - \int_0^1 t^3 \d t - \int_0^1 t^3\cdot \left(t - \frac{1}{2}\right)\d t \cdot 12 \left(t - \frac{1}{2}\right) - \int_0^1 t^3\cdot \left(t^{2} - t + \frac{1}{6}\right) \d t \cdot 180 \left(t^{2} - t + \frac{1}{6}\right)\\
	&= t^3 - \frac{1}{4} - \left(\frac{1}{5} - \frac{1}{8}\right)\cdot 12 \left(t - \frac{1}{2}\right) - \left(\frac{1}{6} - \frac{1}{5} + \frac{1}{24}\right)\cdot 180 \left(t^{2} - t + \frac{1}{6}\right)\\
	&= t^3 - \frac{1}{4} - \left(\frac{9}{10}t - \frac{9}{20}\right) - \left(\frac{3}{2}t^{2} - \frac{3}{2}t + \frac{1}{4}\right)\\
	&= t^3 - \frac{3}{2}t^2 + \frac{15}{10}t - \frac{9}{10}t - \frac{10}{20} + \frac{9}{20}\\
	&= t^{3} - \frac{3 t^{2}}{2} + \frac{3 t}{5} - \frac{1}{20}
\end{align*}

\begin{align*}
	\norm{\tilde{x_4}}^2 &= \int_0^1 \left(t^{3} - \frac{3 t^{2}}{2} + \frac{3 t}{5} - \frac{1}{20}\right)^2\d t\\
	&= \int_0^1 t^{6} - 3t^5 + \frac{6t^4}{5} - \frac{1}{10}t^3 + \frac{9t^4}{4} + \frac{9t^3}{5} + \frac{3t^2}{20} + \frac{9t^2}{25} - \frac{3t}{50} + \frac{1}{400}\d t\\
	&= \int_0^1 t^{6} - 3 t^{5} + \frac{69 t^{4}}{20} - \frac{19 t^{3}}{10} + \frac{51 t^{2}}{100} - \frac{3 t}{50} + \frac{1}{400} \d t\\
	&= \frac{1}{7} - \frac{3}{6} + \frac{69}{100} - \frac{19}{40} + \frac{17}{100} - \frac{3}{100} + \frac{1}{400}\\
	&= \frac{1}{7} + \frac{-200 + 4 \cdot 69 - 190 + 4 \cdot 17 - 12 + 1}{400}\\
	&= \frac{400 - 7 \cdot 57}{2800}\\
	&= \frac{1}{2800}
\end{align*}

$$x_4 = \frac{\tilde{x_4}}{\norm{\tilde{x_4}}} = 20 \sqrt{7} \cdot \left(t^{3} - \frac{3 t^{2}}{2} + \frac{3 t}{5} - \frac{1}{20}\right) = \sqrt{7} \left(20 t^{3} - 30 t^{2} + 12 t - 1\right)$$
\end{document}
