\documentclass{article}

\usepackage[utf8]{inputenc}
\usepackage[T1]{fontenc}
\usepackage[ngerman]{babel}
\usepackage{amsmath, amsfonts, amsthm, mathtools, amssymb}
\usepackage{stmaryrd}
\usepackage{enumerate}
\usepackage{cases}
\usepackage{fancyhdr}
\usepackage{comment}
%\usepackage{xcolor}
\usepackage{tikz}
\usetikzlibrary{calc,intersections,through,backgrounds}
\usepackage{cases}
\usepackage{listings}
\usepackage{siunitx}
\usepackage[left = 2cm, right = 2cm, top=2.5cm, bottom=2.5cm]{geometry}
\usepackage[hidelinks]{hyperref}
\usepackage{subcaption}
\usepackage{gauss}
\newtheorem{satz}{Satz}[section]
\newtheorem{lemma}[satz]{Lemma}
\newtheorem{korollar}[satz]{Korollar}
\newtheorem{proposition}[satz]{Proposition}
\theoremstyle{definition}
\newtheorem{definition}[satz]{Def.}
\newtheorem{axiom}[satz]{Axiom}
\newtheorem{bsp}[satz]{Bsp.}
\newtheorem*{anmerkung}{Anm}
\newtheorem{bemerkung}[satz]{Bem}
\newtheorem*{notatio}{Notation}
\newcommand{\obda}{O.B.d.A. }
\newcommand{\equals}{\Longleftrightarrow}
\newcommand{\N}{\mathbb{N}}
\newcommand{\Q}{\mathbb{Q}}
\newcommand{\R}{\mathbb{R}}
\newcommand{\Z}{\mathbb{Z}}
\newcommand{\C}{\mathbb{C}}
\newcommand{\intd}{\mathrm{d}}
\newcommand{\Pot}{\operatorname{Pot}}
\newcommand{\mychar}{\operatorname{char}}
\newcommand{\myker}{\operatorname{ker}}
\newcommand{\induktion}[3]
{\begin{proof}\ \\
	\noindent\textbf{Induktionsanfang:}\ #1\\
	\noindent\textbf{Induktionsvoraussetzung:}\ #2\\
	\noindent\textbf{Induktionsschluss:}\ #3
\end{proof}}

\newcommand{\rg}{\operatorname{rg}}
\newcommand{\im}{\operatorname{im}}
\newcommand{\End}{\operatorname{End}}
\newcommand{\abb}{\operatorname{Abb}}
\newcommand{\re}{\operatorname{Re}}
\newcommand{\Ima}{\operatorname{Im}}
\newcommand{\Fit}{\operatorname{Fit}}


\newcommand{\lalayout}[1]
{	
	\pagestyle{fancy}
	\fancyhead[L]{Lineare Algebra 1, Blatt #1}
	\fancyhead[R]{David Wesner, Josua Kugler}
	\fancypagestyle{firstpage}{%
		\fancyhf{}
		\lhead{Dozent: Denis Vogel\\
			Tutor: Marina Savarino}
		\rhead{Lineare Algebra 2, Übungsblatt #1\\ David Wesner, Josua Kugler}
		\cfoot{\thepage}
	}
	\thispagestyle{firstpage}
}

\lstset{
    frame=tb, % draw a frame at the top and bottom of the code block
    tabsize=4, % tab space width
    showstringspaces=false, % don't mark spaces in strings
    numbers=left, % display line numbers on the left
    commentstyle=\color{green}, % comment color
    keywordstyle=\color{blue}, % keyword color
    stringstyle=\color{red} % string color
}
\setlength{\headheight}{25pt}
\begin{document}
\lalayout{3}
\section*{Aufgabe 12}
\begin{enumerate}[(a)]
\item Behauptung: Für $m,n\in\Z$ sind äquivalent:
\begin{enumerate}[(i)]
	\item $\overline{m}$ ist eine Einheit in $\Z/n\Z$
	\item ggT($m,n$)$=1$
	\begin{proof}
		\begin{itemize}
			\item (i) $\implies$ (ii): Sei $\overline{m}\in\Z/n\Z$. Da $\overline{m}$ eine Einheit ist, existiert ein $k\in \Z$, sodass $\overline{m}\cdot \overline{k}=\overline{1}.$ Somit ist $mk-1\in \Z$. Weiter existiert ein $l\in \Z$, sodass $mk-1=ln\Leftrightarrow 1=mk-ln$. Sei $d\in \Z$ derart, dass $d|m $ und $d|n$. Dann gilt aber auch $d|(mk-ln)=1$: Weil $d\in\Z$, folgt $d\in\{-1,1\},$ also ggT($m,n$)$=1.$
			\item (ii) $\implies$ (i): Gelte ggT($m,n$)$=1$. Es existieren mit dem erweiterten euklidischen Algorithmus $u,v\in\Z $ mit $um+vn=1$. Daraus folgt $\overline{um}+\overline{vn}=\overline{um}=\overline{1}\implies \overline{u}\cdot \overline{m}=1$. Somit ist $\overline{m}$ eine Einheit in $\Z/n\Z$.
		\end{itemize}
	\end{proof}
\end{enumerate}
\item Es gilt mit dem euklidischen Algorithmus:
\begin{align*}
	51&=1\cdot 42 +9\\
	42 &= 4\cdot 9+6\\
	9&=1\cdot 6*3\\
	6&=2\cdot 3\\
\intertext{Somit ist der ggT($51,42$)$=3$, d.h. $\overline{42}\not\in(\Z/51\Z)^{\times}$.}
	55&=1\cdot 42+13\\
	42&=3\cdot 13+3\\
	13&=4\cdot 3+1\\
	3&=3\cdot 1\\
	\intertext{Somit ist der ggT($55,42$)$=1$, d.h. $\overline{42}\in(\Z/55\Z)^{\times}$. Außerdem gilt:}
	1&= 13-4\cdot 3\\
	&= 13-4(42-3\cdot 13)\\
	&=13\cdot 13-4\cdot 42\\
	&=13(55-42)-4\cdot 42\\
	&= 13\cdot 55+(-17)\cdot 42\\
	\intertext{$55-17=38$, also gilt $\overline{42}\cdot\overline{38}=\overline{1}$ in $\Z/55\Z$.}
\end{align*}


\end{enumerate}
\section*{Aufgabe 13}
\begin{enumerate}[(a)]
	\item Sei $z = c + di$. Dann gilt $|z - (a +bi)| \leq \frac{1}{\sqrt{2}} \equals (c-a)^2 + (d-b)^2 \leq \frac{1}{2}$. Zu jeder reellen Zahl $x$ gibt es eine ganze Zahl $n(x)$, sodass $|n(x) - x| \leq \frac{1}{2}$. Wähle also $a = n(c)$ und $b = n(d)$. Dann ist $(c-a)^2 + (d-b)^2 \leq \left(\frac{1}{2}\right)^2 + \left(\frac{1}{2}\right)^2 = \frac{1}{2}$.
	\item Zu jeder Zahl $z$ gibt es ein $q$, sodass $z$ in einem von $qw$, $(q+1)w$, $(q + i)w$ und $(q + 1+ i)w$, wie in der Skizze abgebildet, begrenzten Quadrat liegt.
	\begin{figure}[ht]
		\centering
		\begin{tikzpicture}[auto]
			\coordinate [label=left:$qw$] (o) at ( 0, 0);
			\coordinate [label=above:$(q+1)w$] (w) at ( 2, 1);
			\coordinate [label=below:$(q+i)w$] (x) at (1, -2);
			\coordinate [label=right:$(q +  i + 1)w$] (y) at (2 + 1, 1 - 2);
			\coordinate [label=right:$z$] (z) at (1.5, -.5);

			\node (O) [fill=black, circle, inner sep=1pt] at (o) {};
			\node (W) [fill=black, circle, inner sep=1pt] at (w) {};
			\node (X) [fill=black, circle, inner sep=1pt] at (x) {};
			\node (Y) [fill=black, circle, inner sep=1pt] at (y) {};
			\node (Z) [fill=black, circle, inner sep=1pt] at (z) {};
			\draw (o) -| node[near start] {$a$} node[near end] {$b$} (w) |- (y) -| (x) |- (o);
			\draw (O) -- node[near end, rotate = 28] {$\sqrt{a^2 + b^2}$} (W) -- (Y) -- (X) -- (O);
		\end{tikzpicture}
	\end{figure}
	Der maximale Abstand von $z$ zu einem Vielfachen von $w$ ist genau dann erreicht, wenn $z$ in der Mitte dieses Quadrats liegt. Dann ist der Abstand von $z$ zu jedem der 4 Endpunkte durch die Hälfte der Länge der Diagonalen gegeben, also nach dem Satz des Pythagoras $|z - qw| = \frac{1}{2}\sqrt{(a^2 + b^2) + (a^2 + b^2)} = \sqrt{\frac{1}{2}}\sqrt{a^2 + b^2}$. Im Allgemeinen ist $|z - qw|$ aber kleiner als dieser Abstand (für geeignetes $q$ natürlich), also $\delta(z - qw) = |z-qw|^2 \leq \frac{1}{2}(a^2 + b^2) = \frac{1}{2}\delta(w)$. Natürlich könnte man das jetzt noch formalisieren, aber das macht nicht so viel Spaß und die Idee ist klar.
	\item Laut Aufgabentext ist $\Z[i]$ nullteilerfrei. Es genügt also zu zeigen, dass  $\forall z, w\in \Z[i]$ mit $w\neq 0 \exists q, r\in \Z[i]$ mit $z = qw + r$, $\delta(r) <\delta(w)$. Wähle $r = z -qw$. Nach Aufgabenteil $(b)$ gilt dann $\deg(r) \leq \frac{1}{2}\deg(w) < \deg(w)$. Daraus folgt sofort die Aussage.
	\item \begin{align*}
		9 &= (1-i)(3+4i) + 2-i &&\delta(2-i) = 5 < 25 = \delta(3+4i)\\
		3 + 4i &= (2i)(2-i) +1 &&\delta(1) = 1 < 5 = \delta(2-i)\\
		2-i &= (2-i)(1) + 0\\
		\implies 1&\in \operatorname{ggT}(9, 3+4i)
	\end{align*}
\end{enumerate}
\section*{Aufgabe 14}
Behauptung: Für einen Ring $\neq 0$ sind folgende Aussagen äquivalent:
\begin{enumerate}[(i)]
	\item $R$ ist ein Körper
	\item $R[t]$ ist ein euklidischer Ring
	\item $R[t]$ ist ein Hauptidealring
\end{enumerate}
\begin{proof}
	\begin{enumerate}[(i)$\implies$ (ii):]
		\item Jeder Polynomring über Körpern ist nach VL euklidisch
		\item [(ii)$\implies$ (iii):]Jeder Euklidische Ring ist nach VL ein Hauptidealring
		\item [(iii) $\implies$ (i):] Sei $R$ ein Körper. Ist $R$ nicht nullteilerfrei, folgt direkt die Behauptung. Sei nun also $R$ nullteilerfrei, weshalb $\exists x\in R\setminus\{0\}$ sodass $xy\neq 1 \forall y\in R$.
	\end{enumerate}
	Behauptung: $(x,t)$ ist kein Hauptideal. 
	\begin{proof}
		Angenommen $\exists f\in R[t]$, sodass $(f)=(x,t)$, dann $\exists h\in R[t],$ sodass $x=fh.$ $R$ ist nullteilerfrei, also gilt 
		$$0=\deg(x)=\deg(f)+\deg(h)\implies \deg(f)=\deg(h)=0.$$
		Es existiert also ein $a\in R$ sodass $f=a$ und ein $g\in R[t]$ sodass $t=fg=ag.$ Somit gilt $\deg(g)=1$ und wegen $e(t)=1$
		$$ 1=e(t)=e(a)\cdot e(g)=a\cdot e(h)\in R.$$
		Folglich ist $a$ eine Einheit auf $R$. Wegen $aa^{-1}=1 $ gilt zudem $1\in (a)=(x,t)$ und es existieren $u,v\in R[t]$ sodass 
		$$1=xu+tv\overset{t=0}{\implies}1=x\cdot u(0)\in R.$$
		Damit ist aber $x\in R^{\times}$, was ein Widerspruch ist.
\end{proof}
\end{proof}

\section*{Aufgabe 15}
\begin{enumerate}[(a)]
	\item \begin{align*}
		\begin{gmatrix}[p]
			0 & 20 & 0\\
			10 & 12 & 6\\
			20 & 12 & 10
			\colops
			\add[-1]{0}{1}
		\end{gmatrix} &\rightsquigarrow
		\begin{gmatrix}[p]
			0 & 20 & 0\\
			10 & 2 & 6\\
			20 & -8 & 10
			\colops
			\swap{0}{1}
		\end{gmatrix}
		\rightsquigarrow
		\begin{gmatrix}[p]
			20 & 0 & 0\\
			2 & 10 & 6\\
			-8 & 20 & 10
			\rowops
			\swap{0}{1}
		\end{gmatrix}\\
		&\rightsquigarrow
		\begin{gmatrix}[p]
			2 & 10 & 6\\
			20 & 0 & 0\\
			-8 & 20 & 10
			\colops
			\add[-5]{0}{1}
			\add[-3]{0}{2}
		\end{gmatrix}
		\rightsquigarrow
		\begin{gmatrix}[p]
			2 & 0 & 0\\
			20 & -100 & -60\\
			-8 & 60 & 34
			\rowops
			\add[-10]{0}{1}
			\add[4]{0}{2}
		\end{gmatrix}\\
		&\rightsquigarrow
		\begin{gmatrix}[p]
			2 & 0 & 0\\
			0 & -100 & -60\\
			0 & 60 & 34
			\rowops
			\add[2]{2}{1}
		\end{gmatrix}
		\rightsquigarrow
		\begin{gmatrix}[p]
			2 & 0 & 0\\
			0 & 20 & 8\\
			0 & 60 & 34
			\rowops
			\add[-4]{1}{2}
		\end{gmatrix}\\
		&\rightsquigarrow
		\begin{gmatrix}[p]
			2 & 0 & 0\\
			0 & 20 & 8\\
			0 & -20 & 2
			\colops
			\swap{1}{2}
			\rowops
			\swap{1}{2}
		\end{gmatrix}
		\rightsquigarrow
		\begin{gmatrix}[p]
			2 & 0 & 0\\
			0 & 2 & -20\\
			0 & 8 & 20
			\rowops
			\add[-4]{1}{2}
		\end{gmatrix}\\
		&\rightsquigarrow
		\begin{gmatrix}[p]
			2 & 0 & 0\\
			0 & 2 & -20\\
			0 & 0 & 100
			\colops
			\add[10]{1}{2}
		\end{gmatrix}
		\rightsquigarrow
		\begin{gmatrix}[p]
			2 & 0 & 0\\
			0 & 2 & 0\\
			0 & 0 & 100
		\end{gmatrix}
	\end{align*}
	Die Elementarteiler sind also $2, 2$ und $100$. Die Fittingideale dieser Matrix sind nach Fittings Lemma gleich den Fittingidealen von $A$. Es gilt daher $\Fit_1(A) = (2,2,100) = (2)$, $\Fit_2(A) = (4, 200, 200) = (4)$ und $\Fit_3(A) = (400)$.
	\item \begin{align*}
		\begin{gmatrix}[p]
			1-t & -1 & 2\\
			-1 & -1 & 3\\
			0 & -1 & 3-t
			\rowops
			\swap{1}{2}
		\end{gmatrix}
		&\rightsquigarrow
		\begin{gmatrix}[p]
			-1 & -1 & 3\\
			1-t & -1 & 2\\
			0 & -1 & 3-t
			\colops
			\add[-1]{0}{1}
			\add[3]{0}{2}
		\end{gmatrix}
		\rightsquigarrow
		\begin{gmatrix}[p]
			-1 & 0 & 0\\
			1-t & -2 +t & 5 -3t\\
			0 & -1 & 3-t
			\rowops
			\add[1-t]{0}{1}
		\end{gmatrix}\\
		&\rightsquigarrow
		\begin{gmatrix}[p]
			-1 & 0 & 0\\
			0 & -2 +t & 5 -3t\\
			0 & -1 & 3-t
			\rowops
			\swap{1}{2}
		\end{gmatrix}
		\rightsquigarrow
		\begin{gmatrix}[p]
			-1 & 0 & 0\\
			0 & -1 & 3-t\\
			0 & -2+t & 5-3t
			\rowops
			\add[-2+t]{1}{2}
		\end{gmatrix}\\
		&\rightsquigarrow
		\begin{gmatrix}[p]
			-1 & 0 & 0\\
			0 & -1 & 3-t\\
			0 & 0 & -1 + 2t -t^2
			\colops
			\add[3-t]{1}{2}
		\end{gmatrix}
		\rightsquigarrow
		\begin{gmatrix}[p]
			-1 & 0 & 0\\
			0 & -1 & 0\\
			0 & 0 & -(t-1)^2
		\end{gmatrix}
	\end{align*}
	Daraus folgt $F_1(B) = (1, 1, (t-1)^2) = \R[t], F_2(B) = (1, (t-1)^2, (t-1)^2) = \R[t], F_3(B) = ((t-1)^2)$.
\end{enumerate}
\end{document}