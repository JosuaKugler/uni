\documentclass{article}

\usepackage[utf8]{inputenc}
\usepackage[T1]{fontenc}
\usepackage[ngerman]{babel}
\usepackage{amsmath, amsfonts, amsthm, mathtools, amssymb}
\usepackage{stmaryrd}
\usepackage{enumerate}
\usepackage{cases}
\usepackage{fancyhdr}
\usepackage{comment}
%\usepackage{xcolor}
\usepackage{tikz}
\usepackage{pgf}
\usepackage{pgfplots}
\pgfplotsset{compat=1.16}
\usepackage{cases}
\usepackage{listings}
\usepackage{siunitx}
\usepackage[left = 3cm, bottom =3cm]{geometry}
\usepackage[hidelinks]{hyperref}
\usepackage{subcaption}
\usepackage{gauss}
\newtheorem{satz}{Satz}[section]
\newtheorem{lemma}[satz]{Lemma}
\newtheorem{korollar}[satz]{Korollar}
\newtheorem{proposition}[satz]{Proposition}
\theoremstyle{definition}
\newtheorem{definition}[satz]{Def.}
\newtheorem{axiom}[satz]{Axiom}
\newtheorem{bsp}[satz]{Bsp.}
\newtheorem*{anmerkung}{Anm}
\newtheorem{bemerkung}[satz]{Bem}
\newtheorem*{notatio}{Notation}
\newcommand{\obda}{O.B.d.A. }
\newcommand{\equals}{\Longleftrightarrow}
\newcommand{\N}{\mathbb{N}}
\newcommand{\Q}{\mathbb{Q}}
\newcommand{\R}{\mathbb{R}}
\newcommand{\Z}{\mathbb{Z}}
\newcommand{\C}{\mathbb{C}}
\newcommand{\intd}{\mathrm{d}}
\newcommand{\Pot}{\operatorname{Pot}}
\newcommand{\mychar}{\operatorname{char}}
\newcommand{\myker}{\operatorname{ker}}
\newcommand{\induktion}[3]
{\begin{proof}\ \\
	\noindent\textbf{Induktionsanfang:}\ #1\\
	\noindent\textbf{Induktionsvoraussetzung:}\ #2\\
	\noindent\textbf{Induktionsschluss:}\ #3
\end{proof}}

\newcommand{\rg}{\operatorname{rg}}
\newcommand{\im}{\operatorname{im}}
\newcommand{\End}{\operatorname{End}}
\newcommand{\abb}{\operatorname{Abb}}
\newcommand{\re}{\operatorname{Re}}
\newcommand{\Ima}{\operatorname{Im}}



\newcommand{\numlayout}[1]
{	
	\pagestyle{fancy}
	\fancyhead[L]{Einführung in die Numerik, Blatt #1}
	\fancyhead[R]{David Wesner, Josua Kugler}
	\fancypagestyle{firstpage}{%
		\fancyhf{}
		\lhead{Professor: Peter Bastian\\
			Tutor: Ernestine Großmann}
		\rhead{Einführung in die Numerik, Übungsblatt #1\\ David, Josua}
		\cfoot{\thepage}
	}
\thispagestyle{firstpage}
}

\newcommand{\analayout}[1]
{	
	\pagestyle{fancy}
	\fancyhead[L]{Analysis 2, Blatt #1}
	\fancyhead[R]{David Wesner, Josua Kugler}
	\fancypagestyle{firstpage}{%
		\fancyhf{}
		\lhead{Professor: Ekaterina Kostina\\
			Tutor: Julian Matthes}
		\rhead{Analysis 1, Übungsblatt #1\\ David Wesner, Josua Kugler}
		\cfoot{\thepage}
	}
	\thispagestyle{firstpage}
}
\newcommand{\lalayout}[1]
{	
	\pagestyle{fancy}
	\fancyhead[L]{Lineare Algebra 2, Blatt #1}
	\fancyhead[R]{David Wesner, Josua Kugler}
	\fancypagestyle{firstpage}{%
		\fancyhf{}
		\lhead{Professor: Denis Vogel\\
			Tutor: Marina Savarino}
		\rhead{Lineare Algebra 2, Übungsblatt #1\\ David Wesner, Josua Kugler}
		\cfoot{\thepage}
	}
	\thispagestyle{firstpage}
}

\lstset{
    frame=tb, % draw a frame at the top and bottom of the code block
    tabsize=4, % tab space width
    showstringspaces=false, % don't mark spaces in strings
    numbers=left, % display line numbers on the left
    commentstyle=\color{green}, % comment color
    keywordstyle=\color{blue}, % keyword color
    stringstyle=\color{red} % string color
}
\setlength{\headheight}{25pt}

\makeatletter \renewcommand\d{\ensuremath{%
		\;\mathrm{d}}}
\makeatother

% maximum norm
\newcommand{\maxnorm}[1]{\left|\left|#1\right|\right|_\infty}
\renewcommand{\epsilon}{\varepsilon}

\begin{document}
\numlayout{3}
\section*{Aufgabe 1}
\begin{enumerate}[(a)]
\item \begin{figure}
	\centering
	\begin{tikzpicture}[scale=2]
		\draw (-1.5,0) edge[-latex] (1.5,0) (0,-1.5) edge[-latex] (0,1.5);
		\node at (-0.2,1) {$1$};	
		\draw (1,0) -- (0,1);
		\draw (0,1) -- (-1,0);
		\draw (-1,0) -- (0,-1);
		\draw (0,-1) -- (1,0);
		\draw[red] (0,0) circle (1);
		\draw[blue] (-1,-1) rectangle (1,1);
		\node at (1, -0.2) {$1$};
		\node at (1.5, -0.2) {$x_1$};
		\node at (-0.2, 1.5) {$x_2$};
	\end{tikzpicture}
\end{figure}
In der Abbildung ist $||x||_1$ schwarz, $||x||_2$ rot und $||x||_\infty$ blau.
\item Sei $x\in \R^n$.
\begin{itemize}
	\item Es gilt $|x|\leq\max{|x|}$ und somit für $x_i\in\R$: $\sum\limits_{i=1}^{n} |x_i|\leq n\cdot \max{|x_i|}$. Wegen $\max{|x|}\leq |x|+k, k\geq 0$ gilt $\max{|x_i|}\leq \sum\limits_{i=1}^{n} |x_i|$. Insgesamt also: 
	 $$\maxnorm{x}\leq ||x||_1 \leq n \cdot \maxnorm{n}$$ Für $n\longrightarrow \infty$ geht $n$ gegen unendlich. Es wächst also $||x||_1$ schneller als $\maxnorm{x}$.
	\item Es gilt $|x|^2\leq \max{|x|^2}$ und für $x_i\in\R$: $\sum\limits_{i=1}^{n} |x_i|^2\leq n\cdot \max{|x_i|^2},$ also insbesondere auch  $\sqrt{\sum\limits_{i=1}^{n} |x_i|^2}\leq \sqrt{n\cdot \max{|x_i|^2}}=\sqrt{n}\cdot\max{|x_i|}$. Wegen $\max{|x|^2}\leq |x|^2+k, k\geq 0$, gilt insgesamt:
	\begin{align*} 
		\max{|x_iw|}&\leq\sqrt{\sum\limits_{i=1}^{n} |x_i|^2}\leq \sqrt{n}\cdot\max{|x_i|}\\
		\intertext{also}
		\maxnorm{x}&\leq ||x||_2\leq \sqrt{n}\cdot\maxnorm{x}
	\end{align*}
	Für $n\longrightarrow \infty$ geht $\sqrt{n}$ gegen unendlich. Es wächst also $||x||_2$ schneller als $\maxnorm{x}$.
	\item Es gilt $$||x||_1=\sum\limits_{k=1}^{n}|x_k\cdot 1| \overset{\text{ Cauchy-Schwarz-Ungl.}}{\leq} ||x||_2\cdot||1||_2=\sqrt{n}||x||_2.$$ 
	Teilen durch $\sqrt{n}$ liefert
	\begin{align*}
	\frac{1}{\sqrt{n}}||x||_1\leq ||x||_2\leq ||x||_1
	\end{align*}
	Für $n\longrightarrow \infty$ geht $\frac{1}{\sqrt{n}}$ gegen Null. Es wächst also $||x||_1$ schneller als $||x||_2$.
\end{itemize}
\end{enumerate}
\section*{Aufgabe 2}
\begin{enumerate}[(a)]
	\item Wir berechnen die Konditionszahl
	\begin{align*}
		k &= \left(\frac{\d f}{\d x} \cdot \frac{x}{f(x)}\right)\\
		&= \frac{\sin(x)x - (1-\cos(x))}{x^2} \cdot \frac{x\cdot x}{1-\cos(x)}\\
		&= \frac{x\sin(x) - (1-\cos(x))}{1-\cos(x)}\\
		&= \frac{x\sin(x)}{1 - \cos(x)} - 1
		\intertext{Wir wenden den Satz von Taylor in zweiter Ordnung genähert einmal nur auf die trigonometrischen Funktionen an und erhalten}
		&= \frac{x (x - \frac{x^3}{3} )}{1 - (1 - \frac{x^2}{2})} - 1\\
		&= \frac{x^2 - \frac{x^4}{3}}{\frac{x^2}{2}} -1\\
		&= 1 - \frac{2}{3}x^2
		\intertext{Nun machen wir es einmal mathematisch korrekt und wenden den Satz von Taylor auf den gesamten Ausdruck an. Dazu benötigen wir zunächst den Funktionswert an der Stelle 0.}
		f(x) &\coloneqq \frac{x\sin(x)}{1 - \cos(x)} - 1\\
		&= \frac{x\sin(x)(1 + \cos(x))}{\sin^2(x)} - 1\\
		&= \frac{x + x\cos(x)}{\sin(x)} - 1
		\intertext{Es gilt}
		2 &=\lim\limits_{x\to0} \frac{1 + \cos(x) - x\sin(x)}{\cos(x)}
		\intertext{Nach der Regel von L'Hospital erhalten wir also}
		2 & = \lim\limits_{x\to0} \frac{x + x\cos(x)}{\sin(x)}\\
		1 &= f(0)
		\intertext{Außerdem müssen wir noch die Ableitung an der Stelle 0 berechnen.}
		f'(x) &= \frac{(1 + \cos(x) - x\sin(x))\sin(x) - (x + x\cos(x))\cos(x)}{\sin^2(x)}
		\intertext{Es gilt}
		(x - 2 \sin(x))\sin(x)&= \frac{\d}{\d x} (1 + \cos(x) - x\sin(x))\sin(x) - (x + x\cos(x))\cos(x)\\
		\intertext{und daher mit der Regel von L'Hospital}
		0 &= \lim\limits_{x\to0} \frac{(x - 2 \sin(x))\sin(x)}{2\sin(x)\cos(x)}\\
		&= \lim\limits_{x\to0}\frac{(1 + \cos(x) - x\sin(x))\sin(x) - (x + x\cos(x))\cos(x)}{\sin^2(x)} = f'(x)
		\intertext{Zusätzlich benötigen wir noch die zweite Ableitung an der Stelle 0}
		f''(x) &= \frac{2 - x \frac{\cos(\frac{x}{2})}{\sin(\frac{x}{2})}}{\cos(x) -1}
		\intertext{Wegen}
		2 &= \lim\limits_{x\to0} \frac{\cos(\frac{x}{2}) - \frac{1}{2} x \sin(\frac{x}{2})}{\frac{1}{2}\cos(\frac{x}{2})}\\
		&= \lim\limits_{x\to0} x \frac{\cos(\frac{x}{2})}{\sin(\frac{x}{2})}
		\intertext{können wir auch bei der zweiten Ableitung die Regel von L'Hospital anwenden}
		\lim\limits_{x\to0} \frac{\frac{\d^4}{\d^4 x} 2 - x \frac{\cos(\frac{x}{2})}{\sin(\frac{x}{2})}}{\frac{\d^4}{\d^4 x} \cos(x) -1} &= \lim\limits_{x\to0} \frac{\frac{\d^3}{\d^3 x} \sin(x) - x}{\frac{\d^3}{\d^3 x} 2\sin(x)\sin^2(\frac{x}{2})}\\
		&= \lim\limits_{x\to0} \frac{\frac{\d^2}{\d^2 x}\cos(x) - 1}{\frac{\d^2}{\d^2 x}\cos(x) - \cos(2x)}\\
		&= \lim\limits_{x\to0} \frac{\frac{\d}{\d x} - \sin(x)}{\frac{\d}{\d x} -\sin(x) + 2\sin(2x)}\\
		&= \lim\limits_{x\to0} \frac{-\cos(x)}{-\cos(x) + 4 \cos(2x)}\\
		&= -\frac{1}{3}
		\intertext{Wir erhalten also}
		f''(0) = - \frac{1}{3}
		\intertext{Nun setzen wir dies in den Satz von Taylor ein}
		f(0) &\approx f(0) + f'(0)\cdot x + \frac{x^2}{2}f''(0) + O(x^3)\\
		&\approx 1 - \frac{x^2}{6} + O(x^3)
	\end{align*}
	Diese Auswertung zeigt zum einen, dass $f(x)$ für $|x| << 1$ gut konditioniert ist, zum anderen wird daraus klar, dass es wirklich einen Unterschied macht, ob man einfach nur die trigonometrischen Funktionen approximiert oder den ganzen Term. Ich würde einfach mal die (erstmal unbegründete) These aufstellen, dass es Funktionen gibt, für die eine Näherung auf die erste Art und Weise eine schlechte Konditionierung, die Näherung nach der zweiten Methode eine gute Konditionierung ergibt. Meine Frage wäre trotzdem: Muss in Zukunft so eine Analyse sein? Oder genügt die sehr viel schnellere Analyse mithilfe einer polynomiellen Näherung der trigonometrischen Funktionen.
	\item Wir berechnen
	\begin{align*}
		\frac{f_{\mathrm{rd}}(x) - f(x)}{f(x)} &= \frac{\frac{(1 - \cos(x)\cdot(1+\epsilon_1))\cdot (1+\epsilon_2)}{x}\cdot (1 + \epsilon_3) - \frac{1 - \cos(x)}{x}}{\frac{1 - \cos(x)}{x}}\\
		&= \frac{(1 + \epsilon_2)(1 + \epsilon_3) \cdot (1 - \cos(x)\cdot(1+\epsilon_1)) - (1 - \cos(x)}{1 - \cos(x)}\\
		&= \frac{(1 + \epsilon_2)(1 + \epsilon_3) \cdot (1 - \cos(x)) - (1 - \cos(x) - (1 + \epsilon_2)(1 + \epsilon_3)\epsilon_1\cos(x)}{1 - \cos(x)}\\
		&= (1 + \epsilon_2)(1 + \epsilon_3) - 1 - \frac{(1 + \epsilon_2)(1 + \epsilon_3)\epsilon_1\cos(x)}{1 - \cos(x)}
		\intertext{Betrachten wir also nur lineare Term in $\epsilon$, so erhalten wir}
		&= \epsilon_2 + \epsilon_3 - \frac{\cos(x)}{1 - \cos(x)}\epsilon_1\\
	\end{align*}
	Da $|x| << 1$, ist der Faktor vor $\epsilon_1$ sehr groß, obwohl der Algorithmus gut konditioniert ist. Dieser Algorithmus ist also instabil.
	\item Es gilt $$f(x) = \frac{1-\cos(x)}{x} = \frac{(1-\cos(x))(1 + \cos(x))}{x(1 + \cos(x))} = \frac{\sin^2(x)}{x(1 + \cos(x))}$$
	Wir erhalten damit 
	$$f_{\mathrm{rd}}(x) = \frac{\sin^2(x)(\epsilon_1 + 1)(\epsilon_2 + 1)\cdot (1 + \epsilon_6)}{(1 + \epsilon_5)(1 + \cos(x)(1 + \epsilon_3)) (1 + \epsilon_4)} = \frac{\sin^2(x)}{x(1 + \cos(x)(1 + \epsilon_3))} \cdot \underbrace{\frac{(1 + \epsilon_1)(1 + \epsilon_2)(1 + \epsilon_6)}{(1 + \epsilon_4)(1 + \epsilon_5)}}_{\xi}$$
	Nun berechnen wir
	\begin{align*}
		\frac{f_{\mathrm{rd}}(x) - f(x)}{f(x)} &= \frac{\frac{\sin^2(x)}{x(1 + \cos(x)(1 + \epsilon_3))} \cdot \xi - \frac{\sin^2(x)}{x(1 + \cos(x))}}{\frac{\sin^2(x)}{x(1 + \cos(x))}}\\
		&= \frac{1 + \cos(x)}{1 + \cos(x)(1 + \epsilon_3)} \cdot \xi - 1\\
		&= \frac{1 + \cos(x)(1 + \epsilon_3)}{1 + \cos(x)(1 + \epsilon_3)} \cdot \xi - 1 - \frac{\cos(x)\epsilon_3}{1 + \cos(x)(1 + \epsilon_3)} \cdot \xi\\
		&= \frac{(1 + \epsilon_1)(1 + \epsilon_2)(1 + \epsilon_6)}{(1 + \epsilon_4)(1 + \epsilon_5)} - 1 - \frac{\cos(x)\epsilon_3}{1 + \cos(x)(1 + \epsilon_3)} \cdot \xi
		\intertext{Als nächstes nutzen wir, dass $f(x) = \frac{a}{1+x}$ in erster Taylor-Näherung gleich $a(1 -x)$ ist und nähern linear in $\epsilon_5$}
		&\approx \frac{(1 + \epsilon_1)(1 + \epsilon_2)(1 + \epsilon_6)(1 - \epsilon_5)}{(1 + \epsilon_4)} - 1 - \frac{\cos(x)\epsilon_3}{1 + \cos(x)(1 + \epsilon_3)} \cdot \xi,\\
		\intertext{in $\epsilon_4$}
		&\approx (1 + \epsilon_1)(1 + \epsilon_2)(1 + \epsilon_6)(1- \epsilon_5)(1 - \epsilon_4) -1 - \frac{\cos(x)\epsilon_3}{1 + \cos(x)(1 + \epsilon_3)} \cdot \xi
		\intertext{und in $\epsilon_3$}
		&\approx (1 + \epsilon_1)(1 + \epsilon_2)(1 + \epsilon_6)(1- \epsilon_5)(1 - \epsilon_4) -1 - \frac{\cos(x)\epsilon_3}{1 + \cos(x)} \cdot \xi (1- \epsilon_3)
		\intertext{Entfernen wir nun alle Terme höherer Potenz in $\epsilon_i$, erhalten wir}
		&\epsilon_1 + \epsilon_2 + \epsilon_6 - \epsilon_5 - \epsilon_4 -\epsilon_3 \frac{\cos(x)}{1 + \cos(x)}
	\end{align*}
	Da alle Vorfaktoren betragsmäßig $\le 1$ sind bzw. der Vorfaktor von $\epsilon_3$ sogar ca. gleich $\frac{1}{2}$, ist dieser Algorithmus numerisch stabil.
\end{enumerate}
\section*{Aufgabe 3}
\begin{enumerate}
	\item Die Matrix $B$ ist eine $(N^2 - 1)\times (2N(N-1))$-Matrix, $A$ ist eine $(N^2-1)\times (N^2-1)$-Matrix.
	\item In der Matrix $B$ gibt es in allen Zeilen, deren Kante nicht am Referenzknoten anliegt, zwei Einträge ungleich 0, sonst nur einen. In der Matrix $A$ lässt sich das nicht so einfach sagen.
	\item Das Gleichungssystem $Ap = b$ ist eindeutig lösbar, da bei einer symmetrischen positiv definiten quadratischen reellen Matrix alle Eigenwerte größer 0 sind und somit auch $\det A > 0$ ist.
	\item $$ B = \begin{pmatrix}
		%1 & 2 & 3 & 4 & 5 & 6 & 7 & 8
		 1 & 0 & 0 & 0 & 0 & 0 & 0 & 0\\ %kante 1
		-1 & 1 & 0 & 0 & 0 & 0 & 0 & 0\\ %kante 2
		 0 & 0 &-1 & 1 & 0 & 0 & 0 & 0\\ %kante 3
		 0 & 0 & 0 &-1 & 1 & 0 & 0 & 0\\ %kante 4
		 0 & 0 & 0 & 0 & 0 &-1 & 1 & 0\\ %kante 5
		 0 & 0 & 0 & 0 & 0 & 0 &-1 & 1\\ %kante 6
		 0 & 0 & 1 & 0 & 0 & 0 & 0 & 0\\ %kante 7
		-1 & 0 & 0 & 1 & 0 & 0 & 0 & 0\\ %kante 8
		 0 &-1 & 0 & 0 & 1 & 0 & 0 & 0\\ %kante 9
		 0 & 0 &-1 & 0 & 0 & 1 & 0 & 0\\ %kante 10
		 0 & 0 & 0 &-1 & 0 & 0 & 1 & 0\\ %kante 11
		 0 & 0 & 0 & 0 &-1 & 0 & 0 & 1\\ %kante 12
	\end{pmatrix}$$
	$$B^T = \left(\begin{array}{cccccccccccc}
		1 &-1 & 0 & 0 & 0 & 0 & 0 &-1 & 0 & 0 & 0 & 0\\
		0 & 1 & 0 & 0 & 0 & 0 & 0 & 0 &-1 & 0 & 0 & 0\\
		0 & 0 &-1 & 0 & 0 & 0 & 1 & 0 & 0 &-1 & 0 & 0\\
		0 & 0 & 1 &-1 & 0 & 0 & 0 & 1 & 0 & 0 &-1 & 0\\
		0 & 0 & 0 & 1 & 0 & 0 & 0 & 0 & 1 & 0 & 0 &-1\\
		0 & 0 & 0 & 0 &-1 & 0 & 0 & 0 & 0 & 1 & 0 & 0\\
		0 & 0 & 0 & 0 & 1 &-1 & 0 & 0 & 0 & 0 & 1 & 0\\
		0 & 0 & 0 & 0 & 0 & 1 & 0 & 0 & 0 & 0 & 0 & 1
	\end{array}\right)$$
	$$ A = B^TB = \begin{pmatrix}
		3 &-1 & 0 &-1 & 0 & 0 & 0 & 0\\
		-1 & 2 & 0 & 0 &-1 & 0 & 0 & 0\\
		0 & 0 & 3 &-1 & 0 &-1 & 0 & 0\\
		-1 & 0 &-1 & 4 &-1 & 0 &-1 & 0\\
		0 &-1 & 0 &-1 & 3 & 0 & 0 &-1\\
		0 & 0 &-1 & 0 & 0 & 2 &-1 & 0\\
		0 & 0 & 0 &-1 & 0 &-1 & 3 &-1\\
		0 & 0 & 0 & 0 &-1 & 0 & 1 & 2
\end{pmatrix}$$
$b$ sieht folgendermaßen aus:
$$b = \begin{pmatrix}
	0\\0\\0\\0\\0\\0\\0\\q_p
\end{pmatrix}$$
Das Gleichungssystem lautet also : 
$$\begin{pmatrix}
	 3 &-1 & 0 &-1 & 0 & 0 & 0 & 0\\
	-1 & 2 & 0 & 0 &-1 & 0 & 0 & 0\\
	 0 & 0 & 3 &-1 & 0 &-1 & 0 & 0\\
	-1 & 0 &-1 & 4 &-1 & 0 &-1 & 0\\
	 0 &-1 & 0 &-1 & 3 & 0 & 0 &-1\\
	 0 & 0 &-1 & 0 & 0 & 2 &-1 & 0\\
	 0 & 0 & 0 &-1 & 0 &-1 & 3 &-1\\
	 0 & 0 & 0 & 0 &-1 & 0 & 1 & 2
\end{pmatrix} \cdot p = \begin{pmatrix}
	0\\0\\0\\0\\0\\0\\0\\q_p
\end{pmatrix}$$
Die Lösung ist daher 
$$p = \left(\begin{matrix}
\frac{q_{p}}{14}\\[0.5em]
\frac{3 q_{p}}{14}\\[0.5em]
- \frac{q_{p}}{14}\\[0.5em]
0\\[0.5em]
\frac{5 q_{p}}{14}\\[0.5em]
- \frac{3 q_{p}}{14}\\[0.5em]
- \frac{5 q_{p}}{14}\\[0.5em]
\frac{6 q_{p}}{7}
\end{matrix}\right)$$
\end{enumerate}
\section*{Aufgabe 4}
\begin{enumerate}[(a)]
	\item Wir nutzen aus, dass man Skalare einfach transponieren darf. 
	$$x^T \frac{1}{2}(A + A^T)x = \frac{1}{2}(x^TAx + x^TA^Tx) = \frac{1}{2}(x^TAx + (x^TA^Tx)^T) = \frac{1}{2}(x^TAx + x^TAx) = x^TAx$$
	Es gilt also $x^TAx> 0 \equals x^TA_Sx > 0$.
	\item Sei $N \coloneqq \{1, \dots n\}$ und $A$ positiv definit. Dann ist $\forall x\in \R^n: x^T Ax = \sum_{i,j \in N} x_i A_{ij}x_j > 0$. Es gilt: $$\forall x' \in \R^{|X|}: \exists x \in \R^n\text{ mit } x_i = 0\; \forall  i \notin X: \sum_{i,j = 1}^{|X|} x'_i (A_X)_{ij}x'_j = \sum_{i,j\in X} x_i a_{ij}x_j,$$ da die $x_i | i \notin X$ überhaupt nicht in der Summe vorkommen und man die restlichen $x_i$ nur umindizieren muss. Allerdings gilt auch $\forall x \in \R^n\text{ mit } x_i = 0\; \forall i \notin X$ $$\sum_{i,j\in X} x_i a_{ij}x_j = \sum_{i,j\in N} x_i a_{ij}x_j \overset{A \text{ positiv definit}}{>} 0,$$ da hier nur Nullterme hinzukommen. Wegen den oben gezeigten Gleichheit gilt also auch 
	$$\forall x' \in \R^{|X|}: \sum_{i,j = 1}^{|X|} x'_i (A_X)_{ij}x'_j >0.$$ Daraus folgt aber sofort, dass $A_X$ positiv definit sein muss.
	\item Wenn $A$ positiv definit ist, dann gilt$\begin{pmatrix}
		x & y
	\end{pmatrix}\cdot \begin{pmatrix}
		2 & 1\\
		-(1+\alpha) & 2
	\end{pmatrix} \cdot \begin{pmatrix}
		x \\y
	\end{pmatrix} = 2x^2 - (1 + \alpha)yx + 2y^2 = 2 (x^2 - \frac{\alpha}{2}xy + y^2) >0$
	Sei nun $\alpha < 4$. Dann gilt $2 (x^2 - \frac{\alpha}{2}xy + y^2) > 2 (x^2 - \frac{4}{2}xy + y^2) = (x-y)^2 \geq 0$. Sei nun $\alpha \geq 4$. Wähle dann $x = y = 1$. Dann erhält man $2(x^2 - \frac{\alpha}{2}xy + y^2) = 2(2 - \frac{\alpha}{2}) \leq 2(2-\frac{4}{2}) = 0$. Also ist $A$ genau für $\alpha <4$ positiv definit.
	\item Es gilt $\overline{x}^TAx = \overline{x}^T\overline{H}^THx = \overline{(Hx)}^T(Hx) = \Vert Hx \Vert_2^2 > 0 \equals Hx \neq 0$. Daraus folgt: $A$ positiv definit $\equals \forall x\in \C^n\setminus\{0\}: Hx \neq 0 \equals \rg(H) = n$.
\end{enumerate}
\end{document}