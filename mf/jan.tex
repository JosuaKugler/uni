\documentclass{article}
\usepackage{josuamathheader}

\begin{document}
\section{Aufgabe 1}
    \noindent Gegeben ist die quadratische Funktion $f$ mit $f(x) = -2x^2 + 12x - 12$.
    Bestimmen Sie mit Hilfe der Scheitelpunktform der Parabelgleichung den Scheitelpunkt $S$.
    \begin{align*}
        y &= -2x^2 + 12x - 12\\
        y &= -2(x^2 - 6x +6)
        \intertext{ Es gilt $a = x$ und $b = 6/2 = 3$.}
        y &= -2[(x^2 - 2 \cdot x \cdot 3 + 3^2) - 3^2 + 6]\\
        y &= -2[(x-3)^2 - 3^2 + 6]\\
        y &= -2[(x-3)^2 - 3]\\
        y &= -2(x-3)^2 + 6
    \end{align*}
    Es folgt $a = -2$ und der Scheitelpunkt liegt bei $S(3|6)$.

\section{Aufgabe 2}
    \noindent Gegeben ist die quadratische Funktion $f$ mit $f(x) = 4x^2 + 12x +9$.
    Bestimmen Sie mit Hilfe der Scheitelpunktform der Parabelgleichung den Scheitelpunkt $S$.
    \begin{align*}
        y &= 4x^2 + 12x +9 \\
        y &= 4(x^2 + 3x + 2,\!25)
        \intertext{ Es gilt $a = x$ und $b = 3/2 = 1,\!5$.}
        y &= 4[(x^2 - 2 \cdot x \cdot 1,\!5 + 1,\!5^2) - 1,\!5^2 + 2.25]\\
        y &= 4[(x-1,\!5)^2 - 1,\!5^2 + 2.25]\\
        y &= 4[(x-1,\!5)^2 + 0]\\
        y &= 4(x-1,\!5)^2 
    \end{align*}
    Es folgt $a = 4$ und der Scheitelpunkt liegt bei $S(3|0)$.
\end{document}