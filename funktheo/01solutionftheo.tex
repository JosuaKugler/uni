\documentclass{article}

\usepackage[utf8]{inputenc}
\usepackage[T1]{fontenc}
\usepackage[ngerman]{babel}
\usepackage{amsmath, amsfonts, amsthm, mathtools, amssymb}
\usepackage{stmaryrd}
\usepackage{enumerate}
\usepackage{cases}
\usepackage{fancyhdr}
\usepackage{comment}
%\usepackage{xcolor}
\usepackage{tikz}
\usepackage{cases}
\usepackage{listings}
\usepackage{siunitx}
\usepackage[left = 3cm]{geometry}
\usepackage[hidelinks]{hyperref}
\usepackage{subcaption}
\usepackage{gauss}
\newtheorem{satz}{Satz}[section]
\newtheorem{lemma}[satz]{Lemma}
\newtheorem{korollar}[satz]{Korollar}
\newtheorem{proposition}[satz]{Proposition}
\theoremstyle{definition}
\newtheorem{definition}[satz]{Def.}
\newtheorem{axiom}[satz]{Axiom}
\newtheorem{bsp}[satz]{Bsp.}
\newtheorem*{anmerkung}{Anm}
\newtheorem{bemerkung}[satz]{Bem}
\newtheorem*{notatio}{Notation}
\newcommand{\obda}{O.B.d.A. }
\newcommand{\equals}{\Longleftrightarrow}
\newcommand{\N}{\mathbb{N}}
\newcommand{\Q}{\mathbb{Q}}
\newcommand{\R}{\mathbb{R}}
\newcommand{\Z}{\mathbb{Z}}
\newcommand{\C}{\mathbb{C}}
\newcommand{\intd}{\mathrm{d}}
\newcommand{\Pot}{\operatorname{Pot}}
\newcommand{\mychar}{\operatorname{char}}
\newcommand{\myker}{\operatorname{ker}}
\newcommand{\induktion}[3]
{\begin{proof}\ \\
	\noindent\textbf{Induktionsanfang:}\ #1\\
	\noindent\textbf{Induktionsvoraussetzung:}\ #2\\
	\noindent\textbf{Induktionsschluss:}\ #3
\end{proof}}

\newcommand{\rg}{\operatorname{rg}}
\newcommand{\im}{\operatorname{im}}
\newcommand{\End}{\operatorname{End}}
\newcommand{\abb}{\operatorname{Abb}}
\newcommand{\re}{\operatorname{Re}}
\newcommand{\Ima}{\operatorname{Im}}


\makeatletter \renewcommand\d[2][]{\ensuremath{%
		\,\mathrm{d}^{#1}#2\@ifnextchar^{}{\@ifnextchar\d{}{\,}}}}
\makeatother

\newcommand{\ipilayout}[1]
{	
	\pagestyle{fancy}
	\fancyhead[L]{Einführung in die praktische Informatik, Blatt #1}
	\fancyhead[R]{Josua Kugler, Jan Metzger, David Wesner}
	\fancypagestyle{firstpage}{%
		\fancyhf{}
		\lhead{Professor: Peter Bastian\\
			Tutor: Frederick Schenk}
		\rhead{Einführung in die praktische Informatik, Übungsblatt #1\\ David, Jan, Josua}
		\cfoot{\thepage}
	}
\thispagestyle{firstpage}
}

\newcommand{\analayout}[1]
{	
	\pagestyle{fancy}
	\fancyhead[L]{Analysis 1, Blatt #1}
	\fancyhead[R]{Alexander Bryant, Josua Kugler}
	\fancypagestyle{firstpage}{%
		\fancyhf{}
		\lhead{Professor: Ekaterina Kostina\\
			Tutor: Philipp Elja Müller}
		\rhead{Analysis 1, Übungsblatt #1\\ Alexander Bryant, Josua Kugler}
		\cfoot{\thepage}
	}
	\thispagestyle{firstpage}
}
\newcommand{\lalayout}[1]
{	
	\pagestyle{fancy}
	\fancyhead[L]{Lineare Algebra 1, Blatt #1}
	\fancyhead[R]{David Wesner, Josua Kugler}
	\fancypagestyle{firstpage}{%
		\fancyhf{}
		\lhead{Professor: Denis Vogel\\
			Tutor: Marina Savarino}
		\rhead{Lineare Algebra 2, Übungsblatt #1\\ David Wesner, Josua Kugler}
		\cfoot{\thepage}
	}
	\thispagestyle{firstpage}
}

\lstset{
    frame=tb, % draw a frame at the top and bottom of the code block
    tabsize=4, % tab space width
    showstringspaces=false, % don't mark spaces in strings
    numbers=left, % display line numbers on the left
    commentstyle=\color{green}, % comment color
    keywordstyle=\color{blue}, % keyword color
    stringstyle=\color{red} % string color
}
\setlength{\headheight}{25pt}
\begin{document}

\section*{Aufgabe 15}
\begin{enumerate}[(a)]
	\item \begin{itemize}
		\item Die Stetigkeit von $\alpha$ folgt unmittelbar aus der Stetigkeit der komplexen Exponentialfunktion. Außerdem ist $\alpha(1) = e^{2\pi i 1} = 1 = e^{2\pi i 0} = \alpha(0)$. Daher ist $\alpha$ auch geschlossen.
		\item Da lineare Funktionen stetig sind, genügt es, die Stetigkeit von $\gamma$ an den Stellen $\frac{1}{3}$ und $\frac{2}{3}$ nachzuweisen.
		\begin{itemize}
			\item[$\frac{1}{3}$:] $\gamma\left(\frac{1}{3}\right) = 1$. Außerdem ist $\lim\limits_{x\nearrow\frac{1}{3}} \gamma(t) = 1 = \lim\limits_{x\searrow \frac{1}{3}}$.
			\item[$\frac{2}{3}$:] $\gamma\left(\frac{2}{3}\right) = i$.  Außerdem ist $\lim\limits_{x\nearrow\frac{2}{3}} \gamma(t) = i = \lim\limits_{x\searrow \frac{2}{3}}$.
		\end{itemize}
		Wegen $\gamma(1) = 0 = \gamma(0)$ ist $\gamma$ außerdem geschlossen.
	\end{itemize}
	\item Es gilt $$\oint_\alpha \frac{1}{z} \d z = \int_0^1 e^{-2\pi i t}\cdot 2\pi e^{2\pi i t} \d t = 2\pi i$$ und
	\begin{align*}
		\oint_\gamma z^2 \d z &= \sum_{\nu = 0}^{2} \int_{\frac{1}{3} \nu}^{\frac{1}{3}(\nu +1)} f(\gamma(t)) \frac{\d \gamma}{\d t}\d t\\
		&= \int_0^\frac{1}{3} (3t)^2 \cdot 3 \d t + \int_\frac{1}{3}^\frac{2}{3} (3 (i-1)t - i+2)^2 \cdot 3(i-1) \d t + \int_\frac{2}{3}^1 (3 i (t-1))^2 \cdot - 3i \d t\\
		&= 9 t^3\bigg|_0^\frac{1}{3}\!\!\!+ 3(i-1)\! \int_\frac{1}{3}^\frac{2}{3}\! 9 (i-1)^2 t^2 - 6(i-1)(i+2)t + (i+2)^2 \d t + 27i\! \int_\frac{2}{3}^1 (t^2 - 2t + 1) \d t\\
		&= \frac{1}{3} + 3(i-1) \int_\frac{1}{3}^\frac{2}{3} 9 (1^2 + i^2 -2i)t^2 - 6(i-3)t + (4 +i^2-2i) \d t + 27i  \bigg[\frac{1}{3}t^3 - t^2 + t\bigg]_\frac{2}{3}^1\\
		&= \frac{1}{3} + 3(i-1) \int_\frac{1}{3}^\frac{2}{3} -18it^2 - 6(i-3)t + (3-2i) \d t + 27i  \bigg[\frac{1}{3} - 1 + 1-\left(\frac{8}{3^4} - \frac{4}{9} + \frac{2}{3}\right)\bigg]\\
		&= \frac{1}{3} + i \left(9 -\frac{8}{3} + 12 - 18\right)+ 3(i-1)  \bigg[-6it^3 - 3(i-3)t^2 + (3-2i)t\bigg]_\frac{1}{3}^\frac{2}{3}\\
		&= \frac{1}{3}(i+1) + (i-1) \left[-i \frac{16}{3} -(i-3)4 + (3-2i)2 - \left(-i \frac{2}{3} -(i-3) + (3-2i)\right) \right]\\
		&= \frac{1}{3}(i+1) + (i-1) \left[-i \frac{14}{3} -(i-3)3 + (3-2i)\right]\\
		&= \frac{1}{3}(i+1) + (i-1) \left[-i \frac{29}{3}+12\right]\\
		&= \frac{1}{3}(i+1) + \frac{29}{3} + 12 i + \frac{29}{3}i - 12\\
		&= 22i -2
	\end{align*}
\end{enumerate}

\section*{Aufgabe 18}
Es gilt $$\frac{\partial v}{\partial y} = \frac{\partial u}{\partial x} = 2x + 2ay.$$ Daraus folgt $$v(x + iy) = 2xy + ay^2 + C(x),$$ wobei $C(x)$ nur von $x$ abhängen darf. Außerdem erhalten wir $$\frac{\partial v}{\partial x} = - \frac{\partial u}{\partial y} = - 2ax - 2by.$$
Also muss gelten $$v(x + iy) = -ax^2 - 2bxy + D(y),$$ wobei $D(y)$ nur von $y$ abhängen darf. 
Gleichsetzen ergibt also $$-ax^2 - 2bxy + D(y) = 2xy + ay^2 + C(x) \equals -2bxy = 2xy \equals b = -1.$$
Für alle $a, b$ mit $b = -1$ ist $x^2 + 2axy + by^2$ der Realteil einer holomorphen Funktion.
\end{document}