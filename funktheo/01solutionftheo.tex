\documentclass{article}

\usepackage[utf8]{inputenc}
\usepackage[T1]{fontenc}
\usepackage[ngerman]{babel}
\usepackage{amsmath, amsfonts, amsthm, mathtools, amssymb}
\usepackage{stmaryrd}
\usepackage{enumerate}
\usepackage{cases}
\usepackage{fancyhdr}
\usepackage{comment}
%\usepackage{xcolor}
\usepackage{tikz}
\usepackage{cases}
\usepackage{listings}
\usepackage{siunitx}
\usepackage[left = 3cm]{geometry}
\usepackage[hidelinks]{hyperref}
\usepackage{subcaption}
\usepackage{gauss}
\newtheorem{satz}{Satz}[section]
\newtheorem{lemma}[satz]{Lemma}
\newtheorem{korollar}[satz]{Korollar}
\newtheorem{proposition}[satz]{Proposition}
\theoremstyle{definition}
\newtheorem{definition}[satz]{Def.}
\newtheorem{axiom}[satz]{Axiom}
\newtheorem{bsp}[satz]{Bsp.}
\newtheorem*{anmerkung}{Anm}
\newtheorem{bemerkung}[satz]{Bem}
\newtheorem*{notatio}{Notation}
\newcommand{\obda}{O.B.d.A. }
\newcommand{\equals}{\Longleftrightarrow}
\newcommand{\N}{\mathbb{N}}
\newcommand{\Q}{\mathbb{Q}}
\newcommand{\R}{\mathbb{R}}
\newcommand{\Z}{\mathbb{Z}}
\newcommand{\C}{\mathbb{C}}
\newcommand{\intd}{\mathrm{d}}
\newcommand{\Pot}{\operatorname{Pot}}
\newcommand{\mychar}{\operatorname{char}}
\newcommand{\myker}{\operatorname{ker}}
\newcommand{\induktion}[3]
{\begin{proof}\ \\
	\noindent\textbf{Induktionsanfang:}\ #1\\
	\noindent\textbf{Induktionsvoraussetzung:}\ #2\\
	\noindent\textbf{Induktionsschluss:}\ #3
\end{proof}}

\newcommand{\rg}{\operatorname{rg}}
\newcommand{\im}{\operatorname{im}}
\newcommand{\End}{\operatorname{End}}
\newcommand{\abb}{\operatorname{Abb}}
\newcommand{\re}{\operatorname{Re}}
\newcommand{\Ima}{\operatorname{Im}}


\makeatletter \renewcommand\d[2][]{\ensuremath{%
		\,\mathrm{d}^{#1}#2\@ifnextchar^{}{\@ifnextchar\d{}{\,}}}}
\makeatother

\newcommand{\ipilayout}[1]
{	
	\pagestyle{fancy}
	\fancyhead[L]{Einführung in die praktische Informatik, Blatt #1}
	\fancyhead[R]{Josua Kugler, Jan Metzger, David Wesner}
	\fancypagestyle{firstpage}{%
		\fancyhf{}
		\lhead{Professor: Peter Bastian\\
			Tutor: Frederick Schenk}
		\rhead{Einführung in die praktische Informatik, Übungsblatt #1\\ David, Jan, Josua}
		\cfoot{\thepage}
	}
\thispagestyle{firstpage}
}

\newcommand{\analayout}[1]
{	
	\pagestyle{fancy}
	\fancyhead[L]{Analysis 1, Blatt #1}
	\fancyhead[R]{Alexander Bryant, Josua Kugler}
	\fancypagestyle{firstpage}{%
		\fancyhf{}
		\lhead{Professor: Ekaterina Kostina\\
			Tutor: Philipp Elja Müller}
		\rhead{Analysis 1, Übungsblatt #1\\ Alexander Bryant, Josua Kugler}
		\cfoot{\thepage}
	}
	\thispagestyle{firstpage}
}
\newcommand{\lalayout}[1]
{	
	\pagestyle{fancy}
	\fancyhead[L]{Lineare Algebra 1, Blatt #1}
	\fancyhead[R]{David Wesner, Josua Kugler}
	\fancypagestyle{firstpage}{%
		\fancyhf{}
		\lhead{Professor: Denis Vogel\\
			Tutor: Marina Savarino}
		\rhead{Lineare Algebra 2, Übungsblatt #1\\ David Wesner, Josua Kugler}
		\cfoot{\thepage}
	}
	\thispagestyle{firstpage}
}

\lstset{
    frame=tb, % draw a frame at the top and bottom of the code block
    tabsize=4, % tab space width
    showstringspaces=false, % don't mark spaces in strings
    numbers=left, % display line numbers on the left
    commentstyle=\color{green}, % comment color
    keywordstyle=\color{blue}, % keyword color
    stringstyle=\color{red} % string color
}
\setlength{\headheight}{25pt}
\begin{document}

\section*{Aufgabe 23}
Sei $F(z) = f(z)g(z)$. Dann ist $F'(z) = f'(z)g(z) + f(z)g'(z)$. Also ist $\oint_\gamma f'(z)g(z) + f(z)g'(z) \d z = \oint_\gamma F'(z) \d z = 0$. Aufgrund der Linearität des Integrals folgt \[\oint_\gamma f(z)g'(z)\d z + \oint_\gamma f'(z)g(z) \d z= 0 \Leftrightarrow \oint_\gamma f(z)g'(z) \d z= - \oint_\gamma f'(z)g(z)\d z.\]

\section*{Aufgabe 26}
Wir wenden die Cauchysche Integralformel an mit $z_0 = a$, da alle Bedingungen nach Aufgabenstellung erfüllt sind und erhalten für $\varphi\colon [0,1] \to D, t \mapsto a + re^{2\pi i t}$ folgende Gleichung
\[f(a) = \frac{1}{2\pi i}\int_\varphi \frac{f(z)}{z-a}\d z = \frac{1}{2\pi i}\int_0^1 \frac{f(a + re^{2\pi i t})}{a + re^{2\pi i t} -a} \cdot 2\pi i\cdot re^{2\pi i t} \d t = \int_0^1 f(a + re^{2\pi i t}) \d t.\]

\section*{Aufgabe 27}
Sei $z_0$ ein Sternmittelpunkt von $E$. Es gilt $b'(z) \neq 0\forall z \in \C$, da eine holomorphe Funktion mit Ableitung 0 lokalkonstant ist. Da aber $b$ insbesondere injektiv ist, erhielten wir daraus einen Widerspruch. Wir definieren $F(z)\coloneqq \frac{1}{b'(z)}\int_{z_0}^{b(z)} f(b^{-1}((\xi))\d \xi$. Nun berechnen wir $F'(z)$. Es gilt 
\begin{align*}
	F'(z) &= \lim\limits_{\epsilon\to 0} \frac{1}{\epsilon}\frac{1}{b'(z)}\left(\int_{z_0}^{b(z + \epsilon)} f(b^{-1}(\xi)) \d \xi - \int_{z_0}^{b(z)} f(b^{-1}(\xi))\d \xi\right)
	\intertext{Nach Eigenschaft E2, und da $E$ sternförmig ist, können wir analog zum Beweis von E3 schreiben}
	&= \lim\limits_{\epsilon\to 0} \frac{1}{\epsilon}\frac{1}{b'(z)}\left(\int_{b(z)}^{b(z+\epsilon)} f(b^{-1}(\xi)) \d \xi \right)
	\intertext{Den Weg $\gamma : [0,1] \to E$ von $b(z)$ nach $b(z + \epsilon)$ parametrisieren wir durch $\gamma(t) = b(z) + t \cdot (b(z+ \epsilon) - b(z))$}
	&= \lim\limits_{\epsilon\to 0} \frac{1}{\epsilon}\frac{1}{b'(z)}\left(\int_{0}^{1} f(b^{-1}(\gamma(t))) \gamma'(t) \d t \right)\\
	&= \lim\limits_{\epsilon\to 0} \frac{(b(z+ \epsilon) - b(z))}{\epsilon}\frac{1}{b'(z)} \int_{0}^{1}  f(b^{-1}(b(z) + t \cdot (b(z+ \epsilon) - b(z)))) \d t\\
	&= \lim\limits_{\epsilon\to 0} \frac{b'(z)}{b'(z)}\cdot \int_{0}^{1}  f(b^{-1}(b(z) + t \cdot (b(z+ \epsilon) - b(z)))) \d t
	\intertext{Aufgrund der Stetigkeit von $f$ gilt}
	&= \int_{0}^{1}  f\left(\lim\limits_{\epsilon\to 0} b^{-1}(b(z) + t \cdot (b(z+ \epsilon) - b(z)))\right) \d t\\
	\intertext{Da $b$ und $b^{-1}$ ebenfalls stetig sind, folgt schließlich}
	&= \int_{0}^{1} f\left(b^{-1}(b(z) + t \cdot (b(\lim\limits_{\epsilon\to 0} z+ \epsilon) - b(z)))\right) \d t\\
	&= \int_{0}^{1} f\left(b^{-1}(b(z) + t \cdot (b(z) - b(z)))\right) \d t\\
	&= \int_{0}^{1} f\left(b^{-1}(b(z))\right) \d t\\
	&= \int_{0}^{1} f\left(z\right) \d t\\
	&= f(z)
\end{align*}
Also ist $F$ eine Stammfunktion von $f$ auf $D$.
\end{document}