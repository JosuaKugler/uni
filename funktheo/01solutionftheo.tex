\documentclass{article}

\usepackage[utf8]{inputenc}
\usepackage[T1]{fontenc}
\usepackage[ngerman]{babel}
\usepackage{amsmath, amsfonts, amsthm, mathtools, amssymb}
\usepackage{stmaryrd}
\usepackage{enumerate}
\usepackage{cases}
\usepackage{fancyhdr}
\usepackage{comment}
%\usepackage{xcolor}
\usepackage{tikz}
\usepackage{cases}
\usepackage{listings}
\usepackage{siunitx}
\usepackage[left = 3cm]{geometry}
\usepackage[hidelinks]{hyperref}
\usepackage{subcaption}
\usepackage{gauss}
\newtheorem{satz}{Satz}[section]
\newtheorem{lemma}[satz]{Lemma}
\newtheorem{korollar}[satz]{Korollar}
\newtheorem{proposition}[satz]{Proposition}
\theoremstyle{definition}
\newtheorem{definition}[satz]{Def.}
\newtheorem{axiom}[satz]{Axiom}
\newtheorem{bsp}[satz]{Bsp.}
\newtheorem*{anmerkung}{Anm}
\newtheorem{bemerkung}[satz]{Bem}
\newtheorem*{notatio}{Notation}
\newcommand{\obda}{O.B.d.A. }
\newcommand{\equals}{\Longleftrightarrow}
\newcommand{\N}{\mathbb{N}}
\newcommand{\Q}{\mathbb{Q}}
\newcommand{\R}{\mathbb{R}}
\newcommand{\Z}{\mathbb{Z}}
\newcommand{\C}{\mathbb{C}}
\newcommand{\intd}{\mathrm{d}}
\newcommand{\Pot}{\operatorname{Pot}}
\newcommand{\mychar}{\operatorname{char}}
\newcommand{\myker}{\operatorname{ker}}
\newcommand{\induktion}[3]
{\begin{proof}\ \\
	\noindent\textbf{Induktionsanfang:}\ #1\\
	\noindent\textbf{Induktionsvoraussetzung:}\ #2\\
	\noindent\textbf{Induktionsschluss:}\ #3
\end{proof}}

\newcommand{\rg}{\operatorname{rg}}
\newcommand{\im}{\operatorname{im}}
\newcommand{\End}{\operatorname{End}}
\newcommand{\abb}{\operatorname{Abb}}
\newcommand{\re}{\operatorname{Re}}
\newcommand{\Ima}{\operatorname{Im}}


\makeatletter \renewcommand\d[2][]{\ensuremath{%
		\,\mathrm{d}^{#1}#2\@ifnextchar^{}{\@ifnextchar\d{}{\,}}}}
\makeatother

\newcommand{\ipilayout}[1]
{	
	\pagestyle{fancy}
	\fancyhead[L]{Einführung in die praktische Informatik, Blatt #1}
	\fancyhead[R]{Josua Kugler, Jan Metzger, David Wesner}
	\fancypagestyle{firstpage}{%
		\fancyhf{}
		\lhead{Professor: Peter Bastian\\
			Tutor: Frederick Schenk}
		\rhead{Einführung in die praktische Informatik, Übungsblatt #1\\ David, Jan, Josua}
		\cfoot{\thepage}
	}
\thispagestyle{firstpage}
}

\newcommand{\analayout}[1]
{	
	\pagestyle{fancy}
	\fancyhead[L]{Analysis 1, Blatt #1}
	\fancyhead[R]{Alexander Bryant, Josua Kugler}
	\fancypagestyle{firstpage}{%
		\fancyhf{}
		\lhead{Professor: Ekaterina Kostina\\
			Tutor: Philipp Elja Müller}
		\rhead{Analysis 1, Übungsblatt #1\\ Alexander Bryant, Josua Kugler}
		\cfoot{\thepage}
	}
	\thispagestyle{firstpage}
}
\newcommand{\lalayout}[1]
{	
	\pagestyle{fancy}
	\fancyhead[L]{Lineare Algebra 1, Blatt #1}
	\fancyhead[R]{David Wesner, Josua Kugler}
	\fancypagestyle{firstpage}{%
		\fancyhf{}
		\lhead{Professor: Denis Vogel\\
			Tutor: Marina Savarino}
		\rhead{Lineare Algebra 2, Übungsblatt #1\\ David Wesner, Josua Kugler}
		\cfoot{\thepage}
	}
	\thispagestyle{firstpage}
}

\lstset{
    frame=tb, % draw a frame at the top and bottom of the code block
    tabsize=4, % tab space width
    showstringspaces=false, % don't mark spaces in strings
    numbers=left, % display line numbers on the left
    commentstyle=\color{green}, % comment color
    keywordstyle=\color{blue}, % keyword color
    stringstyle=\color{red} % string color
}
\setlength{\headheight}{25pt}
\begin{document}

\section*{Aufgabe 24}
\begin{enumerate}[(a)]
	\item Angenommen, es gäbe ein $z \in \C$, sodass $\forall \epsilon > 0: |z - f(\epsilon)| > \epsilon$. Dies können wir umformen zu $$\frac{1}{\epsilon} > \underbrace{\frac{1}{|z - f(\zeta)|}}_{\text{holomorph, da }z\neq f(\zeta)}.$$ Wegen des Satzes von Liouville muss $ \frac{1}{|z - f(\zeta)|} = \text{const}$ sein. Also ist auch $\text{const} = |z - f(\zeta)| = |f(\zeta) - z| \geq |f(\zeta)| - |z|$ und daher $|f(\zeta)| < \text{const}$, also ist $f$ beschränkt und daher nach Liouville konstant. Die Kontraposition war zu zeigen.
	\item Sei $z = x + iy\in \C$. Dann gilt $z = k + i\cdot l + q + i \cdot s$ mit $k, l \in \Z$ und $0 \leq q, s \leq 1$. Aufgrund der in der Aufgabenstellung beschriebenen Eigenschaft, ist also $f(z) = f(k + i\cdot l + q + i \cdot s) = f(q + s\cdot i)$. Aus Holomorphie folgt Stetigkeit, also ist $f(M)$ kompakt für $M = \{q + s \cdot i|0\leq q,s \leq 1\}$. Daher gibt es ein $C \in \C$, sodass $\sup\limits_{\zeta \in M} \zeta < C$. Es gilt folglich $f(z) = f(q + s\cdot i) < C$. Also ist $f$ beschränkt und nach dem Satz von Liouville konstant.
\end{enumerate}
\item Wir betrachten die Funktion $h(z) \coloneqq \frac{f(z)}{g(z)}$. Hat $g$ eine Nullstelle, so nennen wir diese $\zeta$. $h(z)$ ist also wohldefiniert für alle $z \in \C \setminus\{\zeta\}$. In dieser Menge ist auch stets $|h(z)| \leq 1$, da $|f(z)| \leq |g(z)|\; \forall z \in \C$. Soll $h$ holomorph sein, so muss die Cauchy'sche Integralformel gelten: $h(\zeta) = \frac{1}{2\pi i} \oint_\varphi \frac{h(z)}{z - \zeta} \d z$

\end{document}