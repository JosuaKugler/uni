\documentclass{article}

\usepackage[utf8]{inputenc}
\usepackage[T1]{fontenc}
\usepackage[ngerman]{babel}
\usepackage{amsmath, amsfonts, amsthm, mathtools, amssymb}
\usepackage{stmaryrd}
\usepackage{enumerate}
\usepackage{cases}
\usepackage{fancyhdr}
\usepackage{comment}
%\usepackage{xcolor}
\usepackage{tikz}
\usepackage{cases}
\usepackage{listings}
\usepackage{siunitx}
\usepackage[left = 3cm]{geometry}
\usepackage[hidelinks]{hyperref}
\usepackage{subcaption}
\usepackage{gauss}
\newtheorem{satz}{Satz}[section]
\newtheorem{lemma}[satz]{Lemma}
\newtheorem{korollar}[satz]{Korollar}
\newtheorem{proposition}[satz]{Proposition}
\theoremstyle{definition}
\newtheorem{definition}[satz]{Def.}
\newtheorem{axiom}[satz]{Axiom}
\newtheorem{bsp}[satz]{Bsp.}
\newtheorem*{anmerkung}{Anm}
\newtheorem{bemerkung}[satz]{Bem}
\newtheorem*{notatio}{Notation}
\newcommand{\obda}{O.B.d.A. }
\newcommand{\equals}{\Longleftrightarrow}
\newcommand{\N}{\mathbb{N}}
\newcommand{\Q}{\mathbb{Q}}
\newcommand{\R}{\mathbb{R}}
\newcommand{\Z}{\mathbb{Z}}
\newcommand{\C}{\mathbb{C}}
\newcommand{\intd}{\mathrm{d}}
\newcommand{\Pot}{\operatorname{Pot}}
\newcommand{\mychar}{\operatorname{char}}
\newcommand{\myker}{\operatorname{ker}}
\newcommand{\induktion}[3]
{\begin{proof}\ \\
	\noindent\textbf{Induktionsanfang:}\ #1\\
	\noindent\textbf{Induktionsvoraussetzung:}\ #2\\
	\noindent\textbf{Induktionsschluss:}\ #3
\end{proof}}

\newcommand{\rg}{\operatorname{rg}}
\newcommand{\im}{\operatorname{im}}
\newcommand{\End}{\operatorname{End}}
\newcommand{\abb}{\operatorname{Abb}}
\newcommand{\re}{\operatorname{Re}}
\newcommand{\Ima}{\operatorname{Im}}



\newcommand{\ipilayout}[1]
{	
	\pagestyle{fancy}
	\fancyhead[L]{Einführung in die praktische Informatik, Blatt #1}
	\fancyhead[R]{Josua Kugler, Jan Metzger, David Wesner}
	\fancypagestyle{firstpage}{%
		\fancyhf{}
		\lhead{Professor: Peter Bastian\\
			Tutor: Frederick Schenk}
		\rhead{Einführung in die praktische Informatik, Übungsblatt #1\\ David, Jan, Josua}
		\cfoot{\thepage}
	}
\thispagestyle{firstpage}
}

\newcommand{\analayout}[1]
{	
	\pagestyle{fancy}
	\fancyhead[L]{Analysis 1, Blatt #1}
	\fancyhead[R]{Alexander Bryant, Josua Kugler}
	\fancypagestyle{firstpage}{%
		\fancyhf{}
		\lhead{Professor: Ekaterina Kostina\\
			Tutor: Philipp Elja Müller}
		\rhead{Analysis 1, Übungsblatt #1\\ Alexander Bryant, Josua Kugler}
		\cfoot{\thepage}
	}
	\thispagestyle{firstpage}
}
\newcommand{\lalayout}[1]
{	
	\pagestyle{fancy}
	\fancyhead[L]{Lineare Algebra 1, Blatt #1}
	\fancyhead[R]{David Wesner, Josua Kugler}
	\fancypagestyle{firstpage}{%
		\fancyhf{}
		\lhead{Professor: Denis Vogel\\
			Tutor: Marina Savarino}
		\rhead{Lineare Algebra 2, Übungsblatt #1\\ David Wesner, Josua Kugler}
		\cfoot{\thepage}
	}
	\thispagestyle{firstpage}
}

\lstset{
    frame=tb, % draw a frame at the top and bottom of the code block
    tabsize=4, % tab space width
    showstringspaces=false, % don't mark spaces in strings
    numbers=left, % display line numbers on the left
    commentstyle=\color{green}, % comment color
    keywordstyle=\color{blue}, % keyword color
    stringstyle=\color{red} % string color
}
\setlength{\headheight}{25pt}
\begin{document}

\section*{Aufgabe 2}
\begin{enumerate}[(a)]
	\item 
\end{enumerate}
\section*{Aufgabe 3}
Die Lösungen sind $2 \cdot e\left(\frac{\pi}{3}\right), 2 \cdot e\left(\pi\right)$ und $2 \cdot e\left(\frac{5\pi}{3}\right)$, da $\left(2 \cdot e\left(\frac{\pi}{3}\right)\right)^3 = 8 \cdot(e(\pi)) = -8, \left(2 \cdot e\left(\pi\right)\right)^3 = 8 \cdot e(3\pi) = -8$ und schließlich $\left(2 \cdot e\left(\frac{5\pi}{3}\right)\right)^3 = 8 \cdot(e(5\pi)) = -8$. Mehr Lösungen gibt es nicht, da es sich um ein Polynom 3. Grades handelt.
\section*{Aufgabe 4}
\begin{align*}
	(ad - bc) \cdot \Ima(z) &= \Ima(ad\cdot z - bc \cdot z)\\
	 \det(M) \cdot \Ima(z) &= \Ima(ac\cdot z\overline{z} + ad\cdot z + bc \cdot \overline{z} + b\cdot d)&&\Leftrightarrow\\
	 \det(M) \cdot \Ima(z) &= \Ima((az + b) \cdot \overline{cz + d})&&\Leftrightarrow\\
	 \det(M) \cdot \Ima(z) &= \Ima\left(\frac{az + b}{cz + d}\cdot (cz + d) \overline{(cz + d)}\right)&&\Leftrightarrow\\
	 \det(M) \cdot \Ima(z) &= |cz + d|^2 \cdot \Ima\left(\frac{az + b}{cz + d}\right)&&\Leftrightarrow\\
	 \frac{\det(M)}{|cz + d|^2} \cdot \Ima(z) &=  \Ima(M\langle z\rangle)&&\Leftrightarrow
\end{align*}

\section*{Aufgabe 5}
Sei $B'$ der Punkt des Baums. Wähle $W$ als den Ursprung des Koordinatensystems.
Es gilt $A = -i \cdot F$ und $B = B' + i \cdot (F - B')$. Damit erhalten wir für den Schatz
$$\frac{A + B}{2} = \frac{-i \cdot F + B' + i \cdot (F-B')}{2} = \frac{B' - i \cdot B'}{2}.$$
Dieser Ausdruck ist offensichtlich unabhängig von $F$.
\end{document}