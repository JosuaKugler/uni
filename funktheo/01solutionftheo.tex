\documentclass{article}
\usepackage[utf8]{inputenc}
\usepackage[T1]{fontenc}
\usepackage[ngerman]{babel}
\usepackage{amsmath, amsfonts, amsthm, mathtools, amssymb}
\usepackage{stmaryrd}
\usepackage{enumerate}
\usepackage{cases}
\usepackage{fancyhdr}
\usepackage{comment}
%\usepackage{xcolor}
\usepackage{tikz}
\usepackage{cases}
\usepackage{listings}
\usepackage{siunitx}
\usepackage[left = 3cm]{geometry}
\usepackage[hidelinks]{hyperref}
\usepackage{subcaption}
\usepackage{gauss}
\usepackage{environ}
\usepackage{url}
\newtheorem{satz}{Satz}[section]
\newtheorem{lemma}[satz]{Lemma}
\newtheorem{korollar}[satz]{Korollar}
\newtheorem{proposition}[satz]{Proposition}
\theoremstyle{definition}
\newtheorem{definition}[satz]{Def.}
\newtheorem{axiom}[satz]{Axiom}
\newtheorem{bsp}[satz]{Bsp.}
\newtheorem*{anmerkung}{Anm}
\newtheorem{bemerkung}[satz]{Bem}
\newtheorem*{notatio}{Notation}
\newcommand{\obda}{O.B.d.A. }
\newcommand{\equals}{\Longleftrightarrow}
\newcommand{\N}{\mathbb{N}}
\newcommand{\Q}{\mathbb{Q}}
\newcommand{\R}{\mathbb{R}}
\newcommand{\Z}{\mathbb{Z}}
\newcommand{\C}{\mathbb{C}}
\newcommand{\K}{\mathbb{K}}
\newcommand{\intd}{\mathrm{d}}
\newcommand{\Pot}{\operatorname{Pot}}
\newcommand{\mychar}{\operatorname{char}}
\newcommand{\myker}{\operatorname{ker}}
\newcommand{\induktion}[3]
{\begin{proof}\ \\
	\noindent\textbf{Induktionsanfang:}\ #1\\
	\noindent\textbf{Induktionsvoraussetzung:}\ #2\\
	\noindent\textbf{Induktionsschluss:}\ #3
\end{proof}}

\newcommand{\rg}{\operatorname{rg}}
\newcommand{\im}{\operatorname{im}}
\newcommand{\End}{\operatorname{End}}
\newcommand{\abb}{\operatorname{Abb}}
\newcommand{\re}{\operatorname{Re}}
\newcommand{\Ima}{\operatorname{Im}}
\newcommand{\norm}[1]{\left\Vert #1 \right\Vert}

\makeatletter \renewcommand\d{\ensuremath{%
		\;\mathrm{d}\@ifnextchar\d{\!}{}}}
\makeatother

\let\oldstackrel\stackrel
\renewcommand{\stackrel}[2]{%
    \oldstackrel{\mathclap{#1}}{#2}
}%

% maximum norm
\newcommand{\maxnorm}[1]{\left|\left|#1\right|\right|_\infty}
\renewcommand{\epsilon}{\varepsilon}

\newcommand{\dv}[2]{\frac{\d #1 }{\d #2 }}
\newcommand{\pdv}[2]{\frac{\partial #1}{\partial #2}}


\ExplSyntaxOn

% S-tackrelcompatible ALIGN environment
% some might also call it the S-uper ALIGN environment
% uses regular expressions to calculate the widest stackrel
% to put additional padding on both sides of relation symbols
\NewEnviron{salign}
{
    \begin{align}
        \lec_insert_padding:V \BODY
    \end{align}
}
% starred version that does no equation numbering
\NewEnviron{salign*}
{
    \begin{align*}
        \lec_insert_padding:V \BODY
    \end{align*}
}

% some helper variables
\tl_new:N \l__lec_text_tl
\seq_new:N \l_lec_stackrels_seq
\int_new:N \l_stackrel_count_int
\int_new:N \l_idx_int
\box_new:N \l_tmp_box
\dim_new:N \l_tmp_dim_a
\dim_new:N \l_tmp_dim_b
\dim_new:N \l_tmp_dim_needed

% function to insert padding according to widest stackrel
\cs_new_protected:Nn \lec_insert_padding:n
 {
  \tl_set:Nn \l__lec_text_tl { #1 }
  % get all stackrels in this align environment
  \regex_extract_all:nnN { \c{stackrel}{(.*?)}{(.*?)} } { #1 } \l_lec_stackrels_seq
  % get number of stackrels
  \int_set:Nn \l_stackrel_count_int { \seq_count:N \l_lec_stackrels_seq }
  \int_set:Nn \l_idx_int { 1 }
  \dim_set:Nn \l_tmp_dim_needed { 0pt }
  % iterate over stackrels
  \int_while_do:nn { \l_idx_int <= \l_stackrel_count_int }
  {
      % calculate width of text
      \hbox_set:Nn \l_tmp_box {$\seq_item:Nn \l_lec_stackrels_seq { \l_idx_int + 1 }$}
      \dim_set:Nn \l_tmp_dim_a {\box_wd:N \l_tmp_box}
      % calculate width of relation symbol
      \hbox_set:Nn \l_tmp_box {$\seq_item:Nn \l_lec_stackrels_seq { \l_idx_int + 2 }$}
      \dim_set:Nn \l_tmp_dim_b {\box_wd:N \l_tmp_box}
      % check if 0.5*(a-b) > minimum padding, if yes updated minimum padding
      \dim_compare:nNnTF
        { 1pt * \dim_ratio:nn { \l_tmp_dim_a - \l_tmp_dim_b } { 2pt } } > { \l_tmp_dim_needed }
        { \dim_set:Nn \l_tmp_dim_needed { 1pt * \dim_ratio:nn { \l_tmp_dim_a - \l_tmp_dim_b } { 2pt } } }
        { }
      \quad
      % increment list index by three, as every stackrel produces three list entries
      \int_incr:N \l_idx_int
      \int_incr:N \l_idx_int
      \int_incr:N \l_idx_int
  }
  % replace all relations with align characters (&) and add the needed padding
  \regex_replace_all:nnN
      { (\c{iff}&|&\c{iff}|\c{impliedby}&|&\c{impliedby}|\c{implies}&|&\c{implies}|\c{approx}&|&\c{approx}|\c{equiv}&|&\c{equiv}|=&|&=|\c{le}&|&\c{le}|\c{ge}&|&\c{ge}|&\c{stackrel}{.*?}{.*?}|\c{stackrel}{.*?}{.*?}&|&\c{neq}|\c{neq}&) }
      { \c{kern} \u{l_tmp_dim_needed} \1 \c{kern} \u{l_tmp_dim_needed} }
      \l__lec_text_tl
  \l__lec_text_tl
 }
\cs_generate_variant:Nn \lec_insert_padding:n { V }
\ExplSyntaxOff


\newcommand{\ipilayout}[1]
{	
	\pagestyle{fancy}
	\fancyhead[L]{Einführung in die praktische Informatik, Blatt #1}
	\fancyhead[R]{Josua Kugler, Jan Metzger, David Wesner}
	\fancypagestyle{firstpage}{%
		\fancyhf{}
		\lhead{Professor: Peter Bastian\\
			Tutor: Frederick Schenk}
		\rhead{Einführung in die praktische Informatik, Übungsblatt #1\\ David, Jan, Josua}
		\cfoot{\thepage}
	}
\thispagestyle{firstpage}
}

\newcommand{\analayout}[1]
{	
	\pagestyle{fancy}
	\fancyhead[L]{Analysis 2, Blatt #1}
	\fancyhead[R]{David Wesner, Josua Kugler}
	\fancypagestyle{firstpage}{%
		\fancyhf{}
		\lhead{Professor: Ekaterina Kostina\\
			Tutor: Julian Matthes}
		\rhead{Analysis 1, Übungsblatt #1\\ David Wesner, Josua Kugler}
		\cfoot{\thepage}
	}
	\thispagestyle{firstpage}
}
\newcommand{\lalayout}[1]
{	
	\pagestyle{fancy}
	\fancyhead[L]{Lineare Algebra 2, Blatt #1}
	\fancyhead[R]{David Wesner, Josua Kugler}
	\fancypagestyle{firstpage}{%
		\fancyhf{}
		\lhead{Professor: Denis Vogel\\
			Tutor: Marina Savarino}
		\rhead{Lineare Algebra 2, Übungsblatt #1\\ David Wesner, Josua Kugler}
		\cfoot{\thepage}
	}
	\thispagestyle{firstpage}
}

\lstset{
    frame=tb, % draw a frame at the top and bottom of the code block
    tabsize=4, % tab space width
    showstringspaces=false, % don't mark spaces in strings
    numbers=left, % display line numbers on the left
    commentstyle=\color{green}, % comment color
    keywordstyle=\color{blue}, % keyword color
    stringstyle=\color{red} % string color
}
\setlength{\headheight}{25pt}



\begin{document}

\section*{Aufgabe 49}
\begin{enumerate}[(a)]
	\item Sei ein Kompaktum $K$ gegeben, sodass $\forall z \in K\colon |z|<R,\; R\in \R_ {>0}$. Für $n > 2R$ gilt dann 
	\[
		|z^2-n^2| = |z-n||z+n| > \frac{1}{2}n\cdot \frac{1}{2}n = \frac{1}{4}n^2,
	\] also
	\[
		\frac{2z}{z^2 -n^2} < \frac{2R}{\frac{1}{4}n^2} < \frac{8R}{n^2}.
	\]
	Sei nun $r\coloneqq \min_{z\in K} |z|$ (das Minimum existiert, da $K$ kompakt und $|\cdot|$ stetig ist). Dann ist $\left|\frac{1}{z}\right| < \frac{1}{r}$. Damit können wir den gesamten Ausdruck unabhängig von $z$ durch eine konvergente Majorante abschätzen.
	\item Die Ableitung der linken Seite ist gegeben durch
	\[
		-\frac{\pi^2}{\sin^2(z)} \overset{*}{=} \sum_{n \in \Z}	\frac{1}{(z-n)^2}.
	\]
	Wir können die rechte Seite aufgrund der kompakten Konvergenz gliedweise differenzieren und erhalten daher für die Ableitung der rechten Seite
	\[
		-\frac{1}{z^2} - \sum_{n = 1}^{\infty}\frac{2(n^2+z^2)}{(z-n)^2(z+n)^2} = -\frac{1}{z^2} - \sum_{n = 1}^{\infty}\frac{1}{(z-n)^2} + \frac{1}{(z+n)^2} = -\sum_{n\in \N} \frac{1}{(z-n)^2}.	
	\]
	Offensichtlich ist also die Differenz zwischen beiden Ableitungen gleich 0.
	\item Wir berechnen $h\left(\frac{1}{2}\right)$. Wegen $\cot\left(\frac{\pi}{2}\right) = 0$, lassen wir den ersten Teil sofort weg und sehen
	\begin{salign*}
		h\left(\frac{1}{2}\right) &= -\frac{1}{\frac{1}{2}} + \sum_{n = 1}^{\infty}\frac{1}{\frac{1}{4}-n^2}\\
		&= -2 + \sum_{n = 1}^{\infty}\frac{1}{\left(\frac{1}{2}-n\right)\left(\frac{1}{2}+n\right)}\\
		&= -2 + \sum_{n = 1}^{\infty}\frac{1}{n-\frac{1}{2}} + \frac{1}{n + \frac{1}{2}}\\
		&\stackrel{\text{Teleskop}}{=} -2 + \frac{1}{\frac{1}{2}}\\
		&= 0.
	\end{salign*}
\end{enumerate}

\section*{Aufgabe 50}
Wir ermitteln zunächst die ersten 4 Terme der Taylorentwicklung von $\sin^2(\pi z)$, um dann durch Polynomdivision die ersten drei Terme der Laurententwicklung auszurechnen.
Es gilt
\[
	\sin^2(\pi z) = \pi^2z^2 - \frac{1}{3} \pi^4z^4 + \mathcal{O}(z^5).
\]
Mittels endlicher Polynomdivision erhalten wir für die Laurententwicklung daraus
\[
	\frac{1}{\sin^2(\pi z)} = \frac{1}{\pi^2z^2 - \frac{1}{3} \pi^4z^4 + \mathcal{O}(z^5)} = \frac{1}{\pi^2}z^{-2} + \frac{1}{3} + \mathcal{O}(z).	
\]
Multiplikation mit $\pi^2$ ergibt $\frac{\pi^2}{\sin^2(\pi z)} = z^{-2} + \frac{\pi^2}{3} + \mathcal{O}(z)$.
Nun berechnen wir noch die entsprechenden Koeffizienten für die rechte Seite von $(*)$. Da die Reihe kompakt kovergiert und alle Summanden bis auf $\frac{1}{z^2}$ holomorph auf $D_{0,1}(0)$ sind, ist $H(z) = \frac{1}{z^2}$, also $a_{-2} = 1$ und $a_{-1} = 0$. Für $a_0$ können wir den holomorphen Anteil der rechten Seite an der Stelle $z = 0$ auswerten und erhalten $a_0 = \sum_{n \in \Z^*} = 2\cdot \sum_{n = 1}^{\infty}\frac{1}{n^2}$. Vergleich der Laurent-Koeffizienten ergibt schließlich \[\frac{\pi^2}{3} = 2 \cdot \sum_{n = 1}^{\infty}\frac{1}{n^2} \implies \frac{\pi^2}{6} = \sum_{n = 1}^{\infty}\frac{1}{n^2}.\]
\end{document}