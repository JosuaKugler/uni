\documentclass{article}
\usepackage[utf8]{inputenc}
\usepackage[T1]{fontenc}
\usepackage[ngerman]{babel}
\usepackage{amsmath, amsfonts, amsthm, mathtools, amssymb}
\usepackage{stmaryrd}
\usepackage{enumerate}
\usepackage{cases}
\usepackage{fancyhdr}
\usepackage{comment}
%\usepackage{xcolor}
\usepackage{tikz}
\usepackage{cases}
\usepackage{listings}
\usepackage{siunitx}
\usepackage[left = 3cm]{geometry}
\usepackage[hidelinks]{hyperref}
\usepackage{subcaption}
\usepackage{gauss}
\usepackage{environ}
\usepackage{url}
\newtheorem{satz}{Satz}[section]
\newtheorem{lemma}[satz]{Lemma}
\newtheorem{korollar}[satz]{Korollar}
\newtheorem{proposition}[satz]{Proposition}
\theoremstyle{definition}
\newtheorem{definition}[satz]{Def.}
\newtheorem{axiom}[satz]{Axiom}
\newtheorem{bsp}[satz]{Bsp.}
\newtheorem*{anmerkung}{Anm}
\newtheorem{bemerkung}[satz]{Bem}
\newtheorem*{notatio}{Notation}
\newcommand{\obda}{O.B.d.A. }
\newcommand{\equals}{\Longleftrightarrow}
\newcommand{\N}{\mathbb{N}}
\newcommand{\Q}{\mathbb{Q}}
\newcommand{\R}{\mathbb{R}}
\newcommand{\Z}{\mathbb{Z}}
\newcommand{\C}{\mathbb{C}}
\newcommand{\K}{\mathbb{K}}
\newcommand{\intd}{\mathrm{d}}
\newcommand{\Pot}{\operatorname{Pot}}
\newcommand{\mychar}{\operatorname{char}}
\newcommand{\myker}{\operatorname{ker}}
\newcommand{\induktion}[3]
{\begin{proof}\ \\
	\noindent\textbf{Induktionsanfang:}\ #1\\
	\noindent\textbf{Induktionsvoraussetzung:}\ #2\\
	\noindent\textbf{Induktionsschluss:}\ #3
\end{proof}}

\newcommand{\rg}{\operatorname{rg}}
\newcommand{\im}{\operatorname{im}}
\newcommand{\End}{\operatorname{End}}
\newcommand{\abb}{\operatorname{Abb}}
\newcommand{\re}{\operatorname{Re}}
\newcommand{\Ima}{\operatorname{Im}}
\newcommand{\norm}[1]{\left\Vert #1 \right\Vert}

\makeatletter \renewcommand\d{\ensuremath{%
		\;\mathrm{d}\@ifnextchar\d{\!}{}}}
\makeatother

\let\oldstackrel\stackrel
\renewcommand{\stackrel}[2]{%
    \oldstackrel{\mathclap{#1}}{#2}
}%

% maximum norm
\newcommand{\maxnorm}[1]{\left|\left|#1\right|\right|_\infty}
\renewcommand{\epsilon}{\varepsilon}

\newcommand{\dv}[2]{\frac{\d #1 }{\d #2 }}
\newcommand{\pdv}[2]{\frac{\partial #1}{\partial #2}}


\ExplSyntaxOn

% S-tackrelcompatible ALIGN environment
% some might also call it the S-uper ALIGN environment
% uses regular expressions to calculate the widest stackrel
% to put additional padding on both sides of relation symbols
\NewEnviron{salign}
{
    \begin{align}
        \lec_insert_padding:V \BODY
    \end{align}
}
% starred version that does no equation numbering
\NewEnviron{salign*}
{
    \begin{align*}
        \lec_insert_padding:V \BODY
    \end{align*}
}

% some helper variables
\tl_new:N \l__lec_text_tl
\seq_new:N \l_lec_stackrels_seq
\int_new:N \l_stackrel_count_int
\int_new:N \l_idx_int
\box_new:N \l_tmp_box
\dim_new:N \l_tmp_dim_a
\dim_new:N \l_tmp_dim_b
\dim_new:N \l_tmp_dim_needed

% function to insert padding according to widest stackrel
\cs_new_protected:Nn \lec_insert_padding:n
 {
  \tl_set:Nn \l__lec_text_tl { #1 }
  % get all stackrels in this align environment
  \regex_extract_all:nnN { \c{stackrel}{(.*?)}{(.*?)} } { #1 } \l_lec_stackrels_seq
  % get number of stackrels
  \int_set:Nn \l_stackrel_count_int { \seq_count:N \l_lec_stackrels_seq }
  \int_set:Nn \l_idx_int { 1 }
  \dim_set:Nn \l_tmp_dim_needed { 0pt }
  % iterate over stackrels
  \int_while_do:nn { \l_idx_int <= \l_stackrel_count_int }
  {
      % calculate width of text
      \hbox_set:Nn \l_tmp_box {$\seq_item:Nn \l_lec_stackrels_seq { \l_idx_int + 1 }$}
      \dim_set:Nn \l_tmp_dim_a {\box_wd:N \l_tmp_box}
      % calculate width of relation symbol
      \hbox_set:Nn \l_tmp_box {$\seq_item:Nn \l_lec_stackrels_seq { \l_idx_int + 2 }$}
      \dim_set:Nn \l_tmp_dim_b {\box_wd:N \l_tmp_box}
      % check if 0.5*(a-b) > minimum padding, if yes updated minimum padding
      \dim_compare:nNnTF
        { 1pt * \dim_ratio:nn { \l_tmp_dim_a - \l_tmp_dim_b } { 2pt } } > { \l_tmp_dim_needed }
        { \dim_set:Nn \l_tmp_dim_needed { 1pt * \dim_ratio:nn { \l_tmp_dim_a - \l_tmp_dim_b } { 2pt } } }
        { }
      \quad
      % increment list index by three, as every stackrel produces three list entries
      \int_incr:N \l_idx_int
      \int_incr:N \l_idx_int
      \int_incr:N \l_idx_int
  }
  % replace all relations with align characters (&) and add the needed padding
  \regex_replace_all:nnN
      { (\c{iff}&|&\c{iff}|\c{impliedby}&|&\c{impliedby}|\c{implies}&|&\c{implies}|\c{approx}&|&\c{approx}|\c{equiv}&|&\c{equiv}|=&|&=|\c{le}&|&\c{le}|\c{ge}&|&\c{ge}|&\c{stackrel}{.*?}{.*?}|\c{stackrel}{.*?}{.*?}&|&\c{neq}|\c{neq}&) }
      { \c{kern} \u{l_tmp_dim_needed} \1 \c{kern} \u{l_tmp_dim_needed} }
      \l__lec_text_tl
  \l__lec_text_tl
 }
\cs_generate_variant:Nn \lec_insert_padding:n { V }
\ExplSyntaxOff


\newcommand{\ipilayout}[1]
{	
	\pagestyle{fancy}
	\fancyhead[L]{Einführung in die praktische Informatik, Blatt #1}
	\fancyhead[R]{Josua Kugler, Jan Metzger, David Wesner}
	\fancypagestyle{firstpage}{%
		\fancyhf{}
		\lhead{Professor: Peter Bastian\\
			Tutor: Frederick Schenk}
		\rhead{Einführung in die praktische Informatik, Übungsblatt #1\\ David, Jan, Josua}
		\cfoot{\thepage}
	}
\thispagestyle{firstpage}
}

\newcommand{\analayout}[1]
{	
	\pagestyle{fancy}
	\fancyhead[L]{Analysis 2, Blatt #1}
	\fancyhead[R]{David Wesner, Josua Kugler}
	\fancypagestyle{firstpage}{%
		\fancyhf{}
		\lhead{Professor: Ekaterina Kostina\\
			Tutor: Julian Matthes}
		\rhead{Analysis 1, Übungsblatt #1\\ David Wesner, Josua Kugler}
		\cfoot{\thepage}
	}
	\thispagestyle{firstpage}
}
\newcommand{\lalayout}[1]
{	
	\pagestyle{fancy}
	\fancyhead[L]{Lineare Algebra 2, Blatt #1}
	\fancyhead[R]{David Wesner, Josua Kugler}
	\fancypagestyle{firstpage}{%
		\fancyhf{}
		\lhead{Professor: Denis Vogel\\
			Tutor: Marina Savarino}
		\rhead{Lineare Algebra 2, Übungsblatt #1\\ David Wesner, Josua Kugler}
		\cfoot{\thepage}
	}
	\thispagestyle{firstpage}
}

\lstset{
    frame=tb, % draw a frame at the top and bottom of the code block
    tabsize=4, % tab space width
    showstringspaces=false, % don't mark spaces in strings
    numbers=left, % display line numbers on the left
    commentstyle=\color{green}, % comment color
    keywordstyle=\color{blue}, % keyword color
    stringstyle=\color{red} % string color
}
\setlength{\headheight}{25pt}



\begin{document}
\newcommand{\Res}[2][]{\operatorname{Res}_{#1}\!\left(#2\right)}
\section*{Aufgabe 51}
\begin{enumerate}[(a)]
    \item Offensichtlich ist $z^2 \cdot f(z)$ holomorph, sodass höchstens ein Pol 2-ter Ordnung vorliegt. Daher können wir Aufgabe 50 benutzen und erhalten $\Res[0]{f} = \frac{1}{1} \lim\limits_{z\to 0} \dv{}{z}z^2f(z) = \lim\limits_{z\to 0} \dv{}{z} e^{z+1} - e^{2z + 1} = \lim\limits_{z\to 0} e^{z+1} - 2e^{2z+1}  = e - 2e = -e$.
    \item Die Taylorentwicklung von $\cos^2(\frac{z}{2})$ ist $\cos^2(\frac{z}{2}) = \frac{(x-\pi)^2}{4} - \frac{(x-\pi)^4}{48} + \mathcal{O}((x-\pi)^5)$. Daraus erhalten wir mit Polynomdivision
    \[
        \frac{1}{\cos^2(\frac{z}{2})} = 4(x-\pi)^{-2} + \frac{1}{3} + \mathcal{O}((x-\pi)).
    \]
    Also ist $\Res[\pi]{g} = 0$. 
\end{enumerate}
\section*{Aufgabe 52}
\begin{enumerate}[(a)]  
    \item Völlig analog zum Beweis des Satzes von der Gebietstreue betrachten wir nur die Stelle $z = f(z) = 0$ und schließen, dass es eine holomorphe Funktion $h(z)$ mit $f(z) = z^\nu h(z)^\nu$ und $h(0)\neq 0$ auf einer kleinen offenen Kreisscheibe $D_\epsilon$ um $0$ gibt. Die Funktion $zh(z)$ bildet nach dem Satz von der Gebietstreue $D_\epsilon$ auf eine andere offene Kreisscheibe $D_\delta$ um $0$ ab, die dann durch die $\nu$-te Potenz auf eine dritte Kreisscheibe $D'$ um 0 abgebildet wird. Wir wählen dann ein $\zeta \in \C$ so groß, dass $\frac{1}{|\zeta|} < \delta$ und $\frac{1}{\zeta^\nu}\in D$ liegt. Die Gleichung $z^\nu = \frac{1}{\zeta^\nu}$ besitzt dann die $\nu$ verschiedenen Lösungen $\frac{z_1}{\zeta}, \dots, \frac{z_n}{\zeta}$, wobei $z_i$ die $i$-te Einheitswurzel bezeichne. Alle diese Lösungen liegen wegen $\left|\frac{z_i}{\zeta}\right| = \frac{1}{|\zeta|} < \delta$ in $D_\delta$. Da $f$ injektiv ist, muss auch $zg(z)$ injektiv sein, sonst wäre $(zg(z))^\nu$ nicht injektiv. Also gibt es $\nu$ verschiedene Elemente $\zeta_1,\dots, \zeta_\nu\in D_\epsilon$ mit $f(\zeta_i) = \frac{1}{\zeta^\nu}$. Aus der Injektivität von $f$ folgt aber, dass dann $\nu = 1$ sein muss und damit $a_1 \neq 0$ in der Laurent-Entwicklung von $f$. Die Ableitung an der Stelle 0 ist also $\dv{}{z}|_{z=0} \sum_{n = 1}^{\infty}a_nz^n = a_1 \neq 0$.
    \item Nach dem Satz über die Umkehrfunktion aus der reellen Analysis folgt, dass die Jacobimatrix von $f^{-1}$ gegeben ist durch $\left(D_xf(x)\right)^{-1}$. Aufgrund der Cauchy-Riemannschen Differentialgleichungen existieren $a,b$, sodass $D_xf(x) = \begin{pmatrix}
        a & b\\
        -b & a
    \end{pmatrix}$. Daraus erhalten wir $D_xf^{-1}(x) = \frac{1}{a^2+b^2}\begin{pmatrix}
        a & -b\\
        b & a
    \end{pmatrix}$. Offensichtlich erfüllt also $f^{-1}$ ebenfalls die Cauchy-Riemannschen Differentialgleichungen und ist daher auch holomorph.
    \item Nach Aufgabe 44 folgt sofort die Aussage.
\end{enumerate}
\end{document}