\documentclass{article}
\usepackage[utf8]{inputenc}
\usepackage[T1]{fontenc}
\usepackage[ngerman]{babel}
\usepackage{amsmath, amsfonts, amsthm, mathtools, amssymb}
\usepackage{stmaryrd}
\usepackage{enumerate}
\usepackage{cases}
\usepackage{fancyhdr}
\usepackage{comment}
%\usepackage{xcolor}
\usepackage{tikz}
\usepackage{cases}
\usepackage{listings}
\usepackage{siunitx}
\usepackage[left = 3cm]{geometry}
\usepackage[hidelinks]{hyperref}
\usepackage{subcaption}
\usepackage{gauss}
\usepackage{environ}
\usepackage{url}
\newtheorem{satz}{Satz}[section]
\newtheorem{lemma}[satz]{Lemma}
\newtheorem{korollar}[satz]{Korollar}
\newtheorem{proposition}[satz]{Proposition}
\theoremstyle{definition}
\newtheorem{definition}[satz]{Def.}
\newtheorem{axiom}[satz]{Axiom}
\newtheorem{bsp}[satz]{Bsp.}
\newtheorem*{anmerkung}{Anm}
\newtheorem{bemerkung}[satz]{Bem}
\newtheorem*{notatio}{Notation}
\newcommand{\obda}{O.B.d.A. }
\newcommand{\equals}{\Longleftrightarrow}
\newcommand{\N}{\mathbb{N}}
\newcommand{\Q}{\mathbb{Q}}
\newcommand{\R}{\mathbb{R}}
\newcommand{\Z}{\mathbb{Z}}
\newcommand{\C}{\mathbb{C}}
\newcommand{\K}{\mathbb{K}}
\newcommand{\intd}{\mathrm{d}}
\newcommand{\Pot}{\operatorname{Pot}}
\newcommand{\mychar}{\operatorname{char}}
\newcommand{\myker}{\operatorname{ker}}
\newcommand{\induktion}[3]
{\begin{proof}\ \\
	\noindent\textbf{Induktionsanfang:}\ #1\\
	\noindent\textbf{Induktionsvoraussetzung:}\ #2\\
	\noindent\textbf{Induktionsschluss:}\ #3
\end{proof}}

\newcommand{\rg}{\operatorname{rg}}
\newcommand{\im}{\operatorname{im}}
\newcommand{\End}{\operatorname{End}}
\newcommand{\abb}{\operatorname{Abb}}
\newcommand{\re}{\operatorname{Re}}
\newcommand{\Ima}{\operatorname{Im}}
\newcommand{\norm}[1]{\left\Vert #1 \right\Vert}

\makeatletter \renewcommand\d{\ensuremath{%
		\;\mathrm{d}\@ifnextchar\d{\!}{}}}
\makeatother

\let\oldstackrel\stackrel
\renewcommand{\stackrel}[2]{%
    \oldstackrel{\mathclap{#1}}{#2}
}%

% maximum norm
\newcommand{\maxnorm}[1]{\left|\left|#1\right|\right|_\infty}
\renewcommand{\epsilon}{\varepsilon}

\newcommand{\dv}[2]{\frac{\d #1 }{\d #2 }}
\newcommand{\pdv}[2]{\frac{\partial #1}{\partial #2}}


\ExplSyntaxOn

% S-tackrelcompatible ALIGN environment
% some might also call it the S-uper ALIGN environment
% uses regular expressions to calculate the widest stackrel
% to put additional padding on both sides of relation symbols
\NewEnviron{salign}
{
    \begin{align}
        \lec_insert_padding:V \BODY
    \end{align}
}
% starred version that does no equation numbering
\NewEnviron{salign*}
{
    \begin{align*}
        \lec_insert_padding:V \BODY
    \end{align*}
}

% some helper variables
\tl_new:N \l__lec_text_tl
\seq_new:N \l_lec_stackrels_seq
\int_new:N \l_stackrel_count_int
\int_new:N \l_idx_int
\box_new:N \l_tmp_box
\dim_new:N \l_tmp_dim_a
\dim_new:N \l_tmp_dim_b
\dim_new:N \l_tmp_dim_needed

% function to insert padding according to widest stackrel
\cs_new_protected:Nn \lec_insert_padding:n
 {
  \tl_set:Nn \l__lec_text_tl { #1 }
  % get all stackrels in this align environment
  \regex_extract_all:nnN { \c{stackrel}{(.*?)}{(.*?)} } { #1 } \l_lec_stackrels_seq
  % get number of stackrels
  \int_set:Nn \l_stackrel_count_int { \seq_count:N \l_lec_stackrels_seq }
  \int_set:Nn \l_idx_int { 1 }
  \dim_set:Nn \l_tmp_dim_needed { 0pt }
  % iterate over stackrels
  \int_while_do:nn { \l_idx_int <= \l_stackrel_count_int }
  {
      % calculate width of text
      \hbox_set:Nn \l_tmp_box {$\seq_item:Nn \l_lec_stackrels_seq { \l_idx_int + 1 }$}
      \dim_set:Nn \l_tmp_dim_a {\box_wd:N \l_tmp_box}
      % calculate width of relation symbol
      \hbox_set:Nn \l_tmp_box {$\seq_item:Nn \l_lec_stackrels_seq { \l_idx_int + 2 }$}
      \dim_set:Nn \l_tmp_dim_b {\box_wd:N \l_tmp_box}
      % check if 0.5*(a-b) > minimum padding, if yes updated minimum padding
      \dim_compare:nNnTF
        { 1pt * \dim_ratio:nn { \l_tmp_dim_a - \l_tmp_dim_b } { 2pt } } > { \l_tmp_dim_needed }
        { \dim_set:Nn \l_tmp_dim_needed { 1pt * \dim_ratio:nn { \l_tmp_dim_a - \l_tmp_dim_b } { 2pt } } }
        { }
      \quad
      % increment list index by three, as every stackrel produces three list entries
      \int_incr:N \l_idx_int
      \int_incr:N \l_idx_int
      \int_incr:N \l_idx_int
  }
  % replace all relations with align characters (&) and add the needed padding
  \regex_replace_all:nnN
      { (\c{iff}&|&\c{iff}|\c{impliedby}&|&\c{impliedby}|\c{implies}&|&\c{implies}|\c{approx}&|&\c{approx}|\c{equiv}&|&\c{equiv}|=&|&=|\c{le}&|&\c{le}|\c{ge}&|&\c{ge}|&\c{stackrel}{.*?}{.*?}|\c{stackrel}{.*?}{.*?}&|&\c{neq}|\c{neq}&) }
      { \c{kern} \u{l_tmp_dim_needed} \1 \c{kern} \u{l_tmp_dim_needed} }
      \l__lec_text_tl
  \l__lec_text_tl
 }
\cs_generate_variant:Nn \lec_insert_padding:n { V }
\ExplSyntaxOff


\newcommand{\ipilayout}[1]
{	
	\pagestyle{fancy}
	\fancyhead[L]{Einführung in die praktische Informatik, Blatt #1}
	\fancyhead[R]{Josua Kugler, Jan Metzger, David Wesner}
	\fancypagestyle{firstpage}{%
		\fancyhf{}
		\lhead{Professor: Peter Bastian\\
			Tutor: Frederick Schenk}
		\rhead{Einführung in die praktische Informatik, Übungsblatt #1\\ David, Jan, Josua}
		\cfoot{\thepage}
	}
\thispagestyle{firstpage}
}

\newcommand{\analayout}[1]
{	
	\pagestyle{fancy}
	\fancyhead[L]{Analysis 2, Blatt #1}
	\fancyhead[R]{David Wesner, Josua Kugler}
	\fancypagestyle{firstpage}{%
		\fancyhf{}
		\lhead{Professor: Ekaterina Kostina\\
			Tutor: Julian Matthes}
		\rhead{Analysis 1, Übungsblatt #1\\ David Wesner, Josua Kugler}
		\cfoot{\thepage}
	}
	\thispagestyle{firstpage}
}
\newcommand{\lalayout}[1]
{	
	\pagestyle{fancy}
	\fancyhead[L]{Lineare Algebra 2, Blatt #1}
	\fancyhead[R]{David Wesner, Josua Kugler}
	\fancypagestyle{firstpage}{%
		\fancyhf{}
		\lhead{Professor: Denis Vogel\\
			Tutor: Marina Savarino}
		\rhead{Lineare Algebra 2, Übungsblatt #1\\ David Wesner, Josua Kugler}
		\cfoot{\thepage}
	}
	\thispagestyle{firstpage}
}

\lstset{
    frame=tb, % draw a frame at the top and bottom of the code block
    tabsize=4, % tab space width
    showstringspaces=false, % don't mark spaces in strings
    numbers=left, % display line numbers on the left
    commentstyle=\color{green}, % comment color
    keywordstyle=\color{blue}, % keyword color
    stringstyle=\color{red} % string color
}
\setlength{\headheight}{25pt}



\begin{document}
\analayout{9}
\noindent \textbf{Anmerkung:} Wir benutzen für Referenzen unser mit ein paar Kommilitonen zusammen getextes Skript, zu finden unter \url{https://flavigny.de/lecture/pdf/analysis2}.
\section*{Aufgabe 1}
Wir definieren 
\[
	g \colon \begin{pmatrix}
		u\\[0.3em]v\\[0.3em]w
	\end{pmatrix} \mapsto \begin{pmatrix}
		\frac{wv}{w+2u}\\[0.3em]
		\frac{w+2u}{v}\\[0.3em]
		\frac{uv}{w+2u}
	\end{pmatrix}
\]
Es gilt 
\[
	g\circ f(\begin{pmatrix}
		x\\[0.3em]y\\[0.3em]z
	\end{pmatrix}) = g(\begin{pmatrix}
		yz\\[0.3em]x+2z\\[0.3em]xy
	\end{pmatrix}) = \begin{pmatrix}
		\frac{xy(x+2z)}{xy+2yz}\\[0.3em]
		\frac{xy+2yz}{x+2z}\\[0.3em]
		\frac{yz(x+2z)}{xy+2yz}
	\end{pmatrix} = \begin{pmatrix}
		x\\[0.3em]y\\[0.3em]z\\[0.3em]
	\end{pmatrix}
\]
Also ist $f^{-1} = g$. Nun berechnen wir die Jacobi-Matrizen.
\[
	D_f(2,1,0)^{-1} = \left.\begin{pmatrix}
		0&z & y\\[0.3em]
		1 & 0 & 2\\[0.3em]
		y & x & 0
	\end{pmatrix}^{-1}\right|_{(2,1,0)^T} = \begin{pmatrix}
		0 & 0 & 1\\[0.3em]
		1 & 0 & 2\\[0.3em]
		1 & 2 & 0
	\end{pmatrix}^{-1} = \begin{pmatrix}
		-2 & 1 & 0\\[0.3em]
		1 & -\frac{1}{2} & \frac{1}{2}\\[0.3em]
		1 & 0 & 0
	\end{pmatrix},
\] da
\[
	\begin{pmatrix}
		0 & 0 & 1\\[0.3em]
		1 & 0 & 2\\[0.3em]
		1 & 2 & 0
	\end{pmatrix} \cdot \begin{pmatrix}
		-2 & 1 & 0\\[0.3em]
		1 & -\frac{1}{2} & \frac{1}{2}\\[0.3em]
		1 & 0 & 0
	\end{pmatrix} = \begin{pmatrix}
		1 & 0 & 0\\[0.3em]
		0 & 1 & 0\\[0.3em]
		0 & 0 & 1
	\end{pmatrix}	
\]
Außerdem ist 
\begin{align*}
	D_g(2,1,0) &= \left.\begin{pmatrix}
		-2\frac{wv}{(w+2u)^2} & \frac{w}{w+2u} & \frac{v(w+2u)-wv}{(w+2u)^2}\\[0.3em]
		\frac{2}{v} & -\frac{w+2u}{v^2} & \frac{1}{v}\\[0.3em]
		\frac{v(w+2u)-2uv}{(w+2u)^2} & \frac{u}{w+2u} & \frac{uv}{(w+2u)^2}
	\end{pmatrix}\right|_{f(2,1,0)}\\
	&= \left.\begin{pmatrix}
		-2\frac{wv}{(w+2u)^2} & \frac{w}{w+2u} & \frac{2uv}{(w+2u)^2}\\[0.3em]
		\frac{2}{v} & -\frac{w+2u}{v^2} & \frac{1}{v}\\[0.3em]
		\frac{vw}{(w+2u)^2} & \frac{u}{w+2u} & \frac{uv}{(w+2u)^2}
	\end{pmatrix}\right|_{(0,2,2)^T}\\
	&= \begin{pmatrix}
		-2 & 1 & 0\\[0.3em]
		1 & -\frac{1}{2} & \frac{1}{2}\\[0.3em]
		1 & 0 & 0
	\end{pmatrix}
\end{align*}
\section*{Aufgabe 2}
Definiere \[g(x,y,z) = \begin{pmatrix}
	x+y-z-1\\
	x^2+y^2-1
\end{pmatrix}\]
und \[f(x) = x^2+y^2+z^2.\]
Die Aufgabenstellung besteht nun darin, $f$ unter der Bedingung $g = 0$ zu minimieren. Zunächst berechnen wir
\[\nabla f = \begin{pmatrix}
	2x\\2y\\2z
\end{pmatrix},\quad \nabla g_1 = \begin{pmatrix}
	1\\1\\-1
\end{pmatrix},\quad\text{ und } \nabla g_2 = \begin{pmatrix}
	2x\\2y\\0
\end{pmatrix}\]
Nach Lagrange existieren $\lambda_1,\lambda_2\in \R$ mit
\begin{align}
	\lambda_1 \nabla g_1 + \lambda_2 \nabla g_2 &= \nabla f\nonumber
	\intertext{Daraus ergibt sich für jede Komponente eine Gleichung}
	\lambda_1 + 2\lambda_2x &= 2x\label{one}\\
	\lambda_1 + 2\lambda_2y &= 2y\label{two}\\
	-\lambda_1 &= 2z\label{three}
\end{align}
Setzen wir nun Gleichung \refeq{three} in Gleichung \refeq{one} und \refeq{two} ein, so erhalten wir
\begin{align*}
	-2z + 2\lambda_2x &= 2x & -2z + 2\lambda_2y &= 2y\\
	-z + \lambda_2x &= x & -z + \lambda_2y &= y\\
	x(\lambda_2 - 1) &= z & y(\lambda_2-1)&= z
	\intertext{Wir nehmen zunächst $\lambda_2 -1 \neq 0$ an}
	x &= \frac{z}{\lambda_2 -1} & y &= \frac{z}{\lambda_2-1}
\end{align*}
Also ist $y = x$. Da die Punkte auf $Z$ liegen, ist $1 = x^2 + y^2 = 2x^2\Leftrightarrow x^2 = \frac{1}{2}$.
Daraus ergibt sich $x_{1,2} = y_{1,2} = \pm \frac{\sqrt{2}}{2}$. Da die Punkte auch in $P$ liegen, erhalten wir $z = x + y -1 = \pm \sqrt{2}-1$. Damit erhalten wir die Punkte $\vec{x}_1 = (\frac{\sqrt{2}}{2},\frac{\sqrt{2}}{2}, \sqrt{2}-1)$ und $\vec{x}_1 = (-\frac{\sqrt{2}}{2},-\frac{\sqrt{2}}{2},-\sqrt{2}-1)$.
Nun betrachten wir den Fall $\lambda_2 = 1$. Dann erhalten wir $z = 0$. Aus den Zwangsbedingungen ergibt sich dann $x+y = 1$ und $x^2+y^2 = 1$. Einsetzen ergibt $1 = (1-y)^2 + y^2 = 1 - 2y + 2y^2$, also $y^2-y = 0$. Mögliche Lösungen sind also $y = 0$ und $y = 1$, korrespondierend zu $x = 1$ und $x = 0$. Damit ergeben sich zwei weitere Punkte $\vec{x}_3 = (1,0,0)$ und $\vec{x}_4 = (0,1,0)$.
Wir berechnen nun die Werte der Zielfunktion an diesen vier Punkten.
\begin{align*}
	f(\vec{x}_1) &= f(\frac{\sqrt{2}}{2},\frac{\sqrt{2}}{2}, \sqrt{2}-1) = \frac{1}{2} + \frac{1}{2} + 2 - 2\sqrt{2} + 1 = 4 - 2\sqrt{2}\\
	f(\vec{x}_2) &= f(-\frac{\sqrt{2}}{2},-\frac{\sqrt{2}}{2},-\sqrt{2}-1) = \frac{1}{2} + \frac{1}{2} + 2 + 2\sqrt{2} + 1 = 4 + 2\sqrt{2}\\
	f(\vec{x}_3) &= f(1,0,0) = 1\\
	f(\vec{x}_4) &= f(0,1,0) = 1\\
\end{align*}
Der minimale Abstand ist also 1.

\section*{Aufgabe 3}
Wir definieren $f(x) = \frac{1}{2}x^TQx - c^Tx$ mit \[Q = \begin{pmatrix}
	4 & 0 & 2\\0 & 4 & 1\\2 & 1&6
\end{pmatrix},\quad c = \begin{pmatrix}
	-3\\8\\-2
\end{pmatrix}.\] Insbesondere ist $Q$ symmetrisch. Dann gilt
\begin{align*}
	f(x) &= \frac{1}{2}\sum_{j = 1}^{n}\sum_{k = 1}^{n}q_{jk}x_jx_k - \sum_{j = 1}^{n}c_jx_j\\
	&= \frac{1}{2}\sum_{j \neq i}\sum_{k \neq i}q_{jk}x_jx_k + \frac{1}{2}\sum_{k \neq i} q_{ik}x_ix_k + \frac{1}{2}\sum_{j \neq i}q_{ji}x_jx_i + \frac{1}{2}q_{ii}x_i^2 - \sum_{j\neq i}c_jx_j - c_ix_i\\
	\pdv{f}{x_i} &= \frac{1}{2} \sum_{k \neq i}q_{ik}x_k + \frac{1}{2} \sum_{j \neq i}q_{ji}x_j + q_{ii}x_i - c_i\\
	&= \frac{1}{2} \sum_{k \neq i}q_{ik}x_k + \frac{1}{2} \sum_{j \neq i}q_{ij}x_j + q_{ii}x_i - c_i\\
	&= \sum_{j \neq i}q_{ij}x_j + q_{ii}x_i - c_i\\
	&= (Qx - c)_i
\end{align*}
Also ist $\nabla f(x) = Qx-c$.
Weiter definieren wir $g(x) = A\cdot x - b$ mit \[A = \begin{pmatrix}
	-1 & 3 & -2\\-3&2&-1
\end{pmatrix},\quad b = \begin{pmatrix}
	7 \\ 2
\end{pmatrix}\]
Dann gilt \[\partial_i g_j = \partial_i\left(\sum_{k = 1}^na_{jk}x_k - b_j\right) = a_{ji},\] woraus wir schließen $\nabla g = A^T$ (ist glaube ich eher sloppy, funktioniert aber völlig analog).
Durch Ausmultiplizieren erkennt man, dass die oben definierten Matrizen die Kriterien aus der Aufgabenstellung erfüllen, wir können also $f$ s.t. $g = 0$ optimieren.
Nach Lagrange existiert also ein $\lambda \in \R^2$ mit \[\nabla g\cdot \lambda = \nabla f \Leftrightarrow A^T\cdot \lambda = Qx - c \Leftrightarrow Qx= A^T\cdot \lambda + c.\]
Außerdem muss noch $Ax = b$ gelten. Insgesamt erhalten wir demnach \[
	\begin{pmatrix}
		4 & 0 & 2\\0 & 4 & 1\\2 & 1&6
	\end{pmatrix} \cdot \begin{pmatrix}
		x_1\\x_2\\x_3
	\end{pmatrix}=\begin{pmatrix}
		-\lambda_1  -3\lambda_2 - 3\\
		3\lambda_1 + 2\lambda_2 + 8\\
		-2\lambda_1  -\lambda_2 -2
	\end{pmatrix}
\]
und \[
	\begin{pmatrix}
		-1 & 3 & -2\\-3&2&-1
	\end{pmatrix}\cdot \begin{pmatrix}
		x_1\\x_2\\x_3
	\end{pmatrix} = \begin{pmatrix}
		7 \\ 2
	\end{pmatrix}
\]
Wir lösen zunächst die zweite Gleichung. Es gilt
\[
	\begin{gmatrix}[p]
		-1 & 3 & -2 & 7\\
		-3 & 2 & -1 & 2
		\rowops
		\add[-3]{0}{1}
	\end{gmatrix}	\leadsto \begin{gmatrix}[p]
		-1 & 3 & -2 & 7\\
		0 & -7 & 5 & -19
		\rowops
		\mult{0}{-1}\\
		\mult{1}{-\frac{1}{7}}
		\add[3]{1}{0}
	\end{gmatrix} \leadsto \begin{gmatrix}[p]
		1 & 0 & -\frac{1}{7} & \frac{8}{7}\\
		0 & 1 & -\frac{5}{7} & \frac{19}{7}
	\end{gmatrix}.
\]
Damit folgt für $x$
\[
	x \in  \begin{pmatrix}
		\frac{8}{7}\\
		\frac{19}{7}\\
		0
	\end{pmatrix} + \operatorname{Lin} \frac{1}{7}\cdot \begin{pmatrix}
		1\\5\\7
	\end{pmatrix} = \begin{pmatrix}
		1\\2\\-1
	\end{pmatrix} + \operatorname{Lin} \begin{pmatrix}
		1\\5\\7
	\end{pmatrix},\;\text{ also }\exists a\colon x = \begin{pmatrix}
		1 + a\\2 + 5a\\-1+7a
	\end{pmatrix}
\]
Dies setzen wir in die erste Gleichung ein und erhalten
\begin{align*}
	\begin{pmatrix}
		4 & 0 & 2\\0 & 4 & 1\\2 & 1&6
	\end{pmatrix} \cdot \begin{pmatrix}
		1 + a\\2 + 5a\\-1+7a
	\end{pmatrix}=\begin{pmatrix}
		-\lambda_1  -3\lambda_2 - 3\\
		3\lambda_1 + 2\lambda_2 + 8\\
		-2\lambda_1  -\lambda_2 - 2
	\end{pmatrix}
    \intertext{Ausmultiplizieren ergibt}
    \begin{pmatrix}
		2  + 18a\\7 + 27a\\-2+49a
	\end{pmatrix}=\begin{pmatrix}
		-\lambda_1  -3\lambda_2 -3\\
		3\lambda_1 + 2\lambda_2 + 8\\
		-2\lambda_1  -\lambda_2 -2
    \end{pmatrix}
    \intertext{Durch kurzes Umformen sehen wir nun}
	\begin{pmatrix}
		18a+\lambda_1+ 3\lambda_2\\27a-3\lambda_1 - 2\lambda_2\\49a+ 2\lambda_1 + \lambda_2
	\end{pmatrix}=\begin{pmatrix}
		-5\\
		1\\
		0
	\end{pmatrix}\\
	\begin{pmatrix}
		1 &  3 & 18\\ -3 &  - 2 & 27\\ 2 & 1 & 49
	\end{pmatrix}\cdot \begin{pmatrix}
		\lambda_1\\\lambda_2\\a
	\end{pmatrix}=\begin{pmatrix}
		-5\\
		1\\
		0
	\end{pmatrix}\\
\end{align*}
Wir lösen nun auch dieses Gleichungssystem
\begin{align*}
	\begin{gmatrix}[p]
		1 & 3 & 18 &-5\\ 
		-3&-2 & 27 & 1\\ 
		2 & 1 & 49 & 0
		\rowops
		\add[3]{0}{1}
		\add[-2]{0}{2}
	\end{gmatrix}&\leadsto \begin{gmatrix}[p]
		1 & 3 & 18 &-5\\ 
		0 & 7 & 81 & -14\\ 
		0 & -5 & 13 & 10
		\rowops
		\mult{1}{\frac{1}{7}}
		\add[5]{1}{2}
		\add[-3]{1}{0}
	\end{gmatrix}\\
	&\leadsto \begin{gmatrix}[p]
		1 & 0 & \frac{-117}{7} &1\\
		0 & 1 & \frac{81}{7} & -2\\
		0 & 0 & \frac{496}{7} & 0
	\end{gmatrix}
	\intertext{Mit geeigneten Faktoren kann man die 3. Zeile auf alle anderen addieren}
	&\leadsto 
	\begin{gmatrix}[p]
		1 & 0 & 0 &1\\
		0 & 1 & 0 & -2\\
		0 & 0 & 1 & 0
	\end{gmatrix}
\end{align*}
Es folgt $\lambda_1 = 1,\; \lambda_2 = -2$ und $a = 0$. Das setzen wir in unser Ergebnis für $x$ ein und erhalten
\[
	x = \begin{pmatrix}
		1 + a\\2 + 5a\\-1+7a
	\end{pmatrix} = \begin{pmatrix}
		1\\2\\-1
	\end{pmatrix}.	
\]
Es gilt nun $f(x) = -6$. Wir überprüfen noch eine anderen Punkt, der $g = 0$ erfüllt, z.B. mit $a = 1$.
\[
	f\begin{pmatrix}
		2\\7\\-6
	\end{pmatrix} = 86	
\]
Also ist $x$ tatsächlich das gesuchte Minimum von $f$ s.t. $g = 0$.
\section*{Aufgabe 4}
\begin{enumerate}[(a)]
	\item Folgendes Gleichungssystem erster Ordnung ist äquivalent zum Gleichungssystem aus der Aufgabe
	\begin{align*}
		v_2'(t) &= \dv{v_2}{t}(t)& v_2'''(t) &= \dv{v_2''}{t}(t)& \dv{v_2'''(t)}{t} - a\dv{v_1'}{t}(t) &= f(t) \\
		v_2''(t) &= \dv{v_2'}{t}(t)& v_1'(t) &= \dv{v_1}{t}(t) & \dv{v_1'}{t}(t) + bv_2(t) &= g(t)
	\end{align*}
	\item \begin{align*}
		\dv{}{t}W(t) &= -p(t)W(t)\\
		\dv{}{t}\left(u_1(t)\dv{}{t}u_2(t) - \dv{}{t}u_1(t)u_2(t)\right) &= -p(t)\left(u_1(t)\dv{}{t}u_2(t) - \dv{}{t}u_1(t)u_2(t)\right)
		\intertext{Beim Ausschreiben der Ableitung der Wronskideterminante werden sich die gemischten Ableitungen $\dv{}{t}u_1(t)\dv{}{t}u_2(t)$ kürzen, daher schreiben wir sie gar nicht auf.}
		u_1(t)\dv{^2}{t^2}u_2(t) - \dv{^2}{t^2}u_1(t)u_2(t)  &= -p(t)u_1(t)\dv{}{t}u_2(t) + p(t)\dv{}{t}u_1(t)u_2(t)\\
		u_1(t)\left(\dv{^2}{t^2}u_2(t) + p(t)\dv{}{t}u_2(t)\right) &= u_2(t)\left(\dv{^2}{t^2}u_1(t)+ p(t)\dv{}{t}u_1(t)\right)\\
		u_1(t)\left(-q(t)u_2(t)\right) &= u_2(t)\left(-q(t)u_1(t)\right)\\
		q(t)u_1(t)u_2(t) &= q(t)u_2(t)u_1(t)
	\end{align*}
\end{enumerate}

\end{document}