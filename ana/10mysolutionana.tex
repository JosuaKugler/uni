\documentclass{article}
\usepackage[utf8]{inputenc}
\usepackage[T1]{fontenc}
\usepackage[ngerman]{babel}
\usepackage{amsmath, amsfonts, amsthm, mathtools, amssymb}
\usepackage{stmaryrd}
\usepackage{enumerate}
\usepackage{cases}
\usepackage{fancyhdr}
\usepackage{comment}
%\usepackage{xcolor}
\usepackage{tikz}
\usepackage{cases}
\usepackage{listings}
\usepackage{siunitx}
\usepackage[left = 3cm]{geometry}
\usepackage[hidelinks]{hyperref}
\usepackage{subcaption}
\usepackage{gauss}
\usepackage{environ}
\usepackage{url}
\newtheorem{satz}{Satz}[section]
\newtheorem{lemma}[satz]{Lemma}
\newtheorem{korollar}[satz]{Korollar}
\newtheorem{proposition}[satz]{Proposition}
\theoremstyle{definition}
\newtheorem{definition}[satz]{Def.}
\newtheorem{axiom}[satz]{Axiom}
\newtheorem{bsp}[satz]{Bsp.}
\newtheorem*{anmerkung}{Anm}
\newtheorem{bemerkung}[satz]{Bem}
\newtheorem*{notatio}{Notation}
\newcommand{\obda}{O.B.d.A. }
\newcommand{\equals}{\Longleftrightarrow}
\newcommand{\N}{\mathbb{N}}
\newcommand{\Q}{\mathbb{Q}}
\newcommand{\R}{\mathbb{R}}
\newcommand{\Z}{\mathbb{Z}}
\newcommand{\C}{\mathbb{C}}
\newcommand{\K}{\mathbb{K}}
\newcommand{\intd}{\mathrm{d}}
\newcommand{\Pot}{\operatorname{Pot}}
\newcommand{\mychar}{\operatorname{char}}
\newcommand{\myker}{\operatorname{ker}}
\newcommand{\induktion}[3]
{\begin{proof}\ \\
	\noindent\textbf{Induktionsanfang:}\ #1\\
	\noindent\textbf{Induktionsvoraussetzung:}\ #2\\
	\noindent\textbf{Induktionsschluss:}\ #3
\end{proof}}

\newcommand{\rg}{\operatorname{rg}}
\newcommand{\im}{\operatorname{im}}
\newcommand{\End}{\operatorname{End}}
\newcommand{\abb}{\operatorname{Abb}}
\newcommand{\re}{\operatorname{Re}}
\newcommand{\Ima}{\operatorname{Im}}
\newcommand{\norm}[1]{\left\Vert #1 \right\Vert}

\makeatletter \renewcommand\d{\ensuremath{%
		\;\mathrm{d}\@ifnextchar\d{\!}{}}}
\makeatother

\let\oldstackrel\stackrel
\renewcommand{\stackrel}[2]{%
    \oldstackrel{\mathclap{#1}}{#2}
}%

% maximum norm
\newcommand{\maxnorm}[1]{\left|\left|#1\right|\right|_\infty}
\renewcommand{\epsilon}{\varepsilon}

\newcommand{\dv}[2]{\frac{\d #1 }{\d #2 }}
\newcommand{\pdv}[2]{\frac{\partial #1}{\partial #2}}


\ExplSyntaxOn

% S-tackrelcompatible ALIGN environment
% some might also call it the S-uper ALIGN environment
% uses regular expressions to calculate the widest stackrel
% to put additional padding on both sides of relation symbols
\NewEnviron{salign}
{
    \begin{align}
        \lec_insert_padding:V \BODY
    \end{align}
}
% starred version that does no equation numbering
\NewEnviron{salign*}
{
    \begin{align*}
        \lec_insert_padding:V \BODY
    \end{align*}
}

% some helper variables
\tl_new:N \l__lec_text_tl
\seq_new:N \l_lec_stackrels_seq
\int_new:N \l_stackrel_count_int
\int_new:N \l_idx_int
\box_new:N \l_tmp_box
\dim_new:N \l_tmp_dim_a
\dim_new:N \l_tmp_dim_b
\dim_new:N \l_tmp_dim_needed

% function to insert padding according to widest stackrel
\cs_new_protected:Nn \lec_insert_padding:n
 {
  \tl_set:Nn \l__lec_text_tl { #1 }
  % get all stackrels in this align environment
  \regex_extract_all:nnN { \c{stackrel}{(.*?)}{(.*?)} } { #1 } \l_lec_stackrels_seq
  % get number of stackrels
  \int_set:Nn \l_stackrel_count_int { \seq_count:N \l_lec_stackrels_seq }
  \int_set:Nn \l_idx_int { 1 }
  \dim_set:Nn \l_tmp_dim_needed { 0pt }
  % iterate over stackrels
  \int_while_do:nn { \l_idx_int <= \l_stackrel_count_int }
  {
      % calculate width of text
      \hbox_set:Nn \l_tmp_box {$\seq_item:Nn \l_lec_stackrels_seq { \l_idx_int + 1 }$}
      \dim_set:Nn \l_tmp_dim_a {\box_wd:N \l_tmp_box}
      % calculate width of relation symbol
      \hbox_set:Nn \l_tmp_box {$\seq_item:Nn \l_lec_stackrels_seq { \l_idx_int + 2 }$}
      \dim_set:Nn \l_tmp_dim_b {\box_wd:N \l_tmp_box}
      % check if 0.5*(a-b) > minimum padding, if yes updated minimum padding
      \dim_compare:nNnTF
        { 1pt * \dim_ratio:nn { \l_tmp_dim_a - \l_tmp_dim_b } { 2pt } } > { \l_tmp_dim_needed }
        { \dim_set:Nn \l_tmp_dim_needed { 1pt * \dim_ratio:nn { \l_tmp_dim_a - \l_tmp_dim_b } { 2pt } } }
        { }
      \quad
      % increment list index by three, as every stackrel produces three list entries
      \int_incr:N \l_idx_int
      \int_incr:N \l_idx_int
      \int_incr:N \l_idx_int
  }
  % replace all relations with align characters (&) and add the needed padding
  \regex_replace_all:nnN
      { (\c{iff}&|&\c{iff}|\c{impliedby}&|&\c{impliedby}|\c{implies}&|&\c{implies}|\c{approx}&|&\c{approx}|\c{equiv}&|&\c{equiv}|=&|&=|\c{le}&|&\c{le}|\c{ge}&|&\c{ge}|&\c{stackrel}{.*?}{.*?}|\c{stackrel}{.*?}{.*?}&|&\c{neq}|\c{neq}&) }
      { \c{kern} \u{l_tmp_dim_needed} \1 \c{kern} \u{l_tmp_dim_needed} }
      \l__lec_text_tl
  \l__lec_text_tl
 }
\cs_generate_variant:Nn \lec_insert_padding:n { V }
\ExplSyntaxOff


\newcommand{\ipilayout}[1]
{	
	\pagestyle{fancy}
	\fancyhead[L]{Einführung in die praktische Informatik, Blatt #1}
	\fancyhead[R]{Josua Kugler, Jan Metzger, David Wesner}
	\fancypagestyle{firstpage}{%
		\fancyhf{}
		\lhead{Professor: Peter Bastian\\
			Tutor: Frederick Schenk}
		\rhead{Einführung in die praktische Informatik, Übungsblatt #1\\ David, Jan, Josua}
		\cfoot{\thepage}
	}
\thispagestyle{firstpage}
}

\newcommand{\analayout}[1]
{	
	\pagestyle{fancy}
	\fancyhead[L]{Analysis 2, Blatt #1}
	\fancyhead[R]{David Wesner, Josua Kugler}
	\fancypagestyle{firstpage}{%
		\fancyhf{}
		\lhead{Professor: Ekaterina Kostina\\
			Tutor: Julian Matthes}
		\rhead{Analysis 1, Übungsblatt #1\\ David Wesner, Josua Kugler}
		\cfoot{\thepage}
	}
	\thispagestyle{firstpage}
}
\newcommand{\lalayout}[1]
{	
	\pagestyle{fancy}
	\fancyhead[L]{Lineare Algebra 2, Blatt #1}
	\fancyhead[R]{David Wesner, Josua Kugler}
	\fancypagestyle{firstpage}{%
		\fancyhf{}
		\lhead{Professor: Denis Vogel\\
			Tutor: Marina Savarino}
		\rhead{Lineare Algebra 2, Übungsblatt #1\\ David Wesner, Josua Kugler}
		\cfoot{\thepage}
	}
	\thispagestyle{firstpage}
}

\lstset{
    frame=tb, % draw a frame at the top and bottom of the code block
    tabsize=4, % tab space width
    showstringspaces=false, % don't mark spaces in strings
    numbers=left, % display line numbers on the left
    commentstyle=\color{green}, % comment color
    keywordstyle=\color{blue}, % keyword color
    stringstyle=\color{red} % string color
}
\setlength{\headheight}{25pt}



\begin{document}
\analayout{11}
\noindent \textbf{Anmerkung:} Wir benutzen für Referenzen unser mit ein paar Kommilitonen zusammen getextes Skript, zu finden unter \url{https://flavigny.de/lecture/pdf/analysis2}.
\section*{Aufgabe 1}
\begin{enumerate}
	\item Es gilt 
	\begin{align*}
		\norm{f(t,u(t))} &\leq \alpha(t)\norm{u(t)} + \beta(t)\\
		&= \alpha(t)\norm{\int_{t_0}^tu(t')\d t' - u(t_0)} + \beta(t)\\
		&\leq \alpha(t) \int_{t_0}^t\norm{f(t', u(t'))} \d t' + u_0\alpha(t) + \beta(t)
	\end{align*}
	Mit dem Lemma von Gronwall folgt also $\norm{f(t, u(t))} \leq e^{\alpha(t)(t-t_0)} \cdot (u_0 \alpha(t) + \beta(t))$.
	Da $f$ auch noch stetig ist, folgt nach dem Existenzsatz von Peano die lokale Existenz. Es gilt außerdem 
	\[
		\norm{y(t)} \leq y_0 + \int_{t_0}^t e^{\alpha(t)(t-t_0)} \cdot (u_0 \alpha(t) + \beta(t)) \d t
	\]
	Der Integrand ist stetig, daher auf dem kompakten Intervall $[0,t]$ Riemann-integrierbar und die Stammfunktion ist wieder stetig. Folglich ist erhalten wir nach Korollar 5.10 die globale Existenz einer Lösung.
	\item Es gilt 
	\begin{align*}
		\norm{f_1(t, x)} &= \norm{t|x_1|^{\frac{1}{2}} + \sin(t)x_2}
		&\leq |t| (|x_1| + 1) + |x_2|\\
		&\leq (|t|+1)(|x_1| + 1) + (|t|+1)|x_2|\\
		&= (|t|+1)(|x_1| + |x_2|) + |t|+1\\
		&= \alpha(t) \norm{x}_1 + \beta(t)
	\end{align*}
	Also ist $f_1$ linear beschränkt durch $\alpha(t)= \beta(t) = |t|+1$.
	Außerdem gilt auch
	\begin{align*}
		\norm{f_1} &= \norm{e^{-t^2|x_1|} + x_1\frac{1}{1+x_2^2}}\\
		&\leq 1 + |x_1|\\
		&\leq \norm{x}_1 + 1
	\end{align*}
	Daher ist auch $f_2$ linear beschränkt, mit $\alpha(t) = \beta(t) = 1$.
\end{enumerate}
\section*{Aufgabe 2}
Es gilt
\begin{align*}
	u^{(3)}(t) &= \cos(u(t)) \cdot u'(t) - 2u''(t)\\
	&= \cos(u(t)) \cdot u'(t) - 2(-\sin(u(t))-2u'(t))\\
	&= 2 \sin(u(t)) + \cos(u(t))\cdot u'(t) +4 u'(t)\\
\end{align*}
und
\begin{align*}
	u^{(4)}(t) &= 2u'(t)\cos(u(t)) + \cos(u(t))u''(t) - \sin(u(t))u'(t) + 4u''(t)
\end{align*}
Wir benötigen die Ableitungen allerdings nur an der Stelle $0$. Dort ist
\begin{align*}
	u(0) &= 0\\
	u'(0) &= 1\\
	u''(0) &= - \sin(u(0)) - 2u'(0) = -\sin(0) - 2 = -2\\
	u^{(3)}(0) &= 2 \sin(u(0)) + \cos(u(0))\cdot u'(0) +4 u'(0)\\
	&= 2\sin(0) + \cos(0)\cdot 1 + 4 = 5\\
	u^{(4)}(0) &= 2u'(0)\cos(u(0)) + \cos(u(0))u''(0) - \sin(u(0))u'(0) + 4u''(0)\\
	&= 2\cos(0) + \cos(0)\cdot (-2) - \sin(0) + 4(-2) = 2 -2 -8 = -8
\end{align*}
Daraus erhalten wir schließlich
\begin{align*}
	T_4(u,t, t_0 = 0) &= u(0) + u'(0)t + \frac{u''(0)}{2}t^2 + \frac{u^{(3)}(0)}{6}t^3 + \frac{u^{(4)}}{24}t^4\\
	&= t + \frac{-2}{2}t^2 + \frac{5}{6}t^3 -\frac{8}{24}t^4\\
	&= t - t^2 + \frac{5}{6}t^3 - \frac{1}{3}t^4.
\end{align*}
\section*{Aufgabe 3}
$\phi(t) = \begin{pmatrix}
	1+ t & t\\
	t^2 & t ^2 - t + 1
\end{pmatrix}$.
Es gilt $\phi'(t) = A(t)\phi(t)$ und daher
\begin{align*}
	A(t) &= \phi'(t)\cdot \phi^{-1}(t)\\
	&= \begin{pmatrix}
		1 & 1\\
		2t & 2t -1
	\end{pmatrix} \cdot \frac{1}{(1+t)(t^2 + t - 1) - t^2\cdot t} \begin{pmatrix}
		t^2 - t + 1 & -1\\
		-t^2 & 1 + t
	\end{pmatrix}\\
	&= \begin{pmatrix}
		1 & 1\\
		2t & 2t -1
	\end{pmatrix} \cdot \begin{pmatrix}
		t^2 - t + 1 & -1\\
		-t^2 & 1 + t
	\end{pmatrix}\\
	&= \begin{pmatrix}
		t^2  - t + 1 - t^2 & -1 + 1 + t\\
		2t^3 - 2t^2 + 2t - 2t^3 + t^2& -2t + (2t-1)(1+t)
	\end{pmatrix}\\
	&= \begin{pmatrix}
		-t + 1 & t\\
		-t^2 + 2t & 2t^2 - t -1
	\end{pmatrix}
\end{align*}
Um eine partikuläre Lösung zu bestimmen, berechnen wir 
\begin{align*}
	u(t) &= \phi(t) \left(\int_{t_0}^{t} \phi^{-1}(s)b(s) \d s + c\right)\\
	&= \begin{pmatrix}
		1+ t & t\\
		t^2 & t ^2 - t + 1
	\end{pmatrix} \cdot \left(\int_{t_0}^{t} \begin{pmatrix}
		s^2 - s + 1 & -1\\
		-s^2 & 1 + s
	\end{pmatrix}\cdot \begin{pmatrix}
		1\\s
	\end{pmatrix} \d s + c\right)\\
	&= \begin{pmatrix}
		1+ t & t\\
		t^2 & t ^2 - t + 1
	\end{pmatrix} \cdot \left(\int_{t_0}^{t}
	\begin{pmatrix}
		s^2 - 2s + 1\\
		s
	\end{pmatrix}
	\d s + c\right)\\
	&= \begin{pmatrix}
		1+ t & t\\
		t^2 & t ^2 - t + 1
	\end{pmatrix} \cdot \left(
	\begin{pmatrix}
		\frac{1}{3}t^3 - t^2 + t\\
		\frac{1}{2}t^2
	\end{pmatrix} - \begin{pmatrix}
		\frac{1}{3}t_0^3 - t_0^2 + t_0\\
		\frac{1}{2}t_0^2
	\end{pmatrix} + c\right)
\end{align*}
Nun wählen wir $c$ so, dass $u(0) = (1,0)$ ist. Das ergibt die Gleichung
\begin{align*}
	u(0) &= \phi(0) \cdot \left(0- \begin{pmatrix}
			\frac{1}{3}t_0^3 - t_0^2 + t_0\\
			\frac{1}{2}t_0^2
		\end{pmatrix} + c\right)
		\intertext{Es gilt $t_0 = 0$}
	\begin{pmatrix}
		1\\0
	\end{pmatrix} &= \begin{pmatrix}
		1 & 0\\
		0 & 1
	\end{pmatrix} \cdot c = c
\end{align*}
Also wählen wir $c = \begin{pmatrix}
	1\\0
\end{pmatrix}$ und erhalten als Lösung:
\[
	u(t) = \begin{pmatrix}
		1+ t & t\\
		t^2 & t ^2 - t + 1
	\end{pmatrix} \cdot
	\begin{pmatrix}
		\frac{1}{3}t^3 - t^2 + t + 1\\
		\frac{1}{2}t^2
	\end{pmatrix}.	
\]
\section*{Aufgabe 4}
\begin{enumerate}[(a)]
\item 
Wir formulieren zunächst das RWP als System 1. Ordnung. Dann erhalten wir das System
\begin{align*}
	\dv{}{t} \begin{pmatrix}
		y(t)\\y'(t)
	\end{pmatrix} &= \begin{pmatrix}
		0 & 1\\
		-q(t)& - p(t)
	\end{pmatrix} \begin{pmatrix}
		y(t)\\y'(t)
	\end{pmatrix} + \begin{pmatrix}
		0\\\varphi(t)
	\end{pmatrix}\\
	\Leftrightarrow Y'(t) &= A(t) \cdot Y(t) + f(t)
\end{align*} mit $Y(t) = \begin{pmatrix}
	y(t)\\y'(t)
\end{pmatrix}$, $A(t) = \begin{pmatrix}
	0 & 1\\
	-q(t)& - p(t)
\end{pmatrix}\in \mathcal{C}[a,b]$ und $f(t) = \begin{pmatrix}
	0\\\varphi(t)
\end{pmatrix} \in \mathcal{C}[a,b]$, wobei wir noch keine Nebenbedingungen berücksichtigt haben. Sei $U_0 = \begin{pmatrix}
	u_0(t)\\u_0'(t)
\end{pmatrix}$ eine Lösung der AWA $U_0(a) = 0$ nach Bemerkung 5.32 und seien $U_1, U_2$ ein Fundamentalsystem der homogenen Gleichung mit, ebenfalls nach Bemerkung 5.32 
$U(t) = \left[U_1(t),U_2(t)\right]$ und $U(0) = \mathbb{I}_2$. Dann lässt sich eine beliebige Lösung der RWA in der Form $U_0(t) + s_1U_1(t) + s_2U_2(t)$ schreiben. Die Lösung ist eindeutig, wenn $s_1, s_2$ durch die Randbedingungen eindeutig festgelegt sind. Die Randbedingungen schreiben sich in der folgenden Form
\begin{align*}
	\begin{pmatrix}
		\alpha_0 & \alpha_1\\
		0 & 0
	\end{pmatrix} \cdot \left(\begin{pmatrix}
		u_0(a)\\u_0'(a)
	\end{pmatrix} + \begin{pmatrix}
		u_1(a) & u_2(a)\\
		u_1'(a) & u_2'(a)
	\end{pmatrix}\cdot \begin{pmatrix}
		s_1\\s_2
	\end{pmatrix}\right) &= \begin{pmatrix}
		\eta\\
		0
	\end{pmatrix}\\
	\begin{pmatrix}
		\alpha_0 & \alpha_1\\
		0 & 0
	\end{pmatrix}
	\cdot \left(\underbrace{U_0(a)}_{=0} + U(a)\cdot s\right) &= \begin{pmatrix}
		\eta_0\\
		0
	\end{pmatrix}
	\intertext{Und völlig analog}
	\begin{pmatrix}
		0 & 0\\
		\beta_0 & \beta_1
	\end{pmatrix} \cdot \left(U_0(b) + U(b)\cdot s\right) &= \begin{pmatrix}
		0\\
		\eta_1
	\end{pmatrix}
	\intertext{Bringen wir die Terme, die nicht von $s$ abhängen, auf die andere Seite, und vereinfachen, so erhalten wir}
	\begin{pmatrix}
		\alpha_0 u_1(a) + \alpha_1 u_1'(a) & \alpha_0 u_2(a) + \alpha_1 u_2'(a)\\
		0 & 0
	\end{pmatrix}
	\cdot s &= \begin{pmatrix}
		\eta_0\\
		0
	\end{pmatrix}\\
	\intertext{und}
	\begin{pmatrix}
		0 & 0\\
		\beta_0 u_1(b) + \beta_1 u_1'(b) & \beta_0 u_2(b) + \beta_1 u_2'(b)
	\end{pmatrix} 
	\cdot s &= 
	\begin{pmatrix}
		0\\ \eta_1 - \beta_0 u_0(b) - \beta_1 u_0'(b)
	\end{pmatrix}
	\intertext{Addieren der zwei Gleichungen ergibt}
	\begin{pmatrix}
		\alpha_0 u_1(a) + \alpha_1 u_1'(a) & \alpha_0 u_2(a) + \alpha_1 u_2'(a)\\
		\beta_0 u_1(b) + \beta_1 u_1'(b) & \beta_0 u_2(b) + \beta_1 u_2'(b)
	\end{pmatrix} \cdot s&= 
	\begin{pmatrix}
		\eta_0\\
		\eta_1 - \beta_0 u_0(b) - \beta_1 u_0'(b)
	\end{pmatrix}
\end{align*}
Diese Gleichung ist eindeutig lösbar genau dann, wenn die Determinante der linken Matrix $\neq 0$ ist, was zu zeigen war.
\item Es gilt $u_0(t) = -\cos(t) + 1$. Dann ist nämlich $u_0(0) = 0$ und $u_0'(0) = 0$ erfüllt. $u_1(t) = \cos(t), u_2(t) = \sin(t)$ bilden ein Fundamentalsystem, da 
\[
	\left.\begin{pmatrix}
		\cos(t) & \sin(t)\\
		-\sin(t) & \cos(t)
	\end{pmatrix}\right|_{t = 0} = \begin{pmatrix}
		1 & 0\\ 0 & 1
	\end{pmatrix}.
\]
Wir bestimmen daher die Matrix aus Kriterium (**) und setzen sofort $a = 0$ sowie $\alpha_1 = \beta_1 = 0$, da der Definitionsbereich in allen drei Fällen bei $0$ beginnt und die Bedingungen nie von $u'$ abhängen. Außerdem sind alle Vorfaktoren in den Randbedingungen immer 1, sodass wir $\alpha_0 = \beta_0 = 1$ setzen können und daher erhalten
\begin{align*}
	M = \begin{pmatrix}
		\cos(0) & \sin(0)\\
		\cos(b) & \sin(b)
	\end{pmatrix} = \begin{pmatrix}
		1 & 0\\
		\cos(b) & \sin(b)
	\end{pmatrix}
\end{align*}
In Fall (i)ist $\det M = \det \begin{pmatrix}
	1 & 0\\
	0 & 1
\end{pmatrix} = -1$ und daher ist die Lösung eindeutig bestimmt. Im Fall (ii) und (iii) ist $\det M = \det \begin{pmatrix}
	1 & 0\\
	-1 & 0
\end{pmatrix} = 0$. Also gilt Kriterium (**) nicht.
Wir erhalten für (ii) stattdessen die Gleichung
\begin{align*}
	\begin{pmatrix}
		 1 &  0\\
		 -1 &  0
	\end{pmatrix} \cdot s&= 
	\begin{pmatrix}
		\eta_0\\
		\eta_1 - u_0(\pi) 
	\end{pmatrix}\\
	\Leftrightarrow \begin{pmatrix}
		1 &  0\\
		-1 &  0
   \end{pmatrix} \cdot s&= 
   \begin{pmatrix}
	   0\\
	   0 + \underbrace{\cos(\pi) - 1}_{= 0}
   \end{pmatrix}\\
\end{align*}
Offensichtlich sind also alle Vektoren $s = \begin{pmatrix}
	0 \\ x
\end{pmatrix}$ eine Lösung des Gleichungssystems und damit lösen alle Funktionen 
$u(t) = u_0(t) + x \cdot \sin(x) = x\sin(t) - \cos(t) + 1$ die RWA, was man auch leicht durch Einsetzen verifiziert. Es gibt in diesem Fall also unendlich viele Lösungen.
Wir erhalten für (iii) stattdessen die Gleichung
\begin{align*}
	\begin{pmatrix}
		 1 &  0\\
		 -1 &  0
	\end{pmatrix} \cdot s&= 
	\begin{pmatrix}
		\eta_0\\
		\eta_1 - u_0(\pi) 
	\end{pmatrix}\\
	\Leftrightarrow \begin{pmatrix}
		1 &  0\\
		-1 &  0
   \end{pmatrix} \cdot s&= 
   \begin{pmatrix}
	   1\\
	   1 + \underbrace{\cos(\pi) - 1}_{= 0}
   \end{pmatrix}\\
   \Leftrightarrow \begin{pmatrix}
	s_1\\
	-s_1
\end{pmatrix}&= 
\begin{pmatrix}
   1\\
   1
\end{pmatrix}\\
\end{align*}
Dieses Gleichungssystem besitzt keine Lösung, daher hat auch die Randwertaufgabe für diese Anfangsbedingungen keine Lösung.
\end{enumerate}
\end{document}