\documentclass{article}

\usepackage[utf8]{inputenc}
\usepackage[T1]{fontenc}
\usepackage[ngerman]{babel}
\usepackage{amsmath, amsfonts, amsthm, mathtools, amssymb}
\usepackage{stmaryrd}
\usepackage{enumerate}
\usepackage{cases}
\usepackage{fancyhdr}
\usepackage{comment}
%\usepackage{xcolor}
\usepackage{tikz}
\usepackage{cases}
\usepackage{listings}
\usepackage{siunitx}
\usepackage[left = 3cm]{geometry}
\usepackage[hidelinks]{hyperref}
\usepackage{subcaption}
\usepackage{gauss}
\newtheorem{satz}{Satz}[section]
\newtheorem{lemma}[satz]{Lemma}
\newtheorem{korollar}[satz]{Korollar}
\newtheorem{proposition}[satz]{Proposition}
\theoremstyle{definition}
\newtheorem{definition}[satz]{Def.}
\newtheorem{axiom}[satz]{Axiom}
\newtheorem{bsp}[satz]{Bsp.}
\newtheorem*{anmerkung}{Anm}
\newtheorem{bemerkung}[satz]{Bem}
\newtheorem*{notatio}{Notation}
\newcommand{\obda}{O.B.d.A. }
\newcommand{\equals}{\Longleftrightarrow}
\newcommand{\N}{\mathbb{N}}
\newcommand{\Q}{\mathbb{Q}}
\newcommand{\R}{\mathbb{R}}
\newcommand{\Z}{\mathbb{Z}}
\newcommand{\C}{\mathbb{C}}
\newcommand{\intd}{\mathrm{d}}
\newcommand{\Pot}{\operatorname{Pot}}
\newcommand{\mychar}{\operatorname{char}}
\newcommand{\myker}{\operatorname{ker}}
\newcommand{\induktion}[3]
{\begin{proof}\ \\
	\noindent\textbf{Induktionsanfang:}\ #1\\
	\noindent\textbf{Induktionsvoraussetzung:}\ #2\\
	\noindent\textbf{Induktionsschluss:}\ #3
\end{proof}}

\newcommand{\rg}{\operatorname{rg}}
\newcommand{\im}{\operatorname{im}}
\newcommand{\End}{\operatorname{End}}
\newcommand{\abb}{\operatorname{Abb}}
\newcommand{\re}{\operatorname{Re}}
\newcommand{\Ima}{\operatorname{Im}}



\newcommand{\ipilayout}[1]
{	
	\pagestyle{fancy}
	\fancyhead[L]{Einführung in die praktische Informatik, Blatt #1}
	\fancyhead[R]{Josua Kugler, Jan Metzger, David Wesner}
	\fancypagestyle{firstpage}{%
		\fancyhf{}
		\lhead{Professor: Peter Bastian\\
			Tutor: Frederick Schenk}
		\rhead{Einführung in die praktische Informatik, Übungsblatt #1\\ David, Jan, Josua}
		\cfoot{\thepage}
	}
\thispagestyle{firstpage}
}

\newcommand{\analayout}[1]
{	
	\pagestyle{fancy}
	\fancyhead[L]{Analysis 2, Blatt #1}
	\fancyhead[R]{David Wesner, Josua Kugler}
	\fancypagestyle{firstpage}{%
		\fancyhf{}
		\lhead{Professor: Ekaterina Kostina\\
			Tutor: Julian Matthes}
		\rhead{Analysis 1, Übungsblatt #1\\ David Wesner, Josua Kugler}
		\cfoot{\thepage}
	}
	\thispagestyle{firstpage}
}
\newcommand{\lalayout}[1]
{	
	\pagestyle{fancy}
	\fancyhead[L]{Lineare Algebra 2, Blatt #1}
	\fancyhead[R]{David Wesner, Josua Kugler}
	\fancypagestyle{firstpage}{%
		\fancyhf{}
		\lhead{Professor: Denis Vogel\\
			Tutor: Marina Savarino}
		\rhead{Lineare Algebra 2, Übungsblatt #1\\ David Wesner, Josua Kugler}
		\cfoot{\thepage}
	}
	\thispagestyle{firstpage}
}

\lstset{
    frame=tb, % draw a frame at the top and bottom of the code block
    tabsize=4, % tab space width
    showstringspaces=false, % don't mark spaces in strings
    numbers=left, % display line numbers on the left
    commentstyle=\color{green}, % comment color
    keywordstyle=\color{blue}, % keyword color
    stringstyle=\color{red} % string color
}
\setlength{\headheight}{25pt}

\makeatletter \renewcommand\d[1]{\ensuremath{%
		\;\mathrm{d}#1\@ifnextchar\d{\!}{}}}
\makeatother

% maximum norm
\newcommand{\maxnorm}[1]{\left|\left|#1\right|\right|_\infty}
\renewcommand{\epsilon}{\varepsilon}

\begin{document}
\analayout{1}
\section*{Aufgabe 1}
\begin{enumerate}[(a)]
	\item Es ist 
		\begin{align*}
			 \int_{0}^{1}\sqrt{x\sqrt{{x\sqrt{x}}}}dx&=\int_{0}^{1}(x(x(x)^{\frac{1}{2}})^{\frac{1}{2}})^{\frac{1}{2}}dx
			 =\int_{0}^{1}x^{\frac{1}{2}}x^{\frac{1}{4}}x^{\frac{1}{8}}dx \\
			 &=\int_{0}^{1}x^{\frac{7}{8}}dx = [\frac{8}{15}x^{\frac{15}{8}}]_{0}^{1} = \frac{8}{15}
		\end{align*}
	\item Es ist 
		\begin{align*}
			\int_{0}^{1}e^{x}(1-x+x^2)dx &= [e^x(1-x+x^2)]_{0}^{1}-\int_{0}^{1}e^x(-1+2x)dx\\ 
			&=[e^x(1-x+x^2)]_{0}^{1}-[e^x(-1+2x)]_{0}^{1}+\int_{0}^{1}2e^xdx\\
			&=[e^x(1-x+x^2)]_{0}^{1}-[e^x(-1+2x)]_{0}^{1}+[e^x]_{0}^{1}\\
			&=e(1-1+1)-(e-1)+(e-1)=e
		\end{align*}
	\item Substituiere $u=x^2$:
		\begin{align*}
			(u)'=2x &\Rightarrow \frac{du}{dx}=2x\Leftrightarrow dx=\frac{du}{dx} \Rightarrow\int_{0}^{1}e^{x^{2}}x^3dx\\
			&=\int_{0}^{1}\frac{e^{u}xu}{2x}du=\frac{1}{2}\int_{0}^{1}e^{u}udu= \frac{1}{2}([e^{u}u]_{0}^{1}-\int_{0}^{1}e^{u}du)\\
			&=\frac{1}{2}[e^{u}u-e^{u}]_{0}^{1}=-\frac{1}{2}
		\end{align*}
	\item Es gilt $\tan'(x) = \frac{1}{\cos^2(x)}$.
	Partielle Integration führt also auf
	\begin{align*}
		\int_0^{\frac{\pi}{4}} \frac{x}{\cos^2(x)} \d x &= \Bigg[x \cdot \tan(x)\Bigg]_0^\frac{\pi}{4} - \int_0^\frac{\pi}{4} \frac{\sin(x)}{\cos(x)} \d x
		\intertext{Substituiere $u = \cos(x)$}
		&= \frac{\pi}{4} - \int_{\cos(0)}^{\cos\left(\frac{\pi}{2}\right)} \frac{1}{u} \d u\\
		&=\frac{\pi}{4} -\Bigg[\ln(u)\Bigg]_1^\frac{\sqrt{2}}{2}\\
		&= \frac{\pi}{4} - \ln(\frac{\sqrt{2}}{2})
		\end{align*}

\end{enumerate}

\section*{Aufgabe 2}
\begin{enumerate}[(a)]
	\item Da $f$ stetig ist, muss $f$ auf dem kompakten Intervall $[a,b]$ Riemann-integrierbar sein. Es gibt also eine Stammfunktion $F$ mit $\int_{\phi(x)}^{\psi(x)} f(t)\d t = F(\phi(x)) - F(\psi(x))$. Dann gilt auch $$\frac{\d}{\d x} \int_{\phi(x)}^{\psi(x)} f(t)\d t = \frac{\d}{\d x} F(\psi(x)) - F(\phi(x)) \overset{\text{Kettenregel}}{=} f(\psi(x)) \cdot \psi'(x) - f(\phi(x)) \cdot \phi'(x)$$
	\item Es gilt $$G(x) \coloneqq \int_a^x|f'(t)|\d t \overset{\text{Rannacher 1, Korollar 6.3}}{\geq} \left| \int_a^x f'(t) \d t\right| = |f(x)|,$$ womit wir Ungleichung (I) erhalten: 
	$$\left(G(x)^2\right)' = 2 \cdot |f'(x)| \cdot G(x) \geq 2 \cdot |f'(x) \cdot f(x)|.$$ Ferner gilt 
	\begin{align*}
		\int_a^b |f(x)f'(x)| \d x &= \frac{1}{2} \int_a^b 2|f(x)f'(x)| \d x
		\intertext{Nun wenden wir Ungleichung (I) an und folgern}
		&\leq \frac{1}{2} \int_a^b \left(G(x)^2\right)'\d x\\
		&= \frac{1}{2} G(x)^2
		\intertext{Daraufhin benutzen wir, dass in $\R$ stets $x^2 = |x|^2$ gilt und setzen die Definition von $G$ ein}
		&= \frac{1}{2} \left|\int_a^b 1\cdot |f'(t)| \d t\right|^2
		\intertext{Wenden wir schließlich die CSU an, erhalten wir}
		&\leq \frac{1}{2} \int_a^b 1^2 \d x \cdot \int_a^b |f'(x)|^2 \d x\\
		&= \frac{b-a}{2} \cdot \int_a^b f'(x)^2 \d x
	\end{align*}
\end{enumerate}
\section*{Aufgabe 3}
	Es gilt $$\lim\limits_{n\to\infty}\left(\frac{n^2x}{(1+n^2x^2)^2}\right)=\lim\limits_{n\to\infty}\left(\frac{n^2x}{1+2n^2x^2+n^4x^4}\right) \Leftrightarrow \lim\limits_{n\to\infty}\left(\frac{\frac{x}{n^2}}{\frac{1}{n^4}+\frac{2x^2}{n^2}+x^4}\right)\longrightarrow 0$$
	Damit kovergiert die Funktionenfolge $f_n(x)$ gegen die konstante Funktion $f$ mit$f(x)=0$ für alle $x\in\R$.
	Somit ist $$\int_{0}^{1}\lim\limits_{n\to\infty}f_n(x)dx= 0$$
	Für $x^{*}=\frac{1}{\sqrt{3}}$ wird $|f_n(x)|$ nach Hinweis maximal. Es gilt 
	$$|f_n(x^{*})-0|=\frac{n^2\frac{1}{\sqrt{3}n}}{(1^2+n^2\frac{1}{3n^2})^2}=\frac{\frac{n}{\sqrt{3}}}{(1+\frac{1}{3})^2}=\frac{9n}{16\sqrt{3}}=\frac{3\sqrt{3}n}{16}$$
	Und somit 
	$$\lim\limits_{n\to\infty}\left(\frac{3\sqrt{3}n}{16}\right)\longrightarrow \infty$$
	Also konvergiert $f_n(x)$ punktweise gegen die Funktion $f$ mit $f(x)=0$ für alle $x\in \R$
	Satz 1.3.1 bezieht sich nur auf gleichmäßig konvergente Funktionen.\\
	Außerdem gilt mit der Substitution $u=(1+n^2x^2)$:
	\begin{align*}
		(u)'= 2n^2x &\Rightarrow du = dx2n^2x \Rightarrow dx=\frac{du}{2n^2x}\\
		&\implies\int_{0}^{1}f_n(x)dx=\int_{0}^{1}\left(\frac{n^2x}{(1+n^2x^2)^2}\right)\\
		&= \frac{1}{2}\int_{1}^{1+n^2}\frac{1}{u^2}du=\frac{1}{2}[-\frac{1}{u}]_{1}^{1+n^2}=\frac{1}{2}\left(-1-\frac{1}{1+n^2}\right)=\frac{1}{2}-\frac{1}{2+2n^2}\\
		&\Rightarrow \lim\limits_{n\to\infty}\left(\frac{1}{2}-\frac{1}{2+2n^2}\right)\rightarrow \frac{1}{2}
	\end{align*}
	
	

\section*{Aufgabe 4}
Da $\frac{1}{n} e^{-\frac{x}{n}}$ streng monoton fällt, gilt $\forall x \geq 0: \frac{1}{n}e^{-\frac{x}{n}} \leq\frac{1}{n}e^{-\frac{0}{n}} = \frac{1}{n}$. Wähle also für ein beliebiges $\epsilon > 0\; N = \lceil\frac{1}{\epsilon}\rceil$. Dann gilt
$$\forall n \geq N,\; \forall x \geq 0:\; |f_n(x) - 0| \leq \frac{1}{n} \leq \frac{1}{N} \leq \epsilon.$$
Also konvergiert $f_n$ für $n \to \infty$ gleichmäßig gegen $f \equiv 0$ und $\int_0^\infty \lim\limits_{n\to\infty}f_n(x)\d x = 0.$ Andererseits ist $$\lim\limits_{n\to\infty}\lim\limits_{z\to\infty}\int_0^z \frac{1}{n} e^{-\frac{x}{n}} \d x = \lim\limits_{n\to\infty}\lim\limits_{z\to\infty} \left[-e^{-\frac{x}{n}}\right]_0^z = \lim\limits_{n\to\infty}\lim\limits_{z\to\infty} - e^{\frac{z}{n}} + e^0 = \lim\limits_{n\to\infty} 1 = 1.$$
Insgesamt erhalten wir
$$\lim\limits_{n\to\infty}\int_0^\infty f_n(x) \d x = 1 \neq 0 = \int_0^\infty \lim\limits_{n\to\infty}f_n(x)\d x,$$ was nicht im Widerspruch zu Satz 1.3.1 steht, da dort der Definitionsbereich von $f$ als beschränktes Intervall $[a,b]$ vorausgesetzt wird, hier ist aber der Definitionsbereich der $f_n: [0, \infty) \to \R$ unbeschränkt.
\section*{Aufgabe 5}
	Mithilfe von partieller Integration lässt sich das Integral schreiben als
	$$\int \cos(x)\sin(x)\d x = \sin^2(x) - \int \sin(x) \cos(x) \d x.$$ Addieren wir nun $\int \cos(x)\sin(x)\d x$ und dividieren durch 2, so erhalten wir $$\int \cos(x)\sin(x)\d x = \frac{\sin^2(x)}{2} + C.$$
\end{document}
