\documentclass{article}

\usepackage[utf8]{inputenc}
\usepackage[T1]{fontenc}
\usepackage[ngerman]{babel}
\usepackage{amsmath, amsfonts, amsthm, mathtools, amssymb}
\usepackage{stmaryrd}
\usepackage{enumerate}
\usepackage{cases}
\usepackage{fancyhdr}
\usepackage{comment}
%\usepackage{xcolor}
\usepackage{tikz}
\usepackage{cases}
\usepackage{listings}
\usepackage{siunitx}
\usepackage[left = 3cm]{geometry}
\usepackage[hidelinks]{hyperref}
\usepackage{subcaption}
\usepackage{gauss}
\newtheorem{satz}{Satz}[section]
\newtheorem{lemma}[satz]{Lemma}
\newtheorem{korollar}[satz]{Korollar}
\newtheorem{proposition}[satz]{Proposition}
\theoremstyle{definition}
\newtheorem{definition}[satz]{Def.}
\newtheorem{axiom}[satz]{Axiom}
\newtheorem{bsp}[satz]{Bsp.}
\newtheorem*{anmerkung}{Anm}
\newtheorem{bemerkung}[satz]{Bem}
\newtheorem*{notatio}{Notation}
\newcommand{\obda}{O.B.d.A. }
\newcommand{\equals}{\Longleftrightarrow}
\newcommand{\N}{\mathbb{N}}
\newcommand{\Q}{\mathbb{Q}}
\newcommand{\R}{\mathbb{R}}
\newcommand{\Z}{\mathbb{Z}}
\newcommand{\C}{\mathbb{C}}
\newcommand{\intd}{\mathrm{d}}
\newcommand{\Pot}{\operatorname{Pot}}
\newcommand{\mychar}{\operatorname{char}}
\newcommand{\myker}{\operatorname{ker}}
\newcommand{\induktion}[3]
{\begin{proof}\ \\
	\noindent\textbf{Induktionsanfang:}\ #1\\
	\noindent\textbf{Induktionsvoraussetzung:}\ #2\\
	\noindent\textbf{Induktionsschluss:}\ #3
\end{proof}}

\newcommand{\rg}{\operatorname{rg}}
\newcommand{\im}{\operatorname{im}}
\newcommand{\End}{\operatorname{End}}
\newcommand{\abb}{\operatorname{Abb}}
\newcommand{\re}{\operatorname{Re}}
\newcommand{\Ima}{\operatorname{Im}}



\newcommand{\ipilayout}[1]
{	
	\pagestyle{fancy}
	\fancyhead[L]{Einführung in die praktische Informatik, Blatt #1}
	\fancyhead[R]{Josua Kugler, Jan Metzger, David Wesner}
	\fancypagestyle{firstpage}{%
		\fancyhf{}
		\lhead{Professor: Peter Bastian\\
			Tutor: Frederick Schenk}
		\rhead{Einführung in die praktische Informatik, Übungsblatt #1\\ David, Jan, Josua}
		\cfoot{\thepage}
	}
\thispagestyle{firstpage}
}

% integral d sign
\makeatletter \renewcommand\d[1]{\ensuremath{%
		\;\mathrm{d}#1\@ifnextchar\d{\!}{}}}
\makeatother

\newcommand{\analayout}[1]
{	
	\pagestyle{fancy}
	\fancyhead[L]{Analysis 1, Blatt #1}
	\fancyhead[R]{Alexander Bryant, Josua Kugler}
	\fancypagestyle{firstpage}{%
		\fancyhf{}
		\lhead{Professor: Ekaterina Kostina\\
			Tutor: Philipp Elja Müller}
		\rhead{Analysis 1, Übungsblatt #1\\ Alexander Bryant, Josua Kugler}
		\cfoot{\thepage}
	}
	\thispagestyle{firstpage}
}
\newcommand{\lalayout}[1]
{	
	\pagestyle{fancy}
	\fancyhead[L]{Lineare Algebra 1, Blatt #1}
	\fancyhead[R]{David Wesner, Josua Kugler}
	\fancypagestyle{firstpage}{%
		\fancyhf{}
		\lhead{Professor: Denis Vogel\\
			Tutor: Marina Savarino}
		\rhead{Lineare Algebra 2, Übungsblatt #1\\ David Wesner, Josua Kugler}
		\cfoot{\thepage}
	}
	\thispagestyle{firstpage}
}

\lstset{
    frame=tb, % draw a frame at the top and bottom of the code block
    tabsize=4, % tab space width
    showstringspaces=false, % don't mark spaces in strings
    numbers=left, % display line numbers on the left
    commentstyle=\color{green}, % comment color
    keywordstyle=\color{blue}, % keyword color
    stringstyle=\color{red} % string color
}
\setlength{\headheight}{25pt}
\begin{document}
\section*{Aufgabe 1}
\begin{align*}
    \vec \nabla E + \lambda \vec \nabla V &= 0\\
    -2 \cdot \frac{h^2}{8m} \begin{pmatrix}
        \frac{1}{a^3}\\[0.5em]
        \frac{1}{b^3}\\[0.5em]
        \frac{1}{c^3}\\[0.5em]
    \end{pmatrix}
    + \lambda \cdot \begin{pmatrix}
        bc\\[0.5em]
        ac\\[0.5em]
        ab\\[0.5em]
    \end{pmatrix} &= 0\\
    \begin{pmatrix}
        \lambda bc\\[0.5em]
        \lambda ac\\[0.5em]
        \lambda ab\\[0.5em]
    \end{pmatrix} &= 
    \begin{pmatrix}
        \frac{h^2}{4m}\frac{1}{a^3}\\[0.5em]
        \frac{h^2}{4m}\frac{1}{b^3}\\[0.5em]
        \frac{h^2}{4m}\frac{1}{c^3}\\[0.5em]
    \end{pmatrix}
    \intertext{Aus der ersten Zeile erhalten wir}
    \lambda &=  \frac{h^2}{4m} \frac{1}{abc} \frac{1}{a^2}\\
    \intertext{In die zweite Zeile eingesetzt folgt}
    \frac{h^2}{4m} \frac{1}{abc} \frac{1}{a^2} ac &= \frac{h^2}{4m}\frac{1}{b^3}\\
    b^2 &= a^2
    \intertext{Analog kann man schlussfolgern}
    c^2 &= a^2
\end{align*}
Da negative Seitenlängen wenig Sinn ergeben, muss es sich bei dem Quader um einen Würfel handeln.
\section*{Aufgabe 2}
Da die Masse sich nur auf der schiefen Ebene aufhält, muss folgende Zwangsbedingung gelten:
$$f = z \cdot \cos (\alpha) - (x-\xi) \sin(\alpha) = 0$$
Da wir dies später noch benötigen werden, bilden wir gleich auch noch die zweite Ableitung:
$$\ddot f = \ddot z \cdot \cos (\alpha) - (\ddot x - \ddot \xi) \sin(\alpha) = 0$$
Die Lagrange-Gleichungen lauten:
\begin{align*}
    0 &= F_x -m\ddot x + \lambda \frac{\partial f}{\partial x} = -m\ddot x - \lambda \sin(\alpha)\\
    0 &= F_z -m\ddot z + \lambda \frac{\partial f}{\partial z} = -m g_z -m\ddot z + \lambda \cos(\alpha)\\
\end{align*}
Wir stellen die erste Gleichung nach $\lambda$ um und erhalten:
$$\lambda = -\frac{m\ddot x}{\sin(\alpha)}.$$ Eingesetzt in die zweite Gleichung kann man sofort $m$ kürzen und erhält:
$$\ddot z = -\frac{\ddot x\cos(\alpha)}{\sin(\alpha)} - g_z$$
Nun stellen wir die zweite Ableitung der Zwangsbedingung nach $\ddot z$ um:
$$\ddot z = (\ddot x - \ddot \xi)\cdot \tan(\alpha).$$
Eingesetzt in unser obiges Ergebnis erhalten wir nun eine Gleichung für $\ddot x$.
\begin{align*}
    -\frac{\ddot x\cos(\alpha)}{\sin(\alpha)} - g_z &= (\ddot x - \ddot \xi)\cdot \tan(\alpha)\\
    \ddot \xi \cdot \tan(\alpha) -g_z &= \ddot x\left(\frac{\cos(\alpha)}{\sin(\alpha)} + \frac{\sin(\alpha)}{\cos(\alpha)}\right)\\
    \ddot \xi \cdot \sin^2(\alpha) - g_z \sin(\alpha)\cos(\alpha) &= \ddot x \left(\cos^2(\alpha)+ \sin^2(\alpha)\right)\\
    \ddot x &= \ddot \xi \cdot \sin^2(\alpha) - g_z \sin(\alpha)\cos(\alpha)
    \intertext{Wegen $\dot x(0) = 0$ führt Integration auf}
    \dot x &= \dot \xi \cdot \sin^2(\alpha) - g_z \sin(\alpha)\cos(\alpha)\cdot t
    \intertext{Wegen $x(0) = 0$ führt Integration auf}
    x &= \xi \cdot \sin^2(\alpha) - \frac{1}{2} g_z \sin(\alpha)\cos(\alpha)\cdot t^2
    \intertext{Mithilfe der Zwangsbedingung folgt umittelbar}
    z &= \left(\xi \cdot \sin^2(\alpha) - \frac{1}{2} g_z \sin(\alpha)\cos(\alpha)\cdot t^2 - \xi\right) \cdot \tan(\alpha)
    \intertext{Für die Zwangskraft gilt dann:}
    \vec Z &= \lambda \cdot \vec \nabla f\\
    &= m (g\cos(\alpha) - \ddot \xi \sin(\alpha))\begin{pmatrix}
        -\sin(\alpha)\\
        0\\
        \cos(\alpha)\\
    \end{pmatrix}\\
\end{align*}
\section*{Aufgabe 3}
\begin{enumerate}[(a)]
    \item 
    Es gilt der Energiesatz:
    \begin{align*}
        \frac{m}{2}(\dot{\vec{x}}^2) - mgf(x) &= E\\
        \dot x ^2 + \dot z^2 &= \frac{2}{m} (E + mgf(x))\\
        \left(\frac{\d x}{\d t}\right)^2 + \left(\frac{\d f(x)}{\d x}\frac{\d x}{\d t}\right)^2 &= \frac{2}{m} (E + mgf(x))\\
        \left(1 + \left(\frac{\d f(x)}{\d x}\right)^2\right) \cdot \left(\frac{\d x}{\d t}\right)^2 &= \frac{2}{m} (E + mgf(x))\\
        \left(\frac{\d x}{\d t}\right) &= \sqrt{\frac{\frac{2}{m} (E + mgf(x))}{1 + \left(\frac{\d f(x)}{\d x}\right)^2}}\\
    \end{align*}
    Analog zu letztem Semester erhalten wir
    $$\Delta t = \int_{x_0}^{x_E} \frac{\d x}{\sqrt{\frac{\frac{2}{m} (E + mgf(x))}{1 + \left(\frac{\d f(x)}{\d x}\right)^2}}}$$
    \item Hier ist nun $f(x) = x$, $E = 0$, $x_0 = 0$ und $x_E = 1$. Also erhalten wir
    \begin{align*}
        \Delta t &= \int_{x_0}^{x_E} \frac{\d x}{\sqrt{gf(x)}}\\
        &= \frac{2}{\sqrt{g}} \sqrt{x} \Big|^1_0\\
        &= \frac{2}{\sqrt{g}}
    \end{align*}
\end{enumerate}
\end{document}