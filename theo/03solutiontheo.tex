\documentclass{article}

\usepackage[utf8]{inputenc}
\usepackage[T1]{fontenc}
\usepackage[ngerman]{babel}
\usepackage{amsmath, amsfonts, amsthm, mathtools, amssymb}
\usepackage{stmaryrd}
\usepackage{enumerate}
\usepackage{cases}
\usepackage{fancyhdr}
\usepackage{comment}
%\usepackage{xcolor}
\usepackage{tikz}
\usepackage{cases}
\usepackage{listings}
\usepackage{siunitx}
\usepackage[left = 3cm]{geometry}
\usepackage[hidelinks]{hyperref}
\usepackage{subcaption}
\usepackage{gauss}
\newtheorem{satz}{Satz}[section]
\newtheorem{lemma}[satz]{Lemma}
\newtheorem{korollar}[satz]{Korollar}
\newtheorem{proposition}[satz]{Proposition}
\theoremstyle{definition}
\newtheorem{definition}[satz]{Def.}
\newtheorem{axiom}[satz]{Axiom}
\newtheorem{bsp}[satz]{Bsp.}
\newtheorem*{anmerkung}{Anm}
\newtheorem{bemerkung}[satz]{Bem}
\newtheorem*{notatio}{Notation}
\newcommand{\obda}{O.B.d.A. }
\newcommand{\equals}{\Longleftrightarrow}
\newcommand{\N}{\mathbb{N}}
\newcommand{\Q}{\mathbb{Q}}
\newcommand{\R}{\mathbb{R}}
\newcommand{\Z}{\mathbb{Z}}
\newcommand{\C}{\mathbb{C}}
\newcommand{\intd}{\mathrm{d}}
\newcommand{\Pot}{\operatorname{Pot}}
\newcommand{\mychar}{\operatorname{char}}
\newcommand{\myker}{\operatorname{ker}}
\newcommand{\induktion}[3]
{\begin{proof}\ \\
	\noindent\textbf{Induktionsanfang:}\ #1\\
	\noindent\textbf{Induktionsvoraussetzung:}\ #2\\
	\noindent\textbf{Induktionsschluss:}\ #3
\end{proof}}

\newcommand{\rg}{\operatorname{rg}}
\newcommand{\im}{\operatorname{im}}
\newcommand{\End}{\operatorname{End}}
\newcommand{\abb}{\operatorname{Abb}}
\newcommand{\re}{\operatorname{Re}}
\newcommand{\Ima}{\operatorname{Im}}
\newcommand{\Lagrange}[1]{\frac{\d }{\d t}\frac{\partial L }{\partial \dot #1} - \frac{\partial L}{\partial #1}}
\let\oldstackrel\stackrel
\renewcommand{\stackrel}[2]{%
    \oldstackrel{\mathclap{#1}}{#2}
}%


\newcommand{\ipilayout}[1]
{	
	\pagestyle{fancy}
	\fancyhead[L]{Einführung in die praktische Informatik, Blatt #1}
	\fancyhead[R]{Josua Kugler, Jan Metzger, David Wesner}
	\fancypagestyle{firstpage}{%
		\fancyhf{}
		\lhead{Professor: Peter Bastian\\
			Tutor: Frederick Schenk}
		\rhead{Einführung in die praktische Informatik, Übungsblatt #1\\ David, Jan, Josua}
		\cfoot{\thepage}
	}
\thispagestyle{firstpage}
}

% integral d sign
\makeatletter \renewcommand\d[1]{\ensuremath{%
		\;\mathrm{d}#1\@ifnextchar\d{\!}{}}}
\makeatother

\newcommand{\analayout}[1]
{	
	\pagestyle{fancy}
	\fancyhead[L]{Analysis 1, Blatt #1}
	\fancyhead[R]{Alexander Bryant, Josua Kugler}
	\fancypagestyle{firstpage}{%
		\fancyhf{}
		\lhead{Professor: Ekaterina Kostina\\
			Tutor: Philipp Elja Müller}
		\rhead{Analysis 1, Übungsblatt #1\\ Alexander Bryant, Josua Kugler}
		\cfoot{\thepage}
	}
	\thispagestyle{firstpage}
}
\newcommand{\lalayout}[1]
{	
	\pagestyle{fancy}
	\fancyhead[L]{Lineare Algebra 1, Blatt #1}
	\fancyhead[R]{David Wesner, Josua Kugler}
	\fancypagestyle{firstpage}{%
		\fancyhf{}
		\lhead{Professor: Denis Vogel\\
			Tutor: Marina Savarino}
		\rhead{Lineare Algebra 2, Übungsblatt #1\\ David Wesner, Josua Kugler}
		\cfoot{\thepage}
	}
	\thispagestyle{firstpage}
}

\lstset{
    frame=tb, % draw a frame at the top and bottom of the code block
    tabsize=4, % tab space width
    showstringspaces=false, % don't mark spaces in strings
    numbers=left, % display line numbers on the left
    commentstyle=\color{green}, % comment color
    keywordstyle=\color{blue}, % keyword color
    stringstyle=\color{red} % string color
}
\setlength{\headheight}{25pt}
\begin{document}
\section*{Aufgabe 1}
\begin{enumerate}[(a)]
    \item Für die potentielle Energie $V$ erhalten wir
    $$V  = -s g M \cos(\alpha) - m g (L \cos(\alpha) + d\cos(\beta)) = -g(sM + Lm)\cos(\alpha) - dmg\cos(\beta)$$
    Die kinetische Energie ist
    $$T = \frac{M}{2}s^2\dot \alpha^2 + \frac{m}{2}\left(L^2\dot \alpha^2 + d^2\dot \beta^2 + 2Ld\dot \alpha\dot \beta (\cos(\alpha)\cos(\beta) + \sin(\alpha)\sin(\beta))\right)$$
    $$= \frac{M}{2}s^2\dot \alpha^2 + \frac{m}{2}L^2 \dot \alpha^2 + \frac{m}{2}d^2 \dot \beta^2 + mLd\dot \alpha\dot \beta \cos(\alpha - \beta)$$
    Wir erhalten daher folgende Lagrange-Gleichung:
    $$L = \frac{M}{2}s^2\dot \alpha^2 + \frac{m}{2}L^2 \dot \alpha^2 + \frac{m}{2}d^2 \dot \beta^2 + mLd\dot \alpha\dot \beta \cos(\alpha - \beta)+g(sM + Lm)\cos(\alpha) + gdm\cos(\beta)$$
    \item Für die Bewegungsgleichungen folgt also
    \begin{align*}
        0 &= \frac{\d }{\d t}\frac{\partial L }{\partial \dot \alpha} - \frac{\partial L}{\partial \alpha}\\
        &= \frac{\d }{\d t} \left(Ms^2 \dot \alpha + mL^2 \dot \alpha + mLd\dot \beta \cos(\alpha - \beta)\right) - mLd\dot \alpha\dot \beta \sin(\alpha - \beta)+ g(sM + Lm)\sin(\alpha)\\
        &= (Ms^2 + mL^2)\ddot \alpha + mLd\left(\ddot \beta\cos(\alpha - \beta) + (- \dot \beta \dot \alpha + \dot \beta^2+ \dot\alpha \dot \beta) \sin(\alpha - \beta)\right)+ g(sM + Lm)\sin(\alpha)\\
        &= (Ms^2 + mL^2)\ddot \alpha + mLd\left(\ddot \beta\cos(\alpha - \beta) + \dot \beta^2 \sin(\alpha - \beta)\right)+ g(sM + Lm)\sin(\alpha)\\
        0 &= \frac{\d }{\d t}\frac{\partial L }{\partial \dot \beta} - \frac{\partial L}{\partial \beta}\\
        &= \frac{\d }{\d t} \left(md^2\dot \beta + mLd\dot \alpha \cos(\alpha - \beta)\right) - mLd\dot \alpha\dot \beta \sin(\alpha - \beta) + gdm\sin(\alpha)\\
        &= md^2 \ddot \beta + mLd\left(\ddot \alpha \cos(\alpha - \beta) - \dot \alpha (\dot \alpha - \dot \beta)\sin(\alpha - \beta) - \dot \alpha \dot \beta \sin(\alpha - \beta)\right)+ gdm\sin(\alpha)\\
        &= md^2 \ddot \beta + mLd\left(\ddot \alpha \cos(\alpha - \beta) - \dot \alpha^2 \sin(\alpha - \beta)\right)+ gdm\sin(\alpha)\\
    \end{align*}
    \item Zeigen Sie, dass es eine Lösung gibt, bei der der zeitliche Verlauf von $\alpha$ und $\beta$ übereinstimmt.
    Was bedeutet die Lösung anschaulich? Benutzen Sie, dass die Determinante der Koeffizientenmatrix der beiden Differentialgleichungen für den Winkel $\alpha$ für eine nicht-triviale Lösung verschwindet.
    Wir setzen also $\alpha = \beta$. Dann gilt
    \begin{align*}
        0 &= (Ms^2 + mL^2)\ddot \alpha + mLd\left(\ddot \alpha\cos(\alpha - \alpha) + \dot \beta^2 \sin(\alpha - \alpha)\right)+ g(sM + Lm)\sin(\alpha)\\
        0 &= (Ms^2 + mL^2)\ddot \alpha + mLd\ddot \alpha + g(sM + Lm)\sin(\alpha)
        \intertext{Wir multiplizieren diese Gleichung mit $-\frac{dm}{sM + Lm}$}
        0 &= -\frac{dm(Ms^2 + mL^2)}{sM + Lm} \ddot{\alpha} -\frac{d^2m^2L}{sM + Lm} \ddot \alpha - gdm \sin(\alpha)
        \intertext{Die andere Bewegungsgleichung wird zu}
        &= md^2 \ddot \alpha + mLd\left(\ddot \alpha \cos(\alpha - \alpha) - \dot \alpha^2 \sin(\alpha - \alpha)\right)+ gdm\sin(\alpha)\\
        &= md^2 \ddot \alpha + mLd\ddot \alpha+ gdm\sin(\alpha)\\
        \intertext{Addieren wir nun die beiden Bewegungsgleichungen, so erhalten wir}
        0 &= \left(md^2 + mLd - \frac{dm(Ms^2 + mL^2)}{sM + Lm} -\frac{d^2m^2L}{sM + Lm} \right) \ddot \alpha\\
        &= dmMs \frac{d  + L - s}{sM + Lm} \ddot \alpha\\
    \end{align*}
\end{enumerate}
\section*{Aufgabe 2}
\begin{enumerate}[(a)]
    \item $x$ sei die Länge, die über die Tischkante hängt.
    \begin{enumerate}[(i)]
        \item $V(x) = -mgx,\; T(x) = m\dot x^2\implies L = m\dot x^2 + mgx$
        \item $V(x) = -mgx - \int_0^x\frac{M}{L}gx'\d x' = -mgx - \frac{M}{2}gx^2,\; T = m\dot x^2 + \frac{M}{2}\dot x^2\implies L=\left(m + \frac{M}{2}\right)\dot x^2 + mgx + \frac{1}{2}\frac{M}{L}gx^2$
    \end{enumerate}
    \item \begin{enumerate}[(i)]
        \item $\displaystyle 0 = \Lagrange{x} = 2m\ddot x - mg \xRightarrow{v_0 = 0} \dot x = \frac{g}{2}t\xRightarrow{x_0 = l_0} x = \frac{g}{4}t^2 + l_0$
        \item $\displaystyle 0 \Lagrange{x} = (2m + M)\ddot x - mg - \frac{M}{L}gx$
        Eine Lösung der inhomogenen Gleichung ist $x_{\text{inh}} = - \frac{m}{M}L$.
        Die allgemeine Lösung ist also $$x(t) = A\cdot e^{\sqrt{\frac{Mg}{L(2m + M)}}\cdot t} + B \cdot e^{-\sqrt{\frac{Mg}{L(2m + M)}}\cdot t} - \frac{m}{M}L$$
        Mit den Anfangsbedingungen $\dot x(0) = 0$ und $x(0) = l_0$ erhalten wir als Lösung der Bewegungsgleichung
        $$\left(l_0 + \frac{m}{M}L\right)\cosh\left(\sqrt{\frac{Mg}{L(M+ 2m)}\cdot t}\right)$$
    \end{enumerate}
    \item \begin{enumerate}[(i)]
        \item $\frac{\d E}{\d t} = \frac{\d T + \d V}{\d t} = \frac{\d m\dot x^2 - \d m gx}{\d t} = 2m\dot x\ddot x - mg\dot x \overset{\ddot x = \frac{g}{2}}{=} m g\dot x - m g\dot x = 0$
        \item \begin{align*}
            \frac{\d E}{\d t} &= \frac{\d T + \d V}{\d t}\\
            &= \frac{\d \ \left(\left(m + \frac{M}{2}\right)\dot x^2 - mgx - \frac{1}{2} \frac{M}{L}gx^2\right)}{\d t}\\
            &= (2m + M)\dot x\ddot x - mg\dot x - \frac{M}{L}gx\dot x\\
            &\stackrel{(2m + M)\ddot x = mg + \frac{M}{L}gx}{=}\qquad \qquad m g\dot x + \frac{M}{L}gx\dot x - mg\dot x - \frac{M}{L}gx\dot x\\
            &= 0
        \end{align*}
    \end{enumerate}
\end{enumerate}
\section*{Aufgabe 3}
\begin{enumerate}[(a)]
    \item keine Lust zu te$\chi$en
    \item Erhaltungsgrößen sind $p_\varphi,\; p_\psi$ und die Energie.
\end{enumerate}
\end{document}