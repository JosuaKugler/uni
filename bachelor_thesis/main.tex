\documentclass{article}
\usepackage{amsmath, amsthm, amsfonts, amssymb}
\usepackage{mathtools}
\usepackage{tikz-cd}
\usepackage{enumerate}

\theoremstyle{plain}% default
\newtheorem{theorem}{Theorem}[section]
\newtheorem{lemma}{Lemma}[section]
\newtheorem{proposition}{Proposition}[section]
\newtheorem{remark}{Remark}[section]
\newtheorem{corollary}{Corollary}[section]

\theoremstyle{definition}
\newtheorem{definition}{Definition}[section]
\newtheorem{conjecture}{Conjecture}[section]
\newtheorem{example}{Example}[section]

\theoremstyle{remark}
\newtheorem*{note}{Note}
\newtheorem{case}{Case}



\newcommand{\cob}{\mathcal{C}_\mathcal{O}^\bullet}
\newcommand{\co}{\mathcal{C}_\mathcal{O}}
\newcommand{\ann}{\operatorname{Ann}}
\newcommand{\im}{\operatorname{im}}
\newcommand{\fitt}{\operatorname{Fitt}}
\renewcommand{\hom}{\operatorname{Hom}}

\title{Fermat's Last Theorem}
\author{Josua Kugler}

\begin{document}
    \maketitle
    \tableofcontents
    \section{Introduction}
    \section{An Overview of Wiles' proof}
    \newpage
    \section{Wiles' numerical criterion}
    Wiles has discovered a criterion for two rings in a specific category to be isomorphic that only depends on some numerical invariants
    of these rings. The aim of this section is to prove that criterion in its purely algebraic form. 
    %TODO: Explain how this fits into the rest of the proof.
    
    \subsection{Preliminaries}
    Let \(\mathcal{O}\) be the ring of integers of a finite extension \(K\) of \(\mathbb Q_\ell\). 
    As \(K\) is a local field, its ring of integers is a discrete valutation ring (DVR), i.e. 
    \(\mathcal O\) is a local, noetherian Dedekind ring with maximal ideal \(\lambda\). 
    It is complete with resp server usedect to the \(\lambda\)-adic topology, a principal ideal domain (PID) 
    and has residue field \(k \coloneqq \mathcal{O}/\lambda\)  to name some properties that we will use in the course of the proof.
    
    \(\mathbb Z_\ell\) is the ring of integers of \(\mathbb Q_\ell\) and \(\mathbb F_\ell = \mathbb Z_\ell/\ell \mathbb Z_\ell\) its residue field. 
    As \(K/\mathbb{Q}_\ell\) is finite, the residue field of \(\mathcal{O}\) is a finite extension of $\mathbb F_\ell$ and therefore finite. 
    %Not entirely clear to me

    \paragraph{The categories \(\co\) and \(\cob\)}
    In this section, we will mostly deal with very specific rings. Therefore we define the category \(\co\) where objects of \(\co\) are
    local complete noetherian \(\mathcal O\)-algebras with residue field \(k\) and the morphisms are local \(\mathcal{O}\)-algebra morphisms.
    Often, we even need some extra structure. 
    We obtain the category \(\cob\) from \(\co\) by equipping an object \(A\) with an additional surjective map
    \[\pi_A \colon A \twoheadrightarrow \mathcal{O},\]
    the so-called augmentation map. Objects in \(\cob\) are often called \textit{augmented rings}.
    The morphisms in \(\cob\) are local \(\mathcal{O}\)-algebra morphisms that respect the augmentation map structure, i.e. for a morphism
    \(f \colon A \to B\) we have the commutative diagram
    \[
    \begin{tikzcd}[column sep=small]
        A \arrow[rr, "f"] \arrow [dr, "\pi_A", swap, twoheadrightarrow] & & B \arrow [dl, "\pi_B", twoheadrightarrow]\\
        & \mathcal{O} &
    \end{tikzcd}.
    \]

    In order to state Wiles' criterion, we need some more definitions.
    \begin{definition}
        \(A \in \co\) is \textit{finite flat}, if \(A\) is finitely generated and torsion-free as an \(\mathcal{O}\)-module.
        Note that \(\mathcal{O}\) is a PID and therefore being torsion-free is equivalent to being flat as an \(\mathcal{O}\)-module.
    \end{definition}
    
    \begin{definition}[complete intersection]
        A finite flat ring \(A \in \co\) is called a \textit{complete intersection}, if \(A\) is isomorphic as an \(\mathcal{O}\)-algebra
        to a quotient
        \[A \cong \mathcal{O}[[X_1, \dots, X_n]]/(f_1, \dots, f_n),\] where there are as many relations as there are variables.
        %Why is this a good definition? Where does the name come from? see probably remark 5.2
    \end{definition}

    Let's take a look at an example.
    \begin{example}\label{ex:first_1}
        \(A = \{(a,b) \in \mathcal{O}\times\mathcal{O},\; a \equiv b\; (\operatorname{mod} \lambda^n)\} 
        \cong \mathcal{O}[[T]]/(T(T-\lambda^n))\) is a finite flat complete intersection in \(\cob\).
        The projection \(\pi_A\) is given by \(\pi_A(a,b) = a\).
        \begin{proof}
            Consider the map
            \begin{align*}
                \phi\colon \mathcal{O}[[T]]/(T(T-\lambda^n)) &\to A\\
                f &\mapsto (f(0), f(\lambda^n)).
            \end{align*}
            \subparagraph*{\(\phi\) is welldefined and respects the \(\mathcal{O}\)-algebra structure:}
            Let \(f_0\) be the constant term of a polynomial \(f\) and \(f_1 \coloneqq T^{-1}(f-f_0)\), s.t. \(f = f_0 + T\cdot f_1(T)\).
            Because of
            \[f(0) - f(\lambda^n) = (f_0 + 0\cdot f_1(0)) - (f_0 + \lambda^n \cdot f_1(\lambda^n)) = -\lambda^n \cdot f_1(\lambda^n),\]
            \(f(0) \equiv f(\lambda^n)\; (\operatorname{mod} \lambda^n)\) as required.
            Furthermore, \[\phi(T(T-\lambda^n)) = (0(-\lambda^n), \lambda^n(\lambda^n - \lambda^n)) = (0,0).\]
            Finally, we need to think about series in \(\mathcal{O}[[T]]\) with infinitely many terms.
            For the first component \(f(0)\) this doesn't matter, as \(\phi\) just takes the constant term. 
            As \(\mathcal{O}\) is complete with respect to the \(\lambda\)-adic topology, 
            the map \(\tilde\phi_2\colon \mathcal{O}[[T]] \to \mathcal{O},\; f \mapsto f(\lambda^n)\) is clearly welldefined and thus \(\phi\)
            is welldefined.
            %O-algebra structure:
            Let \(o \in \mathcal{O}\). Then 
            \[\phi(of) = ((of)(0),(of)(\lambda^n)) = (of(0), of(\lambda^n)) = o(f(0), f(\lambda^n)) = o\phi(f)\]
            \subparagraph*{Injectivity:}
            Let \(\phi(f) = 0\). Then \(f(0) = 0 \implies T | f\) and \(f(\lambda^n) = 0 \implies (T - \lambda) | f\).
            As a result, \(f \in T(T-\lambda)\).
            \subparagraph{Surjectivity:}
            Let \((a,b) \in A\).
            As \(a \equiv b \operatorname{mod} \lambda^n,\) we can write \(b = a + b' \cdot \lambda^n\). Because of
            \[\phi(\overline{a + b'T}) = (a, a + b' \lambda^n) = (a,b),\] \(\phi\) is surjective.
            \subparagraph*{\(A\in \cob\):}
            \(\mathcal{O}\) is noetherian, so \(\mathcal{O}[T]/(T(T-\lambda^n))\) is noetherian as well.
            \((\lambda, T)\) is a maximal ideal in \(\mathcal{O}[T]/(T(T-\lambda^n))\), because
            \[\left(\mathcal{O}[T]/(T(T-\lambda^n))\right)/(\lambda, T) = \mathcal{O}/(\lambda) = k.\]
            %Korollar 24.13 Alg II
            Therefore, the completion \(\mathcal{O}[T]/(T(T-\lambda^n))^{\wedge(\lambda, T)}\) 
            of \(\mathcal{O}[T]/(T(T-\lambda^n))\) with respect to \((\lambda, T)\) is a local ring with maximal ideal
            \(\widehat{(\lambda, T)}\).
            Consider the SES of finitely generated \(\mathcal{O}\)-modules
            \[
                0 \to (T(T - \lambda^n)) \mathcal{O}[T] \to \mathcal{O}[T] \to \mathcal{O}[T]/(T(T-\lambda^n)) \to 0.
            \]
            %Theorem 24.9
            As completion of finitely generated \(\mathcal{O}\)-modules is exact (because \(\mathcal{O}\) is noetherian), we get the SES
            \[
                0 \to (T(T-\lambda^n))\mathcal{O}[[T]] \to \mathcal{O}[[T]] \to \mathcal{O}[T]/(T(T-\lambda^n))^{\wedge (\lambda, T)}  \to 0.
            \]
            by completing with respect to \((\lambda, T)\).
            As a result, we have
            \[
                \mathcal{O}[T]/(T(T-\lambda^n))^{\wedge (\lambda, T)} = \mathcal{O}[[T]]/(T(T-\lambda^n)).
            \]
            As a result, \(\mathcal{O}[[T]]/(T(T-\lambda^n))\) is a local ring with maximal ideal \((\lambda, T)\).
            Therefore, its residue field is
            \[
                \mathcal{O}[[T]]/(T(T-\lambda^n))/(\lambda, T) = \mathcal{O}[T]/(T(T-\lambda^n))/(\lambda, T) = \mathcal{O}/(\lambda) = k.
            \]
            %Theorem 10.26 in Atiyah-MacDonald
            As \(\mathcal{O}[T]/(T(T-\lambda^n))\) is noetherian, its \((\lambda, T)\)-completion \(\mathcal{O}[[T]]/(T(T-\lambda^n))\)
            is again noetherian.

            In total, we get that \(A \cong \mathcal{O}[[T]]/(T(T-\lambda^n))\) is a local, complete, noetherian \(\mathcal{O}\)-algebra 
            with residue field \(k\; \implies A \in \co\). 
            \subparagraph{\(A\) is a finite flat complete intersection:}
            \(A\) is generated by \((1,1)\) and \(0, \lambda^n\) because
            \[
                (a,b) = a(1,1) + (0, \underbrace{b-a}_{\in \lambda^n}) = a(1,1) + c(0, \lambda^n).
            \]
            Also, \(A\) is torsion-free because \(\mathcal{O}\) is an integral domain. 
            As there is one variable and one relation in \(A \cong \mathcal{O}[[T]]/(T(T-\lambda^n))\), \(A\) is a complete intersection.
        \end{proof}
    \end{example}

    %probably add more examples

    \begin{example}\label{ex:last_1}
        \(U = \mathcal{O}[[X_1, \dots, X_n]]\) with projection \(\pi_U\colon U \to \mathcal{O},\; f \mapsto f(0)\) lies in \(\cob\).
        \begin{proof}
            \(\mathcal{O}\) is noetherian, so \(\mathcal{O}[X_1, \dots, X_{n}]\) is noetherian as well.
            \((\lambda, X_1, \dots, X_n)\) is a maximal ideal in \(\mathcal{O}[X_1, \dots, X_n]\), because
            \[\left(\mathcal{O}[X_1, \dots, X_n]\right)/(\lambda, X_1, \dots, X_n) = \mathcal{O}/(\lambda) = k.\]
            %Korollar 24.13 Alg II
            Therefore, the completion \[\mathcal{O}[X_1, \dots, X_n]^{\wedge(\lambda, X_1, \dots, X_n)} = \mathcal{O}[[X_1, \dots, X_n]]\]
            of \(\mathcal{O}[X_1, \dots, X_n]\) with respect to \((\lambda, X_1, \dots, X_n)\) is a local ring with maximal ideal
            \(\widehat{(\lambda, X_1, \dots, X_n)}\).
            Its residue field is \(\mathcal{O}[X_1, \dots, X_n]/(\lambda, X_1, \dots, X_n) = k\), as required.
            As \(\mathcal{O}[X_1, \dots, X_n]\) is noetherian, its \((\lambda, X_1, \dots, X_n)\)-completion is again noetherian.
        \end{proof}
    \end{example}

    \begin{remark}
        In example~\ref{ex:first_1} we could write \(A\) as a quotient of \(\mathcal{O}[[X]]\). 
        This is possible in a more general setting, in fact every \(A \in \co\) can be written as a quotient of
        \(U = \mathcal{O}[[X_1, \dots, X_n]]\) for suitable \(n\).
    \end{remark}
    \begin{proof}
        As \(A\) is a noetherian ring and \(\ker \pi_A\) is an ideal in \(A\), it is finitely generated and therefore also finitely generated
        as an \(A\)-module. Consider the map
        \begin{align*}
            \Phi\colon U = \mathcal{O}[[X_1, \dots, X_n]] &\to A\\
            X_i &\mapsto a_i,
        \end{align*}
        where \(\ker \pi_A = (a_1, \dots, a_n)\) and \(\pi_U\) is given by \(f \mapsto f(0)\). 
        As \((X_1, \dots, X_n)\) generate the kernel of \(\pi_U\), this is a map in \(\cob\).
        We have the short exact sequences
        \[
         0 \to \ker \pi_A \to A \to \im \pi_A \cong \mathcal{O} \to 0
        \]
        and
        \[
            0 \to \ker \pi_U \to U \to \im \pi_U \cong \mathcal{O} \to 0
        \]
        As both corresponding sequences split via the inclusion \(\mathcal{O} \hookrightarrow A\) resp. \(\mathcal{O} \hookrightarrow U\), 
        we can write \(A \cong \mathcal{O} \oplus \ker \pi_A\) and \(A[[X_1, \dots, X_n]] \cong A \oplus \ker \pi_A\).
        \(\Phi\) by definition induces an equality on the first component, a surjection on the second and therefore is surjective on the
        direct sum.
    \end{proof}

    \begin{definition}
        Let \(A \in \cob\). Then
        \[\phi_A \coloneqq (\ker \pi_A)/(\ker \pi_A)^2.\]
        %is the following helpful?
        The reader with background in algebraic geometry might notice that this can be though of as a tangent space, 
        in particular it is the cotangent space of the scheme \(\operatorname{spec}(A)\) at the point \(\ker \pi_A\).
        However this point of view is not necessary in the following, 
        it might be more a hint of how Wiles came to investigate this specific invariant.
    \end{definition}

    \begin{example}\label{ex:last_phi}
        Remember the definition of \(U\) in example \ref{ex:last_1}.
        The tangent space \(\phi_U = \ker \pi_U/(\ker \pi_U)^2\) is
        \[
            \mathcal{O}X_1 \oplus \dots \oplus \mathcal{O}X_n.
        \]
        Indeed, elements of \(f \in \ker \pi_U\) have no constant term as \(f(0) = 0\) and therefore are multiples of \(X\).
        Elements in \(\ker \pi_U^2\) are multiples of \(X^2\). As a result, we receive elements \(\overline{f} \in \phi_U\) by
        cutting of all higher terms of a power series \(f \in \ker \pi_U\).
    \end{example}

    \begin{remark}
        Write \(A\) as a quotient of \(U,\; A = U/(f_1, \dots, f_n)\).
        We then get \(\phi_A = \phi_U/(\overline{f_1}, \dots, \overline{f_n})\).
        As a quotient of \(\phi_U\) its a finitely generated \(\mathcal{O}\)-module.
    \end{remark}
    \begin{proof}
        Consider the following map of \(\mathcal{O}\)-modules
        \begin{align*}
            \Phi\colon \ker \pi_U = \mathcal{O}X_1 \oplus \dots \oplus \mathcal{O}X_n &\to (\ker \pi_A)/(\ker \pi_A)^2 = \phi_A\\
            a_1X_1 + \dots + a_nX_n &\mapsto [a_1X_1 + \dots + a_nX_n] \; \mod (\ker \pi_A)^2,
        \end{align*}
        where \([f]\) denotes the image of \(f\) in \(A\).
        Then, as \(\pi_A([f]) = f(0)\), we get that \(X_i \in \ker \pi_A \forall i\)
        and therefore \([f] \in \ker \pi_A \forall f \in \ker \pi_U\). Not only is \(\Phi\) welldefined,
        we can conclude that \(X_i \in \ker \pi_A \implies X_i^2 \in (\ker \pi_A)^2\)
        and therefore \(\Phi\) is also surjective and \((\ker \pi_U)^2 \subset \ker \Phi\).
        
        With this knowledge we get a welldefined surjective map
        \begin{align*}
            \tilde \Phi\colon \phi_U &\to \phi_A\\
            a_1X_1 + \dots + a_nX_n \; \mod (\ker \pi_U)^2 &\mapsto [a_1X_1 + \dots + a_nX_n] \; \mod (\ker \pi_A)^2.
        \end{align*}
        Elements in the kernel of this map are either generated by \(X_i^2\) s.t. they become 0 \(\mod (\ker \pi_A)^2\)
        or they become 0 by sending them to \(A = U/(f_i)\).
        As higher order terms of \(f_i\) are vanishing anyways, 
        the kernel of \(\tilde \Phi\) is generated by the \(\overline{f_i}\), i.e.
        \[
            \phi_A \cong \phi_U/(\overline{f_i})  
        \]
    \end{proof}

    \begin{example}\label{ex:first_phi}
        We now compute \(\phi_A\) where \(A\) was defined in example \ref{ex:first_1}.
        Remember that \(f = T(T-\lambda^n) = -\lambda^n T + T^2\). Therefore,
        \[\phi_A = \mathcal{O}T/(-\lambda^n T) = \mathcal{O}/\lambda^n.\]
    \end{example}

    \begin{definition}
        Let \(A \in \cob\). Then
        \[\eta_A \coloneqq \pi_A(\ann_A (\ker \pi_A))\] is an ideal in \(\mathcal{O}\).
    \end{definition}

    \begin{example}
        We now compute \(\eta_U\) for \(U\) from example \ref{ex:last_1}.
        \begin{align*}
            \eta_U &= \pi_U(\ann \ker \pi_U)\\
            &= \pi_U(\ann \mathcal{O}X_1 \oplus \dots \oplus \mathcal{O}X_n)\\
            &= \pi_U(0) = 0.
        \end{align*}
    \end{example}

    \begin{lemma}
        Let \(\mathfrak a \subset \mathcal{O}\) be an ideal. Then
        \[\mathfrak a \neq 0 \implies \mathcal{O}/\mathfrak a \text{ finite}.\]
    \end{lemma}
    \begin{proof}
        As \(\mathcal{O}\) is a DVR, \(\mathfrak a = \lambda^n\) for some \(n \in \mathbb N\) 
        where \(\lambda\) is the maximal ideal in \(\mathcal{O}\).
        Therefore, \(\mathcal{O}/\mathfrak{a} = \mathcal{O}/\lambda^n.\)

        Using the fact that \(\lambda = (t)\) for some uniformizer \(t\), we get \(\forall i \ge 1\) 
        the isomorphism \(\lambda^i/\lambda^{i+1} \cong \mathcal{O}/\lambda = k\) and thereby also the short exact sequence 
        \[0 \to \mathcal{O}/\lambda \cong \lambda^i/\lambda^{i+1} \to \mathcal{O}/\lambda^{i+1} \to \mathcal{O}/\lambda^{i} \to 0.\]
        As \(k = \mathcal{O}/\lambda\) is finite, we can use induction
        \[\# \mathcal{O}/\lambda^{i+1} = \# \mathcal{O}/\lambda \cdot \# \mathcal{O}/\lambda^i = \# k \cdot (\# k)^i = (\# k)^{i+1}\]
        and get \(\# \mathcal{O}/\mathfrak{a} = \# \mathcal{O}/\lambda^n = (\# k)^n\).
    \end{proof}

    \begin{example}\label{ex:first_eta}
        We now compute \(\eta_A\) for \(A\) from example \ref{ex:first_1}.
        \begin{align*}
            \eta_A &= \pi_A(\ann \ker \pi_A)\\
            &= \pi_A(\ann \{(0, b) \subset \mathcal{O}\times \mathcal{O} | b \equiv 0 \mod \lambda^n\})\\
            &= \pi_A(\{(a, 0) \subset \mathcal{O}\times \mathcal{O} | a \equiv 0 \mod \lambda^n\})\\
            &= \pi_A((\lambda^n) \times \mathcal{O})\\
            &= (\lambda^n)
        \end{align*}
    \end{example}

    With these results at hand, we can state 
    \begin{theorem}[Wiles' numerical criterion]\label{thm:wiles_numerical_criterion}
        Let \(R \twoheadrightarrow T\) a surjective morphism of augmented rings, \(T\) finite flat and \(\eta_T \neq 0\) 
        (i.e. \(\mathcal{O}/\eta_T\) finite).
        Then the following are equivalent
        \begin{enumerate}[(a)]
            \item \(\# \phi_R \le \#(\mathcal{O}/\eta_T)\),
            \item \(\# \phi_R = \#(\mathcal{O}/\eta_T)\),
            \item \(R\) and \(T\) are complete intersections, and \(R \to T\) is an isomorphism.
        \end{enumerate}
    \end{theorem}

    \subsection{Basic properties of \(\phi_A\) and \(\eta_A\)}
    In this subsection we prove the equivalence (a) \(\Leftrightarrow\) (b) in Theorem~\ref{thm:wiles_numerical_criterion}
    by investigating the invariants \(\phi_A\) and \(\eta_A\) that we defined last section.

    \begin{lemma}\label{lem:surjectivity_phi}
        A morphism \(f\colon A \to B \in \cob\) induces a homomorphism \(\phi_A \to \phi_B\) of \(\mathcal{O}\)-modules.
        This induced map is surjective if and only if the morphism \(A \to B\) is surjective.
    \end{lemma}
    \begin{proof}
        %TODO
        %see written notes and workflowy citations from matsumura etc.
        We have the commutative diagram
        \[
        \begin{tikzcd}[column sep=small]
            A \arrow[rr, "f"] \arrow [dr, "\pi_A", swap, twoheadrightarrow] & & B \arrow [dl, "\pi_B", twoheadrightarrow]\\
            & \mathcal{O} &
        \end{tikzcd}.
        \]
        It follows from the diagram that the restriction of \(f\) to \(\ker \phi_A\) maps to \(\ker \phi_B\),
        because \(\forall x \in \ker \phi_A\colon\; \pi_B(f(x)) = \pi_A(x) = 0\).
        Concatenating this with the projection to the tangent space, we get a map
        \[
            \tilde f\colon \ker \pi_A \to \ker \pi_B/(\ker \pi_B)^2 = \phi_B.
        \]
        In order to see that \(\tilde f\colon \phi_A \to \phi_B\) is welldefined, we need to show 
        \[
            f(\ker \pi_A)^2 \subset (\ker \pi_B)^2,
        \]
        however this follows from the fact that \(f(\ker \pi_A) \subset \ker \pi_B\) and that \(f\) is an algebra
        homomorphism:
        \[
            f(x^2) = \underbrace{f(x)}_{\in \ker \pi_B}\underbrace{f(x)}_{\in \ker \pi_B} \in (\ker \pi_B)^2 
        \]
        for any \(x \in \ker \pi_A\).

        First, let us assume that \(A \to B\) is a surjective map.
        In this case, every element \(x \in \ker \phi_B\) has a preimage in \(\ker \pi_A\). Indeed, \(\forall y \in f^{-1}(x)\subset A\colon\)
        \[
            \pi_A(y) = \pi_B(f(y)) = \pi_B(x) = 0.
        \]
        As a result, the induced map \(f\colon \ker \pi_A \to \ker \pi_B\) and its concatenation with
        the projection to \(\phi_B\), \(\tilde f\colon \ker \pi_A \to \ker \pi_B/(\ker \pi_B)^2\) are both surjective.
        In total, we obtain a surjective homomorphism \(\tilde f\colon \phi_A \to \phi_B\).
        
        Now, let the induced map \(\phi_A \to \phi_B\) be surjective. 
        Consider the ideal \(\mathfrak{a} = \mathfrak{m}_A B \subset B\). As \(B\) is a local ring
        with maximal ideal \(\mathfrak{m}_B\), by definition \(\mathfrak{a} \subset \mathfrak{m}_B\).
        We know by assumption that \(\ker \pi_A/(\ker \pi_A)^2 \to \ker \pi_B/(\ker \pi_B)^2\) is surjective.
        \textbf{Why is now \(\mathfrak{a} = \mathfrak{m}_B\)?}
        \textbf{Why is \(B\) a finitely generated \(A\)-module?}
        The \(A/\mathfrak{m}_A = k\)-module \(B/\mathfrak{m}_A B = B/\mathfrak{m}_B \cong k\)
        is obviously generated by 1. 
        Therefore, by Nakayama's Lemma, the finitely generated \(A\)-module \(B\) is
        also generated by 1, i.e. the map \(A \to B\) is surjective.
    \end{proof}

    \begin{corollary}\label{cor:surjectivity_phi}
        \(A \to B\) is surjective if and only if \[\phi_A \geq \phi_B.\]
    \end{corollary}

    \begin{lemma}
        If \(f \colon A \to B\) is surjective, then
        \begin{equation}
            \eta_A \subset \eta_B, \quad \text{i.e.,}\quad \#(\mathcal{O}/\eta_A) \geq \#(\mathcal{O}/\eta_B).  
        \end{equation}
    \end{lemma}
    \begin{proof}
        As we have seen in the proof of lemma~\ref{lem:surjectivity_phi}, a surjective map \(f\) induces a surjective
        map on the kernels, \(f\colon \ker \pi_A \to \ker\pi_B\).
        Now let \(x \in \ann_A \ker \pi_A\), i.e. \(x \cdot a = 0\;\; \forall a \in \ker \pi_A\).
        For all \(b \in \ker \pi_B\) and any preimage \(a \in \ker \pi_A\) we have
        \[
            f(x) \cdot b = f(x) \cdot f(a) = f(x \cdot a) = f(0) = 0.
        \]
        As a result, \(f(x) \in \ann_B\ker \pi_B\) and we obtain a map
        \[
            \tilde f\colon\ann_A\ker \pi_A \to \ann_B \ker \pi_B.  
        \]
        In order to show \(\eta_A \subset \eta_B\), let \(x \in \eta_A = \pi_A(\ann_A \ker \pi_A)\), i.e.
        \(x = \pi_A(y)\) for some \(y \in \ann_A \ker \pi_A\). By the commutative diagram
        \[
        \begin{tikzcd}[column sep=small]
            \ann_A \ker \pi_A \arrow[rr, "\tilde f"] \arrow [dr, "\pi_A", swap, twoheadrightarrow] 
            & & \ann_B \ker \pi_B \arrow [dl, "\pi_B", twoheadrightarrow]\\
            & \mathcal{O} &
        \end{tikzcd},
        \]
        we get
        \[
            x = \pi_A(y) = \pi_B(\tilde f(y)) \in \pi_B(\ann_B \ker \pi_B) \implies x \in \eta_B,
        \]
        as desired.
    \end{proof}

    \begin{lemma}\label{lem:standard_ineq}
        Let \(A \in \co\). Then \[\#\phi_A \geq \#(\mathcal{O}/\eta_A).\]
    \end{lemma}
    \begin{proof}
        %TODO
        %probably appendix for Fitting ideals?
        We have \(\# M = \# (\mathcal{O}/\fitt_\mathcal{O}(M))\) for finite \(\mathcal{O}\)-modules and 
        \[
            \fitt_R(M) \subset \ann_R(M)
        \]
        for any \(R\)-module \(M\).
        Furthermore, if \(M\) is a finitely generated \(A\)-module for \(A \in \cob\), then:
        \[
            \pi_A(\fitt_A(M)) = \fitt_\mathcal{O}(M \otimes_A \mathcal{O})
        \]
        and
        \[
            \phi_A = \ker \pi_A \otimes_A\mathcal{O},
        \]
        where \(\mathcal{O}\) becomes an \(A\)-module via the map \(\pi_A \colon A \to \mathcal{O}\).
        Bringing all this together, we get
        \begin{align*}
            \fitt_\mathcal{O}(\phi_A) &= \fitt_\mathcal{O}(\ker \pi_A \otimes_A\mathcal{O})
            \intertext{with \(M = \ker \pi_A\) it follows}
            &= \pi_A(\fitt_A(\ker \pi_A))\\
            &\subset \pi_A(\ann_A(\ker \pi_A))\\
            &= \eta_A
        \end{align*}
        Using \(\# M = \# (\mathcal{O}/\fitt_\mathcal{O}(M))\) for \(M = \phi_A\), we finally obtain
        \[
            \# \phi_A = \# (\mathcal{O}/\fitt_\mathcal{O}(\phi_A)) \geq \# (\mathcal{O}/\eta_A).
        \]
    \end{proof}

    \begin{proposition}
        (a) \(\Leftrightarrow\) (b) in Theorem~\ref{thm:wiles_numerical_criterion}.
    \end{proposition}
    \begin{proof}
        By assumption, \(R \to T\) is a surjective morphism in \(\cob\).
        With corollary~\ref{cor:surjectivity_phi} it follows that \(\#\phi_R \geq \#\phi_T\).
        Lemma~\ref{lem:standard_ineq} tells us that \(\#\phi_T \geq \#(\mathcal{O}/\eta_T)\).
        The inequalities combine to \[\#\phi_R \geq \#(\mathcal{O}/\eta_T).\]
        \begin{itemize}
            \item[(a)\(\implies\)(b)] (a) gives us \(\#\phi_R \leq \#(\mathcal{O}/\eta_T)\), so combined with the 
            inequality \(\#\phi_R \geq \#(\mathcal{O}/\eta_T)\) we have just proven we conclude that (b) must hold.
            \item[(b)\(\implies\)(a)] Obvious.
        \end{itemize}      
    \end{proof}


    \subsection{Regular sequences and the Koszul complex}
    The goal of this section is to prove some technical lemmata and to introduce
    the Koszul complex that we will use to construct two \(\mathcal{O}[[X]]\)-free resolutions for \(A\).
    This will turn out to be very crucial in the next section.

    We start with a few definitions from commutative algebra.
    \begin{definition}[primary ideal]
        Let \(R\) be a local ring and \(\mathfrak{a} \subsetneq R\) an ideal. 
        \(\mathfrak{a}\) is said to be primary if every zero divisor in \(R/\mathfrak{a}\) is nilpotent.
    \end{definition}
    Recall that the dimension of a ring is given by
    \[
        \sup \left\{n | 
                    \mathfrak{p}_0 \subsetneq \dots \subsetneq \mathfrak{p}_n \subsetneq R,\; \mathfrak{p}_i \text{ prime} 
            \right\}.
    \]
    \begin{definition}[system of parameters]
        Let \(x_1, \dots, x_n\) generate a primary ideal of \(R\). If \(n = \dim R\) then \(x_1, \dots, x_n\) is called 
        a system of parameters.
    \end{definition}
    

    \subsection{Complete intersections and the Gorenstein condition}
    The goal of this section is to show that finite flat complete intersection rings in \(\cob\) 
    satisfy a Gorenstein condition, i.e. a specific form of self-duality.
    This fact can then be used to show (c) \(\implies\) (b) in Theorem~\ref{thm:wiles_numerical_criterion}.
    Although there is a very general notion of Gorenstein rings, for
    the purpose of this proof we only need a special case,
    \begin{definition}
        Let \(A \in \co\) be finite flat. \(A\) is called Gorenstein, if there is an isomorphism of \(A\)-modules
        \[
            \Psi\colon \hom_\mathcal{O}(A, \mathcal{O}) \cong A.
        \]
    \end{definition}
    
\end{document}