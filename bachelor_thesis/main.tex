\documentclass{article}
\usepackage{amsmath, amsthm, amsfonts, amssymb}
\usepackage{mathtools}
\usepackage{tikz-cd}
\usepackage{enumerate}

\theoremstyle{plain}% default
\newtheorem{theorem}{Theorem}[section]
\newtheorem{lemma}{Lemma}[section]
\newtheorem{proposition}{Proposition}
\newtheorem*{corollary}{Corollary}

\theoremstyle{definition}
\newtheorem{definition}{Definition}[section]
\newtheorem{conjecture}{Conjecture}[section]
\newtheorem{example}{Example}[section]

\theoremstyle{remark}
\newtheorem*{remark}{Remark}
\newtheorem*{note}{Note}
\newtheorem{case}{Case}



\newcommand{\cob}{\mathcal{C}_\mathcal{O}^\bullet}
\newcommand{\co}{\mathcal{C}_\mathcal{O}}
\newcommand{\ann}{\operatorname{Ann}}

\title{Fermat's Last Theorem}
\author{Josua Kugler}

\begin{document}
    \maketitle
    \tableofcontents
    \section{Introduction}
    \section{An Overview of Wiles' proof}
    \newpage
    \section{Wiles' numerical criterion}
    Wiles has discovered a criterion for two rings in a specific category to be isomorphic that only depends on some numerical invariants
    of these rings. The aim of this section is to prove that criterion in its purely algebraic form. 
    %TODO: Explain how this fits into the rest of the proof.
    
    \subsection{Preliminaries}
    Let \(\mathcal{O}\) be the ring of integers of a finite extension \(K\) of \(\mathbb Q_\ell\). 
    As \(K\) is a local field, its ring of integers is a discrete valutation ring (DVR), i.e. 
    \(\mathcal O\) is a local, noetherian Dedekind ring with maximal ideal \(\lambda\). 
    It is complete with respect to the \(\lambda\)-adic topology, a principal ideal domain (PID) 
    and has residue field \(k \coloneqq \mathcal{O}/\lambda\)  to name some properties that we will use in the course of the proof.
    
    \(\mathbb Z_\ell\) is the ring of integers of \(\mathbb Q_\ell\) and \(\mathbb F_\ell = \mathbb Z_\ell/\ell \mathbb Z_\ell\) its residue field. 
    As \(K/\mathbb{Q}_\ell\) is finite, the residue field of \(\mathcal{O}\) is a finite extension of $\mathbb F_\ell$ and therefore finite. 
    %Not entirely clear to me

    \paragraph{The categories \(\co\) and \(\cob\)}
    In this section, we will mostly deal with very specific rings. Therefore we define the category \(\co\) where objects of \(\co\) are
    local complete noetherian \(\mathcal O\)-algebras with residue field \(k\) and the morphisms are local \(\mathcal{O}\)-algebra morphisms.
    Often, we even need some extra structure. 
    We obtain the category \(\cob\) from \(\co\) by equipping an object \(A\) with an additional surjective map
    \[\pi_A \colon A \twoheadrightarrow \mathcal{O},\]
    the so-called augmentation map. Objects in \(\cob\) are often called \textit{augmented rings}.
    The morphisms in \(\cob\) are local \(\mathcal{O}\)-algebra morphisms that respect the augmentation map structure, i.e. for a morphism
    \(f \colon A \to B\) we have the commutative diagram
    \[
    \begin{tikzcd}[column sep=small]
        A \arrow[rr, "f"] \arrow [dr, "\pi_A", swap, twoheadrightarrow] & & B \arrow [dl, "\pi_B", twoheadrightarrow]\\
        & \mathcal{O} &
    \end{tikzcd}.
    \]

    In order to state Wiles' criterion, we need some more definitions.
    \begin{definition}
        \(A \in \co\) is \textit{finite flat}, if \(A\) is finitely generated and torsion-free as an \(\mathcal{O}\)-module.
        Note that \(\mathcal{O}\) is a PID and therefore being torsion-free is equivalent to being flat as an \(\mathcal{O}\)-module.
    \end{definition}
    
    \begin{definition}[complete intersection]
        A finite flat ring \(A \in \co\) is called a \textit{complete intersection}, if \(A\) is isomorphic as an \(\mathcal{O}\)-algebra
        to a quotient
        \[A \cong \mathcal{O}[[X_1, \dots, X_n]]/(f_1, \dots, f_n),\] where there are as many relations as there are variables.
        %Why is this a good definition? Where does the name come from? see probably remark 5.2
    \end{definition}

    \begin{definition}
        Let \(A \in \cob\). Then
        \[\phi_A \coloneqq (\ker \pi_A)/(\ker \pi_A)^2.\]
        %is the following helpful?
        The reader with background in algebraic geometry might notice that this can be though of as a tangent space, 
        in particular it is the cotangent space of the scheme \(\operatorname{spec}(A)\) at the point \(\ker \pi_A\).
        However this point of view is not necessary in the following, 
        it might be more a hint of how Wiles came to investigate this specific invariant.
    \end{definition}

    \begin{definition}
        Let \(A \in \cob\). Then
        \[\eta_A \coloneqq \pi_A(\ann_A (\ker \pi_A))\] is an ideal in \(\mathcal{O}\).
    \end{definition}

    \begin{lemma}
        Let \(A \in \cob\).
        \[\eta_A \neq 0 \implies \mathcal{O}/\eta_A \text{ finite}.\]
    \end{lemma}
    \begin{proof}
        As \(0 \neq \eta_A\) is an ideal in the DVR \(\mathcal{O}\), \(\eta_A = \lambda^n\) for some \(n \in \mathbb N\) 
        where \(\lambda\) is the maximal ideal in \(O\).
        Therefore, \(\mathcal{O}/\eta_T = \mathcal{O}/\lambda^n.\)

        Using the fact that \(\lambda = (t)\) for some uniformizer \(t\), we get \(\forall i \ge 1\) 
        the isomorphism \(\lambda^i/\lambda^{i+1} \cong \mathcal{O}/\lambda = k\) and thereby also the short exact sequence 
        \[0 \to \mathcal{O}/\lambda \cong \lambda^i/\lambda^{i+1} \to \mathcal{O}/\lambda^{i+1} \to \mathcal{O}/\lambda^{i} \to 0.\]
        As \(k = \mathcal{O}/\lambda\) is finite, we can use induction
        \[\# \mathcal{O}/\lambda^{i+1} = \# \mathcal{O}/\lambda \cdot \# \mathcal{O}/\lambda^i = \# k \cdot (\# k)^i = (\# k)^{i+1}\]
        and get \(\# \mathcal{O}/\eta_A = \# \mathcal{O}/\lambda^n = (\# k)^n\).
    \end{proof}

    With these definitions at hand, we can state 
    \begin{theorem}[Wiles' numerical criterion]\label{thm:wiles_numerical_criterion}
        Let \(R \twoheadrightarrow T\) a surjective morphism of augmented rings, \(T\) finite flat and \(\eta_T \neq 0\).
        Then the following are equivalent
        \begin{enumerate}[(a)]
            \item \(\# \phi_R \le \#(\mathcal{O}/\eta_T)\),
            \item \(\# \phi_R = \#(\mathcal{O}/\eta_T)\),
            \item \(R\) and \(T\) are complete intersections, and \(R \to T\) is an isomorphism.
        \end{enumerate}
    \end{theorem}

    \subsection{Basic properties of \(\phi_A\) and \(\eta_A\)}
    In this subsection we prove the equivalence (a) \(\Leftrightarrow\) (b) in Theorem~\ref{thm:wiles_numerical_criterion}
    by investigating the invariants \(\phi_A\) and \(\eta_A\) that we defined last week.


    %phi_A is a finitely generated O-module


\end{document}