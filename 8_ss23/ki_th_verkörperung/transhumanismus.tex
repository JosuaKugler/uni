\documentclass[12pt]{article}
\linespread{1.5} %regulate line spacing
\usepackage[top=3cm,left=2.5cm, right=2.5cm]{geometry}
\usepackage[T1]{fontenc}
\usepackage[utf8]{inputenc}
\usepackage[ngerman]{babel}
\usepackage{comment}
\title{Essay: KI, Transhumanismus und Verkörperung}
\author{Josua Kugler, 4119892, Mathematik}

\begin{document}
\maketitle
\noindent Kürzliche Entwicklungen im Bereich KI durch die \glqq Transformator\grqq-Architektur und Large Language Models haben großes gesellschaftliches Interesse
am Thema KI geweckt. Durch die große Bandbreite an Themen, auf denen Sprachmodelle wie ChatGPT oft sehr hilfreiche und menschenähnliche Antworten aus einer
riesigen Datenbasis erzeugen können, stellt sich fast von alleine die Frage, was Menschen jetzt eigentlich noch besser können als diese Sprachmodelle.
Die scheinbare Abstraktionsfähigkeit, Adaptabilität und Lernfähigkeit dieser Modelle machen viele Fähigkeiten für Maschinen zugänglich,
die vorher Menschen vorbehalten waren und teils sogar als Unterscheidungsmerkmale angeführt wurden. 
Hinzu kommt oft ein radikaler Materialismus, der Leben einfach als biochemischen Algorithmus auffasst und damit zu weiteren Kontroversen führt.

Kurz könnte man sagen: Zwischen der Reduktion auf eine biologische Maschine und der Hybris, sich selbst kopieren zu können und als Maschine wiederzuerschaffen,
befindet sich die Vorstellung davon, wer der Mensch ist, in einer Krise.
Das Menschenbild schwankt also zwischen Großartigkeit und Selbstverachtung.
Den eigenen Wert auf das von sich gezeichnete übertrieben selbstverherrlichende Bild setzend, 
leidet der Mensch letztlich an der Spannung zwischen dem Glauben an dieses Selbstverständnis und dem gleichzeitigen Anzweifeln jeglichen Selbstwertes.
Der Lösungsversuch basiert auf der Rettung des Selbstbildes, was aber von vornherein zum Scheitern verurteilt ist.
Insbesondere kann das Menschenbild nicht einfach durch Konstruktion immer besserer Maschinen gelöst werden.
Glaubt man den Dystopien einiger Futuristen und Transhumanisten, so führt die beschriebene Entwicklung sogar zur Ablösung des
Menschen durch eine neue Generation künstlich geschaffener Wesen.
Das Menschenbild des Tech-Enthusiasmus und der Transhumanisten trägt also konsequent durchgedacht bereits die Abschaffung des Menschen in sich
und ist damit in gewissem Sinne kein Menschenbild, sondern ein Anti-Menschenbild.

Daher ist es unabdingbar, dieses Menschenbild kritisch zu diskutieren.

Grundsätzlich wird oft, ohne dies zu hinterfragen, davon ausgegangen, dass der \glqq Geist\grqq\ eines Menschen vom Körper getrennt betrachtet
und durch Simulation dieses Geistes bzw. des Gehirns ein Mensch hinreichend abgebildet werden kann. Dabei werden jedoch zwei Aspekte vernachlässigt.
Zum einen handelt es sich bei dieser Vorgehensweise um Zerebrozentrismus, also den ungerechtfertigt starken Fokus auf das Gehirn.
Um den Geist eines Menschen zu verstehen, müssen viele weitere Prozesse berücksichtigt werden, die nicht neuronaler Natur sind (z.B. gewisse Hormone etc.).
Diese können nicht durch Simulation des Gehirns und nicht einmal durch Simulation des gesamten Körpernervensystems adäquat abgebildet werden.
Der daran anknüpfende zweite vernachlässigte Aspekt besteht darin, dass der menschliche Geist nicht notwendigerweise allein auf biologische Prozesse reduziert werden kann.

Die unzulängliche Berücksichtigung dieser beiden Aspekte führt in der Entwicklung zu folgendem Vorgehen.
Basierend auf der Annahme, der Mensch sei schlicht die Summe der biologischen Prozesse, die in seinem Körper ablaufen, wird versucht, 
Subprozesse oder Prozesszusammenhänge näherungsweise nachzubilden.
Beginnen wir beispielsweise mit dem Gehirn, so sind neuronale Netze ein erster Versuch, sich der Komplexität des Gehirns zu nähern.
In der aktuellen Forschung wird immer wieder versucht, die Funktionsweise des Gehirns durch die Architektur der Algorithmen noch genauer abzubilden.
Ein Beispiel: Der \glqq Lernprozess\grqq\ bei klassischen Neuronalen Netzen wird dadurch simuliert, dass Gewichtungen verschiedener Parameter durch Feedback aus den
gegebenen Daten verändert und den korrekten Gewichtungen angenähert werden. Das geschieht durch back-propagation:
Wird dem Algorithmus ein Datenobjekt zur Klassifizierung übergeben, so wird die Differenz zur korrekten Klassifizierung berechnet und dann mithilfe
der Kettenregel zurückgerechnet, welchen Einfluss die einzelnen Gewichte auf diese Abweichung hatten. Stück für Stück rückwärts durch den Algorithmus gehend,
werden dann die Gewichtungen angepasst. Im menschlichen Gehirn allerdings läuft dieser Prozess andersherum ab.
Darum gibt es erste Versuche von KI-Architekturen, die in ebendieser Reihenfolge Gewichte abändern (forward-forward).
Die Resultate sind bisher nicht überzeugend, sicherlich auch deswegen, weil in diesem Bereich noch vergleichsweise wenig Forschung geschehen ist. 
Generell ist es aber gut vorstellbar, dass die Algorithmen zukünftig immer mehr Details der Funktionsweise des Gehirns übernehmen
und damit seiner Simulation graduell näherkommen (auch wenn das natürlich aufgrund der Komplexität des Gehirns sehr spekulativ bleiben muss).
Allerdings muss an dieser Stelle ja nicht aufgehört werden. Stattdessen könnten weitere für das menschliche Denken und Handeln relevante Prozesse integriert werden.
Mit Hormonkreisläufen beginnend, könnte man versuchen, sich graduell der Komplexität der im Körper ablaufenden Prozesse anzunähern.

All dem wird man natürlich und auch zu Recht entgegnen, dass derartige Vorstellungen völlig unrealistisch sind.
Trotz zahlreicher Versuche (z.B. Human Brain Project) hat die Forschung keine signifikanten Resultate in dieser Richtung erzielt.
Tatsächlich ist es sogar so, dass nicht nur keine Fortschritte erzielt werden, sondern im Verlauf der Jahre immer deutlicher wird, 
wie sehr die Komplexität des menschlichen Gehirns im Besonderen und des gesamten menschlichen Körpers im Allgemeinen 
jegliche simulierbare Komplexität weit übersteigt.
Allein schon, dass ein Gehirn analog funktioniert, genügt, um jede digitale Simulation immer nur Näherung bleiben zu lassen.
Als kontinuierliche Prozesse entziehen sich die Abläufe in unserem Körper a priori der digitalen Berechenbarkeit als solcher.
%Jede Zerlegung in diskrete Einzeleinheiten eliminiert das tatsächlich Bewegte. Es kann zwar beliebige hohe Präzision erreicht werden,
%aber nie tatsächliche Übereinstimmung.

Ungeachtet der Frage, ob es jemals technisch möglich sein wird, einen gesamten Menschen auch nur annähernd zu simulieren, muss dennoch
untersucht werden, ob der Unterschied zwischen Mensch und Maschine einzig in der ungleich höheren Komplexität des Menschen begründet ist.
Ist der Mensch am Ende doch einfach eine unglaublich vielschichtige (biologische) Maschine?
Der fundamentale Komplexitätsunterschied mag als Argument genügen, um das Konzept des \glqq mind-uploading\grqq\ zu dekonstruieren,
aus humanistischer Sicht bleibt diese Diskussion jedoch unbefriedigend. Gibt es keinen prinzipiellen Unterschied zwischen Mensch und KI,
so wäre doch zumindest die philosophische Grundlage der Transhumanisten gesichert (auch wenn man an der technischen Umsetzbarkeit ihrer 
futuristischen Ideen große Zweifel äußern kann).

\begin{comment}
Beschäftigt man sich genauer mit KI, Simulation oder Robotern, so fällt auf, dass alle menschenähnlichen bzw. humanoiden Konzepte stets
unabhängig von der Hardware sind. Computerprogramme und Daten können einfach auf anderen Speicherplatten abgelegt werden,
Roboter sind völlig austauschbar, es spielt keine Rolle welche Metallteile genau verwendet werden, da alle verwendeten Teile
auf der relevanten Skala im Wesentlichen gleich sind. Diese Austauschbarkeit ist z.B. wesentlich dafür, dass Algorithmen/Roboter keinen Prozess kennen,
der dem Tod bei Menschen ähnelt. Damit macht sie auch für einige Tech-Enthusiasten den Reiz dieser Technologien aus - der Traum von der Unsterblichkeit ist
so alt wie die Menschheit selber, bereits das Buch Kohelet kennt diese Sehnsucht: 
"Auch hat er [Gott] die Ewigkeit in ihr [der Menschen] Herz gelegt; nur dass der Mensch nicht ergründen kann das Werk, das Gott tut, weder Anfang noch Ende." 
(Kohelet 3,11). Im Gegensatz zum Prediger kommen viele Transhumanisten sehr wohl zu der Überzeugung, das Werk Gottes ergründen zu können
und versuchen sich daran, die Ewigkeit durch Mind-Uploading auf eigene Faust zu erlangen.
Die Unsterblichkeit macht aber die tatsächliche Simulation des Menschen unmöglich, da das Sterben ein wesentlicher Bestandteil des Menschseins ist.
Gefühle wie Angst o.ä. setzen die Sterblichkeit als notwendige Bedingung voraus. Allein schon Hunger oder Durst ergeben nur dann Sinn,
wenn es tatsächlich eine Art Sterblichkeit gibt.

Tatsächlich gibt uns aber genau dieser Sachverhalt noch ein zweites Argument an die Hand, um zu zeigen, 
dass Menschen sich grundsätzlich von Algorithmen/Robotern unterscheiden. Für Menschen ist es nicht egal, 
in welchem Körper sie sich befinden, bzw. tatsächlich ergibt allein schon der vorausgehende Satz nicht wirklich viel Sinn, da ein Mensch sich
nicht in einem Körper befindet, sondern der Körper wesentlich für den Menschen ist. Ein Mensch ist nicht in einem Körper, sondern in einem gewissen Sinne
ist der Mensch sein Körper. Natürlich ist noch nicht ausgeschlossen, dass der Mensch noch mehr ist als nur sein Körper, aber man kann zumindest
einen Menschen nicht von seinem Körper trennen. Die Lebendigkeit, der Geist des Menschen, alles was den Mensch als Mensch auszeichnet ist eben gerade 
nicht als abstrakte Information übertragbar, sondern ist an die Körperlichkeit gekoppelt.

Die Frage ist nun, inwiefern diese Körperlichkeit tatsächlich so ein fundamentaler Unterschied ist. Natürlich ist auch die auf einer Festplatte gespeicherte
Information ein an Materie gekoppelter Zustand. Je nach Speicherart ist die Information darin kodiert,
wo sich gewisse Elektronen befinden bzw. welche Spannungen in einem Leiter vorliegen. Das unterscheidet sich zunächst einmal nicht grundsätzlich
von der Art und Weise, wie biologische Prozesse im Körper ablaufen.
Es gibt sogar sogenanntes neuromorphes Computing, bei dem durch die Hardwarekonfiguration Synapsenzustände des Gehirns abgebildet werden. 
Mit diesen Beispielen will man zeigen, dass Algorithmen und Daten auf eine gewisse Art und Weise auch \glqq verkörpert\grqq\ sind.
Natürlich wird man (zu Recht) einwenden, dass zwischen verschiedenen Körpern wesentlich größere Unterschiede vorliegen als zwischen verschiedenen
Speicherzuständen auf baugleichen Speichermedien.
Insbesondere haben verschiedene Körper nicht einen gemeinsamen Grundzustand (\glqq Werkseinstellungen\grqq), sie unterscheiden sich also nicht nur in Konfiguration,
sondern auch in ihrer Zusammensetzung, Gestalt, Entwicklung, etc.
Nichtsdestotrotz könnte man dem entgegenhalten, dass Maschinen einfach noch nicht die nötige Komplexität erreicht haben.
Prinzipiell könnten Maschinen auch vollständig individualisiert gebaut werden, oder sogar so konfiguriert werden, dass sie ihre eigene Hardware verändern.
Abgesehen von der aktuell unerreichbar scheinenden Komplexität gäbe es kein prinzipielles Argument, das dagegen spricht,
Maschinen zu konstruieren, die auf sehr ähnliche Weise eng verbunden sind mit ihrer Hardware wie Menschen mit ihrem Körper.
Allerdings wären in obiger Argumentation konstruierte Maschinen dann künstlich erzeugtes Leben.
Damit wäre dann Bewusstsein simuliert durch das Nachbauen von Leben und damit eigentlich die Verbindung von Leben und Bewusstsein noch weiter verstärkt.
Zusammenfassend wäre damit die einzige Möglichkeit, wirklich menschlichen Geist zu simulieren, indem effektiv ein lebendiges Wesen gebaut wird,
das dem Menschen in wesentlichen Punkten gleicht, aber gegebenenfalls einfacher oder simpler gebaut ist.
\end{comment}

Aus allen bisher vorgebrachten Argumenten lässt sich folgern, dass menschliche Forschung und Entwicklung sehr weit davon entfernt ist, irgendetwas in Richtung
\glqq mind uploading\grqq\ oder Ähnliches zu erreichen. Vermutlich wird es sogar nie vollständig möglich sein, weil der Komplexitätsunterschied so groß ist.
Trotzdem ist die Frage noch nicht befriedigend geklärt, ob es überhaupt einen prinzipiellen, nicht nur graduellen Unterschied zwischen Mensch und Maschine geben muss.
Kann Bewusstsein künstlich erzeugt werden? Könnten wir theoretisch Maschinen bauen, die bewusst sind?
Wäre Bewusstsein nämlich einfach eine Folge biochemischer Prozesse, dann könnte nach den bisher vorgebrachten Argumenten noch nicht ausgeschlossen werden,
dass auch Maschinen Bewusstsein entwickeln. 
Ist Bewusstsein aber transzendent, so würden sich Menschen an dieser Stelle fundamental von Maschinen unterscheiden.
Das kann aber nicht allein aus der Außensicht geklärt werden. Insbesondere dann, wenn Maschinen eine perfekte Simulation erreichen sollten, wird deutlich, dass diese Frage nicht empirisch klärbar ist.

Daher ist an dieser Stelle eine gesellschaftliche Debatte dringend vonnöten. 
Die oft als selbsverständlich angenommenen Prämissen von z.B. Transhumanisten müssen kritisch hinterfragt und diskutiert werden.
Sollten die aktuellen Entwicklungen tatsächlich dazu führen, dass zwischen Menschen und guten Simulationen nicht mehr unterschieden wird,
so sollte dies zumindest bewusst geschehen und nicht einfach unreflektiert als Folge eines blinden Fortschrittsglaubens.
%Stellt sich dann wirklich heraus, dass ein treibender Faktor hinter diesen Entwicklungen eine Art gesamtgesellschaftlicher Narzissmus ist,
%so muss dringend um ein gesundes Menschenbild gerungen werden.
\end{document}
