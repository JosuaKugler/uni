\documentclass{article}
\usepackage{josuamathheader}
\usepackage{listings}
\newcommand{\Fib}{\operatorname{Fib}}

\begin{document}
	\ipilayout{6}
	\section*{Aufgabe 1}
	siehe .odg Datei.
	\section*{Aufgabe 2}
	\begin{enumerate}[(a)]
		\item \lstinputlisting[language=C++, caption={primfaktor.cc}]{primfaktor.cc}
		\item Zu Beginn gibt es einen konstanten Initialisierungsaufwand.
		Ist $n$ prim, so wird die Schleife \lstinline{sqrt(n)} mal durchlaufen.
		Die Schleifenbedingung der \lstinline{while}-Schleife ist allerdings nie erfüllt.
		Pro Schleifendurchlauf wird also genau einmal die \lstinline{teilt}-Funktion aufgerufen.
		Die Komplexität der \lstinline{teilt}-Funktion sei $C(n)$. 
		Dann ist die Gesamtkomplexität des Programm $\Omega(C(n)\cdot \sqrt{n})$
	\end{enumerate}
	\section*{Aufgabe 3}
	\lstinputlisting[language=C++, caption={taschenrechner.cc}]{taschenrechner.cc}
\end{document}
