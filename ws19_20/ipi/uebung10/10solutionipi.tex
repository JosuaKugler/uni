\documentclass{article}
\setlength{\headheight}{25pt}
\usepackage{josuamathheader}
\begin{document}
    \ipilayout{10}
    \section*{Aufgabe 1}
    \lstinputlisting[language=C++, caption={aufgabe1.cc}]{aufgabe1.cc}
    \section*{Aufgabe 2}
    \begin{itemize}
        \item In Zeile 41 \lstinline{aq = ap;} ist ein Fehler, da \lstinline{aq} private ist.
        \item In Zeile 52 \lstinline{aX();} ist ein Fehler, da \lstinline{aX()} private ist.
        \item In Zeile 60 \lstinline{b.bY();} ist ein Fehler, da \lstinline{bY();} private ist.
        \item In Zeile 64 \lstinline{c.aq = 5;} ist ein Fehler, da \lstinline{aq} private ist.
    \end{itemize}
    \section*{Aufgabe 3}
    \begin{itemize}
        \item \lstinline{A a;} ist ein Fehler, da \lstinline{A} eine abstrakte Klasse ist.
        \item \lstinline{pa->c();} ist ein Fehler, da \lstinline{A::c} private ist.
        \item \lstinline{pa->b();} ist ein Fehler, da \lstinline{b()} eine Methode in \lstinline{B} ist.
        \item \lstinline{pa->vb();} ist ein Fehler, da \lstinline{vb()} eine Methode in \lstinline{B} ist.
        \item \lstinline{pa->a(x);} wird für float überladen und liefert daher keinen Fehler, obwohl keine Methode explizit dafür definiert ist
        \item \lstinline{pb->a();} ist ein Fehler, da in \lstinline{B} die Methode \lstinline{void a()} durch \lstinline{void a(int/double a)} überschrieben wird
        \item \lstinline{pb->c();} ist ein Fehler, da \lstinline{c} nicht in \lstinline{B} implementiert ist
        \item \lstinline{pb->a(x);} ist kein Fehler, da in \lstinline{B} einfach double überladen wird
        \item \lstinline{pa->c();} ist ein Fehler, da \lstinline{A::c} private ist, darf es nicht benutzt werden
        \item \lstinline{pa->b();} ist ein Fehler, da \lstinline{b} nur in \lstinline{B} und \lstinline{C} sichtbar ist
        \item \lstinline{pa->vb();} ist ein Fehler, da \lstinline{vb()} nur in \lstinline{B} und \lstinline{C} sichtbar ist
    \end{itemize}
\end{document}