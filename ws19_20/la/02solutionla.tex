\documentclass{article}
\usepackage{josuamathheader}
\begin{document}
	\lalayout{2}
	\section*{Aufgabe 1}
	Sei $G = (G,\cdot,e)$ eine Gruppe. Auf der Potenzmenge $P(G)$ betrachten wir die Abbildung
	\[(A, B)\mapsto A*B = \{a\cdot b | (a, b) \in  A \times B\}.\]
	Zeigen Sie, dass es sich um eine assoziative Verknüpfung handelt und ein eindeutiges (links- und rechts-)neutrales Element existiert. Zu welchen Teilmengen gibt es inverse Elemente? Ist $(P(G),*)$ jemals eine Gruppe?\\
	\begin{proof}
		\textbf{Assoziativität:}
		\begin{align*}
			(A*B)*C &= \{a\cdot b| (a,b) \in A\times B\}*C\\
					&= \{(a\cdot b)\cdot c| ((a,b), c) \in (A\times B)\times C\}\\
					&= \{(a\cdot b)\cdot c| (a,b) \in (A\times B)\land c \in C\}\\
					&= \{(a\cdot b \cdot c)| a\in A, b\in B, c\in C\}\\
					&= \{a\cdot (b\cdot c)| a\in A, (b,c) \in B\times C\}\\
					&= \{a\cdot (b\cdot c)| (a, (b,c)) \in A\times (B\times C)\}\\
					&= A * \{b\cdot c|(b,c) \in B\times C\}\\
					&= A * (B*C)
		\end{align*}
		\textbf{Es existiert ein neutrales Element $E = \{e\}$.}\\
		Dann ist \[E*A = \{x\cdot a| (x,a) \in E\times A\} = \{x\cdot a | x \in \{1\}\land a \in A\} = \{1\cdot a| a\in A\} = A\]
		und \[A*E = \{a\cdot x| (a,x) \in A\times E\} = \{a\cdot x | a \in A\land x \in \{1\}\} = \{a\cdot 1| a\in A\} = A.\]
		
		\textbf{Das neutrale Element ist eindeutig.}\\
		Seien $E, E'$ zwei verschiedene neutrale Elemente. Daraus folgt aber \[E = E * E' = E'\lightning.\]
		Offensichtlich existiert also nur ein neutrales Element.
		
		\textbf{Es gibt nur zu einelementigen Teilmengen ein Inverses.}
		Sei $A = \{a\}$ eine einelementige Teilmenge von $P(G)$. 
		Dann ist $A^{-1} = \{a^{-1}\}$ das Inverse zu $A$, da 
		\[A*A^{-1} = \{x\cdot y| (x,y)\in A \times A^{-1}\} = \{x\cdot y| (x,y) \in \{a\}\times \{a^{-1}\}\} = \{a \cdot a^{-1}\} = \{e\}\]
		Die leere Menge hat kein Inverses, da $A * \emptyset = \{x\cdot y| (x,y) \in A \times \emptyset\} = \{x\cdot y| (x,y) \in \emptyset\} = \emptyset$.
		Sei $A$ eine Teilmenge von $P(G)$ mit mindestens zwei verschiedenen Elementen $x\in A\subset G$ und $y\in A\subset G$. Angenommen, es gäbe ein $A^{-1}$ zu $A$. Dieses muss mindestens ein Element $z\in A^{-1}\subset G$ enthalten. Damit wissen wir aber, dass $\{x\cdot z, y\cdot z\}\subset A*A^{-1}$. Da $G$ eine Gruppe ist und $x\neq y$ ist auch $x\cdot z \neq y\cdot z$. $A*A^{-1}$ enthält also mindestens zwei Elemente. Damit ist $A*A^{-1}\neq \{e\}$. Dies steht aber im Widerspruch dazu, dass das neutrale Element eindeutig ist. Unsere Annahme ist also falsch und es gibt kein Inverses $A^{-1}$ zu $A$.
		$P(G)$ ist niemals eine Gruppe, da die leere Menge stets ein Element der Potenzmenge ist, es aber kein Inverses zur leeren Menge gibt. 
	\end{proof}
	\section*{Aufgabe 2}
	Es seien $A$, $B$ und $C$ Mengen und $f : A \to B$, $g : B \to C$ Abbildungen
	zwischen ihnen. Zeigen Sie:
	\begin{enumerate}[a)]
		\item Ist $g\circ f$ injektiv, so ist $f$ injektiv.
		\begin{proof}
			Wir zeigen die Kontraposition: $f$ nicht injektiv $\implies$ $g\circ f$ nicht injektiv.\\
			Da $f$ nicht injektiv ist, wählen wir $x, x' \in A$ mit $f(x) = f(x') \in B$.
			Daraus folgt sofort, dass $g(f(x) = g(f(x'))$. Folglich ist auch $g\circ f$ nicht injektiv
		\end{proof}
		\item Ist $g\circ f$ surjektiv, so ist $g$ surjektiv.
		\begin{proof}
			Wir zeigen die Kontraposition: $g$ nicht surjektiv $implies$ $g\circ f$ nicht surjektiv.\\
			Da $g$ nicht surjektiv ist, wählen wir ein $c\in C$, sodass $g^{-1}(\{b\}) = \emptyset$. Es existiert also kein $b \in B$ mit $f(b) = c$.
			Gäbe es ein $a\in A$ mit $(g\circ f)(a) = c$, so wäre $f(b) = c$ mit $b = f(a)$. Da es aber kein solches $b$ geben kann, gibt es auch kein $a \in A$ mit $(g\circ f)(a) = c$ und $(g\circ f)$ ist nicht surjektiv.
		\end{proof}
		\item Sind $f$ und $g$ bijektiv, so ist auch $g\circ f$ bijektiv und es gilt $(g\circ f)^{-1} = f^{-1}\circ g^{-1}$.
		\begin{proof}\ \\
			$f$ bijektiv $\implies$ $\forall a \in A:\exists! b\in B$ mit $f(a) = b$.\\
			$g$ bijektiv $\implies$ $\forall b \in B:\exists! c\in C$ mit $f(b) = c$.\\
			$\implies$ $\forall a \in A: \exists! c\in C$ mit $(g\circ f)(a) = c$ $\implies$ $(g\circ f)$ bijektiv.\\
			$(g\circ f)^{-1} = f^{-1}\circ g^{-1}$ ist äquivalent zu der Aussage: $(g\circ f)\circ f^{-1}\circ g^{-1} = \operatorname{id}$.
			Diese ist leicht zu zeigen: $(g\circ f)\circ f^{-1}\circ g^{-1} = g \circ (f\circ f^{-1}) \circ g^{-1} = g\circ g^{-1} = \operatorname{id}$.
		\end{proof}
	\end{enumerate}
	\section*{Aufgabe 3}
	Sei $a \in \mathbb{N}_0$ . Wir betrachten folgende Abbildung:
	\begin{align*}
		f : \mathbb{N}_0 &\to \mathbb{N}_0\\
		n &\mapsto
		\begin{cases}
		a & \text{falls $n\leq 1$, }\\
		f(n-1) + f(n-2) &\text{sonst.} 
		\end{cases}
	\end{align*}
	Zeigen Sie:
	\begin{enumerate}[a)]
		\item $f$ ist weder injektiv noch surjektiv
		\begin{proof}
			\textbf{Nicht injektiv:} $f(0) = f(1) = a$.\\
			\textbf{Nicht surjektiv:} Dafür zeigen wir zuerst: $f(n+1) \geq f(n)$: $f(n+1) = f(n) + f(n-1) \geq f(n)$.
			Behauptung: $f^{-1}(\{4a\in \mathbb{N}_0\}) = \emptyset$. Beweis: $f(0) = a, f(1) = a, f(2) = 2a, f(3) = 3a, f(4) = 5a$.
			Wir haben gezeigt, dass $f(n+1)\geq f(n-1)$ und somit $\forall n \geq 4: f(n) \geq f(4) \geq 5a > 4a$. Daher hat $4a$ kein Urbild unter $f$ und $f$ ist nicht surjektiv.
		\end{proof}
		\item  $f(n)^2 = f (n - 1)f(n + 1) + ( -1)^n\cdot a^2$ für alle $n\geq 1$.
		\induktion{$n=1$: $f(1)^2 = a^2 = a\cdot 2a -1 \cdot a^2 = f(0)f(2) + (-1)^1\cdot a^2$\\
		$n=2$: $f(2)^2 = (2a)^2 = a\cdot3a +1\cdot a^2 = f(1)f(3) + (-1)^n\cdot a^2$}{Es sei $f(n)^2 = f(n-1)f(n+1) + ( -1)^n\cdot a^2$}
		{$n\to n+1$: \begin{align*}
				f(n+1)^2 &= (f(n-1) + f(n))^2\\
				&= f(n-1)^2 + 2f(n-1)f(n) + f(n)^2\\
				&\overset{I.A.}{=} f(n-2)f(n) + (-1)^{n-1}a^2 + 2f(n-1)f(n) + f(n)^2\\
				&= f(n)(f(n-2) + 2f(n-1)f(n)) + (-1)^{n+1} a^2\\
				&= f(n)(f(n) + f(n+1)) + (-1)^{n+1} a^2 \\
				&= f(n)f(n+2) + (-1)^{n+1} a^2
		\end{align*}}
	\end{enumerate}
	\section*{Aufgabe 4}
	Sei $f : X \to Y$ eine Abbildung. Wir definieren die Relation
	\[R = \{(x_1 , x_2) \in X \times  X | f(x_1) = f(x_2)\}.\]
	Zeigen Sie:
	\begin{enumerate}[a)]
		\item $R$ ist eine Äquivalenzrelation auf $X$.
		\begin{proof} Wir zeigen alle Axiome einer Äquivalenzrelation.
			\begin{enumerate}[Ä1)]
				\item $x\sim_R x$, da $f(x) = f(x)$
				\item $(x_1 \sim_R x_2) \implies (f(x_1) = f(x_2))\implies (f(x_2) = f(x_1))\implies x_2\sim_R x_1$
				\item $(x_1 \sim_R x_2$ und $x_2 \sim_R x_3)$ $\Leftrightarrow$ $(f(x_1) = f(x_2)$ und $f(x_2) = f(x_3)) \implies f(x_1) = f(x_3) \implies x_1\sim_R x_3$
			\end{enumerate}
		\end{proof}
	\newpage
		\item Es bezeichne $p$ die kanonische Projektion $p : X \to X/R$ und
		\[\operatorname{im} f = \{y \in  Y | \exists x \in X: f(x) = y\}\subset Y\]
		das Bild von $f$. Dann existiert eine eindeutige bijektive Abbildung $\overline{f}: X/R \to \operatorname{im} f$ mit der Eigenschaft, dass $\overline{f} \circ p = f$ gilt.
		\begin{proof}Wir führen den Beweis in drei Schritten
			\begin{enumerate}[1)]
				\item \textbf{Es existiert eine Abbildung $\overline{f}$, sodass $\overline{f} \circ p = f$}.\\ 
				\begin{definition}
					Die Abbildung $g(M)$ ordne einer einelementigen Menge $M$ das Element der Menge $M$ zu.
				\end{definition}
				$\overline{f}: A\in X/R\mapsto g(f(A)) \in Y$ erfüllt die Bedingungen.
				Z.Z.: Diese Funktion ist wohldefiniert, d.h. sie ordnet jeder Äquivalenzklasse $A\in X/R$ mit $x\in X$ genau ein $y\in Y$ zu.\\
				Dafür genügt es zu zeigen, dass $\#f(A) = 1$. Wir wählen ein beliebiges $x_0 \in A$. Per Definition von $R$ ist dann $\forall x\in A: f(x) = f(x_0)$. Daher enthält das Bild von $A$ genau ein Element, nämlich $ f(x_0) = g(f(A))\in Y$.
				Z.Z.: $\overline{f} \circ p = f \Leftrightarrow \forall x\in X: \overline{f}(p(x)) = f(x)$\\
				Wir setzen unsere oben definierte Funktion $\overline{f}$ ein: $\forall x\in X: g(f(p(x))) = f(x)$. Die kanonische Projektion eines Elements $x$ entspricht der Menge $p(x) = \{a\in X|f(a)=f(x)\}$. Das Bild $f(p(x))$ ist also die Menge $\{f(a)|a \in p(x)\}$. Da aber $\forall a\in p(x): f(a) = f(x)$ ist $f(p(x)) = \{f(x)\}$. Da offenbar $\#f(p(x)) = 1$ ist $g(f(p(x)))$ wohldefiniert und gleich $g(\{f(x)\}) = f(x)$.
				
				\item \textbf{Es existiert höchstens eine Abbildung $\overline{f}: X/R\to \operatorname{im} f$, sodass $\overline{f} \circ p = f$}\\
				Beweis durch Widerspruch: Wir nehmen an, es gäbe zwei verschiedene Abbildungen $\overline{f}, \overline{f}'$, die diese Bedingung erfüllen.
				Dann $\exists A \in X/R: \overline{f}'(A) \neq \overline{f}(A)$. Wir wählen nun $x$ so, dass $p(x) = A$. Damit erhalten wir $\overline{f}'\circ p(x) \neq \overline{f}\circ p(x)$. Es gilt aber nach Voraussetzung $\overline{f} \circ p = f$ und natürlich auch $\overline{f}' \circ p = f$.
				Mit Einsetzen folgt unmittelbar $f(x) \neq f(x) \lightning$.
				
				\item \textbf{$\overline{f}$ ist bijektiv.}
				Da $\overline{f}$ eindeutig ist, reicht es, die Bijektivität von $\overline{f}: A\in X/R\mapsto g(f(A)) \in Y$ zu zeigen.
				Z.Z.: $\overline{f}$ ist injektiv $\Leftrightarrow$ Für $A \neq A' \in X/R$ ist $g(f(A)) \neq g(f(A'))$.
				Wir wählen wieder ein beliebiges Element $x_0\in A$. Es gilt $\forall x\in A: f(x) = f(x_0)$ und daher $f(A) = \{f(x_0)\} \implies g(f(A)) = g(\{f(x_0)\}) = f(x_0)$.
				
				Analog erhalten wir für ein beliebiges Element $x_0'\in A'$ die Aussage $\forall x'\in A': f(x') = f(x_0') $ und daher $f(A') = \{f(x_0')\} \implies g(f(A')) = g(\{f(x_0')\}) = f(x_0')$. Wäre nun $f(x_0) = f(x_0')$, so müsste $x_0' \in A$ sein. Da Äquivalenzklassen aber stets disjunkt sind, ist $f(x_0) \neq f(x_0')$.\\
				Z.Z.: $\overline{f}$ ist surjektiv.\\
				Aus der Definition von $\operatorname{im} f$ folgt $\forall y \in \operatorname{im} f: \exists x \in X$ mit $f(x) = y$. Wir wissen, dass $\overline{f}(p(x)) = f(x) \forall x\in X$. Damit erhalten wir $\forall y \in \operatorname{im} f:\exists x \in X$ mit $\overline{f}(p(x)) = y$.
				Nun setzen wir $A = p(x)$ und erhalten $\forall y \in \operatorname{im} f:\exists A = p(x)$ mit $x \in X$ und $\overline{f}(p(x)) = y$.
				Das ist äquivalent zu: $\forall y \in \operatorname{im} f:\exists A \in X/R$ mit $\overline{f}(A) = y$. $\overline{f}$ ist also surjektiv.
				
				Aus Injektivität und Surjektivität folgt sofort Bijektivität.
			\end{enumerate}
		\end{proof}
	\end{enumerate} 
\end{document} 