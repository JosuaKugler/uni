\documentclass{article}
\usepackage[utf8]{inputenc}
\usepackage[T1]{fontenc}
\usepackage[ngerman]{babel}
\usepackage{amsmath,amssymb,amstext,amsthm}
\title{Experimentieren und Auswerten}
\author{Josua Kugler}
\begin{document}
	\tableofcontents
	\section{Gültige Ziffern}
	Alle angegebenen Ziffern, abzüglich führender Nullen werden als gültige Ziffern bezeichnet.
	\paragraph{Bsp.:} Sie zählen 7 Umdrehungen eines Rades in 11 Sekunden.\\
	Mathematiker:$\dfrac{7}{11}$\\
	Taschenrechner: $0.63636364$\\
	Physiker: ?
	Anzahl der geltenden Ziffern: eine$\to\ 0.6$
	\section{Fehlerquellen}
	An der Genauigkeit da arbeiten, wo der Fehler am meisten ausmacht.
	\subsection{Fehlerquellen}
	\begin{itemize}
		\item Instrument
		\subitem Auflösungsgenauigkeit
		\subitem Genauigkeit der genutzten Bauteile (ist bei Waagen angegeben)
		\subitem falsche Eichung oder Tarierung
		\item Experimentator(z.B. Reaktionszeit)
		\subitem Ablesefehler
		\subitem menschliche Unzulänglichkeiten
		\subitem Gewöhnungseffekt bei häufiger Wiederholung
		\item Modell (z.B. Vernachlässigung der Reibung)
		\subitem Vernachlässigung von Kräften
		\subitem andere Näherungen
		$\to$ alle drei Fehlerquellen bedenken!
	\end{itemize}
	\section{Fehlerarten}
	\subsection{systematische Fehler}
	\paragraph{systematische Fehler: Offset}
	\begin{itemize}
		\item wert-und zeitunabhängig
		\item z.B. Reaktionszeit beim Stoppen
		\item durch geschicktes Messen eliminieren
		\item oder Formeln anpassen
		\item ansonsten angeben!
	\end{itemize}
	\paragraph{systematische Fehler: Drift}
	\begin{itemize}
		\item zeitabhängig bzw. nutzungsabhängig (Messwerte verschieben sich mit der Zeit)
		\subitem z.B. sich erhitzender Innenwiderstand eines Amperemeters
		\item meist nur schwer kompensierbar(Kalibrationskurve)
	\end{itemize}
	\paragraph{systematischer Fehler: Restfehler}
	\begin{itemize}
		\item nicht korrigierbar
		\item in der Regel am Messgerät oder dessen Dokumentation zu finden
	\end{itemize}
	\subsection{Statistische Fehler}
	\begin{itemize}
		\item zufällig oder unbeeinflussbar
		\item treten immer auf $\to$ immer angeben!
		\item Messwerte oft annähernd gaußverteilt($e^{-\frac{(x-\overline{x})^2}{2\sigma^2}}$)
		\item Standardabweichung $\sigma$ durch mehrere Messungen ermitteln
	\end{itemize}
	\begin{itemize}
		\item Mittelwert der Messwerte:
		\[\overline{x}=\dfrac{1}{n}\sum_{i=1}^{n}x_i\]
		\item Empirische Standardabweichung der Verteilung 
		\[\sigma_{x,n-1} =\sqrt{\dfrac{1}{n-1}\ast \sum_{i=1}^{n}(x_i-\overline{x})^2}\]
		\item Empirische Standardabweichung des Mittelwerts
		\[s_x=\dfrac{1}{\sqrt{n}\ast \sigma_{x,n-1}}\]
	\end{itemize}
	\section{Fehlerfortpflanzung}
	\subsection{Gaußsches Fehlerfortpflanzungsgesetz}
	Ist $G(x,y,...)$ eine Funktion der unabhängigen Messgrößen $x,y,...$ mit den Fehlern $\delta x,\delta y,...$, so erhält man den statistischen Fehler $\delta G$ nach Gauß:
	\[\delta G_{Gauss} = \sqrt{\left(\dfrac{\delta G}{\delta x}\left|_{x_0,y_0,\dots}\delta x\right.\right)^2 +\left(\dfrac{\delta G}{\delta y}\left|_{x_0,y_0,\dots}\delta x\right.\right)^2+\dots}\]
	\subsection{Spezialfälle}
	\begin{itemize}
		\item Eine fehlerbehaftete Messgröße $G(x)$ \[\delta G_{Gauss} = \dfrac{\partial G}{\partial x}\ast \delta x\]
		\item Summe oder Differenz $G(x,y,z...)=x\pm y\pm...$ fehlerbehafteter Messgrößen \[\delta G_{Gauss},\pm =\sqrt{(\delta x)^2+(\delta y)^2+\dots}\]
		\item Produkt und Quotienten $G(x,y,\dots)=\dfrac{x^r\ast y^s}{z^t\ast \dots}\ast \dots$ \[\dfrac{\delta G}{G}=\sqrt{\left(r\frac{\delta x}{\delta x}\right)^2+\left(s\frac{\delta y}{y}\right)^2+\dots}\]
	\end{itemize}
	\section{Graphische Auswertung}
	\begin{itemize}
		\item Ein Graph sagt mehr als 1000 Werte
		\item Achsenbeschriftung
		\item Einfach immer Linearisieren!
	\paragraph{\glqq Geraden werden geraten\grqq}
		\item nur Geraden lassen sich nach Augenmaß fitten
		\item Achsen transformieren, damit sich eine Gerade ergibt
		\item Steigung und Achsenabschnitt der Geraden lassen sich oft nutzen (z.B. Federhärte,...)
		\item Durch drei Messpunkte legt man keine Gerade...
	\paragraph{logarithmische Achsen}
		\item einfache logarithmische Abszisse ($x$-Achse)
		\subitem Logarithmen werden zu Geraden
		\item einfache logarithmische Ordinate
		\subitem Exponentialfunktionen werden zu Geraden
		\item doppelt logarithmische Graphen
		\subitem Potenzfunktionen werden Geraden
		\subitem Steigung entspricht Exponenten
	\end{itemize}

\end{document}