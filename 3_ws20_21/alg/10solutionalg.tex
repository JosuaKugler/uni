\documentclass{article}
\usepackage{josuamathheader}

\newcommand{\ggT}{\operatorname{ggT}}
\newcommand{\id}{\operatorname{id}}
\newcommand{\ord}{\operatorname{ord}}

\begin{document}
\alglayout{10}
\def\headheight{25pt}
    \section*{Aufgabe 1}
    $f$ ist irreduzibel nach Eisenstein und die Erweiterung ist separabel, weil endliche Körper vollkommen sind.
    Sei $\alpha$ eine Nullstelle von $f$, d.h. $f$ ist das Minimalpolynom zu $\alpha$.
    Behauptung: Der Zerfällungskörper von $f$ ist gegeben durch $\mathbb{F}_3(\alpha)$.
    Es gilt 
    \begin{align*}    
        (\alpha^3)^4 + 2(\alpha^3)^2 + 2 &= (\alpha^6)^2 + 2\alpha^6 + 2\\
        &= (2\alpha^2 + 1)^2 + 2(2\alpha^2 + 1) + 2\\
        &= \alpha^4 + \alpha^2 + 1 + \alpha^2 + 2 + 2\\
        &= \alpha^4 + 2\alpha^2 +2\\
        &= 0
    \end{align*}
    und wegen $2^2 = 1$ sind dann offensichtlich auch $2\alpha$ und $2\alpha^3$ Nullstellen von $f$.
    Ein $\sigma \in G\coloneqq \Gal(L/K)$ ist bereits eindeutig bestimmt durch seinen Wert auf $\alpha$, daher besitzt $G$ vier Elemente, 
    \[
        G = \{\sigma_1, \sigma_2, \sigma_3, \sigma_4\},
    \]
    mit $\sigma_1(\alpha) = \alpha$, $\sigma_2(\alpha) = 2\alpha$, $\sigma_3(\alpha) = \alpha^3$ und $\sigma_4(\alpha) = 2\alpha^3$.
    Untergruppen von $G$ haben demnach die Ordnung 2. Jede Untergruppe enthält aber auch $\sigma_1$. Wegen $2^2 = 1$ ist $\sigma_2^2 = \id$. Damit bildet $U = \{\sigma_1, \sigma_2\}$ eine Untergruppe.
    Wegen
    \[ 
        \alpha^9 = \alpha^3 \cdot \alpha^6 = \alpha \cdot \alpha^2 \cdot (2\alpha^2 +1) = \alpha \cdot (2\alpha^4 + \alpha^2) = 2\alpha
    \]
    gilt außerdem $\sigma_3^2 = \sigma_4^2 = \sigma_2$.
    Es kann also keine Untergruppe geben, die $\sigma_3$ oder $\sigma_4$ enthält, aber nicht $\sigma_2$. Daher ist $U$ die einzige Untergruppe von $G$.
    Insbesondere existiert nur ein echter Zwischenkörper von $L/\mathbb{F}_3$. Diese Zwischenerweiterung hat die Ordnung $2$ und ist gegeben durch 
    \begin{align*}
        K \coloneqq L^U = \{x \in L: \sigma_2(x) = x\}
    \end{align*}
    Offensichtlich ist $\mathbb{F}_3(\alpha^2) \subset K$. Es gilt außerdem $[\mathbb{F}_3(\alpha^2) : \mathbb{F}_3] = 2$, da $\alpha^2$ das Minimalpolynom $X^2 + 2X + 2$ besitzt. Daraus folgt $K = \mathbb{F}_3(\alpha^2)$.
    \section*{Aufgabe 2}
    \begin{enumerate}[(a)]
        \item Es gilt $\deg \Phi_{2n} = \varphi(2n) \overset{(2,n) = 1}{=} \varphi(2)\varphi(n) = \varphi(n) = \deg \Phi_n$.
        Ist zudem $\zeta$ eine primitive Einheitswurzel in $\mu_n$, so gilt $\ord_{\mu_n} \zeta = n$. Wir folgern
        \begin{align*}
            \zeta^n = 1 \implies (-\zeta)^n = (-1)^n \zeta^n = -1.
        \end{align*}
        Wegen $\ord_{\mu_{2n}} -\zeta | 2n$, aber $\zeta^n \neq 1$ folgt $\ord_{\mu_{2n}} -\zeta = 2n$. Also ist $-\zeta$ primitive Einheitswurzel in $\mu_{2n}$. Jede Nullstelle $\zeta$ von $\Phi_n$ ist primitive Einheitswurzel in $\mu_n$, also ist stets $-\zeta$ eine primitive Einheitswurzel in $\mu_{2n}$ und damit Nullstelle von $\Phi_{2n}$.
        Da $\Phi_n$ $\varphi(n)$ Nullstellen besitzt und zu jeder Nullstelle $\zeta$ von $\Phi_n$ $-\zeta$ eine Nullstelle von $\Phi_{2n}$ darstellt, besitzt $\Phi_{2n}$ mindestens $\varphi(n)$ Nullstellen. Wegen $\deg \Phi_{2n} = \varphi(n)$ sind damit bereits alle Nullstellen von $\Phi_{2n}$ bestimmt. Es folgt $\Phi_{2n}(-X) = \Phi_n(X)$ oder äquivalent $\Phi_{2n}(X) = \phi_n(-X)$.
        \item Jede $n$-te Einheitswurzel ist primitive $d$-te Einheitswurzel für genau einen Teiler $d$ von $n$. Daher gilt
        \[
            \underbrace{X^n - 1}_{\in \overline{K}[X]} = \Psi_n(X) \prod_{\substack{d|n\\d<n}} \Psi_d(X)
        \]
        Nun argumentieren wir per Induktion über $n$. Der Fall $n = 1$ ist trivial. Sei $n> 1$. Dann gilt
        \begin{align*}
            \underbrace{X^n - 1}_{\in \overline{K}[X]} &= \Psi_n(X) \prod_{\substack{d|n\\d<n}} \Psi_d(X)\\
            \overline{\underbrace{X^n-1}_{\in \Z[X]}} &= \Psi_n(X) \prod_{\substack{d|n\\d<n}} \overline{\Phi}_d(X)\\
            \overline{\Phi_n(X) \prod_{\substack{d|n\\d<n}} \Phi_d(X)} &= \Psi_n(X) \prod_{\substack{d|n\\d<n}} \overline{\Phi}_d(X)
            \intertext{$\overline{\; \cdot \; }$ Homomorphismus}
            \overline{\Phi}_n(X) \prod_{\substack{d|n\\d<n}} \overline{\Phi}_d(X) &= \Psi_n(X) \prod_{\substack{d|n\\d<n}} \overline{\Phi}_d(X)\\
            0 &= (\overline{\Phi}_n(X) - \Psi_n(X)) \prod_{\substack{d|n\\d<n}} \overline{\Phi}_d(X)
            \intertext{$\overline{K}[X]$ nullteilerfrei}
            \implies 0 &= \overline{\Phi}_n(X) - \Psi_n(X)\\
            \Psi_n(X) &= \overline{\Phi}_n(X)
        \end{align*}
    \end{enumerate}
    \section*{Aufgabe 3}
    \begin{enumerate}[(a)]
        \item Genau dann, wenn $f$ nicht separabel, existieren Nullstellen $\alpha_i, \alpha_j$ mit $i\neq j$ aber $\alpha_i = \alpha_j$. Genau dann, wenn es solche zwei Nullstellen gibt, ist ein Faktor von $\delta_f$ Null. Genau dann, wenn ein Faktor von $\delta_f$ null ist, gilt $\delta_f = 0 \Leftrightarrow \Delta_f = 0$.
        \item Sei zunächst $\operatorname{char} K \neq 2$. Dann gilt (bekannt aus der Schule oder auch durch quadratische Erweiterung schnell nachgerechnet) $f(x) = 0 \Leftrightarrow x = \frac{- b \pm \sqrt{b^2 - 4ac}}{2a}$. Die Diskriminante ergibt sich daher zu
        \begin{align*}
            \Delta_f &= \left(\frac{- b + \sqrt{b^2 - 4ac} - (- b - \sqrt{b^2 - 4ac})}{2a}\right)^2\\
            &= \frac{4(b^2 - 4ac)}{4a^2}\\
            &= \frac{b^2 - 4ac}{a^2}
        \end{align*}
        Für $\operatorname{char} K = 2$ gilt $a = 1$, sonst handelt es sich nicht um ein quadratisches Polynom. 
        Daher verbleiben nur 4 Möglichkeiten für $f$. 
        Für $b = c = 0$ gilt $X^2 = X \cdot X$, für $b = 1, c = 0$ gilt $X^2 + X = X(X + 1)$ und für $b = 0, c = 1$ gilt $X^2 + 1 = (X + 1)^2$. Nur das Polynom $X^2 + X + 1$ ist irreduzibel und, weil endliche Körper vollkommen sind auch separabel. Über $\mathbb{F}_4$ besitzt es zwei Lösungen. Sei $\alpha$ eine dieser Lösungen. Dann gilt $\alpha^2 + \alpha + 1 = 0$ und damit auch $(\alpha + 1)^2 + (\alpha + 1) + 1 = \alpha^2 + \alpha + 1 = 0$. Daher ist mit $\alpha$ auch $\alpha + 1$ Nullstelle des Polynoms. Die Differenz der beiden Nullstellen ist daher $1$. Damit ist die Diskriminante durch $1$ gegeben, was genau mit $b^2 - 4ac = b^2 = 1$ übereinstimmt.
        \item Es gilt
        \begin{align*}
            \sigma(\delta_f) &= \prod_{1 \leq i < j \leq n} (\sigma(\alpha_i) - \sigma(\alpha_j))\\
            &= \prod_{1 \leq i < j \leq n} (\alpha_{\varphi(\sigma)(i)} - alpha_{\varphi(\sigma)(j)})\\
            \intertext{Jede Permutation lässt sich schreiben als Produkt von Transpositionen. Jede Transposition führt dazu, dass in einem Faktor von $\delta_f$ das Vorzeichen umgedreht wird. Sei $n$ die Anzahl der Transpositionen. }
            &= (-1)^n \prod_{1 \leq i < j \leq n} (\alpha_i - \alpha_j)
            \intertext{Dann ist per Definition $\operatorname{sgn}(\varphi(\sigma)) = (-1)^n$}
            &= \operatorname{sgn}(\varphi(\sigma)) \delta_f.
        \end{align*}
        \item $\sigma(\Delta_f) = \sigma(\delta_f)^2 = \operatorname{sgn}(\varphi(\sigma))^2 \delta_f^2 = 1 \cdot \Delta_f$ und daher $Delta_f \in L^G = K$.
        \item $\Delta_f \in (K^\times)^2 \Leftrightarrow \delta_f \in K$. Außerdem gilt $\varphi(G) \subset \mathfrak{A}_n \Leftrightarrow \operatorname{sgn}(\varphi(\sigma)) = 1 \forall \sigma \in G $.
        \begin{align*}
            \Delta_f \in (K^\times)^2 &\Leftrightarrow \delta_f \in K\\
            &\Leftrightarrow \delta_f \in L^G\\
            &\Leftrightarrow \sigma(\delta_f) = \delta_f \forall \sigma \in G\\
            &\Leftrightarrow \operatorname{sgn}(\varphi(\sigma)) \delta_f = \delta f \forall \sigma \in G\\
            &\Leftrightarrow \operatorname{sgn}(\varphi(\sigma)) = 1 \forall \sigma \in G\\
            &\Leftrightarrow \varphi(G) \subset \ker(\operatorname{sgn}) = \mathfrak{A}_n
        \end{align*}
    \end{enumerate}
    \section*{Aufgabe 4}
    \begin{enumerate}[(a)]
        \item Es gilt $\overline{a}\coloneqq a \mod q\in \mathbb{F}_q^\times$, da $a$ zu $q$ teilerfremd ist. Nach Aufgabe 6.3(c) gilt 
        \[
            \overline{a}^{\frac{q-1}{2}} = \begin{cases}
            1, &\text{falls }\overline{a} = a \mod q \in    (\mathbb{F}_q^\times)^2,\\
            -1,&\text{falls }\overline{a} = a \mod q \notin (\mathbb{F}_q^\times)^2
        \end{cases}.
        \]
        Daher ist $\big(\frac{a}{q}\big) = \overline{a}^{\frac{q-1}{2}}$. Daraus folgt bereits
        \[
            \biggl(\frac{ab}{q}\biggr) = \overline{ab}^\frac{q-1}{2} = \overline{a}^\frac{q-1}{2} \overline{b}^\frac{q-1}{2} = \biggl(\frac{a}{q}\biggr)\biggl(\frac{b}{q}\biggr).  
        \]
        Für $a = -1$ erhalten wir
        \[
            \biggl(\frac{-1}{q}\biggr) = (-1)^\frac{q-1}{2}.
        \]
        Für $q \equiv 1 \mod 4$ gilt $\frac{q-1}{2} \equiv 0 \mod 2$, für $q\equiv 3 \mod 4$ gilt $\frac{q-1}{2} \equiv 1 \mod 2$. Wegen $(-1)^2 = 1$ genügt es, die Kongruenzen modulo 2 des Exponenten zu betrachten und wir vervollständigen
        \[
            \biggl(\frac{-1}{q}\biggr) = (-1)^\frac{q-1}{2} = \begin{cases}
                1, &\text{falls }q\equiv 1\; (\mod 4)\\
                -1,&\text{falls }q\equiv 3\; (\mod 4).
            \end{cases}
        \]
        \item Für die Diskriminante von $f = X^p -1$ erhalten wir nach der Formel 
        \[ 
            \Delta_f = (-1)^{p(p-1)/2)} p^p(-1)^{p-1} = \left((-1)^{(p-1)/2}p\right)^p.
        \]
        Das Bild von $G$ in $\mathfrak{S}_p$ ist genau dann in $\mathfrak{A}_p$ enthalten, wenn $\Delta_f \in (\mathbb{F}_q^\times)^2$ gilt. Es gilt
        \begin{align*}
            \varphi(G) \subset \mathfrak{A}_p &\Leftrightarrow \Delta_f \in (\mathbb{F}_q^\times)^2\\
            &\Leftrightarrow 1 = \biggl(\frac{\left((-1)^{(p-1)/2}p\right)^p}{q}\biggr)\\
            &\Leftrightarrow 1 = \biggl(\frac{(-1)^{(p-1)/2}p}{q}\biggr)^p
            \intertext{$p$ ungerade, $1^p = 1$, $(-1)^p = -1$.}
            &\Leftrightarrow 1 = \biggl(\frac{(-1)^{(p-1)/2}p}{q}\biggr)\\
            &\Leftrightarrow 1 = \biggl(\frac{(-1)^{(p-1)/2}}{q}\biggr)\biggl(\frac{p}{q}\biggr)\\
            &\Leftrightarrow 1 = \biggl(\frac{(-1)}{q}\biggr)^{(p-1)/2}\biggl(\frac{p}{q}\biggr)\\
            &\Leftrightarrow 1 = \left((-1)^{\frac{q-1}{2}}\right)^{(p-1)/2}\biggl(\frac{p}{q}\biggr)\\
            &\Leftrightarrow 1 = (-1)^{\frac{p-1}{2} \frac{q-1}{2}}\biggl(\frac{p}{q}\biggr)
        \end{align*}
        \item Der Zyklus, in dem die 1 liegt, hat die Gestalt $(1,q,q^2, \dots, q^{k-1})$. Notwendigerweise muss jeder weitere Zyklus die Gestalt $(a,aq,aq^2, \dots, aq^{k-1})$ haben. Jeder Zyklus bricht nämlich nach $k$ Elementen ab, da $q^k = 1$ ist. Gäbe es einen kürzeren Zyklus, so wäre $k$ nicht die Ordnung von $q$. Außerdem sind verschiedene Zyklen natürlich disjunkt. Daher zerfällt $(\Z/p\Z)^\times$ in die disjunkte Vereinigung von Teilmengen mit $k$ Elementen, die jeweils im selben Zyklus liegen. Die Anzahl der Zyklen ist daher gegeben durch $\frac{\# (\Z/p\Z)^\times}{k} = \frac{p-1}{k}$.
        Das Signum eines Zyklus der Länge $k$ ist genau $k-1$, da sich jeder Zyklus der Länge $k$ als Komposition von $k-1$ Transpositionen schreiben lässt (Offensichtlich für $k = 2$, durch Überprüfen für $a_{n-1}$ und $a_n$ sieht man dann leicht $(a_1, \dots, a_n) = (a_1, a_n)(a_1, \dots, a_{n-1})$, woraus per Induktion die Behauptung folgt).
        Die Komposition von $\frac{p-1}{k}$ Zyklen der Länge $k$ lässt sich also als Komposition von $k \frac{p-1}{k}$ Transpositionen schreiben und es gilt $\operatorname{sgn}(\pi) = (-1)^{(k-1)\frac{p-1}{k}}$.
        \item Da der $q$-Frobenius $\sigma$ die Gruppe $G$ erzeugt, gilt $\operatorname{sgn}(\varphi(\sigma')) = 1 \forall \sigma' \in G \Leftrightarrow \operatorname{sgn}(\varphi(\sigma)) = 1$. Wir rechnen also
        \begin{align*}
            1 &= (-1)^{(k-1)\frac{p-1}{k}}\\
            &= (-1)^{k \frac{p-1}{k} - \frac{p-1}{k}}\\
            (-1)^\frac{p-1}{k} &= (-1)^{p-1}
            (-1)^\frac{p-1}{k} &= 1
            \intertext{Das ist äquivalent zur Existenz eines $l \in \Z$ mit}
            \frac{p-1}{k} &= 2 \cdot l\\
            \frac{p-1}{2} &= k \cdot l
        \end{align*}
        $\frac{p-1}{2}$ ist genau dann ein Vielfaches von $k$, wenn $q^\frac{p-1}{2} = 1$ gilt (wegen $k = \ord_{(\Z/p\Z)^\times}(q)$). Aufgrund der Identität
        \[
            \biggl(\frac{q}{p}\biggr) = q^\frac{p-1}{2}
        \]
        erhalten wir schließlich die gesuchte Äquivalenz
        \[
            \varphi(G) \subset \mathfrak{A}_n \Leftrightarrow \biggl(\frac{q}{p}\biggr) = 1.
        \]
        Aus Teilaufgabe (b) wissen wir, dass für $\varphi(G)\subset \mathfrak{A}_n$ auch
        \[
            1 = (-1)^{\frac{p-1}{2} \frac{q-1}{2}}\biggl(\frac{p}{q}\biggr) \Leftrightarrow (-1)^{\frac{p-1}{2}\frac{q-1}{2}} =  \biggl(\frac{p}{q}\biggr)
        \]
        gilt. Wegen $\biggl(\frac{q}{p}\biggr) = 1$ folgern wir im Fall $\varphi(G)\subset \mathfrak{A}_n$ bereits
        \[
            (-1)^{\frac{p-1}{2}\frac{q-1}{2}} =  \biggl(\frac{p}{q}\biggr)\biggl(\frac{q}{p}\biggr).
        \]
        Im Fall $\varphi(G) \subsetneq \mathfrak{A}_n$ erhalten wir aus (b), da der Ausdruck entweder 1 oder -1 sein kann
        \[
            -1 = (-1)^{\frac{p-1}{2} \frac{q-1}{2}}\biggl(\frac{p}{q}\biggr) \Leftrightarrow (-1)^{\frac{p-1}{2}\frac{q-1}{2}} = -\biggl(\frac{p}{q}\biggr)
        \]
        und mit analoger Schlussweise folgern wir aus
        \[
            \varphi(G) \subset \mathfrak{A}_n \Leftrightarrow \biggl(\frac{q}{p}\biggr) = 1.
        \]
        die Äquivalenz
        \[
            \varphi(G) \subsetneq \mathfrak{A}_n \Leftrightarrow \biggl(\frac{q}{p}\biggr) = -1.
        \]
        Multipliziert ergibt sich 
        \[
            (-1)^{\frac{p-1}{2}\frac{q-1}{2}} =  -\biggl(\frac{p}{q}\biggr)\cdot -\biggl(\frac{q}{p}\biggr) = \biggl(\frac{p}{q}\biggr)\biggl(\frac{q}{p}\biggr).
        \]
        Damit haben wir das quadratische Reziprozitätsgesetz für alle Fälle bewiesen.
    \end{enumerate}
\end{document}