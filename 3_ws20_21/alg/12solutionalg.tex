\documentclass{article}
\usepackage{josuamathheader}

\newcommand{\ggT}{\operatorname{ggT}}
\newcommand{\id}{\operatorname{id}}
\newcommand{\ord}{\operatorname{ord}}
\newcommand{\mychar}{\operatorname{char}}
\newcommand{\im}{\operatorname{im}}

\begin{document}
\alglayout{12}
\def\headheight{25pt}
\section*{Aufgabe 1}
\begin{enumerate}[(a)]
    \item Sei $\sigma \in G_\alpha$. Dann gilt $\sigma(\alpha) = \alpha$ nach Definition der Isotropiegruppe.
    Insbesondere ist daher $\sigma \in \operatorname{Aut}_{K(\alpha)}(L) = \Gal(L/K(\alpha))$.
    Sei andererseits $\sigma \in \operatorname{Aut}_{K(\alpha)}(L) = \Gal(L/K(\alpha))$. 
    Dann muss nach Definition von $\operatorname{Aut}_{K(\alpha)}(L)$ und wegen $\alpha \in K(\alpha)$ bereits 
    $\sigma(\alpha) = \alpha$ gelten. Daraus folgt $\sigma \in G_\alpha$. Insgesamt erhalten wir 
    \[
        G_\alpha = \Gal(L/K(\alpha)) \subset G.  
    \]
    \item Wendet man ein beliebiges $\sigma \in G$ auf die Koeffizienten von $f$ an, so erhält man
    \[
        f^{\sigma} = \sigma\left(\prod_{\alpha' \in A} (X-\alpha')\right) = \prod_{\alpha' \in A} (X - \sigma(\alpha')) = f,
    \]
    insbesondere liegen alle Koeffizienten in $L^G = K$ und damit $f\in K[X]$. 
    Außerdem sind alle Nullstellen $\alpha'$ als Elemente von $A$ notwendigerweise verschieden.
    \item Angenommen, $f$ wäre reduzibel. 
    
    Dann gäbe es einen Teiler $f' | f$ in $K[X]$ mit $f'(\alpha) = 0$
    Dann gäbe es eine echte Teilmenge von $A'\subset A$ mit $\alpha \in A'$ derart, dass 
    \[
        f' = \prod_{\alpha' \in A'} (X - \alpha') \in K[X] = L^G[X]
    \]
    Sei nun $\beta \in A \setminus A' \neq \emptyset$ und $\sigma \in G$ mit $\sigma(\alpha) = \beta$. Dann gilt
    \[
        (f')^\sigma = \prod_{\alpha' \in A} (X- \sigma(\alpha')) \neq f',
    \]
    da $(f')^\sigma(\beta) = 0$. 
    Wäre $f' \in K[X]$, so müsste wegen $\sigma|_K = \operatorname{id}$ aber $f' = (f')^\sigma$ gelten, Widerspruch.

    Also ist $f$ irreduzibel. Außerdem gilt $f \in K[X]$ und $f(\alpha) = 0$. 
    Damit ist $f$ das Minimalpolynom von $\alpha$ über $K$.
\end{enumerate}
\section*{Aufgabe 2}
\begin{enumerate}[(a)]
    \item Wir folgern aus Aufgabe 3 auf Zettel 9, dass $[\Q(\zeta_5) \colon \Q(\zeta_5 + \zeta_5^{-1})] = 2$ gilt.
    Somit ist $\Q(\alpha)$ ein echter Zwischenkörper von $\Q(\zeta_5)$.
    %Jedes $\sigma \in G \coloneqq \Gal(\Q(\zeta_5)/\Q)$ permutiert die primitiven Einheitswurzeln $A = \{\zeta_5, \zeta_5^2, \zeta_5^3, \zeta_5^4\}$.
    %Offensichtlich ist $\sigma$ daher durch $\sigma(\zeta_5) \in A$ eindeutig bestimmt. Wir erhalten
    %\[
    %    G = \{\sigma_i | i \in \{1, \dots, 4\}\}
    %\]
    %mit $\sigma_i(\zeta_5) = \zeta_5^i$.
    Nach Satz 4.48 existiert ein Gruppenisomorphismus $\chi\colon G \xrightarrow{\sim} (\Z/5\Z)^\times$. %mit $\chi(\sigma_i) = i$.
    Wegen $\langle 2 \rangle = \{2, 4, 3, 1\} = (\Z/5\Z)^\times$ und $\langle 4 \rangle = \{3, 4, 2, 1\} = (\Z/5\Z)^\times$ folgt, 
    dass $U \coloneqq \{1,4\}$ die einzige echte Untergruppe von $(\Z/5\Z)^\times$ sein kann.
    Insbesondere kann es nur einen echten Zwischenkörper geben. Dieser ist demnach durch $\Q(\alpha)$ gegeben.
    Nachrechnen zeigt
    \[
        (\zeta_5 + \zeta_5^{-1})^2 + (\zeta_5 + \zeta_5^{-1}) -1 
        = \zeta_5^2 + 2 + \zeta_5^{-2} + \zeta_5 + \zeta_5^{-1} -1 
        = \zeta_5^0 + \zeta_5^1 + \zeta_5^2 + \zeta_5^3 + \zeta_5^4 = 0
    \]
    nach Bemerkung 4.45 und Definition 4.43, da 5 eine Primzahl ist.
    Die Nullstellen von $\alpha^2 + \alpha -1$ sind gegeben durch $- \frac{1}{2} \pm \sqrt{\frac{1}{4} +1 } = \frac{-1 \pm \sqrt{5}}{2}$.
    Wegen $\operatorname{Re} \zeta_5 > 0$ folgt 
    $\zeta_5 + \zeta_5^{-1} = \zeta_5 + \overline{\zeta_5} = 2 \operatorname{Re}(\zeta_5) > 0$.
    Nun ist $\frac{\sqrt{5}-1}{2} > 0$ und $\frac{-1-\sqrt{5}}{2} < 0$, sodass $\alpha$ durch die beiden Bedingungen eindeutig als
    \[
      \alpha = \frac{\sqrt{5}-1}{2}  
    \] 
    bestimmt ist.
    \item Alle $\zeta_5^n$ liegen auf dem Einheitskreis. Außerdem schließt $\zeta_5^n$ mit der reellen Achse per Definition
    der Polarkoordinatendarstellung den orientierten Winkel $2\pi n/5$ ein. Insbesondere ist der Winkel zwischen zwei benachbarten
    Einheitswurzeln stets $2\pi /5 \cong 72^\circ$. Daher bilden die Einheitswurzeln ein regelmäßiges Fünfeck.
    Die Seitenlänge des Fünfecks ist nach Pythagoras gegeben durch 
    \[ 
        |\zeta_5 - 1| = \sqrt{(1- \cos(2\pi/5))^2 + \sin^2(2\pi/5)} = \sqrt{\frac{5 - \sqrt{5}}{2}}.
    \]
\end{enumerate}
\section*{Aufgabe 3}
\begin{enumerate}[(a)]
    \item Mit $(12345)$ ist auch das Inverse $(54321)$ in $H$ enthalten.
    Es gilt nach Aufgabe 4(a) auf Blatt 11 $(12)(12345)(12) = (21345)$ und damit $(12)(21345)(54321) = (12)(123) = (23)$.
    Weiter erhalten wir $(23)(12345)(23) = (13245)$ und $(23)(13245)(54321) = (23)(234) = (34)$.
    Außerdem ist $(34)(12345)(34) = (12435)$ und folglich $(34)(12435)(54321) = (34)(345) = (45)$.
    Wir erhalten zunächst $(23)(12)(23) = (13), (34)(13)(34) = (14)$ und $(45)(14)(45) = (15)$.
    Damit ist $(ij) = (1i)(1j)(1i)$ für $i, j \in \{1,2,3,4,5\}$.
    Jede Permutation lässt sich als Produkt von Transpositionen schreiben, daher gilt $H = \mathfrak{S}_5$.
    \item Die komplexe Konjugation wirkt trivial auf $\R$, insbesondere also auf den reellen Nullstellen von $f$.
    Da $f$ genau $5$ komplexe Nullstellen besitzt, gibt es genau 2 Nullstellen von $f$ mit $\operatorname{Im} f \neq 0$.
    Sei $\sigma$ die komplexe Konjugation. Dann ist wegen $f\in \Q[X]$ $f^\sigma = f$.
    Schreibt man $f$ als Produkt seiner Linearfaktoren, so sind die drei Linearfaktoren mit reellem Koeffizienten invariant
    unter komplexer Konjugation. Das gesamte Polynom ist ebenfalls invariant unter komplexer Konjugation.
    Daher muss 
    \[ 
        (X - \overline{\beta_1}) (X - \overline{\beta}_2) = (X - \beta_1)(X- \beta_2)
    \]
    gelten. Daher ist $\beta_1 = \overline{\beta}_2$. Daher liegt die komplexe Konjugation in der Galoisgruppe $G$ des 
    Zerfällungskörpers $L$ von $f$ über $\Q$.
    $G$ ist isomorph zu einer Untergruppe $G'$ von $\mathfrak{S}_5$. 
    Unter diesem Isomorphismus wird $\sigma_c$ auf eine Transposition abgebildet, da genau zwei Elemente der 
    Nullstellemenge vertauscht werden und sonst alles gleich bleibt.

    Sei $\alpha$ eine Nullstelle von $f$. Dann ist $[\Q(\alpha) \colon \Q] = 5$.
    Dann gilt $\# G = [L\colon \Q] = [L \colon \Q(\alpha)] \cdot [\Q(\alpha) \colon \Q] = [L \colon \Q(\alpha)] \cdot 5$ und
    $\# G | \# \mathfrak{S}_5 = 120 = 2^3 \cdot 3 \cdot 5$.
    Also taucht der Primfaktor 5 in der Primfaktorzerlegung von $\# G' = \# G$ genau einmal auf, es existiert also eine
    $5$-Sylowgruppe der Ordnung 5 in $G'$. Diese ist zyklisch von der Ordnung 5 und enthält daher ein Element $\sigma$ der Ordnung 5.
    In Zykel-Schreibweise erhalten wir $\sigma = (1,\sigma(1),\sigma^2(1), \sigma^3(1), \sigma^4(1))$. In der Tat,
    die Länge dieses Zyklus muss genau $\operatorname{ord}(\sigma) = 5$ sein. Alle Einträge des Zyklus müssen notwendigerweise
    verschieden sein, also ist $\sigma(i) \forall i \in \{1,\dots, 5\}$ bestimmt. Folglich ist $\sigma$ durch diesen Zyklus
    vollständig beschrieben.

    Insgesamt folgt, dass $G'$ sowohl eine Transposition als auch einen Fünferzykel enthält.
    Nach Teilaufgabe $(a)$ ist damit bereits $G \cong G' = \mathfrak{S}_5$. 
    \item Wir identifizieren $f$ mir der induzierte Polynomfunktion auf $\R$, die nach Ana 1 insbesondere stetig ist.
    Es gilt 
    \begin{align*}
        2\cdot (-2)^5 - 10\cdot (-2) + 5 &= -39 < 0.\\
        2\cdot (-1)^5 - 10\cdot (-1) + 5 &= 13 > 0.\\
        2 \cdot 1^5 - 10\cdot 1 + 5 &= -3 < 0\\
        2 \cdot 2^5 - 10 \cdot 2 +5 &= 49 > 0\\
    \end{align*}
    Nach dem Zwischenwertsatz besitzt $f$ daher eine reelle Nullstelle zwischen $-2$ und $-1$, eine reelle Nullstelle 
    zwischen $-1$ und $1$ sowie eine reelle Nullstelle zwischen $1$ und $2$.

    Wir betrachten die Ableitung $f'(x) = 10x^4 - 10$ und die zweite Ableitung $f''(x) = 40x^3$.
    Es gilt $f'(x) = 0 \implies 10x^4 = 10 \implies x^4 = 1 \implies x = \pm 1$. Wegen $f''(1) = 40$
    sowie $f''(-1) = -40$ besitzt $f$ einen Hochpunkt bei $-1$, einen Tiefpunkt bei $1$ und sonst keine Extremstellen.
    Eine Funktion mit zwei Extremstellen kann höchstens drei Nullstellen besitzen, zusammen mit den Zwischenwertsatzargumenten
    hat $f$ also genau drei reelle Nullstellen.
    
    Der $\operatorname{ggT}$ aller Koeffizienten von $f$ in $\Z$ ist $1$, insbesondere ist also $f$ primitiv.
    Nach dem Eisensteinkriterium mit $p = 5$ folgt wegen $5 | 10, 5| 5, 5^2\not | 5$ und $5 \not | 2$ die Irreduzibilität von $f$.
    Mit Teilaufgabe (b) folgt, dass die Galoisgruppe der Gleichung $f(x) = 0$ gleich $\mathfrak{S}_5$ ist. $\mathfrak{S}_5$
    ist aber nicht auflösbar. Also ist $f(x) = 0$ nicht durch Radikale auflösbar.
\end{enumerate}

\section*{Aufgabe 4}
Behauptung: Jede Gruppe $H$ der Ordnung $6$ ist auflösbar.
\begin{proof}
    Es gilt $\# H = 6 = 2\cdot 3$. Nach Satz 5.29 existiert eine $3$-Sylow-Gruppe $H'$ der Ordnung 3 in $H$.
    Da $3$ eine Primzahl ist, ist $H'$ nach Satz 5.39 auflösbar. 
    Die Anzahl $s$ der $3$-Sylowgruppen erfüllt $s | 6$ und $s = 1 \mod 3$.
    Da $4$ kein Teiler von $6$ ist, gilt $s = 1$.
    Alle zu $H'$ konjugierten Untergruppen haben ebenfalls die Ordnung $3$ und sind damit $3$-Sylowgruppen.
    Da es aber nur eine $3$-Sylowgruppe gibt, muss $gH'g^{-1} = H' \forall g \in H$ gelten und $H'$ ist ein Normalteiler.
    Wir betrachten daher $H'' = H/H'$. $H''$ hat die Ordnung $6/3 = 2$. Da $2$ und $3$ Primzahlen sind, müssen
    $H'$ und $H''$ nach Satz 5.39 auflösbar sein. Nach Satz 5.48 ist das äquivalent dazu, dass $H$ auflösbar ist.
\end{proof}
Behauptung: Jede Gruppe $H$ der Ordnung $10$ ist auflösbar.
\begin{proof}
    Es gilt $\# H = 10 = 2\cdot 5$. Nach Satz 5.29 existiert eine $5$-Sylow-Gruppe $H'$ der Ordnung 5 in $H$.
    Da $5$ eine Primzahl ist, ist $H'$ nach Satz 5.39 auflösbar. 
    Die Anzahl $s$ der $5$-Sylowgruppen erfüllt $s | 10$ und $s = 1 \mod 5$.
    Da $6$ kein Teiler von $10$ ist, gilt $s = 1$.
    Alle zu $H'$ konjugierten Untergruppen haben ebenfalls die Ordnung $5$ und sind damit $5$-Sylowgruppen.
    Da es aber nur eine $5$-Sylowgruppe gibt, muss $gH'g^{-1} = H' \forall g \in H$ gelten und $H'$ ist ein Normalteiler.
    Wir betrachten daher $H'' = H/H'$. $H''$ hat die Ordnung $10/5 = 2$. Da $2$ und $5$ Primzahlen sind, müssen
    $H'$ und $H''$ nach Satz 5.39 auflösbar sein. Nach Satz 5.48 ist das äquivalent dazu, dass $H$ auflösbar ist.
\end{proof}
\begin{enumerate}[(a)]
    \item Es gilt $\# G = 42 = 7 \cdot 3 \cdot 2$. Nach Satz 5.29 existiert eine $7$-Sylow-Gruppe $G'$ der Ordnung $7$ in $G$.
    Da $7$ eine Primzahl ist, ist $G'$ nach Satz 5.39 auflösbar. 
    Die Anzahl $s$ der $7$-Sylowgruppen erfüllt $s | 42$ und $s = 1 \mod 7$.
    Jeder Teiler von $42$, der nicht durch 7 teilbar ist, lässt sich als Produkt von $2$ und $3$ schreiben.
    Daher kommen für $s$ nur Zahlen $\leq 6$ infrage $\implies s = 1$.
    Alle zu $G'$ konjugierten Untergruppen haben ebenfalls die Ordnung $7$ und sind damit $7$-Sylowgruppen.
    Da es aber nur eine $7$-Sylowgruppe gibt, muss $gG'g^{-1} = G' \forall g \in G$ gelten und $G'$ ist ein Normalteiler.
    Wir betrachten daher $G'' = G/G'$. $G''$ hat die Ordnung $42/7 = 6$. Daher ist $G''$ auch auflösbar.
    Nach Satz 5.48 ist das äquivalent dazu, dass $G$ auflösbar ist.
    \item Es gilt $\# G = 30 = 5 \cdot 3 \cdot 2$. Nach Satz 5.29 existiert eine $5$-Sylowgruppe $G'$ der Ordnung $5$ in $G$.
    Da $5$ eine Primzahl ist, ist $G'$ nach Satz 5.39 auflösbar. 
    Die Anzahl $s_5$ der $5$-Sylowgruppen erfüllt $s_5 | 30$ und $s_5 = 1 \mod 5$.
    Jeder Teiler von $30$, der nicht durch $5$ teilbar ist, lässt sich als Produkt von $2$ und $3$ schreiben.
    Daher kommen für $s_5$ nur Zahlen $\leq 6$ infrage $\implies s_5 \in \{1, 6\}$.

    Sollte $s_5 = 1$ gelten, so besitzen alle zu $G'$ konjugierten Untergruppen ebenfalls die Ordnung $5$ und sind
    damit $5$-Sylowgruppen.
    Da es aber nur eine $5$-Sylowgruppe gibt, muss $gG'g^{-1} = G' \forall g \in G$ gelten und $G'$ ist ein Normalteiler.
    Wir betrachten daher $G'' = G/G'$. $G''$ hat die Ordnung $30/5 = 6$. Daher ist $G''$ auch auflösbar.
    Nach Satz 5.48 ist das äquivalent dazu, dass $G$ auflösbar ist.

    Gelte nun $s_5 = 6$. Dann operiert $G$ vermöge der Konjugation auf der Menge der $5$-Sylowgruppen $S_1, \dots, S_6$.
    Da $5$ prim ist, erzeugt jedes $e \neq x \in S_i$ bereits die Sylowgruppe.
    Da die Gruppen verschieden sind, muss daher $S_i \cap S_j = \{e\} \forall i \neq j$ sein.
    Jedes Element einer $5$-Sylowgruppe hat Ordnung $5^r$ nach Lemma 5.33(ii), aber auch Ordnung $\leq 5$,
    da wir sonst durch $\{a^i, i = 0, \dots, 5\}$ 6 verschiedene Elemente erhalten, die alle in der Untergruppe liegen müssen.
    Also hat jedes Element (außer $e$) einer solchen Sylowgruppe Ordnung $5$.
    Da die Sylowgruppen bis auf das neutrale Element disjunkt sind, gilt
    \[
         \# \bigcup_{i = 1}^6 S_i \setminus \{e\} = 6 \cdot 4 = 24  
    \]
    und es gibt mindestens 24 Elemente der Ordnung 5.

    Nach Satz 5.29 existiert eine $3$-Sylowgruppe $H'$ der Ordnung $3$ in $G$.
    Da $3$ eine Primzahl ist, ist $H'$ nach Satz 5.39 auflösbar. 
    Die Anzahl $s_3$ der $3$-Sylowgruppen erfüllt $s_3 | 30$ und $s_3 = 1 \mod 3$.
    Jeder Teiler von $30$, der nicht durch $3$ teilbar ist, lässt sich als Produkt von $2$ und $5$ schreiben.
    Daher kommen für $s_3$ nur Zahlen $\leq 10$ infrage $\implies s_3 \in \{1, 10\}$.

    Sollte $s_3 = 1$ gelten, so besitzen alle zu $H'$ konjugierten Untergruppen ebenfalls die Ordnung $3$ und sind
    damit $3$-Sylowgruppen.
    Da es aber nur eine $3$-Sylowgruppe gibt, muss $gH'g^{-1} = H' \forall g \in G$ gelten und $H'$ ist ein Normalteiler.
    Wir betrachten daher $H'' = G/H'$. $H''$ hat die Ordnung $30/3 = 10$. Daher ist $H''$ auch auflösbar.
    Nach Satz 5.48 ist das äquivalent dazu, dass $G$ auflösbar ist.

    Gelte nun $s_3 = 10$. Dann operiert $G$ vermöge der Konjugation auf der Menge der $3$-Sylowgruppen $T_1, \dots, T_6$.
    Da $3$ prim ist, erzeugt jedes $e \neq x \in T_i$ bereits die Sylowgruppe.
    Da die Gruppen verschieden sind, muss daher $T_i \cap T_j = \{e\} \forall i \neq j$ sein.
    Jedes Element einer $3$-Sylowgruppe hat Ordnung $3^r$ nach Lemma 5.33(ii), aber auch Ordnung $\leq 3$,
    da wir sonst durch $\{a^i, i = 0, \dots, 3\}$ 4 verschiedene Elemente erhalten, die alle in der Untergruppe liegen müssen.
    Also hat jedes Element (außer $e$) einer solchen Sylowgruppe Ordnung $3$.
    Da die Sylowgruppen bis auf das neutrale Element disjunkt sind, gilt
    \[
         \# \bigcup_{i = 1}^10 T_i \setminus \{e\} = 10 \cdot 2 = 20  
    \]
    und es gibt mindestens 20 Elemente der Ordnung 3.

    Gilt also $s_3 = 1$ oder $s_5 = 1$, so ist $G$ auflösbar. Gilt $s_3 \neq 1$ und $s_5 \neq 1$, so erhalten wir
    mindestens 24 Elemente der Ordnung 5 und 20 Elemente der Ordnung 3.
    $G$ hat aber nur $30$ Elemente, Widerspruch. Also kann dieser Fall nicht eintreten und $G$ ist stets auflösbar.
    \item Wegen $(g,h) \cdot (e \times H) \cdot (g^{-1}, h^{-1}) = gg^{-1} \times hHh^{-1} \subset (e \times H)$ 
    ist $e \times H \triangleleft G \times H$. Weiter folgern wir
    \begin{align*}
        (G \times H) /(e \times H) = \{g \times H : g \in G\} \cong G
    \end{align*}
    Offensichtlich ist außerdem $(e \times H) \cong H$. Mit $H$ und $G$ sind demnach auch 
    $e \times H$ und $(G \times H) /(e \times H)$ auflösbar. Nach Satz 5.48 
    folgt bereits die Auflösbarkeit von $G \times H$.
\end{enumerate} 
\end{document}