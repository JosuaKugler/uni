\documentclass{article}
\usepackage{josuamathheader}

\newcommand{\ggT}{\operatorname{ggT}}
\newcommand{\id}{\operatorname{id}}
\begin{document}
\alglayout{9}
\def\headheight{25pt}
    \section*{Aufgabe 1}
    Ist $\alpha$ eine Nullstelle von $f$, so auch $-\alpha$, da nur gerade Potenzen vorkommen. Daher gilt
    \begin{align*}
        X^4 - 4X^2 + 9 = (X^2 - \alpha^2)(X^2 - \beta^2) = X^4 - (\alpha^2 + \beta^2)X^2 + \alpha^2\beta^2
    \end{align*}
    für zwei Nullstellen $\alpha$ und $\beta$. Durch Koeffizientenvergleich folgt $\beta = \frac{3}{\alpha}$.
    Durch Einsetzen verifiziert man, dass $\alpha = \sqrt{2 + \sqrt{-5}}$ eine Nullstelle von $f$ ist.
    Wegen $\alpha \notin \R$ besitzt das Polynom keine Nullstelle in $\Q$. 
    Um zu zeigen, dass $f$ irreduzibel ist, wählen wir den Ansatz
    \begin{align*}
        X^4 - 4X^2 + 9 = (X^2 + aX + bX^2)(X^2 + cX + d) = X^4 + (a + c)X^3 + (b + d + ac)X^2 + (ad + bc)X + bd
    \end{align*}
    Daraus folgt $c = -a$ und $a(d-b) = 0$. $a = 0$ führt auf $b +d = -4$, $bd = 9$; hat keine Lösung in $\Z$.
    $a\neq 0$ führt auf $d = b \implies b^2 = 9 \implies b = \pm 3$ und $2b -a^2 = -4 \implies a^2 = 2b + 4 = 10$ oder $2$, 
    das sind aber keine Quadrate in $\Z$, Widerspruch. Daher ist $f$ irreduzibel.
    Die Menge der Nullstellen ist dann gegeben durch $\{\pm \alpha, \pm \frac{3}{\alpha}\}$.
    Insbesondere ist ein Zerfällungskörper gegeben durch $L\coloneqq \Q(\alpha)$.
    Da $f$ irreduzibel ist, ist $f$ das Minimalpolynom zu $\alpha$ und die Erweiterung hat somit Grad $4$.
    Daher ist $\sigma_i \in \Gal(L/Q)$ eindeutig bestimmt durch $\sigma(\alpha)$. Daher gilt 
    \begin{align*}
        G \coloneqq \Gal(L/\Q) = \{\sigma_1, \sigma_2, \sigma_3, \sigma_4\}
    \end{align*}
    mit $\sigma_1 \colon \alpha \mapsto \alpha,\; 
    \sigma_2 \colon \alpha \mapsto -\alpha,\; 
    \sigma_3 \colon \alpha \mapsto \frac{3}{\alpha},\; 
    \sigma_4 \colon \alpha \mapsto -\frac{3}{\alpha}$.
    Es gilt $\sigma_1 = \id$ und $\sigma_2 \circ \sigma_3 = \sigma_4 = \sigma_3 \circ \sigma_2$,
    $\sigma_2 \circ \sigma_4 = \sigma_3 = \sigma_4 \circ \sigma_2$ und
    $\sigma_3 \circ \sigma_4 = \sigma_2 = \sigma_4 \circ \sigma_3$. $G$ ist also abelsch.
    Außerdem gilt $\sigma_2^2 = \sigma_3^2 = \sigma_4^2 = \id$. Daher ist die Abbildung
    \begin{align*}
        G &\to \Z/2\Z \times \Z/2\Z,\\
        \sigma_1 &\mapsto (0,0)\\
        \sigma_2 &\mapsto (1,0)\\
        \sigma_3 &\mapsto (0,1)\\
        \sigma_4 &\mapsto (1,1)
    \end{align*}
    ein Gruppenisomorphismus.
    Alle echten Untergruppen von $\Z/2\Z \times \Z/2\Z$ haben Ordnung $2$. Die echten Untergruppen von $G$ sind daher  
    \begin{align*}
        U_1 \coloneqq \{\sigma_1, \sigma_2\}, U_2 \coloneqq \{\sigma_1, \sigma_3\}, U_3 \coloneqq \{\sigma_1, \sigma_4\}
    \end{align*}. Daher erhalten wir drei Zwischenkörper.
    \begin{enumerate}
        \item Der erste Zwischenkörper ist gegeben durch
        \begin{align*}
            K_1 = L^{U_1} = \{x \in L\colon \sigma_2(x) = x\}
        \end{align*}
        Es gilt $\sigma_2(\alpha^2) = \sigma_2(\alpha)\sigma_2(\alpha) = \alpha^2$.
        Damit ist $\Q(\alpha^2) \subset K_1$. 
        Das Minimalpolynom von $\alpha^2$ geht aus dem von $\alpha$ durch $X^2 \mapsto X$ hervor
        und ist folglich gegeben durch $X^2 - 4X + 9$. 
        Damit gilt $[\Q(\alpha^2) \colon \Q] = 2$ und daher $K_1 = \Q(\alpha^2)$.
        \item Der zweite Zwischenkörper ist gegeben durch
        \begin{align*}
            K_2 = L^{U_2} = \{x \in L\colon \sigma_3(x) = x\}
        \end{align*}
        Es gilt $\sigma_3(\alpha + 3/\alpha) = \sigma_4(\alpha) + \sigma_4(3/\alpha) = 3/\alpha + \alpha$.
        Damit ist $\Q(\alpha + 3/\alpha) \subset K_2$. $X^2 - 10$ ist offensichtlich irreduzibel über $\Q$.
        Wegen 
        \begin{align*}
            (\alpha + 3/\alpha)^2 &= \alpha^2 + 6 + 9/\alpha^2\\
            &= 2 + \sqrt{-5} + 6 + \frac{9}{2 + \sqrt{-5}}\\
            &= 8 + \sqrt{-5} + \frac{9(2 - \sqrt{-5})}{(2 + \sqrt{-5})(2 - \sqrt{-5})}\\
            &= 8 + \sqrt{-5} + \frac{9(2 - \sqrt{-5})}{2^2 + 5}
            &= 10
        \end{align*}
        gilt $[\Q(\alpha + 3/\alpha) \colon \Q] = 2$ und daher $K_2 = \Q(\alpha + 3/\alpha)$.
        \item Der dritte Zwischenkörper ist gegeben durch
        \begin{align*}
            K_3 = L^{U_2} = \{x \in L\colon \sigma_4(x) = x\}
        \end{align*}
        Es gilt $\sigma_4(\alpha - 3/\alpha) = \sigma_4(\alpha) + \sigma_4(-3/\alpha) = -3/\alpha + \alpha$.
        Damit ist $\Q(\alpha - 3/\alpha) \subset K_3$. $X^2 + 2$ ist offensichtlich irreduzibel über $\Q$.
        Wegen 
        \begin{align*}
            (\alpha - 3/\alpha)^2 &= \alpha^2 - 6 + 9/\alpha^2\\
            &= 2 + \sqrt{-5} - 6 + \frac{9}{2 + \sqrt{-5}}\\
            &= -4 + \sqrt{-5} + \frac{9(2 - \sqrt{-5})}{(2 + \sqrt{-5})(2 - \sqrt{-5})}\\
            &= -4 + \sqrt{-5} + \frac{9(2 - \sqrt{-5})}{2^2 + 5}
            &= -2
        \end{align*}
        gilt $[\Q(\alpha - 3/\alpha) \colon \Q] = 2$ und daher $K_3 = \Q(\alpha - 3/\alpha)$.
    \end{enumerate}
    \section*{Aufgabe 2}
    \begin{enumerate}[(a)]
        \item Das Polynom $f = X^4 -7$ (irreduzibel nach Eisenstein) hat über $\overline{Q} \subset \C$ die vier 
        Nullstellen $\pm \sqrt[4]{7}, \pm i \sqrt[4]{7}$.
        Insbesondere besitzt es eine Nullstelle in $\Q(\sqrt[4]{7})$. 
        Wegen $\Q(\sqrt[4]{7}) \subset \R$ zerfällt $f$ aber nicht in Linearfaktoren. 
        Daher ist die Erweiterung nicht normal und insbesondere nicht galoissch.
        \item $f$ ist irreduzibel nach Eisenstein. Daher ist $f$ als Polynom über $\Q$ separabel. 
        Also ist $L/\Q$ eine endliche Galoiserweiterung und nach Korollar 4.20 kann
        $\Gal(L/\Q)$ als Untergruppe von $\mathfrak{S}(\{\alpha_1, \dots, \alpha_i\})$ aufgefasst werden, 
        wenn $\{\alpha_1,\dots, \alpha_i\}$ die Menge der Nullstellen von $f$ bezeichnet.
        Da $f$ ein Polynom 4. Grades ist gilt $i \leq 4$.
        Die Untergruppenordnung teilt die Ordnung der Gruppe, wegen 
        $\# \mathfrak{S}(\{\alpha_1, \dots, \alpha_4\}) = 24 | 24, \# \mathfrak{S}(\{\alpha_1, \dots, \alpha_3\}) = 6 | 24$ und 
        $\# \mathfrak{S}(\{\alpha_1, \alpha_2\}) = 2 | 24$ gilt auch $[L\colon \Q] = \# \Gal(L/\Q) | 24$.
        \item Die Diskriminante von $X^3 - 2$ ist gegeben durch $-27 \cdot b^2 = -108 \neq x^2 \forall x \in \Q$.
        Daher ist $\Gal(L/K) \cong \mathfrak{S}_3$ und damit nicht zyklisch.
        \item Für die Galoisgruppe $G$ eines Polynoms dritten Grades gilt entweder $G\cong \mathfrak{S}_3$ oder $G \cong \mathfrak{A}_3$.
        Im zweiten Fall hat die Gruppe nur $3$ Elemente und somit weniger als 4 echte Untergruppen.
        Echte Untergruppen von $\mathfrak{S}_3$ haben die Ordnung $2$ oder $3$, da sie die Gruppenordnung teilen müssen.
        Da jede Transposition selbstinvers sind, erhalten wir durch $\{e, (12)\}, \{3, (23)\}, \{e, (31)\}$ drei Untergruppen.
        Außerdem ist $\mathfrak{A}_3$ ebenfalls eine Untergruppe von $\mathfrak{S}_3$.
        Es gilt nun $(123)^2 = (132)$ und $(132)^2 = (123)$.
        Zu jedem Element $\pi \in \mathfrak{A}_3$ existieren zwei Transpositionen $\tau_1, \tau_2$ mit $\tau_1 \circ \tau_2 = \pi$.
        Genauso gilt $\forall tau_1 \neq \tau_2 \in \mathfrak{S}_3 \setminus \mathfrak{A}_3 \colon \tau_1 \tau_2 \in \mathfrak{A}_3 \setminus \{e\}$.
        Eine Transposition $\tau$ und ein Element $\pi \in \mathfrak{A}_3$ erzeugen daher stets eine weitere Transposition, 
        da $\exists \tau'\colon \pi = \tau' \tau$ und daher $\pi \tau = \tau' \tau \tau = \tau'$.
        Folglich kann es keine weiteren Untergruppen der Ordnung zwei oder drei geben. 
        Die Anzahl der echten Untergruppen ist also durch 4 nach oben beschränkt. 
        Nach dem Hauptsatz der Galoistheorie ist damit die Anzahl der echten Zwischenkörper auch kleiner als 4.
        \item Sei $K = \mathbb{F}_5$ und sei $\alpha$ eine Nullstelle von $X^4 - a$.
        Dann ist wegen $(2^i \alpha)^4 = (2^4)^i \alpha^4 = a$ die Menge der Nullstellen gerade 
        $\{2^i \alpha,\; i \in \{1,2,3,4\}\}$, da diese Menge bereits vier verschiedene Nullstellen enthält.
        Der Zerfällungskörper von $X^4 - a$ ist daher gegeben durch $\L\coloneqq \mathbb{F}_5(\alpha)$ und 
        jedes $\sigma \in \Gal(L/K)$ ist eindeutig bestimmt durch $\sigma(\alpha)$. Daher gilt 
        \[
            \Gal(L/K) = \{\sigma_1, \dots, \sigma_4\}  
        \]
        mit $\sigma_i \colon \alpha \mapsto 2^i \alpha$. Also ist $\Gal(L/K)$ zyklisch mit Erzeuger $\sigma_1$:
        \[
            \sigma_i(\alpha) = 2^i \alpha = (\sigma_1)^i(\alpha).
        \]
        Also ist die Aussage falsch.
        \item Ist $[L\colon K] < \infty$, so ist die Menge aller Untergruppen von $\Gal(L/K)$ ist eine Teilmenge der Potenzmenge, 
        deren Kardinalität durch $2^{\# \Gal(L/K)} = 2^{[L\colon K]}$ gegeben ist.
        Nach dem Hauptsatz der Galoistheorie damit die Anzahl der Zwischenkörper auch höchstens $2^{[L\colon K]}$.
        Ist $L/K$ eine unendliche Erweiterung, so stellt $2^{[L\colon K]} = \infty$ keine Schranke dar und die 
        Aussage ist ebenfalls wahr.
    \end{enumerate}
    \section*{Aufgabe 3}
    \begin{enumerate}[(a)]
        \item Wir betrachten O.B.d.A. den algebraischen Abschluss von $\Q$ in den komplexen Zahlen, $\overline{Q} \subset \C$.
        Nach Analysis 2 hat die Gleichung $\zeta^n = 1$ genau die Lösungen $e^{\frac{2\pi i k}{n}}$ mit $1\leq k \leq n$.
        Wegen $e^{\frac{2\pi i k}{n}} = (e^{\frac{2\pi i}{n}})^k$ ist $e^{\frac{2\pi i}{n}}$ eine primitive Einheitswurzel.
        Setze daher $\zeta_n = e^{\frac{2\pi i}{n}}$. Dann erhalten wir einen Gruppenisomorphismus 
        \begin{align*}
            \Z/n\Z &\to \mu_n\\
            k \mapsto \zeta_n^k = e^{\frac{2\pi i k}{n}}
        \end{align*}
        Es gilt $\overline{e^{\frac{2\pi i k}{n}}} = e^{\frac{2\pi i (-k)}{n}} = e^{\frac{2\pi i (n-k)}{n}}$.
        Also induziert die komplexe Konjugation eine Permutation $\pi_C$ von $\Z/N\Z$ via $\zeta_n^k \mapsto \zeta_n^{\pi_C(k)}$ mit 
        $\pi_C(n) = n$ und $\pi_C(k) = \pi_C(n-k) \forall k \in \{1, \dots, n-1\}$.
        \item Da $L$ gerade der Zerfällungskörper von $X^n - 1$ über $\Q$ ist, 
        kann die Galoisgruppe $\Gal(L/L\cap \R)$ mit einer Untergruppe der $\mathfrak{S}_n$ identifiziert werden,
        wobei $\pi \in \mathfrak{S}_n$ auf $\mu_n$ via $\zeta_n^k \mapsto \zeta_n^{\pi(k)}$ operiert.
        Außerdem gilt 
        \begin{align*}
            \zeta_n^k + \zeta_n^{-k} &= e^{\frac{2\pi i k}{n}} + e^{\frac{2\pi i (n-k)}{n}}\\
            &= \cos(\frac{2\pi k}{n}) + i \sin(\frac{2\pi k}{n}) + \cos(\frac{2\pi (-k)}{n}) + i \sin(\frac{2\pi (-k)}{n})\\
            &= 2\cos(\frac{2\pi k}{n}) \in \R
        \end{align*}
        Sei also $\sigma \in \Gal(L/L\cap \R)$. Dann muss gelten
        \begin{align*}
            \sigma(\zeta_n + \zeta_n^{-1}) &= \zeta_n + \zeta_n^{-1}\\
            \sigma(\zeta_n) + \sigma(\zeta_n)^{-1} &= \zeta_n + \zeta_n^{-1}
            \intertext{Sei $\pi$ die zu $\sigma$ gehörige Permutation}
            \zeta_n^{\pi(1)} + \zeta_n^{-\pi(1)} &= \zeta_n + \zeta_n^{-1}\\
            2 \cos(\frac{2\pi \pi(1)}{n}) &= 2\cos(\frac{2\pi}{n})\\
            \intertext{Für $\pi(1) \in \{1, \dots, n\}$ gibt es hier aufgrund der Symmetrie von $\cos(x)$ bezüglich $x = 1$ zwei Möglichkeiten}
            \pi(1) &\in \{1, n-1\} 
        \end{align*}
        Da es sich bei $\zeta_n$ um eine primitive Einheitswurzel handelt gilt $L = \Q(\zeta_n)$ und $\pi$ ist durch $\pi(1)$ bereits eindeutig bestimmt,
        es gilt dann $\zeta_n^{\pi(k)} = \sigma(\zeta_n^k) = \sigma(\zeta_n)^k = \zeta_n^{k \pi(1)}$.
        Wegen $\pi_C(1) = n-1$ handelt es sich bei einem Automorphismus $\sigma \in \Gal(L/L\cap \R)$ 
        entweder um die Identität oder die komplexe Konjugation, es gilt also
        \begin{align*}
            \Gal(L/L\cap \R) = \{\id, C\},
        \end{align*}
        wobei $C$ die komplexe Konjugation bezeichne.
        Daher erhalten wir $[L\colon L\cap \R] = \# \Gal(L/L\cap \R) = 2$.
        %Außerdem gilt $L \cap \R = L^{\Gal(L/L\cap \R)} = L^{\{\id, C\}}$.
        Sei $\alpha \coloneqq \frac{\zeta_n + \zeta_n^{-1}}{2} = \cos(\frac{2\pi}{n})$.
        Dann gilt $\zeta_n = \cos(\frac{2\pi}{n}) + i \sin(\frac{2\pi}{n}) = \alpha + i\sin(\frac{2\pi}{n})$.
        Es gilt 
        \begin{align*}
            (\zeta_n - \alpha)^2 &= - \sin^2(\frac{2\pi}{n})\\
            &= - (1 - \cos^2(\frac{2\pi}{n}))\\
            &= \alpha^2 - 1
        \end{align*}
        Daher gilt
        \begin{align*}
            (\zeta_n - \alpha)^2 - \alpha^2 + 1 = 0,
        \end{align*}
        es folgt $[L\colon \Q(\zeta_n + \zeta_n^{-1})] = [\Q(\zeta_n) \colon \Q(\alpha)] \leq 2$.
        Wegen $\alpha \in L \cap \R$ ist aber $\Q(\alpha) \subset L\cap \R$. Also gilt 
        $[\Q(\zeta_n) \colon \Q(\alpha)] \geq [L \colon L \cap \R] = 2$ und insgesamt $[L \colon \Q(\alpha)] = 2$.
        Daher ist auch $[L \cap \R \colon \Q] = [\Q(\alpha) \colon \Q]$ und,
        weil es sich um endliche Erweiterungen handelt, ist $L\cap \R$ isomorph zu $\Q(\alpha)$ als $\Q$-VR.
        Wegen $\Q(\alpha) \subset L\cap \R$ ist bereits $L \cap \R = \Q(\zeta_n + \zeta_n^{-1})$.
    \end{enumerate} 
    \section*{Aufgabe 4}
    \begin{enumerate}[(a)]
        \item Nach der universellen Eigenschaft des Polynomrings existiert genau ein Ringhomomorphismus 
        \begin{align*}
            \sigma_a' \colon K[Y] &\to L\\
            Y &\mapsto Y + a\\
            k &\mapsto k \forall k \in K
        \end{align*}
        Diese Abbildung ist als Isomorphismus auf den Unterring $K[Y] \subset K(Y) = L$ injektiv.
        Nach der universellen Eigenschaft des Quotientenkörpers existiert dann genau ein injektiver Körperhomomorphismus
        $\sigma_a \colon Q(K[Y]) = K(Y) \to L$ mit $\sigma_a|_{K[Y]} = \sigma_a'$.
        \item Es gilt $\sigma_0 = \id$. Wegen $\sigma_a \circ \sigma_b(Y) = \sigma_a(Y + b) = Y + b + a = Y + (a + b) = \sigma_{a+b}$ 
        erhalten wir einen Isomorphismus $\Phi \colon (K, +) \to (\operatorname{Aut}_K(L), \circ), a \mapsto \sigma_a$.
        Angenommen, $L^G \neq 0$. Dann existiert ein $0 \neq f \in K(X)$ mit $f(Y + a) = \sigma_a(f(Y)) = f(Y) \forall a\in K$.
        Wir betrachten die rationale Funktion $h \in L(X)$ mit $h(X) \coloneqq f(Y + X) - f(Y)$.
        Diese rationale Funktion besitzt wegen $f(Y + a) = f(Y)$ unendlich Nullstellen in $K$. 
        Eine rationale Funktion hat aber nur endlich viele Nullstellen, Widerspruch.
        \item Angenommen, es gäbe ein Polynom $f \in K[Y]$ mit $\deg f< p$ und $f(Y + a) = f(Y) \forall a \in \mathbb{F}_p$.
        Betrachte dann $h \in L[X]$ mit $h(X) \coloneqq f(Y + X) - f(Y)$. Es gilt $\deg h < p$ wegen $\deg f < p$, 
        allerdings besitzt $h$ $p$ Nullstellen, nämlich alle Elemente von $\mathbb{F}_p$, Widerspruch. 
        Das Polynom $f(Y) = Y^p - Y$ erfüllt $f(Y + a) = (Y + a)^p - (Y + a) = Y^p + a^p - a - Y = Y^p - Y = f(Y)$
        Angenommen, es gäbe noch ein weiteres normiertes Polynom $f'$ vom Grad $p$ mit dieser Eigenschaft, dann folgte
        $(f - f')(Y + a) = f(Y  +a) - f'(Y + a) = (f-f')(Y)$ mit $\deg (f-f') < p$, Widerspruch.
        Jedes Polynom vom Grad $\leq p$ mit der Eigenschaft $f(Y + a) = f(Y) \forall a \in \mathbb{F}_p$ ist also ein Polynom in $Z$.
        Diese Aussage beweisen wir per Induktion für Polynome vom Grad $\leq np$ für beliebiges $n\in \N$ (also für alle).
        Wir nehmen also an, jedes $f\in K[Y]$ mit $\deg f \leq kp$ lässt sich darstellen als Polynom in $Z$ $f(Y) = \tilde{f}(Y^p-Y)$.
        Sei also $g \in K[Y]$ mit $\deg g \leq (k+1)p$ und $g(Y + a) = g(Y)$. Da $K[Y]$ ein euklidischer Ring ist, erhalten wir
        $g(Y) = f(Y) \cdot (Y^p - Y) + q(Y)$ mit $\deg f \leq kp$ und $\deg q < p$. Nach Induktionsvoraussetzung erhalten wir daraus 
        $g(Y) = \tilde{f}(Y^p - Y) \cdot (Y^p - Y) + q(Y) = \tilde{g}(Y^p-Y) + q(Y)$. Es gilt allerdings 
        $g(Y + a) = \tilde{g}((Y + a)^p - (Y + a)) + q(Y + a) = \tilde{g}(Y^p - Y) + q(Y + a) \overset{!}{=} \tilde{g}(Y^p - Y) + q(Y)$.
        Daraus erhalten wir die Forderung $q(Y) = q(Y + a)$ mit $\deg q < p$, woraus wir sofort $q(Y) = 0$ folgern können.
        Es folgt $g(Y) = \tilde{g}(Y^p - Y) = \tilde{g}(Z)$.
        Nun betrachten wir ein Element $(f(Y), g(Y)) \in K(Y)$.
        Wir wählen einen Vertreter mit jeweils eindeutiger Zerlegung in irreduzible Polynome
        $f(Y) = \prod_{i = 1}^r f_i(Y)$ und $g(Y) = \prod_{j=1}^r g_j(Y)$ derart, dass $\forall i, j\colon f_i \neq g_j$.
        Dann gilt $(f(Y + a), g(Y + a)) \sim (f(Y), g(Y))$ genau dann, wenn
        \begin{align*}
            f(Y + a)g(Y) &= f(y)g(Y + a)\\
            \prod_{i = 1}^r f_i(Y+a)\prod_{j=1}^r g_j(Y)&= \prod_{i = 1}^r f_i(Y)\prod_{j=1}^r g_j(Y+a)
        \end{align*} 
        Durch $Y \mapsto Y + a$ bleibt die Irreduzibilität der Faktoren erhalten. 
        Die Gleichheit gilt, wenn $\prod_{i = 1}^r f_i(Y + a) = \prod_{i = 1}^r f_i(Y)$ und $\prod_{j=1}^r g_j(Y+a) = \prod_{j=1}^r g_j(Y)$.
        Ist dies allerdings nicht erfüllt, so müssen $i, j$ existieren mit $f_i(Y +a) = g_j(Y + a)$. 
        Daraus folgt aber sofort $f_i(Y) = g_j(Y)$, was wir ausgeschlossen hatten.
        Daher erhalten wir $f(Y + a) = f(Y)$ und $g(Y + a) = g(Y)$, somit ist aber $(f(Y), g(Y)) \sim (\tilde f(Z), \tilde g(Z))$ für geeignetes $\tilde f, \tilde g$.
        Es gilt also für jedes Element $x \in K(Y)$ mit $\sigma_a(x) = x\forall a \in \mathbb{F}_p$ die Eigenschaft $x\in K(Z)$, was zu zeigen war.
    \end{enumerate}
\end{document}