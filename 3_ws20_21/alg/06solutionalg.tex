%\documentclass{../../theo-lecture/lecture}
\documentclass{article}
\usepackage{josuamathheader}

\newcommand{\ggT}{\operatorname{ggT}}
\begin{document}
\alglayout{6}
\def\headheight{25pt}
    \section*{Aufgabe 1}
    Sei $K$ vollkommen. Angenommen, $\sigma$ wäre nicht surjektiv. Dann gäbe es ein $a\in K$ derart, dass $X^p - a$ keine Nullstelle in $K$ besitzt. 
    Sei nun $\alpha \in \overline{K}$ eine Nullstelle von $X^p-a$, d.h. $\alpha^p = a$. Damit gilt $X^p - a = X^p - \alpha^p = (X- \alpha)^p$.
    Wäre $X^p - a$ reduzibel über $K$, so müsste es in Faktoren der Form $X^p - a = (X-\alpha)^r \cdot (X-\alpha)^{p-r}$. Ein Polynom der Form $(X - \alpha)^r$ kann aber nicht in $K$ liegen.
    Für $r = 1$ würde $\alpha \in K$ folgen.
    Für $1 < r < p$ gilt $(X - \alpha)^r = \sum_{i = 1}^{r} \binom{r}{i} (-\alpha)^iX^{r-i} \in K[X]$. Der Koeffizient vor $X^{r-1}$ ist $\binom{r}{1}(-\alpha)^1 = -r \alpha$.
    Es gilt aber $-r \alpha \in K \implies \alpha \in K$. Das ist ein Widerspruch.
    Das Polynom $X^p - a$ ist also irreduzibel in $K$ und damit Minimalpolynom zu $\alpha$.
    Allerdings hat $X^p - a$ die Mehrfachnullstelle $\alpha$, $X^p - a = (X - \alpha)^p$. Daher ist $\alpha$ nicht separabel. 
    Das ist ein Widerspruch zur Vollkommenheit von $K$. Daher muss $\sigma$ surjektiv sein.
    Sei nun $\sigma$ surjektiv und $f$ ein irreduzibles Polynom in $K[X]$.
    Wir nehmen nun an, dass $f$ eine Mehrfachnullstelle besitzt. Nach Satz 3.81 ist $f$ dann ein Polynom in $X^p$. Es gilt also
    \[
        f = a_n (X^p)^n + a_{n-1}(X^p)^{n-1} + \dots + a_0 = a_n X^{p \cdot n} + a_{n-1} X^{p \cdot (n-1)} + \dots + a_0
    \]
    Da $\sigma$ surjektiv ist, existiert zu jedem $a_i$ ein $c_i$ mit $c_i^p = a_i$. Wir erhalten also
    \[
        f = c_n^p (X^n)^p + c_{n-1}^p (X^{n-1})^p + \dots + c_0^p = (c_n X^n + c_{n-1}X^{n-1})^p + \dots + c_0^p = (c_n X^n + c_{n-1}X^{n-1} + \dots + c_0)^p
    \]
    wobei die letzte Gleichheit induktiv sofort klar ist.
    Das ist aber ein Widerspruch zur Irreduzibilität von $f$. Also kann $f$ keine Mehrfachnullstelle besitzen, jedes irreduzible Polynom über $K$ ist separabel.
    Da $\forall \alpha \in \overline K$ das Minimalpolynom irreduzibel ist, sind alle $\alpha \in \overline{K}$ separabel und folglich ist jede algebraische Körpererweiterung separabel.
    \section*{Aufgabe 2}
    \begin{enumerate}[(a)]
        \item Sei $M/L$ eine beliebige Körpererweiterung. Dann wird $M$ durch $M/L$ und $L/K$ zu einer Erweiterung $M/K$ von $K$. Aufgrund der Vollkommenheit von $K$ ist $M/K$ separabel.
        Nach Lemma 3.93 ist demnach auch $M/L$ separabel. Weil $M$ beliebig gewählt war ist damit die Vollkommenheit von $L$ gezeigt.
        \item Wegen $L \subset \overline{K}$ ist die Erweiterung $\overline{K}/L$ separabel. Sind die Erweiterungen $\overline{K}/L$ und $L/K$ separabel, so auch $\overline{K}/K$.
        Sei $M$ eine algebraische Erweiterung von $K$. Dann sind nach Korollar 3.93 $\overline{K}/M$ und $M/K$ separabel. Also ist jede algebraische Erweiterung von $K$ separabel und damit $K$ vollkommen.
        \item Sei $x \in K_s\subset L$. Dann ist $x^{p^n} \in K_s$, da $K_s$ ein Körper ist. Sei nun $x \in L\setminus K_s$. Dann ist das Minimalpolynom $f$ von $x$ über $K$ nicht separabel
        und nach Lemma 3.81 existiert ein $r \in \N$ (also $r \neq 0$) mit $f(X) = g(X^r)$, wobei $g$ irreduzibel und separabel ist. Wegen $f(x) = 0$ muss auch $g(x^r) = 0$ sein. 
        Daher ist $g$ das Minimalpolynom zu $x^r$ und $x^r$ muss separabel sein. Daher existiert für alle $x\in L$ ein $r \in \N$ derart, dass $x^{p^n} \in K_s \forall n \geq r$. 
        Aufgrund der Endlichkeit von $L/K$ ist aber der Grad eines Minimalpolynoms beschränkt.
        Es existiert daher ein $n\in \N$ derart, dass $\forall x \in L\colon\; x^{p^n} \in K_s$. Da $L$ vollkommen ist, muss der Frobenius-Homomorphismus $\sigma\colon L\to L, a\mapsto a^p$ bijektiv sein.
        Insbesondere existiert daher für alle $x\in L$ ein $\tilde x$ mit $x = \sigma^n(\tilde x) = (\tilde x)^{p^n} \in K_s$. Daher ist $L/K$ separabel.
    \end{enumerate}
    \section*{Aufgabe 3}
    \begin{enumerate}[(a)]
        \item Eine quadratische Erweiterung wird stets von einem Element $\alpha$ erzeugt, da es keine Teilerweiterungen gibt. Das Minimalpolynom zu $\alpha$ ist ein irreduzibles Polynom vom Grad 2 und hat daher die Form $f(X) = X^2 + pX + q$, das eine Nullstelle bei $\alpha$ hat. Jede Nullstelle von $f$ hat die Form $\alpha = -\frac{p}{2} \pm \sqrt{\left(\frac{p}{2}\right)^2 - q}$. Daher ist also $K(\alpha) = K(-\frac{p}{2} \pm \sqrt{\left(\frac{p}{2}\right)^2 - q}) = K(\sqrt{\left(\frac{p}{2}\right)^2 - q})$. Wäre $\left(\frac{p}{2}\right)^2 - q = 0$, so zerfiele $f$ in Linearfaktoren $f(X) = (X + \frac{p}{2})^2$ und folglich nicht irreduzibel. Wäre $\left(\frac{p}{2}\right)^2 - q = a^2$ für ein $a \in K^\times$, so wäre $f(X) = (X + \frac{p}{2} + a)(X + \frac{p}{2} - a)$ und folglich nicht irreduzibel. Daher muss $a \in K^\times \setminus (K^\times)^2$ liegen.
        \item Sei $c^2 = \frac{a}{b}$ mit $c \in K^\times$. Dann gilt $\sqrt{a} = \sqrt{b \cdot \frac{a}{b}} = \sqrt{b} \cdot \sqrt{\frac{a}{b}} = c \cdot \sqrt{b} \in K(\sqrt{a}) \implies K(\sqrt{b}) \subset K(\sqrt{a})$. Analog erhält man die Umkehrung, sodass folgt $K(\sqrt{a}) = K(\sqrt{b})$.
        Nach Lemma 3.40 existiert ein $K$-Isomorphismus zwischen $K(\sqrt{a})$ und $K(\sqrt{b})$ dann, wenn eine Nullstelle des Minimalpolynoms von $\sqrt{a}$ bereits $K(\sqrt{b})$ erzeugt. Das Minimalpolynom von $\sqrt{a}$ ist $X^2 - a$ und hat die beiden Nullstellen $\sqrt{a}$ und $-\sqrt{a}$. Damit muss $K(\sqrt{a}) = K(\sqrt{b})$ und insbesondere $\sqrt{b} \in K(\sqrt{a})$ gelten, wenn ein Isomorphismus existiert. Wegen $\dim_K K(\sqrt{a}) = 2$ und weil $1, \sqrt{a}$ $K$-linear unabhängig sind (sonst wäre $\sqrt{a} \in K$) lässt sich $\sqrt{b}$ als Linearkombination $\sqrt{b} = c + d\sqrt{a}$ schreiben. Ist nun $c \neq 0$, so folgt daraus $b = c^2 + 2cd\sqrt{a} + d^2a \implies \sqrt{a} = \frac{b - c^2 - d^2a}{2cd}$ und damit $\sqrt{a} \in K \lightning$. Also gilt $\sqrt{b} = d\sqrt{a}$. Daraus erhalten wir sofort $\frac{a}{b} = \left(\frac{1}{d}\right)^2 \implies \frac{a}{b} \in (K^\times)^2$.
        \item Ist $a \in (\mathbb{F}_p^\times)^2$, so existiert ein $c \in \mathbb{F}_p^\times$ mit $c^2 = a$. Dann ist $a^{\frac{p-1}{2}} = c^{p-1}$. Wegen $c^p = c$ und $c \in K^\times$ ist $a^{\frac{p-1}{2}} =  c^{p-1} = 1$. Ist $a$ hingegen in $(\mathbb{F}_p^\times \setminus (\mathbb{F}_p^\times)^2)$, so gilt $(a^{\frac{p-1}{2}})^2 = 1$ durch analoge Rechnung wie im ersten Fall, also $a^{\frac{p-1}{2}} = \pm 1$.
        Da $\mathbb{F}_p^\times$ nach Vorlesung zyklisch ist, existiert ein $a\in \mathbb{F}_p^\times$ mit $\mathbb{F}_p^\times = \{a, a^2, \dots, a^{p-1}\}$. Wäre nun $a^{\frac{p-1}{2}} = 1$, so wäre $\# \mathbb{F}_p^\times < p-1$. Also gilt $a^{\frac{p-1}{2}} = -1$. Für eine ungerade Potenz $x$ von $a$ (z.B. $x = a^3$ gilt offensichtlich ebenfalls $x^{\frac{p-1}{2}} = -1$. Da $(K^\times)^2$ eine Untergruppe vom Index 2 in $\mathbb{F}_p^\times$ bildet und alle geraden Potenzen von $a$ in $(\mathbb{F}_p^\times)^2$ liegen, entsprechen die ungeraden Potenzen von $a$ gerade $\mathbb{F}_p^\times\setminus (\mathbb{F}_p^\times)^2$ und es muss gelten
        \[
            \forall a \in \mathbb{F}_p^\times\setminus (\mathbb{F}_p^\times)^2\colon a^{\frac{p-1}{2}} = -1.  
        \]
        Seien $K(\sqrt{a}), K(\sqrt{b})$ zwei quadratische Erweiterungen von $\mathbb{F}_p$ mit $a,  \in K^\times \setminus (K^\times)^2$ und $K(\sqrt{a}) \neq K(\sqrt{b})$, was nach Teilaufgabe b zu $\frac{a}{b} \notin (K^\times)^2$ äquivalent ist. Es gilt aber $1 = \frac{a^{\frac{p-1}{2}}}{b^{\frac{p-1}{2}}} = \left(\frac{a}{b}\right)^{\frac{p-1}{2}}$ und daher $\frac{a}{b} \in (\mathbb{F}_p^\times)^2$. Das ist aber ein Widerspruch. Also kann es keine zwei verschiedenen quadratischen Erweiterungen geben. Die Menge $K^\times \setminus (K^\times)^2$ ist aber für $p \neq 2$ stets nichtleer (siehe Zettel 4, Aufgabe 5 a).
        Daher existiert eine eindeutig bestimmte quadratische Erweiterung von $\mathbb{F}_p$.
    \end{enumerate}
    \section*{Aufgabe 4}
    \begin{enumerate}[(a)]
        \item Sei $f \in \mathbb{F}_p[X]$ ein irreduzibles Polynom vom Grad $d$. Sei $a$ eine Nullstelle von $f$. Dann hat die Körpererweiterung $\mathbb{F}_p(a)$ Grad $d$, da $f$ das Minimalpolynom zu $a$ darstellt. Insbesondere ist $\mathbb{F}_p(a)$ als endlichdimensionaler Vektorraum über einem endlichen Körper endlich. Nach Korollar 3.100 ist $\mathbb{F}_p(a)/\mathbb{F}_p$ aber isomorph zu einer Erweiterung $\mathbb{F}_{q}/\mathbb{F}_p$ mit $q = p^k$. Allerdings muss $[\mathbb{F}_q \colon \mathbb{F}_p] = d$ und damit $q = p^d$ gelten, da der Grad erhalten bleibt.
        Mit Korollar 3.101 folgt, dass die Körpererweiterung und damit auch $f$ separabel sein muss. $f$ besitzt also $d$ verschiedene Nullstellen in $\overline{\mathbb{F}_p}$, die wir mit $a_1, \dots, a_d$ bezeichnen. Nach Lemma 3.40 ist die Anzahl der Nullstellen von $f$ in $\mathbb{F}_q$ gleich der Anzahl der $\mathbb{F}_p$-Automorphismen $\sigma \colon \mathbb{F}_q \to \mathbb{F}_q$. Diese Anzahl ist gleich $\# \operatorname{Aut}_{\mathbb{F}_p}(\mathbb{F}_q) = [\mathbb{F}_q \colon \mathbb{F}_p] = d$ nach Satz 3.102. Daher liegen alle Nullstellen $a_1, \dots, a_d$ von $f$ in $\mathbb{F}_q$, $\mathbb{F}_q$ ist der Zerfällungskörper von $f$.
        $f$ teilt $g \coloneqq X^{p^n} - X$ genau dann, wenn alle Nullstellen von $f$ auch Nullstellen von $g$ sind.
        Gilt also $f|g$, so sind alle Nullstellen von $f$ auch Nullstellen von $g$ und der Zerfällungskörper von $f$ ist im Zerfällungskörper von $g$ enthalten, d.h. $\mathbb{F}_{p^d} \subset \mathbb{F}_{p^n}$. Ist hingegen der Zerfällungskörper von $f$ im Zerfällungskörper von $g$ enthalten, so ist liegt jede Nullstellen von $f$ in $\mathbb{F}_{p^n}$. Jedes Element von $\mathbb{F}_{p^n}$ ist nach Satz 3.99 aber Nullstelle von $X^{p^n} - X$. Daher gilt
        \[
            f|g \Leftrightarrow \mathbb{F}_q \subset \mathbb{F}_{p^n} \Leftrightarrow \mathbb{F}_{p^d} \subset \mathbb{F}_{p^n} \xLeftrightarrow{3.100} d | n \Leftrightarrow \deg f | n  
        \]
        \item Im euklidischen Ring $\mathbb{F}_p[X]$ existiert eine eindeutige Primfaktorzerlegung. Diese ist genau durch alle irreduziblen (normierten) Teiler von $X^{p^n} - X$ gegeben. Nach Teilaufgabe a ist also
        \[
            X^{p^n} - X = \prod_f f(X),
        \]
        wobei $f$ die irreduziblen, normierten Polynome in $\mathbb{F}_p$ mit $\deg (f)|n$ durchlaufe.
        \item In der Produktdarstellung in Aufgabe b addieren sich die Grade der Faktoren auf der rechten Seite zum Grad auf der linken Seite, also $p^n$. Daher gilt
        \[
            p^n = \sum_{\substack{f\text{ irred.}\\\deg f | n}} \deg(f) = \sum_{d | n} \sum_{\substack{f\text{ irred.}\\\deg f = d}} d = \sum_{d | n} d \cdot a_d(p)
        \]
        \item Setzen wir in Aufgabe c $p = 2$ und $n = 6$, so erhalten wir
        \[
            2^6 = a_1(2) + 2\cdot a_2(2) + 3 \cdot a_3(2) + 6 \cdot a_6(2). 
        \]
        Nach Aufgabe 3 auf Blatt 2 gilt aber $a_1(2) = 2,\; a_2(2) = 1,\; a_3(2) = 2$. Einsetzen ergibt
        \[
            64 = 2 + 2 + 6 + 6 a_6(2) \Leftrightarrow 54 = 6 a_6(2) \Leftrightarrow a_6(2) = 9.
        \]
    \end{enumerate}
    \section*{Bonusaufgabe 5}
    \begin{enumerate}[(a)]
        \item 
    \end{enumerate}
\end{document}