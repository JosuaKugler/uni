\documentclass[uebung]{lecture}

\title{Wtheo 0: Übungsblatt 1}
\author{Josua Kugler, Christian Merten}

\newcommand{\IP}{\mathbb{P}}

\usepackage[]{mathrsfs}
\begin{document}

\punkte

\begin{aufgabe}
    \begin{enumerate}[(a)]
        \item Seien $\mathcal{A}_i, i \in I$ $\sigma$-Algebren über $\Omega$.
            Beh.: $\bigcap_{i \in  I} \mathcal{A}_i$ $\sigma$-Algebra über $\Omega$.
            \begin{proof}
                \begin{enumerate}[(i)]
                    \item $\Omega \in \bigcap_{i \in  I} \mathcal{A}_i$, denn
                        $\forall i \in I\colon \Omega \in \mathcal{A}_i$, da $\mathcal{A}_i$ $\sigma$-Algebra.
                    \item Sei $A \in \bigcap_{i \in  I} \mathcal{A}_i$. Dann ist für $i \in I$:
                        $A \in \mathcal{A}_i$. Da $\mathcal{A}_i$ $\sigma$-Algebra,
                        ist $A^{c} \in \mathcal{A}_i$. Damit folgt
                        $A^{c} \in \bigcap_{i \in  I} \mathcal{A}_i$.
                    \item Sei $A_j \in \bigcap_{i \in  I} \mathcal{A}_i$ $\forall j \in \N$. Da
                        für alle $i \in I$, $\mathcal{A}_i$ $\sigma$-Algebra, ist
                        $\bigcap_{j \in  \N} A_j \in \mathcal{A}_i$. Also auch
                        $\bigcap_{j \in \N} A_j \in \bigcap_{i \in I} \mathcal{A}_i$.
                \end{enumerate}
            \end{proof}
        \item Beh.: Die Aussage ist falsch.
            \begin{proof}
                Es sei $\Omega \coloneqq \{ 0, 1, 2\} $,
                $\mathcal{A}_1 \coloneqq \sigma(\{0\}) = \{ \Omega, \emptyset, \{0\} , \{1, 2\} \} $ und \\
                $\mathcal{A}_2 \coloneqq \sigma(\{2\} ) = \{\Omega, \emptyset, \{2\}, \{0, 1\} \} $.
                Dann sind $\mathcal{A}_1$ und $\mathcal{A}_2$ nach VL $\sigma$-Algebren über $\Omega$, aber
                $\mathcal{A}_1 \cup \mathcal{A}_2 = \{\Omega, \emptyset, \{0\} , \{2\} , \{1,2\} , \{0,1\} \} $
                nicht, da $\{0\}  \cup \{2\} = \{0, 2\} \not\in \mathcal{A}_1 \cup \mathcal{A}_2$.
            \end{proof}
        \item Sei $\mathcal{A}$ $\sigma$-Algebra über $\Omega$ und $f\colon \mathcal{X} \to \Omega$ Abbildung.
            Beh.: $f^{-1}(\mathcal{A}) \coloneqq \{ f^{-1}(A) \colon A \in \mathcal{A}\} $ ist $\sigma$-Algebra.
            \begin{proof}
                \begin{enumerate}[(i)]
                    \item $\mathcal{X} \in f^{-1}(\mathcal{A})$, denn $f^{-1}(\Omega) = \mathcal{X}$.
                    \item Sei $B \in f^{-1}(\mathcal{A})$. Dann ex. ein $A \in \mathcal{A}$, s.d.
                        $f^{-1}(A) = B$. Da $\mathcal{A}$ $\sigma$-Algebra ist $A^{c} \in \mathcal{A}$.
                        Damit folgt
                        \[
                            B^{c} = f^{-1}(A)^{c} = f^{-1}(A^{c}) \in f^{-1}(\mathcal{A})
                        .\]
                    \item Seien $B_i \in f^{-1}(\mathcal{A})$ $\forall i \in \N$. Dann ex. $\forall i \in \N$
                        ein $A_i \in \mathcal{A}$, s.d. $f^{-1}(A_i) = B_i$. Da
                        $\mathcal{A}$ $\sigma$-Algebra ist $\bigcup_{i \in \N} A_i \in \mathcal{A}$.
                        Damit folgt
                        \[
                            \bigcup_{i \in \N} B_i = \bigcup_{i \in \N} f^{-1}(A_i)
                            = f^{-1} \left( \bigcup_{i \in \N} A_i \right) \in f^{-1}(\mathcal{A})
                        .\] 
                \end{enumerate}
            \end{proof}
        \item Sei $T \subseteq \Omega$ mit $T \neq \emptyset$ und sei $\mathcal{A}$ $\sigma$-Algebra über
            $\Omega$. Beh.: $A|_T \coloneqq \{ A \cap T \colon A \in \mathcal{A}\} $ $\sigma$-Algebra.
            \begin{proof}
                Betrachte die kanonische Inklusion $\iota \colon T \xhookrightarrow{} \Omega$. Dann
                gilt
                \begin{align*}
                    \iota^{-1}(\mathcal{A}) &= \{ \iota^{-1}(A) \colon A \in \mathcal{A}\} \\
                    &= \{ \{ x \in T \colon \iota(x) \in A \} \colon A \in \mathcal{A}\} \\
                    &= \{ \{ x \in T \colon x \in A \} \colon A \in \mathcal{A}\} \\
                    &= \{ A \cap T \colon A \in \mathcal{A}\} 
                .\end{align*}
                Damit folgt die Behauptung mit (c).
            \end{proof}
    \end{enumerate}
\end{aufgabe}

\begin{aufgabe}
    Sei $(\Omega, \mathcal{A}, \mathbb{P})$ Wahrscheinlichkeitsraum und $A, B, A_n \in \mathcal{A}$ für
    $n \in \N$.
    \begin{enumerate}[(a)]
        \item Beh.: $A \subseteq B \implies \mathbb{P}(A) \le \mathbb{P}(B)$.
            \begin{proof}
                Sei $A \subseteq B$. Dann ist
                \begin{salign*}
                    \mathbb{P}(B) = \mathbb{P}(A \cupdot B \setminus A)
                    &\stackrel{\sigma \text{-Additivität}}{=} \mathbb{P}(A) +
                    \underbrace{\mathbb{P}(B \setminus A)}_{\ge 0} \ge \mathbb{P}(A)
                .\end{salign*}
            \end{proof}
        \item Beh.: $| \mathbb{P}(A) - \mathbb{P}(B)| \le  \mathbb{P}(A \triangle B)$.
            \begin{proof}
                Es ist zunächst
                \begin{salign*}
                    \mathbb{P}(A \triangle B) &= \mathbb{P}(A \setminus B \cupdot B \setminus A) \\
                    &\stackrel{\sigma \text{-Additivität}}{=} \mathbb{P}(A \setminus B) + \mathbb{P}(B \setminus A) \\
                    &= \mathbb{P}(A) - \mathbb{P}(A \cap B) + \mathbb{P}(B) - \mathbb{P}(A \cap B) \\
                \intertext{
                    Sei o.E. $\mathbb{P}(A) \ge \mathbb{P}(B)$ (sonst analog durch Hinzufügen von
                $\mathbb{P}(A) - \mathbb{P}(A)$). Dann folgt}
                \mathbb{P}(A \triangle B) &= \mathbb{P}(A) - \mathbb{P}(B) + \mathbb{P}(B) - \mathbb{P}(A \cap B) + \mathbb{P}(B) - \mathbb{P}(A \cap B) \\
                &= |\mathbb{P}(A) - \mathbb{P}(B)| + \underbrace{2 \mathbb{P}(B \setminus A)}_{\ge 0} \\
                &\ge | \mathbb{P}(A) - \mathbb{P}(B)|
                .\end{salign*}
            \end{proof}
        \item Beh.: $\mathbb{P}(\bigcup_{k \in \N} A_k) \le \sum_{k=1}^{\infty} \mathbb{P}(A_k)$.
            \begin{proof}
                Betrachte $B_n \coloneqq A_n \setminus \left(\bigcup_{k=1}^{n-1}A_k\right)$. Dann
                ist $\forall n \in \N: B_n \subseteq A_n$ also mit (a) $\mathbb{P}(B_n) \le \mathbb{P}(A_n)$.
                Damit folgt
                \begin{salign*}
                    \mathbb{P}\left( \bigcup_{n \in \N} A_n \right)
                    = \mathbb{P}\left( \bigcupdot_{n \in \N} B_n \right)
                    &\stackrel{\sigma \text{Additivität}}{=}
                    \sum_{n=1}^{\infty} \mathbb{P}(B_n) \le \sum_{n=1}^{\infty} \mathbb{P}(A_n)
                .\end{salign*}
            \end{proof}
        \item Beh.: $A_n \subseteq A_{n+1} \forall n \in \N \implies \mathbb{P}(\bigcup_{n \in \N} A_n)
            = \lim_{n \to \infty} \mathbb{P}(A_n)$.
            \begin{proof}
                Sei $A_n \subseteq A_{n+1}$ $\forall n \in \N$. Betrachte
                $B_n \coloneqq A_n \setminus \left( \bigcup_{k=1}^{n-1} A_k \right) $. Da $A_n$ monoton
                wachsend, ist für $n \ge 2\colon B_n = A_n \setminus A_{n-1}$. Damit folgt
                \begin{salign*}
                    \mathbb{P}(\bigcup_{n \in \N} A_n) &= \mathbb{P}\left( \bigcupdot_{n \in \N} B_n \right)\\
                    &\stackrel{\sigma \text{Additivität}}{=} \sum_{n=1}^{\infty} \mathbb{P}(B_n)  \\
                    &= \mathbb{P}(B_1) + \sum_{n=2}^{\infty} \left( \mathbb{P}(A_n) - \mathbb{P}(A_n \cap A_{n-1}) \right) \\
                    &\stackrel{A_n \subseteq A_{n+1}}{=}
                    \mathbb{P}(B_1) + \sum_{n=2}^{\infty} \left( \mathbb{P}(A_n) - \mathbb{P}(A_{n-1}) \right) \\
                    &\stackrel{\text{Teleskopsumme}}{=} \mathbb{P}(B_1) - \mathbb{P}(A_1) + \lim_{n \to \infty} \mathbb{P}(A_n) \\
                    &\stackrel{B_1 = A_1}{=} \lim_{n \to \infty} \mathbb{P}(A_n)
                .\end{salign*}
            \end{proof}
    \end{enumerate}
\end{aufgabe}

\begin{aufgabe}
    Sei $(\Omega, \mathcal{A}, \mathbb{P})$ ein Wahrscheinlichkeitsraum.
    \begin{enumerate}[(a)]
        \item Der Induktionsanfang ist offensichtlich wahr, $\IP(A_1) = (-1)^0 \cdot \IP(A_1)$. Gelte die Behauptung also für ein $n\in \N$. Dann folgern wir
            \begin{align*}
                \IP\left(\bigcup_{j=1}^{n+1} A_j\right) =& \IP\left(\bigcup_{j=1}^{n} A_j\right) + \IP(A_{n+1}) - \IP\left(\bigcup_{j=1}^{n} A_j \cap A_{n+1}\right)\\
                =& \sum_{j = 1}^{n} \left((-1)^{j-1} \cdot \sum_{\{k_1, \dots, k_n\} \subset \{1,\dots, n\}} \IP(A_{k_1} \cap \dots \cap A_{k_j})\right) + \IP(A_{n+1})\\
                &- \IP\left(\bigcup_{j=1}^{n} (A_j \cap A_{n+1})\right)\\
                =& \sum_{j = 1}^{n} \left((-1)^{j-1} \cdot \sum_{\{k_1, \dots, k_n\} \subset \{1,\dots, n\}} \IP(A_{k_1} \cap \dots \cap A_{k_j})\right) + \IP(A_{n+1})\\
                &- \sum_{j = 1}^{n} \left((-1)^{j-1} \cdot \sum_{\{k_1,\dots, k_j\} \subset \{1,\dots, n\}} \IP((A_{k_1} \cap A_{n+1}) \cap \dots \cap (A_{k_j} \cap A_{n+1}))\right)\\
                =& \sum_{j = 1}^{n} \left((-1)^{j-1} \cdot \sum_{\{k_1, \dots, k_n\} \subset \{1,\dots, n\}} \IP(A_{k_1} \cap \dots \cap A_{k_j})\right) + \IP(A_{n+1})\\
                &- \sum_{j = 1}^{n} \left((-1)^{j-1} \cdot \sum_{\{k_1, \dots, k_j\} \subset \{1,\dots, n\}} \IP(A_{k_1} \cap \dots \cap A_{k_j} \cap A_{n+1})\right)\\
                =& \sum_{j = 1}^{n} \left((-1)^{j-1} \cdot \sum_{\substack{\{k_1, \dots, k_n\} \subset \{1,\dots, n+1\}\\\forall i\colon k_i \neq n+1}} \IP(A_{k_1} \cap \dots \cap A_{k_j})\right) + \IP(A_{n+1})\\
                &+ \sum_{j = 2}^{n+1} \left((-1)^{j-1} \cdot \sum_{\substack{\{k_1, \dots, k_j\} \subset \{1,\dots, n+1\}\\\exists i\colon k_i = n+1}} \IP(A_{k_1} \cap \dots \cap A_{k_j})\right)\\
                =& \sum_{j = 1}^{n} \left((-1)^{j-1} \cdot \sum_{\substack{\{k_1, \dots, k_n\} \subset \{1,\dots, n+1\}\\\forall i\colon k_i \neq n+1}} \IP(A_{k_1} \cap \dots \cap A_{k_j})\right)\\
                &+ \sum_{j = 1}^{n+1} \left((-1)^{j-1} \cdot \sum_{\substack{\{k_1, \dots, k_j\} \subset \{1,\dots, n+1\}\\\exists i\colon k_i = n+1}} \IP(A_{k_1} \cap \dots \cap A_{k_j})\right)\\
            \end{align*}
            Für $j = n+1$ gilt $\{k_1,\dots, k_j\} = \{1,\dots, n+1\}$. Daher können wir die beiden Summen im letzten Schritt einfach zusammenfassen und erhalten
            \[
                \IP\left(\bigcup_{j=1}^{n+1} A_j\right) =  \sum_{j = 1}^{n+1} \left((-1)^{j-1} \cdot \sum_{\{k_1, \dots, k_n\} \subset \{1,\dots, n\}} \IP(A_{k_1} \cap \dots \cap A_{k_j})\right),
            \]
            was zu zeigen war.
%        \item Sei $n \in \N$ und $A_1, \ldots, A_n \in \mathcal{A}$.
%            Beh.:
%            \[
%                \mathbb{P}\left( \bigcup_{j=1}^{n} A_n \right)
%                = \sum_{j=1}^{n} \left( (-1)^{j-1} \cdot \sum_{\{k_1, \ldots, k_j\} \subseteq \{1, \ldots, n\} }
%                \mathbb{P}(A_{k_1} \cap \ldots \cap A_{k_j}) \right)
%            .\]
%            \begin{proof}
%                Per Induktion über $n$. Sei $n=1$: Dann ist $\mathbb{P}(\bigcup_{j=1}^{1} A_j) = \mathbb{P}(A_1)$.
%                Sei nun $n \in \N$ und Behauptung gezeigt für $k \le n$. Dann gilt
%                \begin{salign*}
%                    \mathbb{P}\left( \bigcup_{j=1}^{n+1} A_j \right)
%                    =& \mathbb{P}\left(\bigcup_{j=1}^{n} A_j \cup A_{n+1}\right) \\
%                    \stackrel{(*)}{=}& \mathbb{P}\left( \bigcup_{j=1}^{n} A_j \right)
%                    + \mathbb{P}(A_{n+1}) - \mathbb{P}\left( \bigcup_{j=1}^{n} A_j \cap A_{n+1} \right) \\
%                    \stackrel{\text{I.V.}}{=}&
%                    \sum_{j=1}^{n} \left( (-1)^{j-1} \sum_{\{k_1, \ldots, k_j\} \subseteq \{1, \ldots, n\}}
%                    \mathbb{P}(A_{k_1} \cap \ldots \cap A_{k_j})\right)
%                    + \mathbb{P}(A_{n+1}) \\
%                    &- \sum_{j=1}^{n} \left( (-1)^{j-1} \sum_{\{k_1, \ldots, k_j\} \subseteq \{1, \ldots, n\} }
%                    \mathbb{P}(A_{k_1} \cap A_{n+1} \cap \ldots \cap A_{k_j} \cap A_{n+1}) \right) \\
%                    =& \sum_{j=1}^{n+1} \left( (-1)^{j-1} \sum_{\{k_1, \ldots, k_j\}\subseteq \{1, \ldots, n\} }
%                    \mathbb{P}(A_{k_1} \cap \ldots \cap A_{k_j})\right)
%                .\end{salign*}
%            \end{proof}
        \item Beh.: Die Wahrscheinlichkeit für $n \to \infty$ ist $1 - \frac{1}{e}$.
            \begin{proof}
                Setze $\Omega \coloneqq \{ (g_1, \ldots, g_n)  \mid g_1, \ldots, g_n \in \{1, \ldots, n\},
                g_i \neq g_j \text{ für } i \neq j\} $. Dabei bezeichnet ein Ergebnis
                $(g_1, \ldots, g_n) \in \Omega$: ,,Roter Marsmensch $i$ tanzt mit grünem Marsmensch $g_i$
                für $i \in \{1, \ldots, n\} $''. Die ursprüngliche Paarung
                sei dabei $(1, 2, \ldots, n) \in \Omega$.
                Es folgt direkt $\# \Omega = n!$.
                Definiere weiter
                \begin{align*}
                    \mathbb{P}\colon 2^{\Omega} &\to [0,1] \\
                    A &\mapsto \frac{\#A}{n!}
                .\end{align*}
                Wegen $\mathbb{P}(\Omega) = \frac{n!}{n!} = 1$ und $\mathbb{P}(\emptyset) = 0$
                ist $(\Omega, 2^{\Omega}, \mathbb{P})$ ein Wahrscheinlichkeitsraum.

                Damit ist für $i \in \{1, \ldots, n\} $:
                \begin{align*}
                A_i &= \text{,,Roter Marmensch }i\text{ tanzt mit der ursprünglichen Begleitung zusammen''} \\
                    &= \{ (g_1, \ldots, g_n) \in \Omega  \mid g_i = i\} 
                .\end{align*}
                Sei $A_n =$ ,,Mindestens ein ursprüngliches von insgesamt $n$ Paaren tanzt gemeinsam ''.
                Damit folgt
                \begin{salign*}
                    \mathbb{P}(A_n) &= \mathbb{P}\left( \bigcup_{i=1}^{n} A_i \right) \\
                    &\stackrel{\text{(a)}}{=} \sum_{j=1}^{n} \left((-1)^{j-1}
                    \sum_{\{k_1, \ldots, k_j\} \subseteq \{1, \ldots, n\} } \mathbb{P}(A_k \cap \ldots \cap A_{k_j}) \right) \\
                    &= \sum_{j=1}^{n} (-1)^{j-1} \binom{n}{j} \frac{(n-j)!}{n!} \\
                    &= \sum_{j=1}^{n} (-1)^{j-1} \frac{n!}{(n-j)! j!} \frac{(n-j)!}{n!} \\
                    &= \sum_{j=1}^{n} \frac{(-1)^{j-1}}{j!} \\
                    \intertext{Für $n \to \infty$ folgt}
                    \mathbb{P}(A_{\infty}) &= \sum_{j=1}^{\infty} \frac{(-1)^{j-1}}{j!} \\
                    &= - \left( \sum_{j=1}^{\infty} \frac{(-1)^{j}}{j!} \right) \\
                    &= - \left( \sum_{j=0}^{\infty} \frac{(-1)^{j}}{j!} - 1 \right) \\
                    &= - \left( \frac{1}{e} - 1 \right) \\
                    &= 1 - \frac{1}{e}
                .\end{salign*}
            \end{proof}
    \end{enumerate}
\end{aufgabe}

\begin{aufgabe}
    Sei $(\R, \mathscr{B}, \mathbb{P})$ Wahrscheinlichkeitsraum und
    $\mathbb{F}\colon \R \to [0,1]$, $\mathbb{F}(x) \coloneqq \mathbb{P} ((-\infty, x])$ für $x \in \R$.
    \begin{enumerate}[(a)]
        \item Beh.: $\mathbb{F}$ monoton wachsend.
            \begin{proof}
                Seien $x_1, x_2 \in \R$ mit $x_1 \le x_2$. Dann ist
                $(-\infty, x_1] \subseteq (-\infty, x_2]$. Mit 2(a) folgt damit
                $\mathbb{F}(x_1) = \mathbb{P}((-\infty, x_1]) \le \mathbb{P}((-\infty, x_2]) = \mathbb{F}(x_2)$.
            \end{proof}
        \item Beh.: $\lim_{x \to \infty} \mathbb{F}(x) = \R$.
            \begin{proof}
                Sei $(x_n)_{n \in \N}$ Folge mit $x_n \xrightarrow{n \to \infty} \infty$. Dann ist
                $A_n \coloneqq \bigcup_{j=1}^{n} (-\infty, x_n]$ monoton wachsende Folge
                mit $A_n \uparrow \R$. Damit folgt da $\mathbb{P}$ Wahrscheinlichkeitsmaß
                \[
                    \lim_{n \to \infty} \mathbb{F}(x_n) = \lim_{n \to \infty} \mathbb{P}(A_n)
                    \; \stackrel{\text{2(d)}}{=} \;\mathbb{P}(\R) = 1
                .\]
            \end{proof}
            Beh.: $\lim_{x \to -\infty} \mathbb{F}(x) = 0$.
            \begin{proof}
                Analog, betrachte nun $A_n \coloneqq \bigcap_{j=1}^{n} (-\infty, x_n] \downarrow \emptyset$.
            \end{proof}
        \item Beh.: $\mathbb{F}$ rechtsseitig stetig.
            \begin{proof}
                Sei $(x_n)_{n \in \N}$ in $\R$ mit $x_n \downarrow x$. Dann betrachte
                $A_n \coloneqq (-\infty, x_n]$. Es gilt sofort $A_n \downarrow
                \bigcap_{k \in \N} (-\infty, x_k] = (-\infty, x]$. Damit folgt
                \[
                    \lim_{n \to \infty} \mathbb{F}(x_n) = \lim_{n \to \infty} \mathbb{P}(A_n)
                    \stackrel{\text{2(d)}}{=} \mathbb{P}((-\infty, x]) = \mathbb{F}(x)
                .\]
            \end{proof}
        \item Beh.: $\mathbb{F}$ hat höchstens abzählbar viele Sprungstellen.
            \begin{proof}
                Sei $a \in \R$ beliebig. Dann betrachte
                \begin{salign*}
                    \lim_{x \searrow a} \mathbb{F}(x) - \lim_{x \nearrow a} \mathbb{F}(x)
                    &\stackrel{\text{(c) und Hinweis}}{=} \mathbb{P}((-\infty, a])
                    - \mathbb{P}((-\infty, a)) \\
                    &= \mathbb{P}((-\infty, a]) - \mathbb{P}((-\infty, a] \cap (-\infty, a)) \\
                    &= \mathbb{P}((-\infty, a] \setminus (-\infty, a)) \\
                    &= \mathbb{P}( \{ a\} )
                .\end{salign*}
                Die Sprungstellen von $F$ sind also gerade die Atome von $\mathbb{P}$. Da $\mathbb{P}$
                nach VL nur höchstens abzählbar viele Atome auf $\R$ hat, folgt die Behauptung.
            \end{proof}
    \end{enumerate}
\end{aufgabe}

\end{document}
