\documentclass[uebung]{lecture}

\title{Wtheo 0: Übungsblatt 2}
\author{Josua Kugler, Christian Merten}
\usepackage[]{bbm}

%sorry für die vielen Commands xD
\newcommand{\A}{\mathcal{A}}
\newcommand{\B}{\mathcal{B}}
\newcommand{\D}{\mathcal{D}}
\newcommand{\E}{\mathcal{E}}
\newcommand{\X}{\mathcal{X}}
\newcommand{\F}{\mathbb{F}}
\newcommand{\IP}{\mathbb{P}}

\begin{document}

\def\headheight{13pt}

\punkte[5]

\begin{aufgabe}
    \begin{enumerate}[(a)]
        \item Beh.: $\mathcal{D}$ ist ein Dynkinsystem.
            \begin{proof}
                \begin{enumerate}[(i)]
                    \item $\Omega \in \mathcal{D}$, denn $|\Omega| = 2n$ gerade.
                    \item Sei $A \in \mathcal{D}$. Dann ist $|A| = 2k$ für ein $k \in \N_0$ mit
                        $k \le 2n$. Da $\Omega$ endlich folgt
                        \[
                            |A^{c}| = |\Omega| - |A| = 2n - 2k = 2(n-k)
                        .\] Also $A^{c} \in \mathcal{D}$.
                    \item Sei $A_i \in \mathcal{D}$ $\forall i \in \N$ mit $A_i \cap A_j = \emptyset$
                        für $i\neq j$. Dann ex. für $i \in \N$ ein $k_i \in \N_0$ mit
                        $|A_i| = 2 k_i$. Damit folgt, da die $A_i$ disjunkt sind
                        \[
                        \left| \bigcupdot_{i \in  \N} A_i \right|
                        = \sum_{i \in \N} |A_i| = \sum_{i \in \N} 2 k_i
                        = 2 \underbrace{\sum_{i \in \N} k_i}_{\in \N_0}
                        .\] 
                        Also $\bigcup_{i \in \N} A_i \in \mathcal{D}$.
                \end{enumerate}
            \end{proof}
        \item Beh.: Für $n \ge 2$ ist $\mathcal{D}$ keine $\sigma$-Algebra.
            \begin{proof}
                Sei $n \ge 2$. Dann ist $|\Omega| \ge 4$. Seien dann $\omega_1, \omega_2, \omega_3 \in \Omega$
                paarweise
                verschieden. Dann ist
                \[
                    \underbrace{\{\omega_1, \omega_2 \}}_{\in \mathcal{D}}
                    \cap
                    \underbrace{\{\omega_2, \omega_3\}}_{\in \mathcal{D}}
                    = \{w_2\} \not\in \mathcal{D}
                .\] Also $\mathcal{D}$ nicht $\cap $-stabil, also keine $\sigma$-Algebra.
            \end{proof}
    \end{enumerate}
\end{aufgabe}

\begin{aufgabe}
    \begin{enumerate}[(a)]
        \item \textbf{Zu zeigen:} $\IP_1 = \IP_2$.
        \begin{proof}
        Schritt 2 und Schritt 3 sind bereits erledigt. Daher betrachten wir die Menge $\D = \{A\in \A| \IP_1(A) = \IP_2(A)\}$.
        \begin{enumerate}[(i)]
            \item Da $\E$ ein Erzeuger von $\A$ ist, muss $\Omega \in \E$ liegen, somit gilt $\IP_1(\Omega) = \IP_2(\Omega)$ und daraus folgt $\Omega \in \D$. 
            \item Sei $E \in \D$. Dann gilt $\IP_1(E^c) = \IP_1(\Omega \setminus E) = \IP_1(\Omega) - \IP_1(\Omega \cap E) = 1 - \IP_1(E) = 1 - \IP_2(E)$. Mithilfe analoger Umformungsschritte auf der rechten Seite erhält man $\IP_1(E^c) = \IP_2(E^c)$ und damit $E^c \in \D$. $\D$ ist also komplementstabil
            \item Sei $\forall n \in \N\colon E_n \in \D$. Dann gilt 
            \[
                \IP_1\left(\biguplus_{n\in \N} E_n\right) = \sum_{n \in \N} \IP_1(E_n) = \sum_{n \in \N} \IP_2(E_n) = \IP_2\left(\biguplus_{n\in \N} E_n\right).
                \]
                Somit ist auch $\biguplus_{n\in \N}E_n \in \D$.
            \end{enumerate}
            $\D$ ist also ein Dynkin-System. Da $\E$ schnittstabil ist, gilt $\E \subset \D$. Insbesondere folgt unter Benutzung des $\pi-\lambda-$Satzes \[\A = \sigma(\E) = \delta(\E) \subset \D,\] da $\D$ ja ein Dynkin-System ist, das $\E$ enthält. Wegen $\D \subset \A$ erhalten wir die sofort $\A = \D$. Somit gilt $\IP_1(A) = \IP_2(A) \forall A \in \A$.
        \end{proof}
        \item \textbf{Behauptung:} $\sigma(\E) = 2^\Omega$. Außerdem sind die Wahrscheinlichkeitsmaße $\IP_1,\IP_2$ eindeutig gegeben durch
        \[
            \IP_1(\{x\}) = 0.25 \forall x \in \Omega
        \]
        \[
            \IP_2(\{a\}) = \IP_2(\{c\}) = 0.2,\quad \IP_2(\{b\}) = \IP_2(\{d\}) = 0.3
        \] und stimmen auf $\E$ überein, nicht aber auf $2^\Omega$.
        \begin{proof}
            Eine $\sigma$-Algebra enthält stets $\Omega$ und ist stabil bezüglich Schnitt, Vereinigung und Komplement. Daher liegen $\Omega = \{a,b,c,d\},\; \{a\} = A \setminus C,\; \{b\} = A\cap C,\; c = C \setminus A$ und $\{d\} = \Omega \setminus (A \cup C)$  in $\sigma(\E)$. Aus den Mengen $\{a\}, \{b\},\{c\},\{d\}$ erhält man durch disjunkte Vereinigung jede Teilmenge $E \in 2^\Omega$. Daraus folgt auf der einen Seite $\sigma(\E) = \Omega$. Auf der anderen Seite folgt auch, dass $\IP_1$ und $\IP_2$ durch die Werte auf diesen vier einelementigen Mengen bereits eindeutig bestimmt sind, da jeder beliebige Wert als disjunkte Vereinigung aus den Mengen und damit als Summe aus den Werten von $\IP_i$ konstruiert werden kann.
            Offensichtlich ist $\IP_1(\{a\}) \neq \IP_2(\{a\})$. Daher stimmen die beiden Maße auf $2^\Omega$ nicht überein. Es gilt aber $\IP_1(A) = \IP_1(\{a\}\uplus \{b\}) = \IP_1(\{a\}) + \IP_1(\{b\}) = 0.5 = \IP_2(\{a\}) + \IP_2(\{b\}) = \IP_2(A)$ und $\IP_1(B) = \IP_1(\{b\}\uplus \{c\}) = \IP_1(\{b\}) + \IP_1(\{c\}) = 0.5 = \IP_2(\{b\}) + \IP_2(\{c\}) = \IP_2(B)$.
        \end{proof}
        Offensichtlich ist $\E$ einfach nicht schnittstabil, da $A \cap C = \{b\}\notin \E$. Also lässt sich auch der Maßeindeutigkeitssatz nicht anwenden.
    \end{enumerate}
\end{aufgabe}

\begin{aufgabe}[]
    \begin{enumerate}[(a)]
        \item Beh.: $\sum_{\omega \in \Omega} \mathbbm{p}(\omega) = 1$
            \begin{proof}
                \begin{enumerate}[(i)]
                    \item Z.z.: $\binom{\alpha + k -1}{k} = (-1)^{k} \binom{-\alpha}{k}$
                        $\forall \alpha \in \N, k \in \N_0$.

                        Seien $\alpha \in \N, k \in \N_0$. Dann folgt
                        \begin{salign*}
                            \binom{\alpha + k -1}{k} &= \frac{(\alpha + k -1)!}{k!(\alpha -1)!} \\
                            &= \frac{(\alpha + k-1)\cdots (\alpha +1) \alpha }{k!} \\
                            &= (-1)^{k}\frac{(-\alpha -(k-1) \cdot \ldots \cdot (-\alpha -1)(-\alpha)}{k!} \\
                            &= (-1)^{k}\frac{(-\alpha)(-\alpha-1)\cdot \ldots\cdot (-\alpha-(k-1))}{k!} \\
                            &= (-1)^{k} \binom{-\alpha}{k}
                        .\end{salign*}
                    \item Z.z.: $(1+x)^{\alpha} = \sum_{k=0}^{\infty} \binom{\alpha}{k} x^{k}$
                        $\forall \alpha \in \Z, x \in (-1,1)$.

                        Seien $\alpha \in \Z$, $x \in (-1,1)$. Dann betrachte
                        \begin{salign*}
                            f\colon (-1,1) &\to \R \\
                            x&\mapsto (1+x)^{\alpha}
                        .\end{salign*}
                        Dann ist $f^{(k)}(0) = \prod_{j=0}^{k-1} (\alpha-j) $. Damit folgt als Taylorpolynom
                        für $f$ im Entwicklungspunkt $x_0 = 0$:
                        \begin{salign*}
                            T_n(x, 0) &= \sum_{k=0}^{n} \frac{f^{(k)}(0)}{k!} (x-0)^{k} \\
                            &= \sum_{k=0}^{n} \frac{\prod_{j=0}^{k-1} (\alpha-j) }{k!} x^{k} \\
                            &= \sum_{k=0}^{n} \binom{\alpha}{k} x^{k}
                        .\end{salign*}
                        Mit $a_k \coloneqq \binom{\alpha}{k} x^{k}$ folgt
                        \begin{salign*}
                            \left| \frac{a_{k+1}}{a_k} \right| &= \left| \frac{\frac{\prod_{j=0}^{k} (\alpha-j)  }{(k+1)!} x^{k+1}}{\frac{\prod_{j=0}^{k-1} (\alpha-j) }{k!}x^{k}} \right| \\
                            &= \left| \frac{\alpha-k}{k+1} \right| |x| \\
                            &\xrightarrow{k \to \infty} |x| < 1
                        .\end{salign*}
                        $T_n$ ist also konvergent $\forall x \in (-1,1)$:
                        \[
                            T_n(x, 0) \xrightarrow{n \to \infty} f(x) = (1+x)^{\alpha}
                        .\] 
                    \item Damit folgt nun für $r \in \N$ und $p \in (0,1)$:
                        \begin{salign*}
                            \sum_{\omega \in \N_0} \mathbbm{p}(\omega)
                            &= \sum_{\omega \in \N_0} \binom{\omega + r -1}{\omega} p^{r} (1-p)^{\omega} \\
                            &= p^{r} \sum_{\omega \in \N_0} \binom{\omega + r -1}{\omega} (1-p)^{\omega} \\
                            &\stackrel{\text{(i)}}{=} p^{r} \sum_{\omega \in \N_0} (-1)^{\omega}
                            \binom{-r}{\omega}(1-p)^{\omega} \\
                            &= p^{r} \sum_{\omega \in \N_0} \binom{-r}{\omega} (p-1)^{\omega} \\
                            &\stackrel{\text{(ii)}}{=}
                            p^{r}(1+p-1)^{-r} \\
                            &= p^{r} p^{-r} \\
                            &= 1
                        .\end{salign*}
                \end{enumerate}
            \end{proof}
            Mit Hilfe dieser Zähldichte kann modelliert werden, dass eine Münze bei $\omega + r$ Würfen
            genau im $\omega + r$-ten Wurf $r$ mal Kopf gezeigt hat.
        \item Es soll nach dem 30. Zug genau zum 6. Mal gewonnen werden, d.h. $r=6$, damit
            \[
            \omega +r = 30 \implies \omega = 24
            .\] Mit $p=0.2$ und der (a) folgt
            \[
                \mathbb{P}(\{\omega\}) = \binom{24 + 6 -1}{24} 0.2^{6}(1-0.2)^{24} \approx 0.625
            .\]
    \end{enumerate}
\end{aufgabe}

\begin{aufgabe}
    \begin{enumerate}[(a)]
        \item Es gilt 
        \begin{align*}
            \scriptstyle{\IP}_{\text{Hyp}_{(N, M, n)}}(\omega) &= \frac{\binom{N-M}{n-\omega}\binom{M}{\omega}}{\binom{N}{n}}\\
            &= \frac{\frac{(N-M)!}{(N-M-(n-\omega))! \cdot (n-\omega)!} \cdot \frac{M!}{(M-\omega)! \cdot \omega!}}{\frac{N!}{(N-n)!\cdot n!}}\\
            &= \frac{n!}{(n-\omega)!\cdot \omega!} \cdot \frac{M!}{(M-\omega)!} \cdot \frac{(N-n)!}{N!} \cdot \frac{(N-M)!}{(N-M-(n - \omega))!}\\
            &= \binom{n}{\omega} \cdot \frac{M^\omega \cdot \prod_{i=1}^\omega (1 - \frac{i}{M})}{N^\omega \cdot \prod_{i = 1}^\omega (1 - \frac{i}{N})} \cdot \frac{(N-M)^{n - \omega - 1} \prod_{i = 1}^{n - \omega - 1}\left(1 - \frac{i}{N-M}\right)}{N ^{n-1 - \omega} \prod_{i = \omega}^{n - 1} (1 - \frac{i}{N})}
            \intertext{Bilden wir nun den Grenzwert $\lim\limits_{N, M \to \infty}$, so erhalten wir}
            &= \lim\limits_{N, M \to \infty} \binom{n}{\omega} \cdot \left(\frac{M}{N}\right)^\omega \cdot \left(\frac{N- M }{N}\right)^{n - \omega - 1}
            \intertext{Wegen $M/N \to p$ erhalten wir daraus}\\
            &= \binom{n}{\omega} \cdot \left(p\right)^\omega \cdot \left(1-p\right)^{n - \omega - 1}\\
            &= \scriptstyle{\IP}_{\text{Bin}_{(n, p)}}(\omega)
        \end{align*}
        \item Die Situation kann durch eine hypergeometrische Verteilung $\text{Hyp}_{(N, M , n)}$ mit $N = 1000, M = 200, n = 10$ modelliert werden. Daher erhalten wir als exaktes Ergebnis
        \[
            \scriptstyle{\IP}_{\text{Hyp}_{(1000, 200, 10)}}(2) = \frac{\binom{800}{8}\binom{200}{2}}{\binom{1000}{10}} \approx 0.304
        \] und für die Näherung durch $\text{Bin}_{(10,0.2)}$ ergibt sich
        \[
            \scriptstyle{\IP}_{\text{Bin}_{(10, 0.2)}}(2) = \binom{10}{2} (0.2)^2 (0.8)^2 \approx 0.302.
        \]
        \item Die Zähldichte entspricht genau einer Binomialverteilung $\text{Bin}_{(n, p)}$ mit $n = 100$ und $p = 0.01$. Es gilt nun für das eindeutig bestimmte Wahrscheinlichkeitsmaß 
        \[
            \IP(\{x | 2 \leq x \leq 100\}) = \IP(\{1,\dots, 100\} \setminus \{0,1\}) = 1 - \scriptstyle{\IP}_{\text{Bin}_{(100, 0.01)}}(0) - \scriptstyle{\IP}_{\text{Bin}_{(100, 0.01)}}(1).
        \]
        Wegen $\scriptstyle{\IP}_{\text{Bin}_{(100, 0.01)}}(0) = \binom{100}{0} \cdot 0.01^0 \cdot 0.99^100 \approx 0.366$ und $\scriptstyle{\IP}_{\text{Bin}_{(100, 0.01)}}(1) = \binom{100}{1} \cdot 0.01^1 \cdot 0.99^99 = 0.370$
        erhalten wir damit als exakte Wahrscheinlichkeit $\IP(\{x | 2 \leq x \leq 100\}) \approx 1 - 0.366 - 0.370 = 0.264$.
        Wir nähern nun die Binomialverteilung durch eine Poisson-Verteilung. Wegen $p \cdot n = 0.01 \cdot 100 = 1$ wählen wir $\lambda = 1$ und erhalten $\scriptstyle{\IP}_{\text{Bin}_{(100, 0.01)}}(0) \approx \scriptstyle{\IP}_{\text{Poi}_1}(0) = e^{-1}\frac{1^0}{0!} = \frac{1}{e}$ und $\scriptstyle{\IP}_{\text{Bin}_{(100, 0.01)}}(1) \approx \scriptstyle{\IP}_{\text{Poi}_1}(1) = e^{-1}\frac{1^1}{1!} = \frac{1}{e}$. Für die genäherte Wahrscheinlichkeit ergibt sich damit $\IP(\{x | 2 \leq x \leq 100\}) \approx 1 - \frac{2}{e} = 0.264$.
    \end{enumerate}
\end{aufgabe}

\end{document}
