\begin{frame}
\begin{block}{Beweis.}
    \vspace*{-0.5cm}
    \begin{align*}
        \pi^{-s/2} \Gamma(s/2)\zeta(s) &= R_\infty(s) + R_0(s)\\
        \visible<3->{\xi(s)}\visible<2->{&= R_\infty(s) + R_\infty(1-s) - \frac{1}{s} - \frac{1}{1-s}}
    \end{align*}%
    \begin{itemize}
        \item<4-> $R_\infty$ ist ganz
        \item<5-> $\xi$ ist holomorph auf $\C \setminus\{0,1\}$.
        \item<6-> $\xi$ genügt der Funktionalgleichung $\xi(s) = \xi(1-s)$
    \end{itemize}
\end{block}
\end{frame}
\begin{frame}
    \begin{block}{Beweis.}
        \begin{align*}
            \pi^{-s/2} \Gamma(s/2)\zeta(s) &= \xi(s)\\
            \visible<2->{\zeta(s) &= \frac{\pi^{s/2}}{\Gamma(s/2)}\xi(s)}
        \end{align*}
        \visible<3->{$\Gamma$ ist meromorph auf $\C$ und besitzt keine Nullstellen.\\}
        \visible<4->{$\implies$ $\frac{\pi^{s/2}}{\Gamma(s/2)}$ ist holomorph auf $\C$.\\}
        \visible<5->{$\implies$ $\frac{\pi^{s/2}}{\Gamma(s/2)}\xi(s)$ ist holomorph auf $\C\setminus\{0,1\}$.}
    \end{block}
\end{frame}
\begin{frame}
    \begin{block}{Beweis.}
        \begin{align*}
            & \lim\limits_{s \to 0} \frac{\pi^{s/2}}{\Gamma(s/2)}\bigg(\underbrace{R_\infty(s) + R_\infty(1-s) - \frac{1}{1-s}}_{\implies \text{beschränkt für } s \to 0} - \frac{1}{s}\bigg)\\
            \visible<2->{\intertext{Wegen $\lim\limits_{s \to 0} \Gamma(s/2) = \infty$ erhalten wir}}
            \visible<3->{=& 0 - \lim\limits_{s \to 0} \frac{\pi^{s/2}}{s\cdot \Gamma(s/2)} = - \lim\limits_{s \to 0} \frac{\pi^{s/2}}{2\cdot s/2\cdot \Gamma(s/2)}\\}
            \visible<4->{=& -\lim\limits_{s \to 0} \frac{\pi^{s/2}}{2\cdot \Gamma(s/2 + 1)}= -\frac{\pi^0}{2\cdot \Gamma(1)} = -\frac{1}{2}}
        \end{align*}
    \end{block}
\end{frame}
\begin{frame}
    \begin{block}{Beweis.}
        Die Funktion \[
            \frac{\pi^{s/2}}{\Gamma(s/2)}\left(R_\infty(s) + R_\infty(1-s) - \frac{1}{s} - \frac{1}{1-s}\right)
        \] ist daher
        \begin{itemize}
            \item<2-> holomorph auf $\C \setminus \{1\}$. 
            \item<3-> stimmt für $\Re s > 1$ mit $\zeta(s)$ überein
        \end{itemize}
        \visible<4->{$\implies$ stellt die gesuchte analytische Fortsetzung für die Riemannsche $\zeta$-Funktion dar!!!}
    \end{block}
\end{frame}