\begin{frame}
\begin{block}{Beweis.}
    \vspace*{-0.5cm}
    \begin{align*}
        \pi^{-s/2} \Gamma(s/2)\zeta(s) &= R_\infty(s) + R_0(s)\\
        \visible<3->{\xi(s)}\visible<2->{&= R_\infty(s) + R_\infty(1-s) - \frac{1}{s} - \frac{1}{1-s}}
        %\visible<3->{\zeta(s) &= \frac{\pi^{s/2}}{\Gamma(s/2)} \left(R_\infty(s) + R_\infty(1-s) - \frac{1}{s} - \frac{1}{1-s}\right)}
    \end{align*}%
    \begin{itemize}
        \item<4-> $R_\infty$ ist ganz
        \item<5-> $\xi$ ist ganz bis auf einfache Polstellen bei $s = 0$ und $s = 1$
        \item<6-> $\xi$ genügt der Funktionalgleichung $\xi(s) = \xi(1-s)$
    \end{itemize}
    %\begin{itemize}
    %    \item<4-> $R_\infty$ und $\Gamma$ sind ganz
    %    \item<5-> $\Gamma$ besitzt keine Nullstellen
    %    \item<6->[$\Rightarrow$] Einzige Singularitäten bei $s = 1$ und $s = 0$. Für $s = 0$ gilt:
    %    \[\lim\limits_{s \to 0} \frac{1}{s\cdot \Gamma(s/2)} = \lim\limits_{s \to 0} \frac{1}{2\Gamma(s/2 + 1)} = \frac{1}{2\Gamma(1)} = \frac{1}{2}\]
    %    \vspace*{-0.5cm}
    %\end{itemize}
\end{block}
\end{frame}

\begin{frame}
    \begin{block}{Beweis.}
        \vspace*{-0.5cm}
        \begin{align*}
            \pi^{-s/2} \Gamma(s/2)\zeta(s) &= \xi(s)\\
            \visible<2->{\zeta(s) &= \frac{\pi^{s/2}}{\Gamma(s/2)}}
            \begin{overprint}
                \only<2>{\xi(s)}
                \only<3->{\left(R_\infty(s) + R_\infty(1-s) - \frac{1}{s} - \frac{1}{1-s}\right)}
            \end{overprint}
        \end{align*}%
        \vspace*{-0.5cm}
        \begin{itemize}
            \item<4-> Die $\Gamma$-Funktion ist meromorph auf $\C$ und besitzt einfache Polstellen genau bei $s = 0, -1,\dots$
            \item<5-> Es gilt \[\lim\limits_{s \to 0} \frac{1}{s\cdot \Gamma(s/2)} = \lim\limits_{s \to 0} \frac{1}{2\Gamma(s/2 + 1)} = \frac{1}{2\Gamma(1)} = \frac{1}{2},\] die Polstelle der $\Gamma$-Funktion bei $s = 0$ gleicht also die Polstelle der $\xi$-Funktion aus
        \end{itemize}
        %\begin{itemize}
        %    \item<4-> $R_\infty$ und $\Gamma$ sind ganz
        %    \item<5-> $\Gamma$ besitzt keine Nullstellen
        %    \item<6->[$\Rightarrow$] Einzige Singularitäten bei $s = 1$ und $s = 0$. Für $s = 0$ gilt:
        %    \[\lim\limits_{s \to 0} \frac{1}{s\cdot \Gamma(s/2)} = \lim\limits_{s \to 0} \frac{1}{2\Gamma(s/2 + 1)} = \frac{1}{2\Gamma(1)} = \frac{1}{2}\]
        %    \vspace*{-0.5cm}
        %\end{itemize}
    \end{block}
\end{frame}
\begin{frame}
    \begin{block}{Beweis.}
        Die Funktion \[
            \frac{\pi^{s/2}}{\Gamma(s/2)}\left(R_\infty(s) + R_\infty(1-s) - \frac{1}{s} - \frac{1}{1-s}\right)
        \] ist daher
        \begin{itemize}
            \item<2-> holomorph auf $\C \setminus \{1\}$. 
            \item<3-> stimmt für $\Re s > 1$ mit $\zeta(s)$ überein
        \end{itemize}
        \visible<4->{$\implies$ stellt die gesuchte analytische Fortsetzung für die Riemannsche $\zeta$-Funktion dar!!!}
        %\begin{itemize}
        %    \item<4-> $R_\infty$ und $\Gamma$ sind ganz
        %    \item<5-> $\Gamma$ besitzt keine Nullstellen
        %    \item<6->[$\Rightarrow$] Einzige Singularitäten bei $s = 1$ und $s = 0$. Für $s = 0$ gilt:
        %    \[\lim\limits_{s \to 0} \frac{1}{s\cdot \Gamma(s/2)} = \lim\limits_{s \to 0} \frac{1}{2\Gamma(s/2 + 1)} = \frac{1}{2\Gamma(1)} = \frac{1}{2}\]
        %    \vspace*{-0.5cm}
        %\end{itemize}
    \end{block}
\end{frame}