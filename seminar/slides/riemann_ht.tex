\begin{frame}
    \begin{theorem}[Riemannsche Hypothese]
        Abgesehen von den \glqq trivialen\grqq\ Nullstellen bei $s = -2n,\; n\in \N$ haben alle Nullstellen Realteil $\frac{1}{2}$.
    \end{theorem}
    de la Vallée-Poussin: $\pi(x) - \operatorname{Li}(x) = \mathcal{O}(xe^{-a\sqrt{\log(x)}})$\\
    mit RH: $\pi(x) - \operatorname{Li}(x) = \mathcal{O}(\sqrt{x}\log(x))$
\end{frame}