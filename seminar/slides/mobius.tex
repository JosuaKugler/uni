\begin{frame}
    \frametitle{Konsequenzen der Riemannschen Hypothese}

    \begin{definition}
        \[
            \mu(n) = \begin{cases}
                (-1)^k & n \text{ ist quadratfrei und hat $k$ Primfaktoren}\\
                0 & \text{sonst}
            \end{cases}  
        \]
    \end{definition}
    \begin{lemma}
        Es gilt \[
            \frac{1}{\zeta(s)} = \sum_{n = 1}^{\infty} \frac{\mu(n)}{n^s}  
        \] für $\Re s > 1$ mit
    \end{lemma}
\end{frame}
\begin{frame}
    \frametitle{Konsequenzen der Riemannschen Hypothese}
    \begin{proof}\vspace*{-0.5cm}
        \[
            \frac{1}{\zeta(s)} = \prod_{k=1}^\infty \left(1 - \frac{1}{p_k^s}\right) = \left(1 - \frac{1}{2^s}\right)\left(1 - \frac{1}{3^s}\right) \dots = \sum_{n = 1}^{\infty} \frac{\mu(n)}{n^s} 
        \]
    \end{proof}
    \begin{theorem}
        Die Gleichung \[
            \frac{1}{\zeta(s)} = \sum_{n = 1}^{\infty} \frac{\mu(n)}{n^s}
        \] für $\Re s > \frac{1}{2}$ ist äquivalent zur Riemannschen Hypothese.
    \end{theorem}
\end{frame}