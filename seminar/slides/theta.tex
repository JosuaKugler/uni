\begin{frame}

    \begin{lemma}[$\theta$-Funktion]
        Die Thetafunktion, gegeben durch
        \[
            \theta(z) = \sum_{n = -\infty}^{\infty} e^{i\pi n^2z},
        \] konvergiert für $\Im z > 0$.
    \end{lemma}
    \begin{proof}
        \begin{align*}
            \visible<2->{\sum_{n = -\infty}^{\infty} \left|e^{i\pi n^2 (\Re z + i (\Im z)}\right| &= \sum_{n = -\infty}^{\infty} \left|e^{i\pi n^2 \Re z}\right|\cdot \left| e^{i \pi n^2 \cdot i (\Im z)}\right|\\}
            \visible<3->{&= \sum_{n = -\infty}^{\infty} e^{- \pi n^2 \Im z}\\}
            \visible<4->{&\leq C \cdot \sum_{n = -\infty}^{\infty} \frac{1}{n^2}\\}
            \visible<5>{&<\infty}
        \end{align*}
    \end{proof}
    %\begin{itemize}
    %    \item<3-> erfüllt die Thetatransformationsformel ($\sqrt{\cdot}$ bezeichne den Zweig mit positivem Realteil): 
    %    \begin{align*}
    %        \theta(z) &= \theta\left(-\frac{1}{z}\right)\cdot \sqrt{\frac{i}{z}}\\
    %        \Leftrightarrow \theta(it) &= \theta\left(-\frac{1}{it}\right) \cdot \sqrt{\frac{i}{it}} = \theta\left(it^{-1}\right)\cdot t^{-\frac{1}{2}}.
    %    \end{align*}
    %    \end{itemize}
\end{frame}
\begin{frame}
    \begin{behauptung}
        $\theta(z)$ erfüllt die Thetatransformationsformel ($\sqrt{\cdot}$ bezeichne den Zweig mit positivem Realteil): 
            \begin{align*}
                \theta(z) &= \theta\left(-\frac{1}{z}\right)\cdot \sqrt{\frac{i}{z}}\\
                \Leftrightarrow \theta(it) &= \theta\left(-\frac{1}{it}\right) \cdot \sqrt{\frac{i}{it}} = \theta\left(it^{-1}\right)\cdot t^{-\frac{1}{2}}.
            \end{align*}
    \end{behauptung}
\end{frame}