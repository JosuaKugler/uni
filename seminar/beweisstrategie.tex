\begin{frame}
    Beweisstrategie: 
    \begin{itemize}
        \item Fallunterscheidung je nach Eigenschaften des Randes (positiv berandet / nullberandet).
        \item \textbf{Konstruiere jeweils eine injektive holomorphe Funktion $f\colon X \to \overline{\C}$.}
        \item Erhalte eine biholomorphe Abbildung $X \cong f(X) \subseteq \C$. 
        \item Ist $X \subsetneq \C$, so folgt mit dem Riemann'schen Abbildungssatz $X \cong \C$ oder $X\cong \E$.
    \end{itemize}
\end{frame}