\documentclass{article}
\usepackage[T1]{fontenc}
\usepackage[utf8]{inputenc}
\usepackage[ngerman]{babel}
\usepackage{amsmath}
\usepackage{amssymb}
\usepackage{amsthm}
\usepackage{mathtools}
%\usepackage{fancyhdr}


\newtheorem*{satz}{Satz}
\newtheorem*{theorem}{Theorem}
\newtheorem*{definition}{Definition}

\title{Endliche separable $K$-Algebren}
\author{}
\date{\vspace*{-1cm}28.04.2022}

\begin{document}
    \maketitle
    \begin{satz}[Struktur endlicher $K$-Algebren]
        Sei $B$ eine endlichdimensionale $K$-Algebra. Dann gilt
        \[
            B \cong \prod_{i=1}^t B_i
        \]
        für ein $t \geq 0$ und lokale $K$-Algebren $B_i$ mit nilpotentem Maximalideal.
    \end{satz}
    \begin{definition}
        Eine endliche $K$-Algebra $B$ heißt separabel, wenn der Homomorphismus
        \begin{align*}
            \phi\colon B &\to \hom_K(B, K)\\
            x &\mapsto (y \mapsto \operatorname{Sp}(xy))
        \end{align*}
        ein Isomorphismus ist.
    \end{definition}
    \begin{satz}[Struktur separabler Algebren]
        Sei $K$ ein Körper mit algebraischen Abschluss $\overline{K}$ und $B$ eine endlichdimensionale $K$-Algebra.
        Wir definieren außerdem die $K$-Algebra $\overline B \coloneqq B \otimes_K \overline K$.
        Dann sind die folgenden Aussagen äquivalent:
        \begin{enumerate}
            \item $B$ ist separabel über $K$
            \item $\overline{B}$ ist separabel über $\overline{K}$
            \item $\overline{B} \cong \overline{K}^n$ als $K$-Algebren für ein $n \geq 0$
            \item $B \cong \prod_{i=1}^t B_i$ als $K$-Algebren für ein $t \geq 0$ und endlich separablen Körpererweiterungen $B_i/K$.
        \end{enumerate}
    \end{satz}
    \begin{definition}
        Eine $\pi$-Menge ist eine endliche Menge $E$ mit einer stetigen Gruppenwirkung $\pi \curvearrowright E$.
        Dabei trägt $\pi$ die proendliche und $E$ die diskrete Topologie. 
    \end{definition}
    Das Hauptresultat dieses Vortrags deutet schon auf die Kategorienantiäquivalenz hin, die wir insgesamt im Seminar zeigen wollen:
    \begin{theorem}
        Sei $K$ ein Körper und $\pi = \operatorname{Gal}(K_s/K)$ die absolute Galoisgruppe. Dann sind die Kategorien $\operatorname{SepAlg}_K$ und $\pi-\operatorname{Fin}$ antiäquivalent.
    \end{theorem}
\end{document}