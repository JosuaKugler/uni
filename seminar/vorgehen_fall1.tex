\begin{frame}
    \frametitle{Vorgehen}
    \begin{itemize}
        \item Es existiert eine holomorphe Funktion $F_a\colon X \to \C$ mit $|F_a(x)| = e^{-G_a(x)}$ für $x \neq a$.
        \item $F_a$ ist injektiv.
        \item Wir erhalten eine bijektive holomorphe (und damit direkt biholomorphe nach Funktheo 1) Abbildung von $X$ auf $F_a(X)$.
        \item $F_a(X)$ ist beschränkt ($|F_a(x)| < 1$) und einfach zusammenhängend.
        \item Mit dem Riemann’schen Abbildungssatz folgt $X \cong \mathbb{E}$.
    \end{itemize}
\end{frame}