\begin{frame}
    \begin{lemma}
        Auf nullberandeten Flächen gilt der Satz von Liouville.
    \end{lemma}
    \begin{lemma}
        Auf einer Riemann'schen Fläche existiert eine harmonische Funktion $u \coloneqq u_{a,b} \colon X \setminus \{a, b\} \to \C$ mit folgenden Eigenschaften:
        \begin{itemize}
            \item $u$ ist logarithmisch singulär bei $a$.
            \item $-u$ ist logarithmisch singulär bei $b$.
            \item $u$ ist beschränkt im Unendlichen, d.h. in $X \setminus [U (a) \cup U (b)]$, wobei $U(a)$ und $U(b)$ zwei beliebige Umgebungen von $a, b$ seien.
        \end{itemize}
    \end{lemma}
\end{frame}