\documentclass{article}

\usepackage{josuamathheader}

\newcommand{\A}{\mathcal{A}}
\newcommand{\B}{\mathcal{A}}
\newcommand{\X}{\mathcal{X}}
\newcommand{\IP}{\mathbb{P}}
\newcommand{\F}{\mathbb{F}}
\begin{document}
\def\headheight{25pt}
\wtheolayout{1}
    \section*{Aufgabe 1}
    \begin{enumerate}[(a)]
        \item Sei $\A = \bigcap_{i \in I} \A_i$. Dann gilt
        \begin{enumerate}[(i)]
            \item  $\Omega \in \A_i \forall i \in I \implies \Omega \in \A$.
            \item $A\in \A, B\in \A \implies A, b \in \A_i \forall i \in I \implies A^c \in \A_i \forall i\in I \implies A^c \in \A$.
            \item $A_j \in \A \forall j \in J \implies A_j \in A_i \forall i \in I \forall j \in J \implies \bigcup_{j\in J} A_j \in \A_i \forall i \in I \implies \bigcup_{j\in J} A_j \in \A$.
        \end{enumerate}
        \item Sei $\Omega = \{1,2,3\}$, $\A = \{\emptyset, \{1\}, \{2,3\}, \{1,2,3\}\}$ und $\B = \{\emptyset, \{2\}, \{1,3\}, \{1,2,3\}\}$. Dann sind $\A$ und $\B$ $\sigma$-Algebren (wie man durch genaues Hinschauen sehr schnell sieht). Allerdings ist $\A \cup \B = \{\emptyset, \{1\}, \{2\}, \{1,3\}, \{2,3\}, \{1,2,3\}\}$ keine $\sigma$-Algebra, da $\{1,3\}\cap \{2,3\} = \{3\}$ nicht in $\A \cup \B$ enthalten ist.
        \item \begin{enumerate}[(i)]
            \item $f^{-1}(\Omega) = \X$.
            \item Sei $f^{-1}(A) \in f^{-1}(\A)$. Es gilt $f^{-1}(A) \in f^{-1}(\A) \implies A\in \A \implies A^c \in \A \implies f^{-1}(A^c) \in f^{-1}(\A) \implies f^{-1}(A)^c \in f^{-1}(\A)$.
            \item Seien $f^{-1}(A_n) \in f^{-1}(\A) \forall n \in \N$. Es gilt $f^{-1}(A_n) \in f^{-1}(\A) \forall n \in \N \implies A_n \in \A \forall n \in \N \implies \bigcup_{n \in \N} A_n \in \A \implies f^{-1}\left(\bigcup_{n\in \N} A_n\right) \in f^{-1}(\A) \implies \bigcup_{n\in \N} f^{-1}(A_n) \in f^{-1}(\A)$.
        \end{enumerate}
        \item \begin{enumerate}
            \item Wegen $\Omega \cap T = T$ liegt $T$ in $\A_{|T}$.
            \item Sei $A\in \A_{|T}$. Dann $\exists B \in \A$ mit $B \cap T = A$. Mit $B$ liegt auch $B^c$ in $\A$.
            Dann liegt aber auch $B^c \cap T = T \setminus B$ in $\A_{|T}$. Dabei ist $T\setminus B$ das Komplement von $B$ bezüglich $T$.
            \item Sei $A_n \in \A_{|T} \forall n \in \N$. Dann $\exists B_n \in \A$ mit $\B_n \cap T = A_n \forall n \in \N$. Da $\A$ eine $\sigma$-Algebra ist, liegt auch $\bigcup_{n\in \N} B_n$ in $\A$ und somit $\left(\bigcup_{n\in \N} B_n\right) \cap T = \bigcup_{n\in \N} (B_n \cap T) = \bigcup_{n\in \N} A_n$ in $\A_{|T}$. 
        \end{enumerate}
    \end{enumerate}
    \section*{Aufgabe 2}
    \begin{enumerate}[(a)]
        \item Zunächst gilt $A, B \in \A \implies B\setminus A \in \A$. Da $A \subset B$ dürfen wir $B$ als disjunkte Vereinigung schreiben, $\IP(B) = \IP((B\setminus A) \uplus A) = \IP(B\setminus A) + \IP(A) \geq \IP(A)$, woraus schon die Behauptung folgt.
        \item Es gilt $|\IP(A) - \IP(B)| = |\IP(A\setminus B) + \IP(A \cap B) - (\IP(B\setminus A) + \IP(B\cap A))| = |\IP(A\setminus B) - \IP(B\setminus A)| \leq \IP(A\setminus B) + \IP(B\setminus A) = \IP((A \setminus B) \cup (B\setminus A)) = \IP(A \triangle B)$.
        \item Wir greifen die Definition aus dem Hinweis auf, $B_n \coloneqq A_n \setminus \left(\bigcup_{k=1}^{n-1} A_k\right)$ und können damit $\bigcup_{n=1}^\infty A_n$ als disjunkte Vereinigung der $B_n$ darstellen,
        \[
            \bigcup_{n=1}^\infty A_n = \biguplus_{n = 1}^\infty B_n.
        \]
        Insbesondere erhalten wir also
        \[
            \IP\left(\bigcup_{n=1}^\infty A_n\right)  = \IP \left(\biguplus_{n = 1}^\infty B_n\right) = \sum_{n = 1}^{\infty} \IP(B_n).
        \]
        Nun können wir für jedes $\IP(B_N)$ die Monotonieeigenschaft ausnutzen, $\IP(B_n) \leq \IP(A_n)$, da $B_n \subset A_n$ und schließen also
        \[
            \IP\left(\bigcup_{n=1}^\infty A_n\right) \leq  \sum_{n = 1}^{\infty} \IP(A_n).
        \]
        \item Analog zu gerade eben schreiben wir
        \begin{align*}
            \IP\left(\bigcup_{n=1}^\infty A_i\right) &= \IP\left(\biguplus_{n=1}^\infty B_i\right)\\
            &= \sum_{n = 1}^{\infty} \IP(B_i)\\
            &= \lim\limits_{n \to \infty} \sum_{i = 1}^{n} \IP(B_i)\\
            &= \lim\limits_{n \to \infty} \IP\left(\biguplus_{i=1}^n B_i\right)\\
            &= \lim\limits_{n \to \infty} \IP(\bigcup_{i=1}^n A_n)\\
            &= \lim\limits_{n \to \infty} \IP(A_n),
        \end{align*}
        wobei wir im letzten Schritt benutzen, dass $A_{n-1}\subset A_n$.
    \end{enumerate}
    \section*{Aufgabe 3}
    \begin{enumerate}[(a)]
        \item Der Induktionsanfang ist offensichtlich wahr, $\IP(A_1) = (-1)^0 \cdot \IP(A_1)$. Gelte die Behauptung also für ein $n\in \N$. Dann folgern wir
        \begin{align*}
            \IP\left(\bigcup_{j=1}^{n+1} A_j\right) =& \IP\left(\bigcup_{j=1}^{n} A_j\right) + \IP(A_{n+1}) - \IP\left(\bigcup_{j=1}^{n} A_j \cap A_{n+1}\right)\\
            =& \sum_{j = 1}^{n} \left((-1)^{j-1} \cdot \sum_{\{k_1, \dots, k_n\} \subset \{1,\dots, n\}} \IP(A_{k_1} \cap \dots \cap A_{k_j})\right) + \IP(A_{n+1})\\
            &- \IP\left(\bigcup_{j=1}^{n} (A_j \cap A_{n+1})\right)\\
%
            =& \sum_{j = 1}^{n} \left((-1)^{j-1} \cdot \sum_{\{k_1, \dots, k_n\} \subset \{1,\dots, n\}} \IP(A_{k_1} \cap \dots \cap A_{k_j})\right) + \IP(A_{n+1})\\
            &- \sum_{j = 1}^{n} \left((-1)^{j-1} \cdot \sum_{\{k_1,\dots, k_j\} \subset \{1,\dots, n\}} \IP((A_{k_1} \cap A_{n+1}) \cap \dots \cap (A_{k_j} \cap A_{n+1}))\right)\\
%
            =& \sum_{j = 1}^{n} \left((-1)^{j-1} \cdot \sum_{\{k_1, \dots, k_n\} \subset \{1,\dots, n\}} \IP(A_{k_1} \cap \dots \cap A_{k_j})\right) + \IP(A_{n+1})\\
            &- \sum_{j = 1}^{n} \left((-1)^{j-1} \cdot \sum_{\{k_1, \dots, k_j\} \subset \{1,\dots, n\}} \IP(A_{k_1} \cap \dots \cap A_{k_j} \cap A_{n+1})\right)\\
%
            =& \sum_{j = 1}^{n} \left((-1)^{j-1} \cdot \sum_{\substack{\{k_1, \dots, k_n\} \subset \{1,\dots, n+1\}\\\forall i\colon k_i \neq n+1}} \IP(A_{k_1} \cap \dots \cap A_{k_j})\right) + \IP(A_{n+1})\\
            &+ \sum_{j = 2}^{n+1} \left((-1)^{j-1} \cdot \sum_{\substack{\{k_1, \dots, k_j\} \subset \{1,\dots, n+1\}\\\exists i\colon k_i = n+1}} \IP(A_{k_1} \cap \dots \cap A_{k_j})\right)\\
%
            =& \sum_{j = 1}^{n} \left((-1)^{j-1} \cdot \sum_{\substack{\{k_1, \dots, k_n\} \subset \{1,\dots, n+1\}\\\forall i\colon k_i \neq n+1}} \IP(A_{k_1} \cap \dots \cap A_{k_j})\right)\\
            &+ \sum_{j = 1}^{n+1} \left((-1)^{j-1} \cdot \sum_{\substack{\{k_1, \dots, k_j\} \subset \{1,\dots, n+1\}\\\exists i\colon k_i = n+1}} \IP(A_{k_1} \cap \dots \cap A_{k_j})\right)\\
        \end{align*}
        Für $j = n+1$ gilt $\{k_1,\dots, k_j\} = \{1,\dots, n+1\}$. Daher können wir die beiden Summen im letzten Schritt einfach zusammenfassen und erhalten
        \[
            \IP\left(\bigcup_{j=1}^{n+1} A_j\right) =  \sum_{j = 1}^{n+1} \left((-1)^{j-1} \cdot \sum_{\{k_1, \dots, k_n\} \subset \{1,\dots, n\}} \IP(A_{k_1} \cap \dots \cap A_{k_j})\right),
        \]
        was zu zeigen war.
        \item Wir stellen zunächst alle roten Marsmenschen in eine Reihe und nummerieren die Plätze von 1 bis $n$. Danach teilen wir die grünen Marsmenschen zufällig auf. Wir erhalten also eine Permutation von $n$ grünen Marsmenschen. Daher definieren wir $\Omega = \mathfrak{S}_n$ (Permutationsgruppe von $\{1,\dots, n\}$). Dann ist $A_i = \{\sigma \in \mathfrak{S}_n | \sigma(i) = i\}$. Die Wahrscheinlichkeit, dass $\{\sigma\}$ für ein $\sigma \in \mathfrak{S}_n$ eintritt, beträgt genau $\frac{1}{|\mathfrak{S}_n|} = \frac{1}{n!}$, da alle Permutationen gleichwahrscheinlich sein sollen. Wir definieren also $\IP(\{\sigma\}) = \frac{1}{n!}$. Da es sich um einen diskreten Raum handelt, lässt sich jedes Element von $\A \coloneqq \mathcal{P}(\Omega)$ als disjunkte Vereinigung von den Elementarereignissen $\{\sigma\}, \; \sigma \in \mathfrak{S}_n$ schreiben. Daher erhalten wir $\IP(A) = \frac{|A|}{n!}$. Die Wahrscheinlichkeit, dass mindestens ein ursprüngliches Paar miteinander tanzen wird, beträgt daher
        \begin{align*}
            \IP\left(\bigcup_{j=1}^{n} A_j\right) &= \sum_{j = 1}^{n} \left((-1)^{j-1} \cdot \sum_{\{k_1, \dots, k_n\} \subset \{1,\dots, n\}} \IP(A_{k_1} \cap \dots \cap A_{k_j})\right)\\
            &= \sum_{j = 1}^{n} \left((-1)^{j-1} \cdot \sum_{\{k_1, \dots, k_n\} \subset \{1,\dots, n\}} \IP(\{\sigma \in \mathfrak{S}_n| \sigma(k_j) = k_j \forall 1 \le j \le n\})\right)
            \intertext{Es sind also $j$ Werte der Permutation bereits festgelegt, daher dürfen die restlichen $n-j$ Werte beliebig gewählt werden. Dafür gibt es $(n-j)!$ verschiedene Möglichkeiten}
            &= \sum_{j = 1}^{n} \left((-1)^{j-1} \cdot \sum_{\{k_1, \dots, k_n\} \subset \{1,\dots, n\}} \frac{(n-j)!}{n!}\right)\\
            \intertext{Es ist allgemein bekannt, dass es $\binom{n}{j}$ verschiedene Möglichkeiten gibt, $j$ Elemente aus einer Menge von $n$ Elementen auszuwählen}
            &= \sum_{j = 1}^{n} \left((-1)^{j-1} \cdot \binom{n}{j} \frac{(n-j)!}{n!}\right)\\
            &= \sum_{j = 1}^{n} \left((-1)^{j-1} \cdot \frac{n!}{(n-j)! \cdot j!} \frac{(n-j)!}{n!}\right)\\
            &= \sum_{j = 1}^{n} \frac{(-1)^{j-1}}{j!}
        \end{align*}
        Der Grenzwert dieser Wahrscheinlichkeit für $n\to \infty$ beträgt
        \[
            - \sum_{j = 1}^{\infty} \frac{(-1)^j}{j!} = 1 - \sum_{j = 0}^{\infty} \frac{(-1)^j}{j!} = 1 - e^{-1} = 1 - \frac{1}{e}.
        \]
    \end{enumerate}
    \section*{Aufgabe 4}
    \begin{enumerate}[(a)]
        \item $\F(x_2) = \IP((-\infty, x_2]) = \IP((-\infty, x_1] \uplus (x_1, x_2]) = \F(x_1) + \IP((x_1, x_2]) \geq \F(x_1)$.
        \item $\lim\limits_{x \to -\infty} \F(x) = \lim\limits_{x \to -\infty} \IP((-\infty, x]) = \IP(\emptyset) = 0$\\
            $\lim\limits_{x \to \infty} \F(x) = \lim\limits_{x \to \infty} \IP((-\infty, x]) = \IP((-\infty, \infty)) = \IP(\R) = 1$
        \item $\lim\limits_{x_n \searrow x} \F(x_n) = \lim\limits_{x_n \searrow x} \IP((-\infty, x] \uplus (x, x_n]) = \F(x) + \lim\limits_{x_n \searrow x} \IP((x, x_n]) = \F(x) + \IP(\emptyset) = \F(x)$
        \item Sei $x$ eine Sprungstelle. Für $x_n \searrow x$ erhalten wir $\lim\limits_{x_n \to x} \F(x_n) = \IP((-\infty, x])$. Außerdem gilt $\lim\limits_{x_n \nearrow x} \F(x) = \IP(X < x) = \IP((-\infty, x))$. 
        Da $x$ eine Sprungstelle ist, gilt 
        \begin{align*}
            \lim\limits_{x_n \nearrow x} \F(x) &\neq \lim\limits_{x_n \searrow x} \F(x)\\
            \IP((-\infty, x)) &\neq \IP((-\infty, x]) = \IP((-\infty, x)) + \IP(\{x\})\\
            0 &\neq \IP(\{x\}) \implies x \text{ Atom}
        \end{align*}
        Nach Vorlesung besitzt ein Wahrscheinlichkeitsraum aber höchstens abzählbar viele Atome, woraus sofort die Behauptung folgt. 
    \end{enumerate}
\end{document}