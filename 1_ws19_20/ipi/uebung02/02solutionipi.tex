\documentclass{article}
\usepackage{../josuamathheader}
\newcommand{\p}{\operatorname{potenz}}
\newcommand{\q}{\operatorname{quadrat}}
\newcommand{\cond}{\operatorname{cond}}
\newcommand{\quadrat}[1]{#1 * #1}
\newcommand{\potenz}[2]{\cond (#2 == 1, #1, \cond( #2 \% 2 == 0, \q(\p(#1, #2/2)), #1 \cdot \p(#1, #2 - 1)))}
\newcommand{\syntax}[3]{\textbf{Syntax:} (#1)\begin{center}#2\ ::=\ #3\end{center}}

\begin{document}
\ipilayout{1}
\section*{Aufgabe 1}
\begin{enumerate}[(a)]
    \item 
    \hspace*{2cm}\begin{itemize}
        \item[Preorder] $+(*(:(8,3),7),-(4,*(+(1,5),2)))$
        \item[Inorder] $(((8:3)*7)+(4-((1+5)*2)))$    
        \item[Postorder] $(((8,3):, 7)*, (4, ((1, 5)+, 2)*)-)+$
    \end{itemize}
    \item Preorder und Postorder sind auch ohne Klammerung und Operator-Rangfolge eindeutig
    \item Ja, das ist einfach Postorder rückwärts gelesen
\end{enumerate}
\section*{Aufgabe 2}
\syntax{Zahl}{<zahl>}{[$+$|$-$]$\{0|1|2|3|4|5|6|7|8|9\}^{+}$}
\syntax{Index}{<index>}{$\{_0|_1|_2|_3|_4|_5|_6|_7|_8|_9\}^+$}
\syntax{Unterterm}{<unterterm>}{<zahl>\underline{x}<index>}
\syntax{Term}{<term>}{$\{\text{unterterm}\}^+$}
\syntax{Gleichung}{<gleichung>}{<term>\underline{=}<zahl>}
\syntax{Gleichungssystem}{<gleichungssystem>}{$\{\text{<gleichung>}\underline{\backslash n}\}^+$}
Nein, in der ersten Zeile des Gleichungssystems können beliebig viele Variablen $x_n$ vorkommen. Es gibt keine Möglichkeit, zu kontrollieren, dass höchstens $n$ Gleichungen darunter vorkommen.
\section*{Aufgabe 3}
\lstinputlisting[language=C++, caption={potenz.cc}]{potenz.cc}
applikativ
\begin{align*}
    &\p(2,4)\\
    &=\potenz{2}{4}\\
    &=\cond(0==0, \q(\p(2,2)), 2\cdot \p(2,3))\\
    &=\q(\p(2,2))\\
    &=\q(\potenz{2}{2})\\
    &=\q(\cond(0==0, \q(\p(2,1)), 2\cdot \p(2,1)))\\
    &=\q(\q(\p(2,1)))\\
    &=\q(\q(\potenz{2}{1}))\\
    &=\q(\q(2))
    =\q(\quadrat{2})
    =\q(4)
    =\quadrat{4}
    =16
\end{align*}
normal
\begin{align*}
    &\p(2,4)\\
    &=\potenz{2}{4}\\
    &=\cond(0==0, \q(\p(2,(4/2))), 2\cdot \p(2,4-1))\\
    &=\q(\p(2,(4/2)))\\
    &=\quadrat{\p(2,(4/2))}\\
    &=\quadrat{\potenz{2}{(4/2)}\\&}
     =\quadrat{\cond(0==0, \q(\p(2,(4/2)/2), 2\cdot\p(2,(4/2)-1)\\&}
     =\quadrat{\q(\p(2,(4/2)/2))}\\
    &=\quadrat{\quadrat{\p(2,(4/2)/2)}}\\
    &=\quadrat{\quadrat{\potenz{2}{(4/2)/2)}\\&}}
     =\quadrat{\quadrat{2}} =16 
\end{align*}
Wird der Code normal ausgewertet, so erhöht sich die Effizienz durch unseren Algorithmus überhaupt nicht. Wird der Code stattdessen applikativ ausgewertet, so erhält man die gewünschte Effizienzsteigerung.
\newpage
\section*{Aufgabe 4}
\lstinputlisting[language=C++, caption={vollkommenezahlen.cc}]{vollkommenezahlen.cc}
\end{document}
