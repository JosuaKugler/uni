\documentclass{article}
\usepackage{josuamathheader}
\newcommand{\mylim}{\lim\limits_{n\to \infty}}
\renewcommand{\epsilon}{\varepsilon}
\usepackage{graphicx}
\setlength{\headheight}{25pt}

\begin{document}
	\analayout{7}
	\section*{Aufgabe 1}
	\begin{enumerate}[(a)]
		\item $(a_k)_{k\in \N_0}$ mit $a_{k} = \frac{5}{3^{k+1}}$ ist offensichtlich eine Nullfolge. Daher ist die Reihe $\displaystyle \sum_{k = 0}^{\infty} (-1)^k \frac{5}{3^{k+1}}$ nach Leibniz-Kriterium konvergent. Daher ist $\displaystyle \sum_{k = 0}^{\infty} (-1)^k \frac{5}{3^{k+1}} = \frac{5}{3} \cdot \sum_{k=0}^{\infty} \left(\frac{-1}{3}\right)^k$. Mit der geometrischen Summenformel erhalten wir als Grenzwert der Reihe $\frac{5}{3}\cdot \frac{1}{1-\frac{-1}{3}} = \frac{5}{3} \cdot \frac{1}{\frac{4}{3}} = \frac{5}{4}$.
		\item Diese Folge lässt sich abschätzen durch $$\sum_{k=0}^{\infty} \frac{k^k}{(k+1)^k} = \sum_{k=0}^{\infty} \left(\frac{k}{(k+1)}\right)^k = \sum_{k=0}^{\infty} \left(1 + \frac{1}{k+1}\right)^k > \sum_{k=0}^{\infty} 1$$ und divergiert daher.
		\item Es gilt $$\sum_{k=1}^{\infty} \frac{1}{k(k+1)} = \sum_{k=1}^{\infty} \frac{k+1}{k(k+1)} - \frac{k}{k(k+1)} = \sum_{k=1}^{\infty} \frac{1}{k} - \frac{1}{k+1} \overset{\text{Teleskopsumme}}{=} 1$$
		\item Es gilt $a_n = \frac{3^{2k-2} \cdot 5^{1-k}}{2^{k-2}} = \frac{9^{k-1}}{5^{k-1} \cdot 2^{k-2}} = \frac{20}{9}\cdot \left(\frac{9}{10}\right)^{k}$. Folglich ist $\frac{|a_{n+1}|}{a_n} = \frac{|\frac{20}{9}\cdot \left(\frac{9}{10}\right)^{k+1}|}{|\frac{20}{9}\cdot \left(\frac{9}{10}\right)^{k}|} = \frac{9}{10} \leq 1$. Also ist die Reihe konvergent. Gemäß der geometrischen Summenformel ist der Grenzwert der Reihe $$\mylim \sum_{k=0}^{k = n} \frac{3^{2k-2} \cdot 5^{1-k}}{2^{k-2}} = \frac{20}{9}\cdot \sum_{k=0}^{k = n} \left(\frac{9}{10}\right)^{k} = \frac{20}{9} \mylim \cdot \frac{1-\left(\frac{9}{10}\right)^n}{1 - \frac{9}{10}} = \frac{20}{9} \cdot \frac{1}{\frac{1}{10}} = \frac{200}{9}$$
	\end{enumerate}
	\section*{Aufgabe 2}
	\begin{enumerate}[(a)]
		\item Annahme: $\mylim a_n \neq 0$, also $\mylim a_n \cdot n \geq k > 0$. Dann gilt $\forall \epsilon > 0: \exists N \in \N: \forall n > N:$
		\begin{align*}
			|a_n \cdot n -k| &< \epsilon\\
			|k-a_n\cdot n| &< \epsilon
			\intertext{Ist $k - a_n\cdot n$ positiv, so spielt der Betrag keine Rolle, ansonsten gilt die Ungleichung sowieso}
			k - a_n\cdot n &< \epsilon\\
			- a_n \cdot n &< \epsilon - k\\
			a_n \cdot n &< k - \epsilon\\
			a_n &< \frac{k - \epsilon}{n}
		\end{align*} 
		Es gilt also $\forall \epsilon > 0: \displaystyle \sum_{n= 0}^{\infty} a_n = \sum_{n = 0}^{N} a_n + \sum_{N+1}^{\infty} a_n > \sum_{n = 0}^{N} a_n + \sum_{N+1}^{\infty} \frac{k - \epsilon}{n}$
		Wähle $\epsilon < k$ und setze $k - \epsilon \coloneqq \alpha > 0$. Dann erhalten wir 
		$$\sum_{n= 0}^{\infty} a_n > \sum_{n = 0}^{N} a_n + \sum_{N+1}^{\infty} \frac{\alpha}{n}$$
		Die Reihe auf der rechten Seite der Ungleichung divergiert. Folglich divergiert auch $a_n$ und unsere Annahme kann nicht stimmen. Insgesamt folgt also $\mylim a_n\cdot n = 0$.
		\item Es gilt $$\sum_{n = 1}^{\infty} a_n = \sum_{n = 1}^{\infty} \sum_{k = 2^{n-1}}^{2^n-1}a_k.$$ Aufgrund der Monotonie ist $\sum_{k = 2^{n-1}}^{2^n-1}a_k \geq 2^{n-1} \cdot a_{2^n}$. Konvergiert also $\sum_{n = 1}^{\infty} a_n$, so auch $\sum_{n = 1}^{\infty} 2^{n-1} \cdot a_{2^n}$ und folglich auch $2 \cdot \sum_{n = 1}^{\infty} 2^{n-1} \cdot a_{2^n} = \sum_{n = 1}^{\infty} 2^n \cdot a_{2^n}$ und, da sich durch addieren endlich vieler Elemente nichts ändert, auch $\sum_{n = 0}^{\infty} 2^n \cdot a_{2^n}$. Ist hingegen $\sum_{n = 1}^{\infty} a_n$ divergent, so können wir mithilfe von $$\sum_{n = 1}^{\infty} a_n = \sum_{n = 1}^{\infty} \sum_{k = 2^{n-1}}^{2^n-1}a_k \overset{\text{Monotonie}}{\leq} \sum_{n = 1}^{\infty} 2^{n-1} \cdot a_{2^{n-1}} = \sum_{n = 0}^{\infty} 2^{n} \cdot a_{2^{n}}.$$ auf die Divergenz von $\sum_{n = 0}^{\infty} 2^{n} \cdot a_{2^{n}}$ schließen.
		\item Betrachte zunächst den Fall $\liminf\limits_{n\to \infty} a_n = k$.
		Sei $\epsilon_1 > 0$. Dann existiert ein $n_1\in \N$ mit $\forall n > n_1: \frac{a_{n+1}}{a_n} > k -\epsilon_1$.
		Folglich gilt $\forall n > n_1: a_n \geq (k-\epsilon_1)^{n - n_1}\cdot a_{n_1} = (k-\epsilon_1)^{n - n_1}\cdot a_{n_1} \cdot\frac{(k-\epsilon_1)^{n-n_1}}{(k-\epsilon_1)^{n-n_1}} = (k-\epsilon_1)^n \cdot \underbrace{\frac{a_{n_1}}{(k-\epsilon_1)^{n_1}}}_{C}$.
		Also ist $\sqrt[n]{a_n} \geq (k-\epsilon_1)\cdot \sqrt[n]{C}$ und mit $\mylim (k-\epsilon_1)\cdot \sqrt[n]{C} = (k-\epsilon_1)$ erhalten wir $\mylim \sqrt[n]{a_n} \geq (k-\epsilon_1)$.
		Sei nun außerdem $\epsilon_2 > 0$. Dann $\exists n_2 \in \N: \forall n > n_2: \sqrt[n]{a_n} \geq (k-\epsilon_1)-\epsilon_2$ und daher $\liminf\limits_{n\to \infty} \sqrt[n]{a_n} \geq k = \liminf\limits_{n\to \infty} \frac{a_{n+1}}{a_n}$. Im Fall $\liminf\limits_{n\to \infty} a_n = -\infty$ ist die Ungleichung offensichtlich. Gilt $\liminf\limits_{n\to \infty} \frac{a_{n+1}}{a_n}= \infty$, so wächst die Folge $\frac{a_{n+1}}{a_n}$ und damit auch $\sqrt[n]{a_n}$ unbeschränkt. Folglich ist $\liminf\limits_{n\to \infty} \sqrt[n]{a_n} = \infty$ und die Ungleichung ist auch in diesem Fall erfüllt.
		Der Beweis für $\limsup\limits_{n\to \infty} \sqrt[n]{a_n} \leq \limsup\limits_{n\to \infty} \frac{a_{n+1}}{a_n}$ erfolgt völlig analog. $\liminf\limits_{n\to \infty} \sqrt[n]{a_n} \leq \limsup\limits_{n\to \infty} \sqrt[n]{a_n}$ ist nach Konstruktion klar.
		Das Wurzelkriterium ist daher stärker als das Quotientenkriterium, d.h. es kann Reihen geben, für die man durch das Quotientenkriterium nichts über das Konvergenzverhalten aussagen kann, mit dem Wurzelkriterium aber schon.
	\end{enumerate}

	\section*{Aufgabe 3}
	\begin{enumerate}[(a)]
		\item $\sum_{k= 1}^{\infty} \frac{1}{\sqrt{k(k+1)}} > \sum_{k= 1}^{\infty} \frac{1}{\sqrt{(k+1)\cdot (k+1)}} = \sum_{k = 1}^{\infty} \frac{1}{k+1} = -\frac{1}{k} + \sum_{k = 1}^{\infty} \frac{1}{k}$. Diese Reihe ist divergent, da $\sum_{k = 1}^{\infty} \frac{1}{k}$ divergent ist. Offensichtlich ist sie daher auch absolut divergent.
		\item Es gilt $$\frac{|a_{n+1}|}{|a_n|} = \left|\frac{\frac{k}{2^{k+1}}}{\frac{k-1}{2^k}}\right| = \frac{k}{2(k-1)}.$$ Für $k > 3$ ist $\frac{k}{2(k-1)} \leq \frac{3}{4} < 1$. Daher ist die Reihe absolut konvergent.
		\item Es gilt $\mylim \left|\sqrt[n]{n\cdot \left(\frac{n-3}{7n}\right)}\right| = \mylim \sqrt[n]{n}\cdot \left|\frac{1}{7} - \frac{3}{7n}\right| \overset{Lemma 2.5}{=} \frac{1}{7}$. Also existiert ein $n_0\in \N: \forall n > n_0: \left|\sqrt[n]{n\cdot \left(\frac{n-3}{7n}\right)}\right| < \frac{1}{2} < 1$ und die Folge ist nach Wurzelkriterium absolut konvergent.
		\item Es gilt $\left|\frac{(k!)^2}{(k+1!)^2}\right| = \frac{1}{(k+1)^2} \overset{k > 0}{<} \frac{1}{2}$. Also gilt nach Quotientenkriterium, dass die Folge absolut konvergiert.
		\item Es gilt $$\sum_{k= 1}^{\infty}\left| (-1)^k \left(\sqrt[k]{k} - 1\right) \right| = \sum_{k= 1}^{\infty} k^{\frac{1}{k}} -1 = \sum_{k = 1}^{\infty} e^{\ln(k) \cdot \frac{1}{k}} - 1 \geq \sum_{k = 1}^{\infty} 1 + \frac{1}{k}\ln\left(k\right) - 1 = \sum_{k = 1}^{\infty} \frac{1}{k}\ln\left(k\right)$$ Die Folge $\frac{1}{k} \cdot \ln\left(\frac{1}{k}\right)$ ist monoton fallend und, da $k\cdot \frac{1}{k}\ln(k) = \ln(k)$ keine Nullfolge ist, muss die Reihe laut Aufgabe 2.1 divergieren.
	\end{enumerate}

	\section*{Aufgabe 4}
	Sei $\displaystyle a_0 + \sum_{k=1}^{\infty} a_k - a_{k-1}$ nicht konvergent, aber absolut konvergent: $\displaystyle a_0 + \sum_{k=1}^{\infty} |a_k - a_{k-1}|$.
	Zunächst ist die Folge der Partialsummen $(a_n)_{n\in \N_0}$ nicht konvergent.
	Z.Z.: $\displaystyle  \forall \epsilon > 0: \exists N\in \N:\sum_{k=N}^{\infty} |a_k -a_{k-1}| < \epsilon$.
	\begin{proof}
		Sei $S\in \R$ eine obere Schranke für die Reihe. Wähle $n_S = \lceil\frac{S}{\epsilon}\rceil + 1$.
		Angenommen, die zu zeigende Aussage ist falsch. Dann $\exists \epsilon > 0: \forall N\in \N: \sum_{k=N}^{\infty} |a_k -a_{k-1}| > \epsilon$. Insbesondere ist also $\sum_{k=0}^{\infty} |a_k -a_{k-1}| > \epsilon$. Betrachte nun den kleinsten Index $N_1\in \N: \sum_{k=0}^{N_1} |a_k -a_{k-1}| > \epsilon$. Nun ist aber auch $\sum_{k=N_1+1}^{\infty} |a_k -a_{k-1}| > \epsilon$. Wähle also analog $N_2\in \N: \sum_{k=N_1 + 1}^{N_2} |a_k -a_{k-1}| > \epsilon$.
		Verfahre so weiter bis $N_{n_S}$. Dann ist $\sum_{k=0}^{N_{n_S}} |a_k -a_{k-1}| > n_S \cdot \epsilon > S$. Das ist aber ein Widerspruch zur Schrankeneigenschaft von $S$. Folglich ist unsere Annahme falsch und die Aussage ist wahr. 
	\end{proof}
	Z.Z.: $(a_n)_{n\in \N_0}$ ist eine Cauchy-Folge.
	\begin{proof}
		Sei $\epsilon > 0$. Dann existiert gemäß unserer Aussage oben ein $N\in \N$ sodass $\displaystyle \sum_{k=N}^{\infty} |a_k -a_{k-1}| < \epsilon$. 
		Insbesondere gilt für alle $n, m > N$, o.B.d.A. $n > m:$ gemäß Dreiecksungleichung $|a_n-a_m| < \sum_{k = m-1}^n|a_k - a_{k-1}| < \sum_{m-1}^{\infty} |a_k - a_{k-1}| < \sum_{N}^{\infty} |a_k - a_{k-1}| < \epsilon$.
	\end{proof}
	Folglich ist, wenn 
\end{document}