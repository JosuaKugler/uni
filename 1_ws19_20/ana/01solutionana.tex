\documentclass{article}
\usepackage{josuamathheader}

\begin{document}
	\analayout{1}
	\section{Aufgabe}
	Die $n$-te Zahl der Fibonacci-Folge wird rekursiv durch
	\[F_n = F_{n-1} + F_{n-2},\quad \text{für $n\geq 3$}\]
	mit den Anfangswerten 
	\[F_1 = F_2 = 1\]
	definiert. Beweisen Sie mit Hilfe vollständiger Induktion
	\begin{enumerate}[(a)]
		\item $1 + \displaystyle \sum_{k=1}^{n-1} F_k = F_{n+1},\quad n\in \mathbb{N}, n\geq 2$
		\induktion{$n=2$: $1 + \sum_{k=1}^{1} F_k = 1 + F_1 = F_2$}
		{Für ein beliebiges, aber festes $n\in \mathbb{N}$ mit $n\geq 2$ gelte $1 + \displaystyle \sum_{k=1}^{n-1} F_k = F_{n+1}$}
		{$n\to n+1$:\quad $1 + \sum_{k=1}^{n}F_k = 1 + \sum_{k=1}^{n-1}F_k + F_n = F_{n+1} + F_n = F_{n+2}$}
		\item $\displaystyle F_{n-1}F_{n+1} = F_n^2+1,\quad n\in \mathbb{N}, n \text{ gerade}$
		\induktion{$n=2$:\quad $F_1F_3 = 1\cdot 2 = 1^2 + 1 = F_2^2+1$}
		{Für ein beliebiges, aber festes $n\in \mathbb{N}$, $n\text{ gerade}$, gelte $\displaystyle F_{n-1}F_{n+1} = F_n^2+1$}
		{$n \to n+2$:\begin{align*}
				F_{n+1}F_{n+3} &= (F_{n+2} - F_n)(F_{n+2}+F_{n+1})\\
				&= F_{n+2}^2+F_{n+2}F_{n+1} - F_{n+2}F_n - F_{n+1}F_n\\
				&= F_{n+2}^2+F_{n+1}^2 + F_{n+1}F_n - F_{n+2}F_n - F_{n+1}F_n\\
				&= F_{n+2}^2+F_{n+1}F_n + F_{n+1}F_{n-1} - F_{n+2}F_n\\
				&\overset{I.A.}{=} F_{n+2}^2+F_{n+1}F_n + F_nF_n + 1 - F_{n+2}F_n\\
				&= F_{n+2}^2 + 1 + F_n(F_{n+1} + F_n - F_{n+2})\\
				&= F_{n+2}^2 + 1
			\end{align*}}
		\item $\displaystyle F_{2n+1} = F_{n+1}^2 + F_n^2, \quad n\in \mathbb{N}$
		\induktion{	$n=1$:\quad $F_{3} = 2 = 1^2+1 ^2 = F_2^2+ F_1^2$\\
					$F_{5} = 5 = 2^2+ 1^2= F_3^2+F_2^2$}
		{Für ein beliebiges, aber festes $n\in \mathbb{N}$, gelte $\displaystyle F_{2n+1} = F_{n+1}^2+F_{n}^2$ und $\displaystyle F_{2n+3} = F_{n+2}^2+F_{n+1}^2$}
		{$n\to n+1$: $F_{2n+3} = F_{n+2}^2 + F_{n+1}^2$ folgt direkt aus der Induktionsannahme.
			\begin{align*}
			F_{2n+5} &= F_{2n+4} + F_{2n+3}\\
			&= F_{2n+3} + F_{2n+3} + F_{2n+2}\\
			&= F_{2n+3} + F_{2n+3} + F_{2n+3} - F_{2n+1}\\
			&\overset{I.A.}{=} 3(F_{n+2}^2 + F_{n+1}^2) - F_{n+1}^2-F_n^2\\
			&=2F_{n+2}^2 + (F_{n+1} +F_n)^2 + 2F_{n+1}^2 - F_n^2\\
			&=2F_{n+2}^2 + F_{n+1}^2 + 2F_{n+1}F_{n} + 2 F_{n+1}^2 + F_{n}^2 - F_n^2\\
			&=2F_{n+2}^2 + F_{n+1}^2 + 2F_{n+1}(F_{n} + F_{n+1})\\
			&=F_{n+2}^2 + F_{n+2}^2 + 2F_{n+2}F_{n+1} + F_{n+1}^2\\
			&=F_{n+2}^2 + F_{n+3}^2
		\end{align*}}
	\end{enumerate}
	\section{Aufgabe}
	Zeigen Sie:
	\begin{enumerate}[(a)]
		\item $\displaystyle \sum_{k=1}^{n}(2k-1) = n^2,\quad \forall n\in \mathbb{N}$ (Achtung, geht nicht ab $k=0$)
		\induktion{$n=1$:\quad $\sum_{k=1}^{1} (2k-1) = 1 = 1^2$}
		{Für ein beliebiges, aber festes $n\in \mathbb{N}$, gelte $\displaystyle \sum_{k=1}^{n}(2k-1) = n^2$}
		{$n\to n+1$: \[\sum_{k=1}^{n+1} (2k-1)= \sum_{k=1}^{n} + (2(n+1)-1 \overset{I.A.}{=} n^2 + 2n + 1 = (n+1)^2 \]}
		\item $\displaystyle\sum_{k=0}^{n}\binom{n}{k} = 2^n,\quad \forall n\in \mathbb{N}$
		\begin{proof}
			Laut dem binomischen Lehrsatz gilt:
			$(1+1)^n = \sum_{k=0}^{n}\binom{n}{k} 1 ^k \cdot 1 ^{n-k} = \sum_{k=0}^{n}\binom{n}{k}$
		\end{proof}
		\item $\displaystyle\sum_{k=1}^{n}\frac{1}{k(k+1)} = 1-\frac{1}{n+1},\quad \forall n\in \mathbb{N}$
		\begin{proof}
			\begin{align*}
				\sum_{k=1}^{n}\frac{1}{k(k+1)}
				&= \sum_{k=1}^{n}\frac{k+1-k}{k(k+1)} \\
				&=\sum_{k=1}^{n}\left(\frac{k+1}{k(k+1)} -
				\frac{k}{k(k+1)}\right) \\
				&= \sum_{k=1}^{n} \left(\frac{1}{k} - \frac{1}{k+1}\right)
				\intertext{Dies ist eine Teleskopsumme und daher}
				&= 1 - \frac{1}{n+1}
			\end{align*}
		\end{proof}
		\item $\displaystyle\binom{l+1}{k+1} = \sum_{m=k}^{l}\binom{m}{k}\quad \forall k,l \in \mathbb{N}: k\leq l$
		\induktion{$l=1$: $$\binom{2}{k+1} =\binom{2}{2} = 1 =  \binom{1}{1} = \sum_{m=k}^{1}\binom{m}{k}$$}
		{Für ein beliebiges, aber festes $l \in \mathbb{N}$ gilt $$\displaystyle\binom{l+1}{k+1} = \sum_{m=k}^{l}\binom{m}{k}\quad \forall k \in \mathbb{N}: k\leq l$$}
		{$l\to l+1$: \begin{align*}
				\binom{l+2}{k+1} &= 
				\begin{cases}
					1 = \binom{l+1}{l+1} = \sum_{m=k}^{l+1}\binom{m}{k}&k= l+1\quad \checkmark\\
					\binom{l+1}{k} + \binom{l+1}{k+1} &k \leq l
				\end{cases}\\
				&\overset{I.A.}{=} \binom{l+1}{k} + \sum_{m=k}^{l}\binom{m}{k}\\
				&=\sum_{m=k}^{l+1}\binom{m}{k}
		\end{align*}}
\end{enumerate}
\section{Aufgabe}
	Gegeben die Buchstaben $a,b$ formen wir \glqq Worte\grqq\ $W$ wie folgt:
	\[W_1\coloneqq a, \quad W_2\coloneqq b,\quad W_{n+1} \coloneqq W_nW_{n-1}, \quad n\in \mathbb{N}, n\geq 2\]
	Das heißt es gilt beispielsweise
	\[W_3 = ba, \quad W_4 = bab\]
	Zeigen Sie mit Hilfe von vollständiger Induktion:
	\begin{enumerate}[(a)]
		\item $W_n$ besteht aus $F_n$ Buchstaben, $n\in \mathbb{N}$. ($F_n = n$-te Fibonacci-Zahl)
		\begin{definition}
			Wir bezeichnen mit $L(W_n)$ die Länge des Wortes $W_n$.
		\end{definition}
		\induktion{$n=2$:\quad $L(W_1) = L(a) = 1  = F_1$. $L(W_2) = L(b) = 1 = F_2$}{Für ein beliebiges, aber festes $n\in N$ sei $L(W_{n-1}) = F_{n-1}$ und  $L(W_n) = F_n$.}
		{$n\to n+1$: Aus der Induktionsannahme folgt unmittelbar $L(W_n) = F_n$. $L(W_{n+1}) = L(W_{n+1}) = L(W_{n}W_{n-1}) = L(W_n) + L(W_{n -1}) \overset{I.A.}{=} F_n + F_{n-1} = F_{n+1}$}
		\item Die Buchstabenkombination $aa$ ist kein Bestandteil des Wortes $W_n$ für alle $n\in \mathbb{N}$.
		\begin{definition}
			$*$ bezeichne eine Abfolge von $a$s und $b$s beliebiger Reihenfolge und Länge (auch Länge 0 ist zugelassen).
		\end{definition}
		\begin{lemma}
			Alle $W_{2n}$ beginnen und enden mit $b$.
		\end{lemma}
		\induktion{$n=1$:\quad $W_{2} = b$}{Für ein beliebiges, aber festes $n\in \mathbb{N}$ sei $W_{2n} = b*b$.}
		{$n\to n+1$. $W_{2n+2} = W_{2n+1}W_{2n} = W_{2n}W{2n-1}W_{2n} = b*bW_{2n-1}b*b = b*b$.}
		Nun zeigen wir die eigentliche Behauptung.
		\induktion{$n=2$:\quad $aa$ ist offensichtlich weder in $W_1$ noch in $W_2$ enthalten}{Für ein beliebiges, aber festes $n\in \mathbb{N}$ gelte, dass $aa$ weder in $W_{n-1}$ noch in $W_n$ enthalten ist.}{$n\to n+1$:\quad Dass $aa$ nicht in $W_n$ enthalten ist, folgt sofort aus der Induktionsannahme. $W_{n+1} = W_{n}W_{n-1}$. Weder $W_n$ noch $W_{n-1}$ enthalten $aa$. Die einzige Möglichkeit, so dass $aa$ in $W_nW_{n-1}$ vorkommen kann, ist dass der letzte Buchstabe von $W_n$ und der erste Buchstabe von $W_{n-1}$ gleich $a$ sind. Entweder $n$ oder $n-1$ sind aber gerade und fangen nicht nur mit $b$ an, sondern hören auch auf $b$ auf. Also folgt, dass entweder der letzte Buchstabe von $W_n$ oder der erste Buchstabe von $W_{n-1}$ gleich $b$ sind, $W_{n+1}$ enthält $aa$ daher nicht. }
	\end{enumerate}
\section{Aufgabe}
Seien $n, k\in \mathbb{N}$. Zeigen Sie, dass die Anzahl $A(n)$ aller $k-$Tupel $(a_1,\dots, a_k)\in \mathbb{N}^k$ mit\\
$(*)\quad 1\leq a_1\leq a_2\leq \dots \leq a_k\leq n$\\
gegeben ist durch
\[A(n) = \binom{n+k-1}{k}.\]
Formal ausgedrückt, ist also zu zeigen, dass
\[\binom{n+k-1}{k} = \Bigg|\Bigg\{(a_1,\dots, a_k)\in \mathbb{N}^k\Bigg|1\leq a_1\leq a_2\leq\dots\leq a_k\leq n\Bigg\}\Bigg|\]
\begin{definition}
	$B(n,k) \coloneqq \Bigg|\Bigg\{(a_1,\dots, a_k)\in \mathbb{N}^k\Bigg|1\leq a_1\leq a_2\leq\dots\leq a_k\leq n\Bigg\}\Bigg|$
\end{definition}
\begin{lemma}
	\[\sum_{l=1}^{l=n}\binom{l+k-1}{k} =\binom{n+k}{k+1} \]
\end{lemma}
\induktion{$n=1$: $\sum_{l=1}^{l=1}\binom{l+k-1}{k} = \binom{k}{k} = \binom{1+k}{k+1} = \binom{n+k}{k+1}$}
{Für ein beliebiges, aber festes $n \in \mathbb{N}$ gelte $\sum_{l=1}^{l=n}\binom{l+k-1}{k} =\binom{n+k}{k+1}$}
{$n\to n+1$: $$\sum_{l=1}^{l=n+1}\binom{l+k-1}{k} = \sum_{l=1}^{l=n}\binom{l+k-1}{k} + \binom{n+k}{k} \overset{I.A.}{=}\binom{n+k}{k+1} +  \binom{n+k}{k} = \binom{n+k+1}{k+1}$$}
\noindent Z.Z.: $B(n,k) = \binom{n+k-1}{k}$
\induktion{$k=1$: Es gibt nur ein Element in unserem Tupel, dieses kann folglich $n = \binom{n}{1} = \binom{n+k-1}{k}$ Werte annehmen.}
{Für ein beliebiges, aber festes $k\in \mathbb{N}$ sei $\forall n\in \mathbb{N}: B(n,k) = \binom{n+k-1}{k}$}
{$k\to k+1$:\quad Setzt man $a_{k+1} = n$, so gibt es $A(n,k)$ Möglichkeiten, die restlichen $a_i$ ($1\leq i\leq k$) zu wählen. Setzt man $a_{k+1} = n-1$, so gibt es $B(n-1,k)$ Möglichkeiten, die restlichen $a_i$ ($1\leq i\leq k$) zu wählen. Allgemein gilt: Setzt man $a_{k+1} = l\in \mathbb{N}$ mit $l\leq n$, so gibt es $B(l, k)$ Möglichkeiten, die restlichen $a_i$ ($1\leq i\leq k$) zu wählen. Da für zwei verschiedene $l$ $a_k$ ebenfalls verschieden ist, überschneiden sich keine dieser Möglichkeiten, es gilt:
\[B(n,k+1) = B(n,k) + B(n-1, k) + \dots + B(1,k) = \sum_{l=1}^{l=n}B(l, k) = \sum_{l=1}^{l=n}\binom{l+k-1}{k}\overset{Lemma}{=} \binom{n+k}{k+1}\]
}
\end{document}