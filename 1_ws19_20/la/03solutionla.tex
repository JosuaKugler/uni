\documentclass{article}
\usepackage{josuamathheader}
\begin{document}
	\lalayout{3}
	\section*{Aufgabe 1}
	\begin{enumerate}[a)]
		\item \[(3,2,1)\cdot(4,5) = \begin{pmatrix}1&2&3&4&5\\3&1&2&4&5\end{pmatrix}\cdot \begin{pmatrix}1&2&3&4&5\\1&2&3&5&4\end{pmatrix} = \begin{pmatrix}1&2&3&4&5\\3&1&2&5&4\end{pmatrix}\]
		\[(5,4,3,2,1)\cdot(3,5) = \begin{pmatrix}1&2&3&4&5\\5&1&2&3&4\end{pmatrix}\cdot \begin{pmatrix}1&2&3&4&5\\1&2&5&4&3\end{pmatrix} = \begin{pmatrix}1&2&3&4&5\\3&1&2&5&4\end{pmatrix}\]
		\item Z.Z.: Sind $\sigma = (a_1,\dots,a_d), \tau = (b_1,\dots,b_e)$ zwei gleiche Zyklen, dann ist $e=d$.\\
		\begin{proof} Zwei gleiche Abbildungen haben die gleiche Ursprungs-und Zielmenge, es ist also $\sigma \in \mathfrak{S}_n \implies \tau\in \mathfrak{S}_n$.
		Sein nun $N =  \{k\in \N|1 \leq k\leq n\}$,  $M_\sigma = \{k \in N|\sigma(k)\neq k\}$ und $M_\tau = \{k \in N|\tau(k)\neq k\}$.
		Aus $\tau = \sigma$ folgt $M_\tau = M_\sigma$.
		Ferner gilt $\sigma(a_i) = a_j$ mit $1\leq i, j\leq d, i\neq j$ und daraus folgt mit $\forall 1\leq i, j\leq d, i \neq j: a_i\neq a_j$ sofort $\sigma(a_i)\neq a_i$. Für alle anderen $k\in N$ ist aber nach Definition des Zyklus $\sigma(k) = k$. Daraus folgt $M_\sigma = \{a_i|1\leq i\leq d\}$ und schließlich $\# M_\sigma = d$. Analog erhalten wir $\# M_\tau = e$. Mit $M_\tau = M_\sigma$ folgt sofort $e = d$.\end{proof}
		\item Z.Z.: Für einen Zyklus $\sigma$ der Länge $d$ gibt es genau $d$ Darstellungen $\sigma = (a_1, \dots, a_d)$.\\
		\begin{proof} Im folgenden bezeichne $\forall k\in \N: R(k)$ den Rest von $k$ modulo $d$, insbesondere ist also $R(d+1) = 1$.
			\begin{itemize}
				\item Z.Z.: Es gibt $d$ verschiedene Darstellungen.\\
				Bew.: Wir bezeichnen mit $(a_1, a_2, \dots, a_d)$ eine Darstellung von $\sigma$ (es existiert auf jeden Fall eine).  Alle Darstellungen $b_1, \dots b_d$ für die gilt:
				$\forall 1\leq i, j, k\leq d: b_i = a_{R(i+k)} \implies b_{j} = a_{R(j+k)}$. Es gibt genau $d$ verschiedene solche $d$-Tupel. Man erhält sie, wenn man $k=1,2,\dots, d$ wählt. Sobald $k= d+1$ ist die Darstellung äquivalent zu der Darstellung für $R(k) = R(d+1) = 1$.\\
				Damit aber alle diese $d$-Tupel den gleichen Zyklus darstellen, muss gelten $\forall 1\leq i\leq d: \sigma(a_i) = a_{R(i+1)}$. (Im Fall $i = d$ wird das zu $\sigma(a_d) = a_{R(d+1)} = a_1$).
				Es gilt: $$\sigma(a_i) \overset{\text{Es existiert stets ein geeignetes }k}{=} \sigma(b_{R(i+k)}) = b_{R(i+1+k)} = a_{R(i+1)}$$.
				\item Es gibt höchstens $d$ verschiedene Darstellungen.\\
				Beweis durch Widerspruch: Annahme: Es gibt zusätzlich zu den $d$ oben beschriebenen Darstellungen noch mindestens eine weitere Darstellung $(b_1, \dots, b_d)$. Da diese Darstellung nicht mehr der obigen Bedingung genügen kann, müssen \obda $b_i = a_{R(i+k)}, b{i+1}\neq a_{R(i+1+k)}$ existieren. Damit erhalten wir allerdings $\sigma(a_i) = \sigma(b_{R(i+k)}) = b_{R(i+1+k)} \neq a_{R(i+1)}$, es folgt $\sigma \neq (b_1, \dots, b_d)$.
			\end{itemize}
		\end{proof}
		\item Z.Z. Jede Permutation $\sigma = \begin{pmatrix}1 & 2&\dots &n\\ \sigma_1 &\sigma_2 &\dots &\sigma_n\end{pmatrix}\in \mathfrak{S}_n$ lässt sich als Produkt von Zyklen der Länge 2 schreiben.
		\induktion{siehe Aufgabenblatt}{Jedes Element der $\mathfrak{S}_n$ lässt sich  als Produkt von Zyklen der Länge 2 schreiben.}
		{Wir betrachten die Permutation $\sigma = \begin{pmatrix}1 & 2&\dots &n&n+1\\ \sigma_1 &\sigma_2 &\dots &\sigma_n&\sigma_{n+1} \end{pmatrix} \in \mathfrak{S}_{n+1}$.\\
		Fall 1: $n+1 = \sigma_{n+1}$.\\ Die Permutation $\begin{pmatrix}1 & 2&\dots &n\\ \sigma_1 &\sigma_2 &\dots &\sigma_n\end{pmatrix}\in \mathfrak{S}_n$ lässt sich nach Induktionsvoraussetzung als Produkt $\prod_{i=1}^{k}p_i$ mit $k\in \N$ und $p_i\in \mathfrak{S}_n$ schreiben, wobei jedes $p_i$ ein Zyklus der Länge zwei ist. Sei $f: \mathfrak{S}_n \to \mathfrak{S}_{n+1}, z \mapsto z$ ein Homomorphismus. Dann ist $f(p)_i$ ebenfalls ein Zyklus der Länge zwei und $\prod_{i=1}^{k}f(p_i) = \begin{pmatrix}1 & 2&\dots &n&n+1\\ \sigma_1 &\sigma_2 &\dots &\sigma_n&n+1 \end{pmatrix} \in \mathfrak{S}_{n+1} = \sigma$ ist das gesuchte Produkt.\\
		Fall 2: $\sigma_{n+1} \neq n+1$.\\
		Der letze Zyklus in dem zu konstruierenden Produkt von Zweierzyklen  sei $z_1 = (n+1, \sigma_{n+1})\in \mathfrak{S}_{n+1}$.
		Es existiert genau ein $k$, sodass $\sigma(k) = n+1$. (Das wird sofort aus unserer Darstellungsweise einer Permutation ersichtlich). Dieses $k$ ist außerdem $< n+1$. Zudem ist $\sigma_{n+1} < n+1$.
		Nach Induktionsvoraussetzung wissen wir, dass 
		$\begin{pmatrix}1 & 2&\dots&k &\dots &n\\ 1 &2 &\dots&\sigma_{n+1}&\dots &n\end{pmatrix} = \prod_{i=1}^{k}p_i$ mit $k\in \N$ und $p_i$ ein Zyklus der Länge zwei aus $\mathfrak{S}_n$.
		Analog zu Fall 1 erhalten wir $z_2 = \prod_{i=1}^{k}f(p_i) = \begin{pmatrix}1 & 2&\dots&k&\dots &n&n+1\\ \sigma_1 &\sigma_2 &\dots&\sigma_{n+1}&\dots &\sigma_n&n+1 \end{pmatrix} \in \mathfrak{S}_{n+1}$.
		Das gesuchte Produkt ist $z_2\cdot z_1 = \begin{pmatrix}1 & 2&\dots&k&\dots &n&n+1\\ \sigma_1 &\sigma_2 &\dots&\sigma_{n+1}&\dots &\sigma_n&n+1 \end{pmatrix}\cdot \begin{pmatrix}1 & 2&\dots&\sigma_{n+1} &\dots &n&n+1\\ 1 &2 &\dots&n+1&\dots &n&\sigma_{n+1} \end{pmatrix}$\\ $= \begin{pmatrix}1 & 2&\dots&k&\dots &n&n+1\\ \sigma_1 &\sigma_2 &\dots&n+1&\dots &\sigma_n&\sigma_{n+1} \end{pmatrix}$. Da $\sigma(k) = n+1$, ist $n+1$ = $\sigma_k$.\\
		$z_2\cdot z_1=\begin{pmatrix}1 & 2&\dots&k&\dots &n&n+1\\ \sigma_1 &\sigma_2 &\dots&\sigma_k&\dots &\sigma_n&\sigma_{n+1} \end{pmatrix} = \sigma$
		}
	\end{enumerate}
	\section*{Aufgabe 2}
	Nach Bemerkung 1.28 im Skript ist $gH = \{g'\in G|g' = g*h, h\in H\}$.
	\begin{itemize}
		\item Z.Z.: Es existiert eine bijektive Abbildung $f:H \to gH, h \to g*h$.
		\begin{proof}
			Die Abbildung ist offensichtlich wohldefiniert. Angenommen, $f$ wäre nicht surjektiv. Dann gäbe es ein $g' = g*h \in gH$ und $h\in H$, sodass es kein $h\in H$ mit $g*h = g' \in gH$ gäbe. $\lightning$.
			Angenommen, die Abbildung wäre nicht injektiv, dann gäbe es $g, g'\in G$ \textbf{mit $h \neq h'$}, sodass $f(h)= f(h') \implies g*h = g*h'$. Sei $g^{-1}$ das Inverse zu $g$ (es existiert laut den Gruppenaxiomen). Dann ist $g^{-1}*g*h = g^{-1} * g *h'$ und damit $h= h'\lightning$.
		\end{proof}
		Da es eine bijektive Abbildung von $H \to gH$ gibt, ist $\# H = \# gH$.
		\item Z.Z.: $\# G = \#H\cdot\#(G/H)$.
		\begin{proof}
			\begin{align*}
				G &= \dot{\bigcup\limits_{M\in G/H}}&&\text{folgt aus den Axiomen für Äquivalenzrelationen}\\
				\implies \#G &= \sum_{M\in G/H} \# M&&\# M = \# gH = \# H\quad  \forall M\in G/H\\
				\#G &= \sum_{M\in G/H} \# H &&\text{Diese Summe addiert $\#(G/H)$ mal $\# H$}\\
				\# G &= \#(G/H) \cdot \# H
			\end{align*}
		\end{proof} 
	\end{itemize}
	\section*{Aufgabe 3}
	\begin{enumerate}[a)]
		\item \begin{proof}
			\begin{align}
				a+b &= (a+b)^2 = a^2 + ab + ba + b^2 = a + ab + ba + b\nonumber\\
				0 &= ab + ba\nonumber\\
				ab &= -ba\\
				a+a &= (a+a)^2 = a^2 + a^2 + a^2 + a^2 = a + a + a+ a\nonumber\\
				0 &= a + a\nonumber\\
				a &= -a\\
				\intertext{Wir setzen (2) in (1) ein und erhalten}
				ab &= ba
			\end{align}
		\end{proof}
		\item \begin{proof}
			In einem Körper $K$ gilt $\forall x \neq 0\exists x^{-1}$ mit $x \cdot x^{-1} = 1_K$.
			Sei $K$ ein Körper.
			Sei $x\in R$.\\
			Fall 1: $x = 0_R$. $0_R^2 = 0_R\checkmark$\\
			Fall 2: $x\neq 0_R$. Da $R$ ein Körper ist, existiert ein $x^{-1}$ mit $x\cdot x^{-1} = 1_R$.
			Ferner ist \begin{align*}
				x*x &= x &&|*x^{-1}\\
				x*x*x^{-1} &= x*x^{-1}\\
				x &= 1_R
			\end{align*}
			$R$ enthält also nur $0_R$ und $1_R$.
		\end{proof}
	\end{enumerate}
	\section*{Aufgabe 4}
	\begin{enumerate}[a)]
		\item Wir zeigen die Ringaxiome.
			\begin{proof}
			\begin{enumerate}[R1)]
				\item Da die Addition komponentenweise definiert ist und $(\mathbb{Q}, +, 0)$ eine abelsche Gruppe ist, muss auch $(\mathbb{Q}\times \mathbb{Q}, +_{K_d}, (0,0))$ eine abelsche Gruppe sein. 
				\item \begin{align*}
					 &((a_0, a_1)\cdot_{K_d} (b_0, b_1)) \cdot_{K_d} (c_0, c_1)\\
					=&(a_0b_0 + a_1b_1d, a_1b_0 + a_0b_1)\cdot_{K_d} (c_0, c_1)\\
					=&\left((a_0b_0 + a_1b_1d)c_0 + (a_1b_0 + a_0b_1)c_1d, (a_1b_0 + a_0b_1)c_0 + (a_0b_0 + a_1b_1d)c_1\right)\\
					=&\left(a_0b_0c_0+a_1b_1c_0d + a_1b_0c_1d + a_0b_1c_1d, a_1b_0c_0 + a_0b_1c_0 + a_0b_0c_1 + a_1b_1c_1d\right)\\
					=&\left(a_0(b_0c_0 + b_1c_1d) + a_1(b_1c_0 + b_0c_1)d, a_0(b_1c_0 + b_0c_1) + a_1(b_0c_0 + b_1c_1d)\right)\\
					=&(a_0,a_1)\cdot_{K_d}(b_0c_0 + b_1c_1d, b_1c_0 + b_0c_1)\\
					=&(a_0,a_1)\cdot_{K_d}((b_0,b_1)\cdot_{K_d}(c_0,c_1))
				\end{align*}
				\item \begin{align*}
					 &(a_0, a_1)\cdot_{K_d}((b_0,b_1) +_{K_d} (c_0, c_1))\\
					=&(a_0, a_1)\cdot_{K_d}(b_0+c_0, b_1+c_1)\\
					=&(a_0(b_0+c_0) + a_1(b_1+c_1)d, a_1(b_0+c_0) + a_0(b_1+c_1))\\
					=&(a_0b_0 + a_1b_1d + a_0c_0 + a_1c_1d, a_1b_0 + a_0b_1 + a_1c_0 + a_0c_1)\\
					=&(a_0b_0 + a_1b_1d, a_1b_0 + a_0b_1) +_{K_d} (a_0c_0 + a_1c_1d, a_1c_0 + a_0c_1)\\
					=&(a_0, a_1) \cdot_{K_d} (b_0,b_1) +_{K_d} (a_0, a_1)\cdot_{K_d} (c_0, c_1)
				\end{align*}
			\end{enumerate}
			Damit der Ring unitär ist, muss es ein neutrales Element $1_{K_d}$ bezüglich der Multiplikation geben. Behauptung $1_{K_d} = (1,0)$.\\
			Bew.:
				\[(1,0)\cdot_{K_d}(a_0, a_1) = (1\cdot a_0 + 0\cdot a_1 \cdot d, 1\cdot a_1 + 0\cdot a_0) = (a_0, a_1)  = (a_0\cdot 1 + a_1\cdot 0 \cdot d, a_1\cdot 1 + a_0\cdot 0) = (a_0, a_1)\cdot_{K_d}(1,0)\]
			\end{proof}
		\item Z.Z.: $\iota: \mathbb{Q} \to K_d, x\mapsto (x,0)$ ist ein unitärer Ringhomomorphismus.
			\begin{proof}
			$\mathbb{Q}$ und $K_d$ sind beides unitäre Ringe. Es gilt $\forall p, q\in \mathbb{Q}:$
			\[\iota(p+q) = (p+q, 0) = (p, 0) +_{K_d} (q, 0) = \iota(p) +_{K_d} \iota(q)\]
			\[\iota(p\cdot q) = (pq, 0) = (pq + 0\cdot 0\cdot d, 0\cdot q + p\cdot 0) = (p, 0)\cdot_{K_d} (q,0) = \iota(p)\cdot_{K_d}\iota(q)\]
			\[\iota(1_\mathbb{Q}) = \iota(1) = (1,0) = 1_{K_d}\]\end{proof}
			Z.Z.: $X^2 -\iota(d) = 0$ hat eine Lösung in $K_d$.
			\begin{proof}
				Behauptung $X = (0,1)$ löst die Gleichung.
				\begin{align*}
					 &X^2 -d\\
					=&(0, 1)\cdot_{K_d}(0, 1) - (d,0)\\
					=&(0^2 + 1^2d, 1\cdot 0 + 0\cdot 1) -(d, 0)\\
					=&(0, 0)\\
				\end{align*}
			\end{proof}
	\item\ \begin{itemize}
		\item[(i)$\to$(ii)] Das folgt sofort aus Lemma 1.15.
		\item[(ii)$\to$(iii)] Wir zeigen die Kontraposition: Wenn die Gleichung $X^2-d = 0$ eine Lösung in $\mathbb{Q}$ hat, dann gibt es $a, b\in K_d\setminus\{0_{K_d}\}$ mit $a\cdot b = 0$. Wenn die Gleichung $X^2-d = 0$ eine Lösung in $\mathbb{Q}$ hat, können wir $a_0, a_1\in \mathbb{Q}, a_1\neq 0$ so wählen, dass $\left(\frac{a_0}{a_1}\right)^2-d = 0$. Sei außerdem $b \neq 0$.\\
		Behauptung: Dann ist $(a_0, a_1) \cdot (b, -\frac{a_1}{a_0} b) = (0,0)$
		\begin{proof}
			\begin{align*}
				 &(a_0, a_1) \cdot \left(b, -\frac{a_1}{a_0} b\right)\\
				=&\left(a_0b - a_1\cdot\frac{a_1}{a_0} b d, - a_0 \cdot \frac{a_1}{a_0}\cdot b + a_1 \cdot b\right)\\
				=&\left(b\frac{a_1^2}{a_0} \left(\frac{a_0^2}{a_1^2} - d\right), -a_1b+ a_1b\right)
				\intertext{Nach unserer Wahl von $a_0, a_1$ gilt $\left(\frac{a_0}{a_1}\right)^2 - d = 0$.}
				=&\left(b\frac{a_1^2}{a_0} \left(0\right), 0\right) = (0,0)
			\end{align*}
		\end{proof}
		$a_1\implies (a_0, a_1) \neq (0,0)$ und $b\neq 0 \implies \left(b, -\frac{a_1}{a_0} b\right) \neq (0,0)$. Es gibt also $a,b\in K_d\setminus\{0_{K_d}\}$ mit $a\cdot b = 0_{K_d}$. Damit ist die Kontraposition $\neg (iii) \to \neg (ii)$ und somit auch die Implikation $(ii) \to (iii)$ bewiesen.
		\item[(iii) $\to$ (i)] Behauptung: $K_d$ ist ein Körper, wenn die Gleichung $X^2-d = 0$ keine Lösung in $\mathbb{Q}$ hat. 
		\begin{proof}
			Da $K_d$ ein unitärer Ring ist, müssen wir nur noch zeigen, dass es zu jedem $(a_0, a_1) \neq (0,0) \in K_d$ ein multiplikatives Inverses $(a_0, a_1)^{-1} \in K_d$ gibt.
			Fall 1: $a_1 = 0\implies a_0\neq 0$. $(a_0, 0) \cdot\left(\frac{1}{a_0}, 0\right) = \left(a_0\frac{1}{a_0} + 0\cdot 0\cdot d, 0\cdot \frac{1}{a_0} + a_0\cdot 0 \right) = (1, 0)\checkmark $. Das Inverse existiert stets, da $a_0\neq 0$.\\
			Fall 2: $a_1\neq 0$: 
			\begin{align*}
				&(a_0, a_1)\cdot_{K_d}\left(\frac{a_0}{a_1^2\left(\left(\frac{a_0}{a_1}\right)^2-d\right)}, \frac{1}{-a_1\left(\left(\frac{a_0}{a_1}\right)^2-d\right)}\right)\\
				=&(a_0, a_1)\cdot_{K_d}\left(\frac{a_0}{a_0^2-a_1^2\cdot d}, \frac{a_1}{a_1^2\cdot d - a_0^2}\right)\\
				=&\left(\frac{a_0^2}{a_0^2-a_1^2\cdot d} +\frac{a_1^2\cdot d}{a_1^2\cdot d - a_0^2}, \frac{a_0a_1}{a_1^2\cdot d - a_0^2} + \frac{a_1a_0}{a_0^2-a_1^2\cdot d}\right)\\
				=&\left(\frac{a_0^2 - a_1^2\cdot d}{a_0^2-a_1^2\cdot d}, \frac{a_0a_1 -a_1a_0}{a_1^2\cdot d - a_0^2}\right)\\
				=&(1,0)\checkmark
			\end{align*}
			Da die Gleichung $X^2-d = 0$ keine Lösung in $\mathbb{Q}$ hat, ist stets  $\left(\frac{a_0}{a_1}\right)^2-d \neq 0$. Da zusätzlich $a_1\neq 0$, ist außerdem stets $a_1^2\left(\left(\frac{a_0}{a_1}\right)^2-d\right) \neq 0$ und $-a_1\left(\left(\frac{a_0}{a_1}\right)^2-d\right)\neq 0$.
			Daher gibt es stets ein Inverses $\left(\frac{a_0}{a_1^2\left(\left(\frac{a_0}{a_1}\right)^2-d\right)}, \frac{1}{-a_1\left(\left(\frac{a_0}{a_1}\right)^2-d\right)}\right)$.
		\end{proof}
	Aus $(i)\implies (ii)\implies (iii) \implies (i)$ folgt $(i)\equals (ii)\equals (iii)$.
	\end{itemize}
	\end{enumerate}
\end{document}