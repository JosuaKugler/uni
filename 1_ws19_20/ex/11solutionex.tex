\documentclass{article}
\usepackage{josuamathheader}
\begin{document}
    \section*{Aufgabe 4}
    \begin{enumerate}[(a)]
        \item Das verdrängte Volumen der Luft ist gleich dem Gesamtvolumen der Luft nach der Ausdehnung minus dem Volumen, das sich noch im Glas befindet. Der Anteil der verdrängten Luft beträgt also $$\frac{V_1-V_0}{V_0} = \frac{N\cdot \frac{k}{p_0} \cdot T_1 - N\cdot \frac{k}{p_0} \cdot T_0}{N\cdot \frac{k}{p_0} \cdot T_0} = \frac{T_1-T_0}{T_0} = \frac{78}{273.16 + 22} = 0.2643$$
        \item Bei konstantem Volumen ist der Druck proportional zur Temperatur. Daher gilt $$p_0 = \lambda \cdot T_1 \implies \lambda = \frac{p_0}{T_1}$$
        $$p_1 = \lambda \cdot T_0$$
        $$\Delta p = p_0 - p_1 = \lambda ( T_1 - T_0) = \lambda \cdot \Delta T = \frac{p_0}{T_1} \cdot \Delta T$$
        $$F = \Delta p \cdot A = \frac{p_0}{T_1} \cdot \Delta T \cdot \pi \frac{d^2}{4} = 101300 \frac{88}{373,16} \cdot 3.14 \cdot 2.5 \cdot 10^{-3} \mathrm{N} = 166 \mathrm{N}$$
    \end{enumerate}
\end{document}